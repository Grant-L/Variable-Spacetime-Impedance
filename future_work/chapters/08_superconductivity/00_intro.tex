\section{Introduction}
The Meissner Effect describes how superconductors completely expel external magnetic fields. Standard BCS theory models this via the abstract formation of Cooper pairs interacting via phonon exchange.

The AVE framework offers a strictly mechanical interpretation. If fundamental electric charges (electrons) are physical, spinning graph-topological gears (flywheels) embedded within the metric, a Cooper Pair is geometrically equivalent to two gears meshing together. In a macroscopic superconducting state or Bose-Einstein Condensate, the thermal vibration (noise) of the lattice drops low enough that billions of these individual flywheels structurally interlock into a single macroscopic, rigid, phase-locked gear-train.

When an external magnetic flux (a torque) attempts to penetrate the superconductor, it physically cannot; turning a single "gear" would require breaking the immense phase-locked shear-modulus rigidity of the entire macroscopic array. Superconductivity is thus modeled as a classical phase transition into mechanical rigidity.
