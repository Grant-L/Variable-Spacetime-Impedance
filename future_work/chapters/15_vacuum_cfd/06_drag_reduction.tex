\section{Engineering Implications: Metric Drag Reduction}
\label{sec:drag_reduction}

If the vacuum behaves as a viscous fluid ($Re_{vac} < \infty$), then any object moving through it experiences **Inductive Drag**. To reach relativistic speeds without infinite energy cost, we must apply the principles of Vacuum Hydrodynamics.

\subsection{The Inductive Drag Coefficient ($C_d$)}
Standard relativity treats inertia as an immutable scalar ($m$). VCFD reveals it as a drag force dependent on geometry:
\begin{equation}
    F_{drag} = \frac{1}{2} \rho_{vac} v^2 C_d A_{cross}
\end{equation}
Where $C_d$ is the Metric Drag Coefficient.
\begin{itemize}
    \item \textbf{Blunt Bodies (High $C_d$):} A standard mass (proton/sphere) creates a large turbulent wake (Back-EMF), maximizing inertia.
    \item \textbf{Streamlined Bodies (Low $C_d$):} A hull shaped to guide vacuum flux around it laminarly can reduce its effective mass.
\end{itemize}

\subsection{Active Flow Control: The Metric "Dimple"}
Just as golf balls use dimples to energize the boundary layer and reduce wake separation, a relativistic vessel can use **Metric Actuators**.
\begin{itemize}
    \item \textbf{Mechanism:} High-frequency toroidal emitters ($\omega \gg \omega_{plasma}$) placed at the leading edge can "pre-stress" the vacuum, lowering the local viscosity.
    \item \textbf{Result:} The vacuum fluid adheres to the hull surface (Laminar Flow) rather than separating into a turbulent wake. This effectively "lubricates" the spacetime trajectory, reducing the inertial mass of the vessel.
\end{itemize}

\begin{figure}[ht]
    \centering
    \includegraphics[width=1.0\textwidth]{metric_streamlining_cfd.png}
    \caption{\textbf{Vacuum Aerodynamics.} Comparison of vacuum flow around a standard mass (Top) vs a Metrically Streamlined hull (Bottom). The blunt body creates a chaotic wake of gravitational turbulence (high inertia). The streamlined body maintains laminar flow, minimizing Inductive Drag.}
    \label{fig:metric_streamlining}
\end{figure}