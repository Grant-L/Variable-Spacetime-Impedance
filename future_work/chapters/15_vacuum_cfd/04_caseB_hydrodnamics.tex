\section{Case Study B: Warp Drive Hydrodynamics}
\label{sec:warp_drive_cfd}

The Alcubierre Warp Drive is often described geometrically. In VCFD, it is a **Supersonic Pressure Vessel**.

\subsection{The Moving Pressure Gradient}
A warp drive functions by creating a localized pressure gradient: High Pressure (Compression) in the front, Low Pressure (Rarefaction) in the rear. This propels the bubble through the fluid.
\begin{equation}
    v_{bubble} \propto \Delta P = P_{rear} - P_{front}
\end{equation}

\subsection{The Bow Shock (Cherenkov Radiation)}
When the bubble velocity $v_b$ exceeds the vacuum sound speed $c$ ($Mach > 1$), a conical **Bow Shock** forms at the leading edge.
\begin{itemize}
    \item \textbf{Hazard:} This shockwave continuously accumulates high-energy vacuum fluctuations (Hawking Radiation).
    \item \textbf{The Wake:} Behind the bubble, a turbulent low-pressure wake forms. In standard physics, we detect these as Gravitational Waves.
\end{itemize}

\begin{figure}[ht]
    \centering
    \includegraphics[width=1.0\textwidth]{warp_wake_cfd.png}
    \caption{\textbf{Warp Drive Hydrodynamics.} Simulation of a superluminal pressure source moving through the vacuum fluid. \textbf{A:} The Bow Shock (Mach Cone) where fluid piles up. \textbf{B:} The Laminar Bubble where the ship resides. \textbf{C:} The Turbulent Wake trailing the vessel.}
    \label{fig:warp_wake}
\end{figure}