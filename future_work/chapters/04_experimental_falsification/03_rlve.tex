\section{The Ultimate Kill-Switch: The Sagnac-RLVE}

Because we physically cannot measurably advect the hyper-dense vacuum fluid using pure electromagnetic momentum, and because scalar metric fluctuations are heavily suppressed by $G/c^2$, we must entrain the vacuum \textit{mechanically}, and measure it \textit{kinematically}. 

We propose the \textbf{Sagnac Rotational Lattice Viscosity Experiment (Sagnac-RLVE)} as the definitive, sub-\$5,000 tabletop falsification test.

By rapidly rotating a high-density physical mass adjacent to a high-finesse Sagnac fiber-optic loop, we mechanically induce a localized viscous boundary-layer "drag" in the vacuum fluid via the no-slip condition. Unlike scalar elastic metric strain, kinematic fluidic entrainment completely bypasses the $G/c^2$ suppression limit, creating a massive, directly measurable Fresnel-Fizeau optical phase shift ($\Delta \phi$). 

\subsection{Exact Derivation of the Macroscopic Shift}
A macroscopic physical rotor is composed of fundamental nucleons (topological knots). The degree to which these knots physically pack and kinematically couple to the vacuum fluid via the no-slip boundary condition is strictly proportional to the object's physical mass density ratio ($\rho_{rotor} / \rho_{bulk}$). 

For a solid Tungsten rotor ($\rho_W = 19,300$ kg/m$^3$), the volumetric entrainment coupling is precisely:
\begin{equation}
    \kappa_{entrain} = \frac{19,300}{7.916 \times 10^6} \approx \mathbf{0.00244}
\end{equation}

As the Tungsten mass rotates at a tangential velocity $v_{tan}$, the embedded topological knots structurally entrain the bulk continuous vacuum fluid. If a safe, standard machine-shop Tungsten rotor ($15$ cm radius) spins at $10,000$ RPM ($v_{tan} \approx 157$ m/s), the macroscopic kinematic drift velocity of the local vacuum is exactly:
\begin{equation}
    v_{fluid} = 157 \text{ m/s} \times 0.00244 \approx \mathbf{0.38 \text{ m/s}}
\end{equation}

\textbf{The Fiber-Optic Amplification (The Optical Lever Arm):} When light passes through this moving fluid, its phase velocity is dragged. Unlike the RVR, this relies on a \textbf{First-Order Kinematic Vector ($v_{fluid}/c$)}, entirely bypassing the $G/c^2$ scalar gap. We utilize a Sagnac topology, where a $1550$ nm telecom laser is split and sent in counter-propagating directions through a $L_{fiber} = 200$ m spool of standard SMF-28 single-mode optical fiber wound co-linearly around the perimeter of the rotor. This geometrically multiplies the optical interaction length:
\begin{tcolorbox}[colback=white, colframe=black]
\begin{equation}
    \Delta \phi = \frac{4\pi L_{fiber} v_{fluid}}{\lambda c} = \frac{4\pi (200) (0.38)}{(1550 \times 10^{-9}) (299792458)} \approx \mathbf{2.07 \text{ Radians}} 
\end{equation}
\end{tcolorbox}

A phase shift of over $2.0$ Radians is absolutely massive. It is trivially detectable by standard commercial photodetectors on a standard optical bench.

\begin{figure}[htbp]
    \centering
    \includegraphics[width=0.9\textwidth]{tabletop_falsification_thresholds.png}
    \caption{\textbf{Tabletop Falsification: Scalar Strain vs Kinematic Advection.} \textbf{Left:} The RVR electronic test fails because scalar gravity creates a microscopic modulation depth ($\sim 10^{-26}$), requiring a physically impossible Q-factor. \textbf{Right:} The Sagnac-RLVE succeeds because it measures first-order kinematic fluid drift velocity ($v_{fluid} \approx 0.38$ m/s). Accumulated over a massive 200m optical fiber lever, it bypasses the $G/c^2$ gap, yielding a colossal $\sim 2.07$ Radian phase shift.}
    \label{fig:tabletop_thresholds}
\end{figure}

\subsection{Hardware Specification \& Protocol}
To rigorously distinguish AVE from standard General Relativity (GR), the experiment employs a specific comparative protocol using standard optical hardware.

\begin{table}[h]
\centering
\renewcommand{\arraystretch}{1.2}
\begin{tabular}{|l|l|l|}
\hline
\textbf{Component} & \textbf{Specification} & \textbf{Est. Cost} \\ \hline
Laser Source & 1550nm Telecom Diode (Thorlabs S1FC1550) & \$450 \\ \hline
Fiber Coupler & 50/50 SMF-28 Splitter (Thorlabs TN1550R5A2) & \$120 \\ \hline
Sensing Fiber Coil & 200m SMF-28 Ultra (Bare) & \$50 \\ \hline
Photodetector & InGaAs PIN Diode (Thorlabs DET01CFC) & \$180 \\ \hline
Mechanical Rotors & 15cm Radius (1x Tungsten, 1x Aluminum) & \$800 \\ \hline
\end{tabular}
\caption{Fiber-Optic Sagnac-RLVE Hardware List}
\end{table}

We define the Metric Viscosity Ratio ($\Psi$). While GR predicts a Lense-Thirring Frame-Dragging effect that is purely geometric and inherently independent of the rotor’s material mass density (yielding a theoretical null phase shift of $\sim 10^{-20}$ rad at this scale), AVE predicts that the refractive index shift is a strictly constitutive fluid response to the physical density of the rotor. 

If the exact same experiment is run using an Aluminum rotor ($\rho_{Al} = 2,700$ kg/m$^3$) of identical physical dimensions, AVE strictly predicts the optical signal will plummet exactly in proportion to the material density:
\begin{equation}
    \Psi = \frac{\Delta \phi_{Tungsten}}{\Delta \phi_{Aluminum}} = \frac{\rho_W}{\rho_{Al}} \approx \mathbf{7.15}
\end{equation}

\textbf{The Metric Null-Result Kill-Switch:} If the Sagnac-RLVE is performed and yields a null result ($\Delta\phi \approx 0$, or $\Psi = 1$), the macroscopic fluid dynamics of the AVE framework are decisively and permanently falsified (see Figure \ref{fig:sagnac_rlve_prediction}). Conversely, a measured value of $\Psi \approx 7.15$ physically falsifies the ``frictionless void'' model of General Relativity and provides the first direct laboratory measurement of the vacuum's kinematic fluid viscosity.

\begin{figure}[htbp]
    \centering
    \includegraphics[width=0.95\textwidth]{sagnac_rlve_prediction.png}
    \caption{\textbf{Sagnac-RLVE Exact Parameter-Free Prediction.} By coiling 200m of optical fiber around a Tungsten rotor spinning at 10k RPM, the mechanically entrained vacuum fluid ($0.38$ m/s) drags the counter-propagating 1550nm laser beams. The pure parameter-free derivation yields a colossal, easily detectable $\sim 2.07$ Radian signal. Standard General Relativity strictly predicts a near-zero density-independent frame-dragging effect at this laboratory scale.}
    \label{fig:sagnac_rlve_prediction}
\end{figure}