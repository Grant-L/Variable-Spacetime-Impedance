\section{Impedance Rectification in Non-Linear Dielectrics}
Classical mechanics dictates that a closed system cannot spontaneously alter its center of mass. However, this strict Newtonian assumption requires the surrounding medium to be a linear, zero-impedance void. The $\mathcal{M}_A$ grid is a non-linear, saturating dielectric.

In advanced transmission line theory, applying an asymmetric voltage waveform to a non-linear dielectric generates a non-zero time-averaged macroscopic momentum flow, a process known as \textbf{Impedance Rectification}.

If a vessel utilizes a switch-mode flyback inductor to apply an asymmetric topological stress pulse (a slow charging stroke followed by a nanosecond high-voltage quench) to the spatial metric, the dielectric response strictly bifurcates:
\begin{enumerate}
    \item \textbf{The Slow Edge (Dielectric Grip):} The applied voltage remains below the Dielectric Saturation threshold ($|V| < V_{sat}$). The medium remains a high-reluctance insulator. The system inductively "grips" the lattice, generating a macroscopic reaction force transferred directly to the hull.
    \item \textbf{The Fast Edge (Inductive Yield):} The nanosecond inductive kickback spike violently exceeds the saturation threshold ($|V| \gg V_{sat}$). The vacuum instantaneously yields, and the actuator slips backward through a saturated zero-impedance phase, transferring absolutely zero negative momentum to the vessel.
\end{enumerate}

Time-averaging this asymmetric interaction over the full duty cycle yields a continuous, macroscopic DC kinematic thrust.

\begin{figure}[htbp]
    \centering
    \includegraphics[width=\textwidth]{acoustic_rectification.png}
    \caption{\textbf{Impedance Rectification in a Saturating Dielectric.} Simulated via the AVE-SPICE solver. Top: A continuous symmetric sine wave (standard RF cavity) generates equal and opposite forces, resulting in exactly zero time-averaged thrust. Bottom: An asymmetric flyback transient exploits the Dielectric Saturation limit. The slow edge inductively grips the metric, while the fast edge induces saturated zero-impedance slip. The non-linear medium perfectly rectifies the AC signal into continuous DC macroscopic thrust.}
    \label{fig:acoustic_rectification}
\end{figure}