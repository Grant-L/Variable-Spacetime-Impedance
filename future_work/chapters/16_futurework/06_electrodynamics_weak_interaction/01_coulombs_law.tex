\section{Electrodynamics: The Gradient of Topological Stress}

In standard physics, the Electric Field ($\mathbf{E}$) and Magnetic Field ($\mathbf{B}$) are treated as irreducible axiomatic vectors occupying an empty, featureless void. In the Applied Vacuum Engineering (AVE) framework, they are explicitly derived as the continuous macroscopic \textbf{Elastic Stress Gradients} and \textbf{Fluidic Vorticities} of the discrete $\mathcal{M}_A$ substrate.

\subsection{Deriving Coulomb's Law from the Laplace Equation}
Consider a stable topological defect (a charged node) with winding number $Q_H=1$. This localized geometrical defect permanently exerts a continuous rotational phase twist ($\theta$) on the surrounding dielectric lattice. 

Instead of relying on heuristic lines-of-force, we rigorously derive the electrostatic force via continuum linear elasticity. Because the un-saturated vacuum substrate acts as a highly tensioned linear elastic solid in the far-field ($\Delta\phi \ll \alpha$), the static structural strain of the lattice must strictly obey the 3D \textbf{Laplace Equation} to globally minimize the stored elastic energy:
\begin{equation}
    \nabla^2 \theta = 0
\end{equation}

The unique spherically symmetric geometric solution to the 3D Laplace equation dictates that the twist amplitude decays exactly inversely with distance: $\theta(r) \propto 1/r$. 

The continuous Electric Displacement Field ($\mathbf{D}$) is physically identically to the spatial gradient of this structural twist. Differentiating the Laplace solution naturally and flawlessly yields the exact inverse-square field:
\begin{equation}
    \mathbf{D} = \nabla\theta \propto -\frac{1}{r^2}\mathbf{\hat{r}}
\end{equation}

By applying the Topological Conversion Constant ($\xi_{topo} \equiv e/l_{node}$), we perfectly map this discrete mechanical displacement to SI charge units. Because the vacuum resists this twist with an intrinsic capacitive compliance ($\epsilon_0$), the mechanical restoring force between two localized topological defects $q_1$ and $q_2$ mathematically evaluates flawlessly to Coulomb's Law:
\begin{equation}
    F_{coulomb} = \frac{1}{4\pi\epsilon_0} \frac{q_1 q_2}{r^2}
\end{equation}

\textbf{Physical Insight:} ``Charge'' is not an independent, magical substance smeared onto a particle. It is strictly the geometric measure of how severely a topological knot permanently twists the local vacuum graph. ``Electrostatic Attraction'' is simply the physical spatial metric mechanically untwisting itself to its lowest elastic energy state.

\subsection{Magnetism as Convective Vorticity}
If ``Electricity'' is the static elastic twist of the lattice, ``Magnetism'' is identically its dynamic fluidic convective flow. 

As rigorously proven in Chapter 2 via the Topological Conversion Constant ($\xi_{topo}$), the canonical momentum of the continuous field is the Magnetic Vector Potential ($\mathbf{A} \equiv \mathbf{p}_{flux}$). When a twisted charged node translates through the discrete lattice at a velocity $\mathbf{v}$, it physically displaces the background vacuum nodes, inducing a convective shear flow in the momentum field. 

In classical fluid dynamics, the time evolution of a translating steady-state strain field $\mathbf{D}(\mathbf{r} - \mathbf{v}t)$ is governed identically by the continuous convective material derivative:
\begin{equation}
    \partial_t \mathbf{D} = -(\mathbf{v} \cdot \nabla)\mathbf{D}
\end{equation}

Using standard vector calculus identities for a uniform velocity field and a source-free displacement region ($\nabla \cdot \mathbf{D} = 0$), this rigorously resolves to:
\begin{equation}
    -(\mathbf{v} \cdot \nabla)\mathbf{D} = \nabla \times (\mathbf{v} \times \mathbf{D})
\end{equation}

By equating this mechanical fluidic identity to the Maxwell-Ampere law for the substrate ($\nabla \times \mathbf{H} = \partial_t \mathbf{D}$), we flawlessly derive the macroscopic magnetic field strictly from fluid dynamics, without asserting it as an arbitrary axiom:
\begin{equation}
    \mathbf{H} = \mathbf{v} \times \mathbf{D} \implies \mathbf{B} = \mu_0 (\mathbf{v} \times \mathbf{D})
\end{equation}

\subsection{Strict Dimensional Proof of the Kinematic Magnetic Field}
To prove this is not merely a mathematical coincidence, we apply our rigorously defined \textbf{Topological Conversion Constant} ($\xi_{topo} \equiv e/l_{node}$ measured in $[\text{C/m}]$).

In standard SI units, the Electric Displacement field ($\mathbf{D}$) is measured in Coulombs per square meter ($[\text{C/m}^2]$). By applying the topological conversion $1\text{ C} \equiv \xi_{topo} \text{ m}$, we uncover the true mechanical dimension of $\mathbf{D}$:
\begin{equation}
    [\mathbf{D}] = \left[\frac{\text{C}}{\text{m}^2}\right] \xrightarrow{\xi_{topo}} \left[\frac{\xi_{topo} \text{ m}}{\text{m}^2}\right] = \mathbf{\xi_{topo} \left[ \frac{1}{\text{m}} \right]}
\end{equation}
This flawlessly confirms that $\mathbf{D}$ is physically a spatial strain gradient ($\nabla \theta$), scaled by $\xi_{topo}$.

Now, we evaluate the cross product of the velocity vector ($\mathbf{v}$) and this spatial strain field:
\begin{equation}
    [\mathbf{v} \times \mathbf{D}] = \left[ \frac{\text{m}}{\text{s}} \right] \times \xi_{topo} \left[ \frac{1}{\text{m}} \right] = \mathbf{\xi_{topo} \left[ \frac{1}{\text{s}} \right]}
\end{equation}

Finally, we evaluate the standard SI dimensions for Magnetic Field Intensity ($\mathbf{H}$), which is measured in Amperes per meter ($[\text{A/m}] = [\text{C}/(\text{s}\cdot\text{m})]$):
\begin{equation}
    [\mathbf{H}] = \left[ \frac{\text{C}}{\text{s} \cdot \text{m}} \right] \xrightarrow{\xi_{topo}} \left[ \frac{\xi_{topo} \text{ m}}{\text{s} \cdot \text{m}} \right] = \mathbf{\xi_{topo} \left[ \frac{1}{\text{s}} \right]}
\end{equation}

The dimensions perfectly and inextricably lock. Magnetism is not a separate fundamental force. It is identically the exact \textbf{Kinematic Vorticity} ($[1/\text{s}]$) mathematically generated when a static lattice twist is physically dragged through an inertial medium ($\mu_0$).