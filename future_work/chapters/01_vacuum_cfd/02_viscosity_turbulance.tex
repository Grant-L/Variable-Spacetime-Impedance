\section{Mutual Inductance and Turbulence: The Origin of $h$}
\label{sec:vacuum_viscosity}

In standard quantum mechanics, the Planck constant ($h$) is a fundamental scalar of unknown origin. In VCFD, we identify it as the **Eddy Mutual Inductance** of the vacuum network.

\subsection{The $k-\epsilon$ Turbulence Model}
At the microscopic scale ($Kn \sim 1$), the vacuum is not smooth; it is a frothing sea of nodal interactions. We model this using the standard $k-\epsilon$ turbulence model:
\begin{equation}
    \eta_{eddy} = \rho C_{\mu} \frac{k^2}{\epsilon} \approx h
\end{equation}
Where $k$ is the turbulent kinetic energy and $\epsilon$ is the dissipation rate.
\begin{itemize}
    \item \textbf{Implication:} "Quantum Uncertainty" is simply the isotropic turbulence of the background network. A particle cannot have a definite position and momentum simultaneously because it is being buffeted by the "Brownian Motion" of the vacuum nodes.
    \item \textbf{The Laminar Transition:} At low energies, the turbulence averages out, and the vacuum appears smooth (Classical Physics). At high energies (Planck scale), the Reynolds number increases, and the flow becomes chaotic (Quantum Foam).
\end{itemize}