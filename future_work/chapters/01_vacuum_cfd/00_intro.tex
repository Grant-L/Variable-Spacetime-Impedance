
\section{Introduction: From Lattice to Liquid}
\label{sec:vcfd_intro}

Throughout this text, we have treated the vacuum as a discrete graph of nodes ($M_A$). However, just as the discrete collisions of water molecules average out to form a smooth network, the stochastic interactions of vacuum nodes average out to form a "Spacetime Network."

\subsection{The Continuum Hypothesis}
Standard General Relativity assumes the vacuum is a continuum at all scales ($l_P \to 0$). VCFD adopts the **Knudsen Number ($Kn$)** criterion used in hydrodynamics:
\begin{equation}
    Kn = \frac{l_P}{L}
\end{equation}
\begin{itemize}
    \item \textbf{Macroscopic ($Kn \ll 1$):} At astrophysical scales, the discrete lattice behaves as a continuous, inviscid network. We can use the Navier-Stokes equations to solve for gravity and warp mechanics.
    \item \textbf{Microscopic ($Kn \sim 1$):} At the Planck scale, the network approximation breaks down, and we must return to the discrete node mechanics (Quantum behavior).
\end{itemize}

\subsection{The Vacuum Reynolds Number}
We define the flow regime of the vacuum using the Reynolds Number ($Re_{vac}$):
\begin{equation}
    Re_{vac} = \frac{\rho_{E} \cdot v \cdot L}{\eta_{vac}}
\end{equation}
Where $\eta_{vac} \approx \alpha$ (Fine Structure Mutual Inductance).
\begin{itemize}
    \item \textbf{Laminar Flow ($Re < 1$):} Empty space. Signals propagate linearly (Light).
    \item \textbf{Turbulent Flow ($Re \gg 1$):} High energy density (Mass). The network creates self-sustaining vortices (Particles) and chaotic wakes (Gravity).
\end{itemize}