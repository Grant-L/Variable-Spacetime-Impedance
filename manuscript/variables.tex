\chapter*{Nomenclature and Fundamental Constants}
\addcontentsline{toc}{chapter}{Nomenclature and Fundamental Constants}

\section*{Universal Hardware Constants}
The following constants define the constitutive, physical properties of the \textit{Discrete Amorphous Manifold} ($M_A$). In the **Vacuum Engineering framework**, these are the bulk engineering specifications of the vacuum substrate.

\begin{tabularx}{\textwidth}{|l|X|l|l|}
\hline
\textbf{Symbol} & \textbf{Name} & \textbf{Value (SI)} & \textbf{Hardware Role} \\ \hline
$\Lvac$ & Lattice Inductance & $\approx 1.26 \times 10^{-6}$ H/m & $\mu_0$ (Inertial Density) \\ \hline
$\Cvac$ & Lattice Capacitance & $\approx 8.85 \times 10^{-12}$ F/m & $\epsilon_0$ (Elastic Potential) \\ \hline
$\Zvac$ & Characteristic Impedance & $\approx 376.73\,\Omega$ & $\sqrt{\Lvac/\Cvac}$ (Base Load) \\ \hline
$\lp$ & Lattice Pitch & $\approx 1.62 \times 10^{-35}$ m & Nodal Spacing (Nyquist Limit) \\ \hline
$\Wcut$ & Saturation Frequency & $2 / \sqrt{\Lvac \Cvac \lp^2}$ & Global Slew Rate ($c/\lp$) \\ \hline
\end{tabularx}

\section*{Emergent Variables and Tensors}
These variables describe the behavior of topological defects (matter) and signal propagation within the hardware layers.

\begin{itemize}
    \item \textbf{$\epsilon_{\mu\nu}$ (Metric Strain Tensor)}: Quantifies the physical displacement of manifold nodes from ground-state equilibrium. \citestart This recasts the abstract "curvature" of General Relativity as a measurable mechanical strain of the hardware nodes\cite{5}\citeend.
    \item \textbf{$h$ (Topological Helicity)}: An integer representing the quantized phase-twist of a defect. \citestart This is the mechanical identity of \textbf{Electric Charge} ($q$)\cite{7}\citeend.
    \item \textbf{$\nu_{sat}$ (Saturation Threshold)}: The frequency at which a node enters a non-linear regime, clamping energy into a stationary standing wave. \citestart This "trapped flux" is measured as \textbf{Rest Mass}\cite{8}\citeend.
    \item \textbf{$v_g$ (Group Velocity)}: The propagation speed of energy through the lattice nodes, subject to frequency-dependent attenuation:
    \begin{equation}
        v_g = c \sqrt{1 - \left(\frac{\omega}{\Wcut}\right)^2}
    \end{equation}
\end{itemize}

\section*{Operational Acronyms}
\begin{tabularx}{\textwidth}{|l|l|X|}
\hline
\textbf{Acronym} & \textbf{Full Term} & \textbf{Mechanical Mapping} \\ \hline
\textbf{AVE} & Applied Vacuum Electrodynamics & The methodology of treating gravity as a dielectric phenomenon. \\ \hline
\textbf{B-EMF} & Back-Electromotive Force & \citestart The hardware origin of \textbf{Inertia}\cite{9}\citeend. \\ \hline
\textbf{CBE} & Chiral Bias Equation & \citestart The law governing spin-dependent impedance (Matter/Antimatter)\cite{10}\citeend. \\ \hline
\textbf{MA} & Amorphous Manifold & \citestart The stochastic, discrete hardware substrate\cite{11}\citeend. \\ \hline
\textbf{TVS} & Transient Voltage Suppressor & Analogy for high-frequency \textbf{Weak Force} clamping. \\ \hline
\end{tabularx}