\chapter{The Neutrino Sector: Chiral Defects}
\label{ch:neutrino_sector}

\section{The Twisted Unknot ($0_1$)}
\label{sec:twisted_unknot}

Neutrinos are the most abundant matter particles in the universe, yet they interact weakly with everything. In Vacuum Engineering, we identify them not as "Matter Knots" but as \textbf{Twisted Unknots} ($0_1$).

\subsection{Mass Without Charge}
A fundamental question is: How can a particle have mass but zero electric charge?

\begin{itemize}
    \item \textbf{Charge ($q$):} Defined by the Winding Number ($N$) around a singularity. A knot must cross itself to trap flux.
    \item \textbf{Mass ($m$):} Defined by the stored Lattice Stress energy.
\end{itemize}
The Neutrino is a simple closed loop with \textbf{Internal Twist} (Torsion) but \textbf{No Knot} (Crossing Number $C=0$).
\begin{equation}
    q_\nu = 0 \quad (\text{No Crossings})
\end{equation}
\begin{equation}
    m_\nu \propto \tau_{twist}^2 \ll m_e \quad (\text{Torsional Stress only})
\end{equation}
Because torsional stress stores far less energy than the inductive bending of a knot, the neutrino mass is orders of magnitude smaller than the electron mass ($\approx 0.1$ eV vs $0.5$ MeV).

\subsection{Ghost Penetration}
Why do neutrinos pass through light-years of lead?
\begin{itemize}
    \item \textbf{Cross-Section:} A knotted particle (Electron/Proton) has a large "Inductive Cross-Section" due to its magnetic moment. It drags on the vacuum.
    \item \textbf{Twist Soliton:} The neutrino is a localized twist without a magnetic moment. It slides through the lattice impedance ($Z_0$) without generating a wake. It only interacts when it hits a node directly (Weak Interaction).
\end{itemize}