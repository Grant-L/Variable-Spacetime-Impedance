\section{Gravity as Metric Refraction}
\label{sec:gravity}

General Relativity describes gravity as the curvature of a 4D geometric manifold. SVF describes it as a gradient in the **Variable Spacetime Impedance**.

\section{Derivation of the Vacuum Impedance Tensor ($\mathbf{L}_{ij}$)}

\subsection{Objective}
To derive the components of the Inductance Tensor $\mathbf{L}$ such that the propagation of electromagnetic flux through the VSI lattice follows the exact geodesics predicted by the Schwarzschild metric of General Relativity.

\subsection{The Schwarzschild Metric}
The invariant line element around a static, spherically symmetric mass $M$ is given by:
$$ ds^2 = -c^2 \left(1 - \frac{r_s}{r}\right) dt^2 + \left(1 - \frac{r_s}{r}\right)^{-1} dr^2 + r^2 d\Omega^2 $$
Where $r_s = \frac{2GM}{c^2}$ is the Schwarzschild radius.

\subsection{The VSI Tensor Correspondence}
In the Variable Spacetime Impedance framework, the metric tensor $g_{\mu\nu}$ is mapped to the electromagnetic material properties of the vacuum lattice. Specifically, the spatial curvature $g_{rr}$ corresponds to the \textbf{Radial Inductance} ($L_{rr}$), while time dilation $g_{tt}$ corresponds to the \textbf{Geometric Mean Impedance} of the node.

To mimic the optical geometry of General Relativity, the vacuum behaves as an anisotropic dielectric (a "Plebanski Medium"). We define the local Inductance Tensor $\mathbf{L}$ in spherical coordinates $(r, \theta, \phi)$ as:

$$ \mathbf{L} = \begin{pmatrix} L_{rr} & 0 & 0 \\ 0 & L_{\theta\theta} & 0 \\ 0 & 0 & L_{\phi\phi} \end{pmatrix} $$

\subsection{Component Derivation}
The effective refractive index $n$ for a wave traveling through the lattice is derived from the impedance $Z = \sqrt{L/C}$. For a vacuum with constant capacitance $C_0$ (to satisfy the Charge Scaling patch), all variations must be borne by $\mathbf{L}$.

\textbf{1. Radial Component ($L_{rr}$):}
The radial metric component is $g_{rr} = \left(1 - \frac{r_s}{r}\right)^{-1}$.
This represents the "stretching" of space in the radial direction. In the lattice, this manifests as an increased inductance (inertia) for flux moving radially.
$$ L_{rr} = L_0 \cdot g_{rr} = L_0 \left(1 - \frac{2GM}{rc^2}\right)^{-1} $$

\textbf{2. Transverse Components ($L_{\theta\theta}, L_{\phi\phi}$):}
In the standard Schwarzschild metric, the transverse space is not stretched locally (it scales with $r^2$ purely geometrically). However, relative to the radial stretching, the lattice exhibits anisotropy. 
$$ L_{\theta\theta} = L_{\phi\phi} = L_0 $$

\subsection{The Resulting Tensor Field}
Thus, the gravity of a point mass $M$ is described not by a scalar field, but by the deformation of the Inductance Tensor:

$$ \mathbf{L}_{Schwarzschild} = L_0 \begin{pmatrix} \frac{1}{1 - \frac{2GM}{rc^2}} & 0 & 0 \\ 0 & 1 & 0 \\ 0 & 0 & \sin^2\theta \end{pmatrix} $$
*(Note: The $\sin^2\theta$ term accounts for the spherical coordinate basis. In a local orthonormal frame, the transverse terms are unity.)*

\subsection{Implication for Light Bending}
The total bending angle $\delta$ of a light ray passing the sun is the sum of the bending due to time dilation (scalar potential) and spatial curvature (tensor deformation).
$$ \delta = \delta_{time} + \delta_{space} $$
In VSI v2.0 (Scalar $L$), $\delta_{space} = 0$, yielding $\delta = \frac{2GM}{rc^2}$.
In VSI v3.0 (Tensor $\mathbf{L}$), the anisotropy $L_{rr} \neq L_{\theta\theta}$ creates a birefringence effect that accounts for the missing spatial curvature term.
$$ \delta_{VSI} = \frac{2GM}{rc^2} + \frac{2GM}{rc^2} = \frac{4GM}{rc^2} $$
This successfully recovers the Einsteinian prediction and passes the Solar Eclipse audit.