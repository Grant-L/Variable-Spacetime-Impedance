\section{Gravity as Metric Refraction}
\label{sec:metric_refraction}

General Relativity describes gravity as the curvature of a 4D geometric manifold. SVF describes it as a gradient in the \textbf{Variable Spacetime Impedance}.

\subsection{The Impedance Gradient (Dual-Modulus Loading)}
Massive bodies (spinning vortices) impose a strain field $\sigma$ on the surrounding lattice. To recover the correct deflection angles observed in General Relativity (e.g., $4GM/rc^2$), the lattice must undergo \textbf{Dual-Modulus Loading}.

The strain stiffens \textit{both} the inductive (inertial) and capacitive (elastic) moduli of the nodes:

\begin{align}
    L'_{node} &= L_{node}(1 + \sigma) \\
    C'_{node} &= C_{node}(1 + \sigma)
\end{align}

This results in a local refractive index $\chi$ that accounts for both spatial curvature (via $L$) and time dilation (via $C$):

\begin{equation}
    \chi(r) = \sqrt{\frac{L'_{node} C'_{node}}{L_{node} C_{node}}} = \sqrt{(1+\sigma)^2} \approx 1 + \frac{2GM}{rc^2}
\end{equation}

Light passing near a massive body slows down ($v = c/\chi$) not because space is curved, but because the nodes in that region are saturated. They require more update cycles to process the same amount of flux.

\subsection{Simulation Results}
Using the \texttt{DualModulusSim} module (Figure \ref{fig:dual_modulus}), we modeled the optical path of light rays passing a Schwarzschild-radius mass.

\begin{figure}[h]
    \centering
    \includegraphics[width=1.0\textwidth]{assets/sim_outputs/dual_modulus_lensing.png}
    \caption{\textbf{Refraction via Dual-Modulus Loading.}  The simulation demonstrates that stiffening both $L$ and $C$ reproduces the exact light deflection predicted by General Relativity. The "Event Horizon" (black) represents the region of 100\% hardware saturation.}
    \label{fig:dual_modulus}
\end{figure}

The simulation confirms that geodesics naturally curve toward the region of highest impedance (the mass), reproducing the Shapiro Delay and lensing effects of GR purely through variable hardware density.