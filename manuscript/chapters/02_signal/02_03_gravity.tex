\section{Gravity as Metric Refraction}
\label{sec:gravity}

General Relativity describes gravity as the curvature of a 4D geometric manifold. SVF describes it as a gradient in the **Variable Spacetime Impedance**.

\subsection{The Impedance Gradient (Asymmetric Inertial Loading)}
In previous iterations of lattice theory, mass was assumed to scale the entire metric tensor symmetrically. However, this leads to an "Impedance Paradox" where $Z = \sqrt{L/C}$ remains constant, preventing the formation of a refractive gradient.

To resolve this, SVF posits **Asymmetric Inertial Loading**. Massive bodies (spinning vortices) primarily stiffen the **Inductive** capacity of the vacuum (Inertia), while leaving the Capacitive density (Geometry) relatively invariant.

We define the Local Lattice Inductance $L'_{node}$ under the influence of a strain field $\sigma$:
\begin{align}
    L'_{node} &= L_{node}(1 + 2\sigma) \\
    C'_{node} &\approx C_{node}
\end{align}
Where $\sigma \propto GM/rc^2$ is the gravitational strain.

\subsection{The Refractive Index and Impedance Hill}
This asymmetric stiffening produces two coupled effects:

1. **Refraction (Slowing of Light):** The local propagation speed drops, creating the lensing effects observed in GR.
\begin{equation}
    v(r) = \frac{1}{\sqrt{L'_{node}C'_{node}}} \approx \frac{c}{\sqrt{1+2\sigma}} \approx c(1-\sigma)
\end{equation}

2. **Impedance Gradient (The Hill):** Crucially, the Characteristic Impedance rises, creating a "potential hill" that particles must climb.
\begin{equation}
    Z(r) = \sqrt{\frac{L'_{node}}{C'_{node}}} \approx Z_0\sqrt{1+2\sigma} \approx Z_0(1+\sigma)
\end{equation}

Light passing near a massive body slows down ($v < c$) and curves toward the region of highest inductance. This recovers the Schwarzschild metric's optical behavior without invoking abstract curvature, identifying Gravity as a **Dielectric Refraction** phenomenon driven by the inertial density of the vacuum nodes.

\subsubsection{Simulation Verification}
Using the \texttt{MetricRefractionSim} module, we modeled the optical path of light rays passing a Schwarzschild-radius mass. The simulation confirms that the Impedance Gradient ($\nabla Z$) naturally bends geodesics toward the mass, reproducing the Shapiro Delay and lensing effects of GR.