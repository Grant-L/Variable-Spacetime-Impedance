\section{The Origin of Inertia as Back-EMF}
\label{sec:inertia}

In classical mechanics, inertia is an axiom ($F=ma$). In the SVF framework, inertia is an emergent \textbf{Back-Electromotive Force (B-EMF)}.

Because the manifold is inductive ($L_{node} = \mu_0$), any attempt to change the flux state of a node (acceleration) is met with an opposing force generated by the lattice.

\begin{axiombox}[The Inertial B-EMF]
    Inertia is the manifold's inductive resistance to the change in flux density associated with an accelerating topological defect. The "Force" required to move a mass is simply the work required to overcome the lattice B-EMF:
    \begin{equation}
        \mathcal{E}_{back} = -L_{node} \frac{d\Phi}{dt}
    \end{equation}
\end{axiombox}