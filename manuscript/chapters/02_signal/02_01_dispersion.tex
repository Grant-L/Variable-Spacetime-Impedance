\section{The Vacuum Dispersion Relation}
\label{sec:dispersion}

In the Standard Model, the speed of light $c$ is an axiomatic constant. In VSI, it is the \textbf{Global Slew Rate Limit} of the hardware.

\subsection{Mode 1: Linear Flux (Light)}
Photons represent sub-saturation perturbations of the vacuum potential ($U \ll U_{sat}$). For wavenumbers $k$ below the Nyquist limit ($k \ll \pi/l_P$), the lattice behaves as a linear transmission line with constant group velocity:
\begin{equation}
    v_g = \frac{1}{\sqrt{L_{node}C_{node}}} = c
\end{equation}
This confirms that $c$ is the maximum signaling rate of the dielectric medium.

\subsection{Mode 2: Topological Defects (Matter)}
Matter particles are stable **Topological Knots** (vortices) in the field. Unlike free flux, these structures impose a continuous computational load on the nodes, defined as the \textbf{Intrinsic Spin Frequency} ($\omega_{spin}$).

As a defect accelerates, its update rate approaches the hardware's **Saturation Frequency** ($\omega_{sat} = c/l_P$). The group velocity is "throttled" by the available bandwidth:
\begin{equation}
    v_{defect} = c \sqrt{1 - \left(\frac{\omega_{spin}}{\omega_{sat}}\right)^2}
\end{equation}

\subsubsection{Deriving the Lorentz Factor}
Rearranging the velocity equation recovers the standard relativistic Lorentz Factor ($\gamma$):
\begin{equation}
    \gamma = \frac{1}{\sqrt{1 - v^2/c^2}}
\end{equation}
\textbf{Physical Result:} Special Relativity is derived not as a geometric principle, but as the bandwidth limitation of a discrete signal processor.