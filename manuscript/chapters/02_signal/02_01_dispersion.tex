\section{The Vacuum Dispersion Relation}
\label{sec:dispersion}

In the Standard Model, the speed of light $c$ is an axiomatic constant. In the SVF, we distinguish between two distinct modes of propagation within the $M_{A}$ substrate: **Linear Flux** and **Topological Defects**.

This bifurcation resolves the "Lattice Trap" common to discrete theories, ensuring that high-energy cosmic rays obey Lorentz invariance while massive particles exhibit relativistic saturation.

\subsection{Mode 1: Linear Flux (Light)}
Photons represent sub-saturation perturbations of the vacuum potential. Because the amplitude of a flux signal is small compared to the saturation threshold of the nodes, the lattice behaves as a linear transmission line. 

For all wavenumbers $k$ below the hard Nyquist limit ($k \ll \pi/l_P$), the dispersion relation is linear:
\begin{equation}
    \omega_{flux}(k) = c \cdot k
\end{equation}

Consequently, the Group Velocity $v_g$ remains constant:
\begin{equation}
    v_g = \frac{d\omega}{dk} = c = \frac{1}{\sqrt{L_{node}C_{node}}}
\end{equation}

This derivation confirms that the speed of light is the **Global Slew Rate Limit** of the hardware in its linear regime. High-energy photons do not "see" the granularity of the lattice until their wavelength approaches the Planck scale ($l_P$), preventing the violation of Lorentz invariance observed in simple cosine-dispersion models.

\subsection{Mode 2: Topological Defects (Matter)}
Matter particles are not transient waves, but stable **Topological Knots** (vortices) in the field. Unlike free flux, these structures impose a continuous, high-frequency load on the local nodes, defined as the particle's **Intrinsic Spin Frequency** ($\omega_{spin}$).

As a defect accelerates, its effective update rate approaches the hardware's **Saturation Frequency** ($\omega_{sat}$):
\begin{equation}
    \omega_{sat} = \frac{c}{l_P} = \frac{1}{l_P\sqrt{L_{node}C_{node}}}
\end{equation}

When $\omega_{spin} \to \omega_{sat}$, the node enters a non-linear saturation regime. It can no longer update fast enough to translate the pattern transversely. The group velocity is effectively "throttled" by the available bandwidth:

\begin{equation}
    v_{defect} = c \sqrt{1 - \left(\frac{\omega_{spin}}{\omega_{sat}}\right)^2}
\end{equation}

\subsubsection{Deriving the Lorentz Factor}
Rearranging Eq 2.4 recovers the standard relativistic Lorentz Factor ($\gamma$):
\begin{equation}
    \gamma = \frac{1}{\sqrt{1 - v^2/c^2}} = \frac{\omega_{sat}}{\sqrt{\omega_{sat}^2 - \omega_{spin}^2}}
\end{equation}

This reveals the physical definition of **Inertial Mass**:
\begin{quote}
    \textbf{Mass is Hardware Latency.} It is the drag induced when a topological pattern's internal spin frequency competes with the lattice's global refresh rate.
\end{quote}