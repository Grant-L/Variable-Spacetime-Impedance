\section{The Signal Layer}

In Chapter 1, we established the vacuum as a physical transmission medium composed of discrete LC nodes. We now derive the relationship between signal frequency and propagation velocity, identifying the mechanical origin of rest mass and relativistic scaling as a direct result of hardware bandwidth limitations.

\subsection{Time Dilation as Lattice Latency}
Time is the rate of nodal updates. In a high-impedance zone (high gravity or high velocity), nodes must dedicate a higher percentage of their "hardware cycles" to maintaining the saturation state of the mass. Consequently, fewer cycles are available for external signal propagation.

An observer in a high-strain zone perceives time moving slower because the hardware is running at a higher \textbf{Lattice Latency}. The "flow" of time is the global clock-rate of the manifold minus the local processing load.