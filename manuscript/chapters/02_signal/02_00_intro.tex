\subsection{The Vacuum Dispersion Relation}
In the Standard Model, the speed of light $c$ is an axiomatic constant. In the SVF, we distinguish between two modes of propagation within the $M_{A}$ substrate: \textbf{Linear Flux} and \textbf{Topological Defects}.

\subsubsection{Mode 1: Linear Flux (Light)}
Photons represent sub-saturation perturbations of the vacuum potential. Because the amplitude of these signals is small compared to the saturation threshold of the nodes, the lattice behaves as a linear transmission line. The dispersion relation is linear:
\begin{equation}
    \omega_{flux}(k) = c \cdot k
\end{equation}
Consequently, the Group Velocity $v_g$ remains constant for all frequencies up to the Nyquist limit:
\begin{equation}
    v_g = \frac{d\omega}{dk} = c = \frac{1}{\sqrt{L_{node}C_{node}}}
\end{equation}
This ensures Lorentz invariance for high-energy cosmic rays, resolving the "Cosine Trap" common in discrete lattice theories.

\subsubsection{Mode 2: Topological Defects (Matter)}
Matter particles are not simple waves, but stable topological knots (vortices) in the field. These structures impose a continuous, high-frequency load on the local nodes, defined as the Intrinsic Spin Frequency ($\omega_{spin}$).

As a defect accelerates, its effective frequency approaches the hardware Saturation Frequency ($\omega_{sat}$). The nodes can no longer update fast enough to translate the pattern, resulting in a non-linear velocity drop:
\begin{equation}
    v_{defect} = c \sqrt{1 - \left(\frac{\omega_{spin}}{\omega_{sat}}\right)^2}
\end{equation}
This derivation recovers the relativistic Lorentz Factor ($\gamma$), revealing that \textbf{Inertial Mass is a measure of Hardware Latency}. Mass is the drag induced when a topological pattern's update rate competes with the lattice's global refresh rate.