\section{The Vacuum Dispersion Relation}
\label{sec:vacuum_dispersion}

In the Standard Model, the speed of light $c$ is an axiomatic constant. In the SVF, it is a derived property of the substrate's impedance. We treat the vacuum as a 3D transmission line grid where each node satisfies the discrete Kirchhoff equations.

\subsection{Discrete Kirchhoff Derivation}
Consider a 1D chain of nodes separated by lattice pitch $l_P$. The voltage $V_n$ (vacuum potential) and current $I_n$ (flux) are governed by:

\begin{align}
    L_{node} \frac{dI_n}{dt} &= V_{n-1} - V_n \label{eq:inductance} \\
    C_{node} \frac{dV_n}{dt} &= I_n - I_{n+1} \label{eq:capacitance}
\end{align}

Substituting a plane-wave solution $V_n = V_0 e^{i(\omega t - n k l_P)}$, we obtain the exact dispersion relation for the vacuum substrate:

\begin{equation}
    \omega(k) = \frac{2}{\sqrt{L_{node}C_{node}}} \sin\left(\frac{k l_P}{2}\right)
    \label{eq:dispersion}
\end{equation}

\subsection{The Group Velocity Limit}
The speed at which information (energy) propagates through the lattice is the Group Velocity $v_g = \frac{d\omega}{dk}$. Differentiating Eq. \ref{eq:dispersion}:

\begin{equation}
    v_g(k) = \frac{l_P}{\sqrt{L_{node}C_{node}}} \cos\left(\frac{k l_P}{2}\right) = c \cdot \cos\left(\frac{k l_P}{2}\right)
\end{equation}

This reveals the fundamental mechanism of Relativity: \textbf{Bandwidth Limiting}. As the wavenumber $k$ increases (higher energy/momentum), the cosine term drops. When $k \rightarrow \pi/l_P$ (the Nyquist limit), $v_g \rightarrow 0$.

Standard Special Relativity is therefore the low-frequency approximation ($k l_P \ll 1$) of this discrete hardware limit. The "Lorentz Factor" $\gamma$ is simply the non-linear approach to the lattice saturation frequency.

\subsection{Relativistic Scaling: The Rotational Origin of Mass}
We rewrite the velocity relation in terms of frequency:

\begin{equation}
    v_g = c \sqrt{1 - \left(\frac{\omega_{spin}}{\omega_{sat}}\right)^2}
\end{equation}

In the SVF, a particle is not a static point but a dynamic \textbf{Topological Vortex}. The fundamental property of matter is its Intrinsic Spin Frequency ($\omega_{spin}$).

As $\omega_{spin} \to \omega_{sat}$, the hardware node enters a saturation regime. It can no longer process transverse updates (motion) because its bandwidth is consumed by maintaining the rotational state of the vortex. This "locking" of the lattice is what we macroscopically perceive as \textbf{Inertial Mass}.

If the spin were to stop ($\omega_{spin} \to 0$), the saturation would vanish, and the "mass" would evaporate into radiation.