\section{The Bullet Cluster: Shockwave Dynamics}
\label{sec:bullet_cluster}

The Bullet Cluster is often cited as the "smoking gun" for particulate Dark Matter because the gravitational lensing center is separated from the visible gas.
Vacuum Engineering identifies this not as "collisionless particles," but as a \textbf{Refractive Shockwave}.

\subsection{Metric Separation}
When two galactic clusters collide, they create a massive pressure wave in the substrate.
\begin{itemize}
    \item \textbf{Baryonic Matter (Gas):} interacts via electromagnetism and slows down (viscous drag).
    \item \textbf{The Metric Shock (Gravity):} is a longitudinal compression wave in the vacuum. It passes through the collision zone unimpeded.
\end{itemize}

\subsection{Lensing without Mass}
Gravitational lensing is caused by the refractive index of the vacuum ($n$).
\begin{equation}
    n = \sqrt{\permeability \permittivity}
\end{equation}
A compression shockwave locally increases the density ($\permeability$) of the vacuum. This increases $n$, causing light to bend \textbf{even in the absence of mass}.
The "Dark Matter" map of the Bullet Cluster is simply a map of the \textbf{residual stress} left in the vacuum after the collision.