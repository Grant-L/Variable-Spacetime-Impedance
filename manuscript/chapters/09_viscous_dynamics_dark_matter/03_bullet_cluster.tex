\section{The Bullet Cluster: Shockwave Dynamics}

The Bullet Cluster is frequently cited as the ``smoking gun'' for particulate Dark Matter because the gravitational lensing center is physically separated from the visible baryonic gas. Vacuum Engineering identifies this phenomenon not as ``collisionless dark particles,'' but as a \textbf{Refractive Shockwave}.

When two massive galactic clusters collide, they create a colossal pressure wave in the underlying $\mathcal{M}_A$ substrate. The baryonic matter (gas) interacts via electromagnetism and slows down due to viscous drag. However, the metric shock is a longitudinal compression wave in the vacuum lattice itself. It passes through the collision zone unimpeded.

Because gravitational lensing is caused exclusively by the refractive index of the vacuum ($n = \sqrt{\mu \epsilon}$), a compression shockwave locally increases the lattice density, increasing $n$. This causes light to bend even in the complete absence of physical matter. The ``Dark Matter'' map of the Bullet Cluster is simply an optical mapping of the residual acoustic stress ringing in the vacuum after the collision.