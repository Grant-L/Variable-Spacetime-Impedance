\section{The Rheology of Space: The Bingham Transition}

A critical classical objection to any hydrodynamic or discrete substrate model of the vacuum is the ``Viscosity Paradox'': if space is a physical substance dense enough to drag galaxies together (Dark Matter), its viscosity should effectively stop the Earth in its orbit around the Sun within millions of years.

We rigorously resolve this by treating the vacuum substrate ($\mathcal{M}_A$) identically to a solid-state \textbf{Bingham Plastic}---a non-Newtonian shear-thinning material.

In solid mechanics, a Bingham Plastic behaves as a rigid solid at low stress but physically fractures and flows as a zero-drag fluid when subjected to a high shear rate ($\nabla g \gg \text{Yield}$). The discrete topological edges of the vacuum lattice physically break and relink when sheared beyond their critical relaxation threshold.

\subsection{The Two Regimes of Gravity}

This exact rheological property creates two distinct dynamic regimes natively dependent on the scale of the system:

\textbf{Regime I: High Shear (Solar System Stability)} \\
Near a dense stellar mass like the Sun, the gravitational gradient (shear rate) is immense. The extreme curvature continuously liquefies the local lattice boundaries, effectively reducing the structural viscosity to zero ($\eta \to 0$). This localized \textbf{Superfluid} transition ensures that planetary orbits are perfectly conservative and stable over billions of years, flawlessly matching General Relativity and pulsar timing observations.

\textbf{Regime II: Low Shear (Galactic Rotation)} \\
In the deep outer reaches of a galaxy, the gravitational gradient is minuscule. The shear stress falls below the critical threshold required to break the local $\mathcal{M}_A$ lattice bonds. The lattice relaxes back into its rigid state, exhibiting its full baseline structural viscosity ($\eta \approx \eta_0$). This macroscopic network stiffness physically drags on the orbiting stars, manifesting macroscopically as the phenomenon of ``Dark Matter.''

