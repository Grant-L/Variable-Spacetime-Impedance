\section{The Rheology of Space: Why Planets Don't Crash}
\label{sec:viscosity_rheology}

A critical objection to any hydrodynamic model of the vacuum is the "Viscosity Paradox": if space is viscous enough to drag galaxies (Dark Matter), it should effectively stop the Earth in its orbit within millions of years.

We resolve this by defining the vacuum substrate ($M_A$) not as a Newtonian fluid, but as a \textbf{Non-Newtonian Shear-Thinning Superfluid}.

\subsection{The Bingham Plastic Vacuum}
Standard fluids have constant viscosity. The vacuum lattice, however, is a structured solid that yields under stress. We propose the constitutive relation:
\begin{equation}
\eta(\dot{\gamma}) = \frac{\eta_0}{1 + (\frac{\dot{\gamma}}{\dot{\gamma}_c})^2}
\end{equation}
Where:
\begin{itemize}
    \item $\eta_0$: The base vacuum viscosity (Dark Matter limit).
    \item $\dot{\gamma}$: The local shear rate (Gravitational Gradient $\nabla g$).
    \item $\dot{\gamma}_c$: The critical shear threshold (Transition point).
\end{itemize}

\begin{figure}[h]
    \centering
    \includegraphics[width=\textwidth]{chapters/09_viscous_dynamics_dark_matter/simulations/rheology_2d_curve.png}
    \caption{The Rheological Shield. The log-log plot demonstrates the dual nature of the vacuum. In the Solar System (Green Zone), the high gravitational shear liquefies the lattice, reducing drag to zero (Superfluid). In the Galactic outskirts (Blue Zone), the low shear allows the lattice to relax into a high-viscosity gel, creating the Dark Matter effect.}
    \label{fig:rheology_curve}
\end{figure}

\subsection{The Two Regimes of Gravity}
This rheology creates two distinct dynamic regimes based on the scale of the system:

\subsubsection{Regime I: High Shear (Solar System Stability)}
Near a star or planet, the gravitational gradient is immense ($\dot{\gamma} \gg \dot{\gamma}_c$).
\begin{equation}
\eta_{local} \approx \frac{\eta_0}{(\dot{\gamma}/\dot{\gamma}_c)^2} \to 0
\end{equation}
The intense curvature "liquefies" the local lattice boundaries, effectively reducing drag to zero. This ensures that planetary orbits are conservative and stable over billions of years, matching observations of the Earth and Hulse-Taylor binary pulsars.

\begin{figure}[h]
    \centering
    \includegraphics[width=0.9\textwidth]{chapters/09_viscous_dynamics_dark_matter/simulations/rheology_3d_well.png}
    \caption{Volumetric View of the Viscosity Well. A 3D slice of the viscosity field $\eta(x,y,z)$ around a star. The star creates a "hole" in the cosmic viscosity fluid. Planets orbiting inside this well feel no drag, while the galaxy outside floats on the viscous plateau.}
    \label{fig:rheology_3d}
\end{figure}

\subsubsection{Regime II: Low Shear (Galactic Rotation)}
In the outer reaches of a galaxy, the gravitational gradient is minuscule ($\dot{\gamma} \ll \dot{\gamma}_c$).
\begin{equation}
\eta_{local} \approx \eta_0
\end{equation}
The lattice relaxes into its "gel" state, exhibiting the full structural viscosity derived in Eq 9.1. This macroscopic drag forces the rotation curve to flatten, manifesting as "Dark Matter."

\textbf{Conclusion:} Dark Matter is not a particle halo; it is the phase transition of the vacuum fluid from a local superfluid (near stars) to a global viscous gum (interstellar space).