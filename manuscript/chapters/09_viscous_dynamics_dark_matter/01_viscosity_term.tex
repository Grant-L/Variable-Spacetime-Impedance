\section{The Viscosity of Space}
\label{sec:viscosity_term}

The Standard Model assumes the vacuum is a frictionless superfluid. Vacuum Engineering asserts that the Discrete Amorphous Manifold ($M_A$) possesses a finite \textbf{Lattice Viscosity} ($\eta_{vac}$).

Just as water resists the motion of a spoon, the vacuum lattice resists the motion of topological defects (mass). This resistance is not constant; it depends on the scale and coherence of the moving object.

\subsection{Deriving Vacuum Viscosity from Alpha}
We propose that the viscosity coefficient is determined by the geometric coupling constant $\alpha$ (derived in Chapter 3):
\begin{equation}
    \eta_{vac} \approx \alpha \cdot \frac{\hbar}{l_P^3}
\end{equation}
This viscosity implies that gravity is not merely a static field, but a \textbf{Fluid Dynamic} phenomenon. At solar system scales, viscosity is negligible ($Re \gg 1$). At galactic scales, it dominates.

\subsubsection{Dimensional Analysis Proof} 
To verify the validity of this constitutive relation, we execute a rigorous dimensional analysis:
\begin{itemize}
    \item The Fine Structure Constant ($\alpha$) is a dimensionless geometric ratio: $[1]$.
    \item The Planck Action ($\hbar$) possesses units of angular momentum: $[\text{kg} \cdot \text{m}^2 / \text{s}]$.
    \item The Lattice Pitch cubed ($l_P^3$) possesses units of volume: $[\text{m}^3]$.
\end{itemize}
Dividing Action by Volume yields:
\begin{equation}
    [\eta_{vac}] = \frac{\text{kg} \cdot \text{m}^2 / \text{s}}{\text{m}^3} = \left[ \frac{\text{kg}}{\text{m} \cdot \text{s}} \right] \equiv \text{Pa} \cdot \text{s}
\end{equation}
The standard SI unit for Dynamic Viscosity (Pascal-seconds) is defined exactly as $\text{Pa} \cdot \text{s} = (\text{N}/\text{m}^2)\text{s} = [\text{kg} / (\text{m} \cdot \text{s})]$. 

\textbf{Result:} The dimensional mapping is exact. We have successfully derived classical fluid viscosity purely from the fundamental quantum properties of the discrete substrate.