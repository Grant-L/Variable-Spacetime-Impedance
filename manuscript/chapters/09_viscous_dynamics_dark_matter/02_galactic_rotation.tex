\section{Galactic Rotation: The Vortex Model}
\label{sec:galactic_rotation}

The "Galaxy Rotation Problem" is the primary evidence for Dark Matter. Stars at the edge of galaxies orbit faster than Newtonian gravity allows.
Standard Physics adds invisible mass to fix the equation. \textbf{Vacuum Engineering} adds fluid viscosity.

\subsection{The Galaxy as a Superfluid Vortex}
A galaxy is not just a collection of rocks in empty space; it is a \textbf{driven vortex} in the vacuum substrate.
The central supermassive black hole is not just a heavy object; it is the "impeller" of the system, dragging the local manifold with it (Frame Dragging).

\subsection{Viscous Coupling}
In a perfect superfluid ($\eta = 0$), velocity drops off as $1/r$ (irrotational vortex). However, if the vacuum has a non-zero \textbf{Lattice Viscosity} ($\eta > 0$):
\begin{equation}
    \tau = \eta \frac{dv}{dr}
\end{equation}
The rotating core transfers angular momentum to the outer layers of the vacuum. This "viscous drag" keeps the outer metric spinning, carrying the stars with it.

\subsection{The Flat Rotation Curve}
We model the galaxy using the Navier-Stokes equations for the substrate. The tangential velocity $v(r)$ becomes:
\begin{equation}
    v(r) = \sqrt{\frac{GM}{r} + \frac{\eta}{\rho} r}
\end{equation}
\begin{itemize}
    \item \textbf{Inner Region ($r \to 0$):} Gravity dominates ($v \propto r^{-1/2}$).
    \item \textbf{Outer Region ($r \to \infty$):} Viscosity dominates ($v \to \text{constant}$).
\end{itemize}
\textbf{Result:} The rotation curve flattens naturally. We do not need "Dark Matter"; we simply need to account for the friction of space itself.