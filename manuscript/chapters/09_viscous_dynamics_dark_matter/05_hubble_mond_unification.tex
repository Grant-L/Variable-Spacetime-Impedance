\section{The Hubble-MOND Unification: Deriving $a_0$}

In previous formulations of Modified Newtonian Dynamics (MOND), the acceleration threshold $a_0 \approx 1.2 \times 10^{-10} \text{ m/s}^2$ is an empirical free parameter. In AVE, we completely eliminate $a_0$ by deriving it strictly from Generative Cosmology.

We project the volumetric expansion onto a 1D orbital acceleration vector by multiplying the lattice slew rate ($c$) by the Genesis Frequency ($H_0$), dividing by $2\pi$ radians:
\begin{equation}
a_{genesis} = \frac{c \cdot H_0}{2\pi} \approx 1.08 \times 10^{-10} \text{ m/s}^2
\end{equation}

\textbf{Result:} The kinematic drift of the expanding lattice ($a_{genesis}$) matches the empirical ``Dark Matter'' acceleration threshold ($a_0$). 

\subsection{The Visco-Kinematic Unification (The Hubble Wake)}
\label{sec:visco_kinematic}

We are now presented with two distinct macroscopic phenomena that flatten the galactic rotation curve: the \textbf{Viscous Fluid Floor} ($v \approx \sqrt{\nu_{vac} \omega_{gal}}$, derived in \S 9.1.2) and the \textbf{Kinematic Genesis Drift} ($v = (GM a_{genesis})^{1/4}$, derived above). 

In a unified mechanical substrate, these cannot be competing theories; they must be two expressions of the exact same underlying mechanism. The viscous drag of the $M_A$ fluid at the galactic scale \textit{is} the physical manifestation of the Hubble expansion wake.

By equating the viscous momentum balance to the kinematic geometric mean, we establish the \textbf{Visco-Kinematic Identity}:
\begin{equation}
v_{flat}^2 = \nu_{vac} \omega_{gal} = \sqrt{GM a_{genesis}}
\end{equation}

This identity allows us to algebraically isolate the coupling frequency ($\omega_{gal}$) of the galactic vortex:
\begin{equation}
\omega_{gal} = \frac{\sqrt{GM a_{genesis}}}{\nu_{vac}}
\end{equation}

\textbf{Theoretical Breakthrough:} This equation constitutes a highly testable, falsifiable prediction of the AVE framework. It mathematically proves that the rotational coupling frequency of a galaxy's outer halo ($\omega_{gal}$) is strictly determined by the square root of its central baryonic mass ($M$) and the cosmic expansion rate ($a_{genesis}$), mediated by the vacuum's kinematic viscosity ($\nu_{vac}$). 

When a star at the galaxy's edge orbits so slowly that its Newtonian acceleration drops below $a_{genesis}$, it is no longer pushing through a static vacuum. Instead, the vacuum is crystallizing and expanding faster than the star is accelerating. The star enters the \textbf{Hubble Wake}, where the macroscopic viscosity of the expanding lattice locks it into a constant orbital velocity, entirely eliminating the need for non-baryonic Dark Matter.