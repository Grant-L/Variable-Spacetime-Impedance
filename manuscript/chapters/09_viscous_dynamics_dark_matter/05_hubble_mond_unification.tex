\section{The Hubble-MOND Unification: Deriving $a_0$}

In previous formulations of Modified Newtonian Dynamics (MOND), the acceleration threshold $a_0 \approx 1.2 \times 10^{-10} \text{ m/s}^2$ is an empirical free parameter. In AVE, we completely eliminate $a_0$ by deriving it strictly from Generative Cosmology.

We project the volumetric expansion onto a 1D orbital acceleration vector by multiplying the lattice slew rate ($c$) by the Genesis Frequency ($H_0$), dividing by $2\pi$ radians:
\begin{equation}
a_{genesis} = \frac{c \cdot H_0}{2\pi} \approx 1.08 \times 10^{-10} \text{ m/s}^2
\end{equation}

\textbf{Result:} The kinematic drift of the expanding lattice ($a_{genesis}$) matches the empirical ``Dark Matter'' acceleration threshold ($a_0$). 

\subsection{The Hubble Wake}
This represents the \textbf{Hubble-MOND Unification}. When a star orbits so slowly that its Newtonian acceleration drops below $a_{genesis}$, the vacuum is crystallizing faster than the star is accelerating. The star enters the \textbf{Hubble Wake}. To ensure impedance matching, the effective acceleration scales with the geometric mean: $a_{AVE} = \sqrt{a_{Newton}^2 + a_{genesis} a_{Newton}}$.