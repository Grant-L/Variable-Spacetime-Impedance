\section{The Hubble-MOND Unification: Deriving $a_{genesis}$}
In previous formulations of Modified Newtonian Dynamics (MOND), the acceleration threshold $a_{0} \approx 1.2 \times 10^{-10} m/s^2$ is an empirical free parameter. In AVE, we completely eliminate $a_{0}$ by deriving it strictly from Generative Cosmology.

We project the volumetric expansion onto a 1D orbital acceleration vector by multiplying the lattice slew rate ($c$) by the Genesis Frequency ($H_{0}$), dividing by $2\pi$ radians (cyclic conversion):

\begin{equation}
    a_{genesis} = \frac{c \cdot H_{0}}{2\pi} \approx 1.11 \times 10^{-10} m/s^2
\end{equation}

Result: The kinematic drift of the expanding lattice ($a_{genesis}$) naturally matches the empirical "Dark Matter" acceleration threshold ($a_{0}$) without fitting.

\subsubsection{Simulation: Physics-Derived Rotation Curve}
Using this derived acceleration, we simulate the rotation curve of a Milky Way-like galaxy ($M \approx 10^{11} M_{\odot}$). The viscous velocity floor is calculated via the Visco-Kinematic Identity:
$$ v_{flat} = (G M a_{genesis})^{1/4} \approx 196 \text{ km/s} $$
This predicted value aligns with observational data, confirming that the "Dark Matter" halo is the hydrodynamic wake of the visible mass interacting with the expanding vacuum lattice.

\begin{figure}[ht]
\centering
\includegraphics[width=0.9\textwidth]{chapters/09_viscous_dynamics_dark_matter/simulations/galaxy_rotation_derived.png}
\caption{\textbf{Physics-Derived Rotation Curve.} Unlike curve-fitting models, the velocity floor (Green Dotted Line) is derived entirely from $G, c, H_0$ and the visible mass. The AVE prediction (Blue Line) naturally flattens to match observations (Red Dots).}
\label{fig:galaxy_derived}
\end{figure}