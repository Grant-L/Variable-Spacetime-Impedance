\section{The Measurement Effect: Impedance Loading}
\label{sec:measurement_effect}

The "Measurement Problem"---where observation induces the "collapse" of the wavefunction---is treated by the Copenhagen interpretation as a metaphysical discontinuity. In Vacuum Engineering, it is formally resolved as a thermodynamic circuit problem: \textbf{Impedance Loading}.

\subsection{Deriving the Born Rule ($P \propto |\psi|^2$)}
To measure a quantum state, a macroscopic detector must couple to the vacuum lattice. A detector is not a passive mathematical observer; it is a physical entity with an activation energy threshold $E_{thresh}$. It functions as a resistive load ($R_{load}$) drawing power from the local $\mathcal{M}_A$ substrate.

From classical electrodynamics, the energy density $u_n$ of a wave in a dielectric medium is proportional to the square of its amplitude: $u_n \propto |\psi(x_n)|^2$. The power dissipated into the detector over a measurement interval $\Delta t$ is governed by Joule's Law:
\begin{equation}
    W_{extracted} = \int P_{load} dt \propto \frac{|\psi(x_n)|^2}{R_{load}} \Delta t
\end{equation}

Because the vacuum lattice possesses a fundamental stochastic noise floor (due to the amorphous geometry and zero-point vibrations), the exact energy transferred fluctuates. A physical detector requires a minimum threshold of energy ($E_{thresh}$) to register a discrete ``click'' (e.g., ionizing an atom, triggering a photomultiplier cascade) against this noise floor.

In signal detection theory, the probability of an analog signal triggering a discrete threshold logic gate is strictly proportional to the Signal-to-Noise Ratio (SNR). Because the available signal power is proportional to $|\psi|^2$, the statistical probability that the extracted work exceeds the detector's deterministic threshold ($W_{extracted} \ge E_{thresh}$) scales identically with the squared amplitude:
\begin{equation}
    P(click | x_n) = \frac{|\psi(x_n)|^2}{\int |\psi(x)|^2 dx}
\end{equation}
\textbf{Conclusion:} The Born Rule is not an axiomatic postulate of probability. It is the deterministic thermodynamic equation for energy extraction from a wave-bearing lattice by a thresholded resistive load.

\subsection{Decoherence as Ohmic Dissipation}
Standard quantum mechanics utilizes non-unitary Lindblad equations to model wavefunction collapse via environmental decoherence. AVE provides the direct physical mechanism for this mathematical structure.

Prior to measurement, the pilot wave evolves unitarily according to the energy-conserving discrete Lagrangian (Section \ref{sec:dielectric_lagrangian}). The insertion of the detector introduces a non-conservative Ohmic damping term (friction) to the local lattice nodes.

The ``Collapse of the Wavefunction'' is nothing more than rapid critical damping. By draining the pilot wave's energy to gain information, the detector acts as an electrical short-circuit. The spatial interference fringes (the off-diagonal coherence terms of the density matrix) decay exponentially to zero as energy is extracted, causing the particle to decouple from the wave and resume localized ballistic motion. 

The transition from quantum to classical physics is identically the transition from an isolated, lossless $LC$ circuit to a dissipative $RLC$ circuit.