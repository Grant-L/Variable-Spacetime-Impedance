\section{The Measurement Effect: Impedance Loading}
\label{sec:measurement_effect}

The "Measurement Problem" in quantum mechanics—where observation collapses the wavefunction—is treated by Copenhagen interpretation as a metaphysical event. In Vacuum Engineering, it is a simple circuit load problem.

\subsubsection{The Observer as a Resistor}
To measure a quantum system, one must couple to it. In circuit theory, this coupling acts as a resistive load ($R_{load}$) that dissipates energy from the oscillating pilot wave.

The total energy extracted during the measurement interval $\tau$ is the integral of the instantaneous power:
\begin{equation}
    E_{measured} = \int_{0}^{\tau} P_{load}(t) \, dt = \int_{0}^{\tau} I(t)^2 R_{load} \, dt
\end{equation}

If we approximate the pilot wave current as a pulse of duration $\Delta t$ with mean square amplitude $\langle I^2 \rangle$, this simplifies to:
\begin{equation}
    E_{measured} \approx \langle I^2 \rangle R_{load} \Delta t
\end{equation}

The "Collapse of the Wavefunction" is therefore not a metaphysical event, but the rapid damping of the lattice oscillation ($L-R$ decay) caused by the sudden insertion of the detector's impedance.