\section{Deriving the Quantum Formalism from Signal Bandwidth}
\label{sec:quantization_as_bandwidth}

Standard Quantum Mechanics posits its formalism---complex Hilbert spaces, unitary evolution, and non-commuting operators---as axiomatic. In the Applied Vacuum Electrodynamics (AVE) framework, these are not axioms. They are the rigorous mathematical consequences of transmitting signals across a discrete, band-limited mechanical graph ($\mathcal{M}_A$).

\subsection{The Paley-Wiener Hilbert Space ($\mathcal{H}$)}
A frequent critique of discrete spacetime models is that a grid cannot support the continuous, complex Hilbert space required for quantum mechanics. This objection fails to account for the mathematics of band-limited signal reconstruction.

Because the $\mathcal{M}_A$ lattice has a fundamental pitch $l_0$, it acts as a spatial Nyquist sampling grid. The maximum spatial frequency (wavenumber) the lattice can support without aliasing is the Nyquist limit: $k_{max} = \pi / l_0$.

By the \textbf{Whittaker-Shannon Interpolation Theorem}, any physical signal $\phi(x)$ on this discrete lattice that is perfectly band-limited can be reconstructed uniquely and continuously everywhere in space using a superposition of orthogonal sinc functions:
\begin{equation}
    \phi(x) = \sum_{n=-\infty}^{\infty} \phi(n l_0) \text{sinc}\left( \frac{x - n l_0}{l_0} \right)
\end{equation}
Mathematically, the set of all such band-limited functions formally constitutes a Reproducing Kernel Hilbert Space known as the \textbf{Paley-Wiener Space} ($PW_{\pi/l_0}$). 

To map the real, physical lattice potential $\phi(x,t)$ to the complex quantum state vector $\psi(x,t)$, we apply the standard signal-processing \textbf{Analytic Signal} representation using the Hilbert Transform ($\mathcal{H}_{transform}$):
\begin{equation}
    \psi(x,t) = \phi(x,t) + i \mathcal{H}_{transform}[\phi(x,t)]
\end{equation}
\textit{Conclusion:} The complex Hilbert space of Quantum Mechanics is identically the Paley-Wiener signal space of the discrete vacuum lattice. The ``wavefunction'' is simply the complex analytic representation of the physical mechanical strain, tracking both the capacitive potential (real) and inductive momentum (imaginary) phases simultaneously.

\subsection{Operator Non-Commutativity ($[\hat{x}, \hat{p}] = i\hbar_{AVE}$)}
In standard QM, the non-commutativity of position and momentum is an unexplained axiom. In AVE, it is a strict geometric consequence of the discrete shift operator acting on a lattice with finite pitch $l_0$.

On a discrete graph, continuous translation is physically impossible. The minimum possible physical translation is a shift to the adjacent node, governed by the discrete shift operator $\hat{T}$, where $\hat{T} \psi(x) = \psi(x + l_0)$.

We evaluate the commutator of the position operator $\hat{x}$ and the shift operator $\hat{T}$:
\begin{equation}
    [\hat{x}, \hat{T}] \psi(x) = \hat{x} (\hat{T} \psi(x)) - \hat{T} (\hat{x} \psi(x))
\end{equation}
\begin{equation}
    = x \psi(x + l_0) - (x + l_0)\psi(x + l_0) = -l_0 \psi(x + l_0) = -l_0 \hat{T} \psi(x)
\end{equation}
Thus, the exact discrete commutator is:
\begin{equation}
    [\hat{x}, \hat{T}] = -l_0 \hat{T}
\end{equation}

By Stone's theorem, the momentum operator $\hat{p}$ is the generator of translations. We relate the discrete shift operator to the continuous momentum generator via $\hat{T} = \exp(i \hat{p} l_0 / \hbar_{AVE})$.
Expanding the exponential to first order in the continuum limit ($l_0 \to 0$):
\begin{equation}
    \hat{T} \approx \mathbb{I} + i \frac{l_0}{\hbar_{AVE}} \hat{p}
\end{equation}
Substitute this expansion into our exact discrete commutator:
\begin{equation}
    \left[ \hat{x}, \left(\mathbb{I} + i \frac{l_0}{\hbar_{AVE}} \hat{p}\right) \right] = -l_0 \left( \mathbb{I} + i \frac{l_0}{\hbar_{AVE}} \hat{p} \right)
\end{equation}
Because the identity operator $\mathbb{I}$ commutes with $\hat{x}$, this simplifies to:
\begin{equation}
    i \frac{l_0}{\hbar_{AVE}} [\hat{x}, \hat{p}] = -l_0 - i \frac{l_0^2}{\hbar_{AVE}} \hat{p} + \mathcal{O}(l_0^3)
\end{equation}
Dividing both sides by $i l_0 / \hbar_{AVE}$ (and noting that $-l_0 / (il_0/\hbar_{AVE}) = i\hbar_{AVE}$):
\begin{equation}
    [\hat{x}, \hat{p}] = i\hbar_{AVE} - l_0 \hat{p} + \mathcal{O}(l_0^2)
\end{equation}

\textit{Conclusion:} In the macroscopic continuum limit ($l_0 \to 0$), we flawlessly recover the Heisenberg Commutation Relation $[\hat{x}, \hat{p}] = i\hbar_{AVE}$. Furthermore, the residual $-l_0 \hat{p}$ term explicitly generates the Generalized Uncertainty Principle (GUP) required by quantum gravity, proving that operator non-commutativity is strictly the algebraic manifestation of finite-difference aliasing on a discrete spatial grid.

\subsection{Unitary Evolution: Deriving the Schr\"odinger Equation}
To prove that the classical lattice dynamics perfectly reproduce quantum evolution, we derive the Schr\"odinger equation directly from the classical wave equation of the substrate.

In Section \ref{sec:dielectric_lagrangian}, we established that a topological defect (mass $m$) on the lattice obeys the classical Klein-Gordon wave equation:
\begin{equation}
    \nabla^2 \phi - \frac{1}{c^2}\frac{\partial^2 \phi}{\partial t^2} - \frac{m^2 c^2}{\hbar_{AVE}^2}\phi = 0
\end{equation}
Let us map this real field to its complex analytic signal $\psi$. For a particle moving at non-relativistic speeds ($v \ll c$), the vast majority of the signal's temporal evolution is simply the rest-mass Compton frequency $\omega_m = mc^2/\hbar_{AVE}$. We factor this out by defining a slowly varying envelope function $\Psi(x,t)$:
\begin{equation}
    \psi(x,t) = \Psi(x,t) e^{-i \omega_m t}
\end{equation}
Substituting this into the wave equation, the second time derivative becomes:
\begin{equation}
    \frac{\partial^2 \psi}{\partial t^2} = \left( \frac{\partial^2 \Psi}{\partial t^2} - 2i\omega_m \frac{\partial \Psi}{\partial t} - \omega_m^2 \Psi \right) e^{-i \omega_m t}
\end{equation}
In the non-relativistic limit, the envelope varies slowly compared to the Compton frequency, allowing us to mathematically apply the Paraxial Approximation by dropping the tiny second derivative term ($\partial_t^2 \Psi \approx 0$). Substituting this back into the wave equation yields:
\begin{equation}
    \nabla^2 \Psi - \frac{1}{c^2} \left( - 2i\omega_m \frac{\partial \Psi}{\partial t} - \omega_m^2 \Psi \right) - \frac{m^2 c^2}{\hbar_{AVE}^2}\Psi = 0
\end{equation}
Recognizing that $\frac{\omega_m^2}{c^2} = \frac{m^2 c^2}{\hbar_{AVE}^2}$, the rest-mass terms cancel identically:
\begin{equation}
    \nabla^2 \Psi + \frac{2i\omega_m}{c^2} \frac{\partial \Psi}{\partial t} = 0
\end{equation}
Substituting $\omega_m = mc^2/\hbar_{AVE}$ and rearranging:
\begin{equation}
    i\hbar_{AVE} \frac{\partial \Psi}{\partial t} = -\frac{\hbar_{AVE}^2}{2m} \nabla^2 \Psi
\end{equation}
\textit{Conclusion:} The Schr\"odinger Equation is mathematically identical to the \textbf{Paraxial Approximation} of a classical macroscopic wave propagating through the discrete LC circuits of the $\mathcal{M}_A$ vacuum. The evolution is unitary strictly because the underlying LC lattice is lossless prior to measurement.