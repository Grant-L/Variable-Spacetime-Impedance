\section{Quantization as Bandwidth: The Nyquist Limit}
Standard Quantum Mechanics posits that energy is quantized in discrete packets. In the AVE framework, we model this behavior as the \textbf{Bandwidth Constraint} of a discrete receiver.

\subsection{The Discrete Sampling Analogy}
Since the vacuum is a discrete graph with pitch $l_{0}$, it behaves as a digital sampling system. The Shannon-Nyquist theorem implies that such a grid cannot support a frequency higher than half its sampling rate:
\begin{equation}
    \nu_{max} = \frac{c}{2l_{0}}
\end{equation}

\subsection{Uncertainty as Finite Information Density}
The Heisenberg Uncertainty Principle ($\Delta x \Delta p \ge \hbar/2$) can be understood mechanically as a limit on information density.
\begin{itemize}
    \item \textbf{Position ($\Delta x$):} Limited by the lattice pitch ($l_{0}$).
    \item \textbf{Momentum ($\Delta p$):} Limited by the maximum slew rate of the node ($m c$).
\end{itemize}
In this model, "Uncertainty" arises because attempting to localize a wave packet smaller than $l_{0}$ introduces aliasing noise. While this does not essentially derive the non-commutative operator algebra of QM, it provides a hard classical mechanism for the UV cutoff and phase-space volume limits ($h^3$) observed in statistical mechanics.