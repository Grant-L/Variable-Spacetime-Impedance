\section{Deriving the Quantum Formalism from Signal Bandwidth}
\label{sec:quantization_as_bandwidth}

Standard Quantum Mechanics posits its formalism---complex Hilbert spaces and non-commuting operators---as axiomatic. In the AVE framework, these are not axioms. They are the rigorous mathematical consequences of transmitting signals across a discrete, band-limited mechanical graph ($\mathcal{M}_A$).

\subsection{The Paley-Wiener Hilbert Space ($\mathcal{H}$)}
Because the $\mathcal{M}_A$ lattice has a fundamental pitch $l_0$, it acts as a spatial Nyquist sampling grid. The maximum spatial frequency the lattice can support without aliasing is the Nyquist limit: $k_{max} = \pi / l_0$.

By the \textbf{Whittaker-Shannon Interpolation Theorem}, any physical signal $\mathbf{A}(x)$ on this discrete lattice that is perfectly band-limited can be reconstructed uniquely and continuously everywhere in space using a superposition of orthogonal sinc functions. Mathematically, the set of all such band-limited functions formally constitutes a Reproducing Kernel Hilbert Space known as the \textbf{Paley-Wiener Space} ($PW_{\pi/l_0}$). 

To map the real, physical lattice potential $\mathbf{A}(x,t)$ to the complex quantum state vector $\psi(x,t)$, we apply the standard signal-processing \textbf{Analytic Signal} representation using the Hilbert Transform ($\mathcal{H}_{transform}$):
\begin{equation}
    \psi(x,t) = \mathbf{A}(x,t) + i \mathcal{H}_{transform}[\mathbf{A}(x,t)]
\end{equation}
\textit{Conclusion:} The complex Hilbert space of Quantum Mechanics is identically the Paley-Wiener signal space of the discrete vacuum lattice.

\subsection{Operator Algebra on the Discrete Manifold}
In standard QM, the non-commutativity of position and momentum ($[\hat{x}, \hat{p}] = i\hbar$) is an assumed axiom. On a discrete graph with pitch $l_0$, continuous translation is physically impossible. Furthermore, continuous momentum $\hat{p}_c$ is not infinite; it is strictly bounded by the Brillouin zone $p_c \in [-\pi\hbar/l_0, \pi\hbar/l_0]$. 

The exact physical lattice momentum operator $\hat{P}$ must be defined via the symmetric central finite-difference operator across the adjacent nodes:
\begin{equation}
    \hat{P} = \frac{\hbar}{i 2l_0} \left( \exp\left(i \frac{\hat{p}_c l_0}{\hbar}\right) - \exp\left(-i \frac{\hat{p}_c l_0}{\hbar}\right) \right) = \frac{\hbar}{l_0} \sin\left(\frac{l_0 \hat{p}_c}{\hbar}\right)
\end{equation}

We evaluate the exact commutator of the position operator with the lattice momentum using the identity $[\hat{x}, f(\hat{p}_c)] = i\hbar f'(\hat{p}_c)$:
\begin{equation}
    [\hat{x}, \hat{P}] = \left[ \hat{x}, \frac{\hbar}{l_0} \sin\left(\frac{l_0 \hat{p}_c}{\hbar}\right) \right] = \frac{\hbar}{l_0} \left( i \hbar \frac{l_0}{\hbar} \cos\left(\frac{l_0 \hat{p}_c}{\hbar}\right) \right) = i\hbar \cos\left(\frac{l_0 \hat{p}_c}{\hbar}\right)
\end{equation}

\subsection{The Authentic Generalized Uncertainty Principle}
Applying the generalized Robertson-Schr\"odinger relation, taking the expectation value yields the rigorously exact \textbf{Generalized Uncertainty Principle (GUP)} for the discrete vacuum:
\begin{tcolorbox}[colback=white, colframe=black]
\begin{equation}
    \Delta x \Delta P \ge \frac{\hbar}{2} \left| \left\langle \cos\left(\frac{l_0 \hat{p}_c}{\hbar}\right) \right\rangle \right|
\end{equation}
\end{tcolorbox}
\textbf{Proof of Limit:} In the low-energy continuum limit where particle momentum is extremely small compared to the grid cutoff ($p_c \ll \hbar/l_0$), the cosine evaluates to exactly $1$, natively recovering Heisenberg's principle $\Delta x \Delta p \ge \hbar/2$ flawlessly. At extreme momenta approaching the Brillouin zone boundary, the expectation value of the cosine shrinks, establishing a strict physical cutoff length directly from exact graph mathematics, without any heuristic Taylor approximations.

\subsection{Unitary Evolution: Deriving the Schr\"odinger Equation}
To map the classical wave equation to quantum evolution, we apply the \textbf{Paraxial Approximation}. Factoring out the ultra-fast rest-mass Compton frequency $\omega_m = mc^2/\hbar$ via an envelope function $\psi(x,t) = \Psi(x,t) e^{-i \omega_m t}$, the second time derivative drops out for non-relativistic speeds ($v \ll c$):
\begin{equation}
    i\hbar \frac{\partial \Psi}{\partial t} = -\frac{\hbar^2}{2m} \nabla^2 \Psi
\end{equation}
The Schr\"odinger Equation is mathematically identical to the paraxial envelope of a classical macroscopic wave propagating through the discrete LC circuits of the $\mathcal{M}_A$ vacuum.