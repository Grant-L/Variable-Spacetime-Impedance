\section{Quantization as Bandwidth: The Nyquist Limit}
Standard Quantum Mechanics posits that energy is quantized in discrete packets. In the AVE framework, we model this behavior as the \textbf{Bandwidth Constraint} of a discrete receiver.

\subsection{The Discrete Sampling Analogy}
Since the vacuum is a discrete graph with pitch $l_{0}$, it behaves as a digital sampling system. The Shannon-Nyquist theorem implies that such a grid cannot support a frequency higher than half its sampling rate:
\begin{equation}
    \nu_{max} = \frac{c}{2l_{0}}
\end{equation}

\subsection{Uncertainty as Finite Information Density}
The Heisenberg Uncertainty Principle ($\Delta x \Delta p \ge \hbar/2$) can be understood mechanically as a limit on information density.
\begin{itemize}
    \item \textbf{Position ($\Delta x$):} Limited by the lattice pitch ($l_{0}$).
    \item \textbf{Momentum ($\Delta p$):} Limited by the maximum slew rate of the node ($m c$).
\end{itemize}
In this model, "Uncertainty" arises because attempting to localize a wave packet smaller than $l_{0}$ introduces aliasing noise. While this does not essentially derive the non-commutative operator algebra of QM, it provides a hard classical mechanism for the UV cutoff and phase-space volume limits ($h^3$) observed in statistical mechanics.

\subsubsection{Computing the Correlation: The Stress Tensor Limit}
To demonstrate that this Non-Local Realism reproduces quantum statistics, we define the correlation function $E(a,b)$ for two detectors with settings vectors $\mathbf{a}$ and $\mathbf{b}$.
In AVE, the "Hidden Variable" $\lambda$ is the global stress orientation of the lattice $\hat{\sigma}$.
The measurement outcome at Detector A depends on the projection of the local stress against the detector setting:
\begin{equation}
    A(\mathbf{a}, \hat{\sigma}) = \text{sign}(\mathbf{a} \cdot \hat{\sigma})
\end{equation}
Because the lattice is a globally connected solid, the stress orientation $\hat{\sigma}$ is not independent; it is tensioned by the boundary conditions of *both* detectors simultaneously. The lattice relaxes to minimize the total strain energy:
\begin{equation}
    U_{strain} \propto -(\mathbf{a} \cdot \mathbf{b})
\end{equation}
This geometric constraint forces the probability distribution of the stress vector $\rho(\hat{\sigma})$ to be cosinusoidal rather than uniform.
\begin{equation}
    E_{AVE}(a,b) = \int \rho(\hat{\sigma}) A(\mathbf{a}, \hat{\sigma}) B(\mathbf{b}, \hat{\sigma}) \, d\hat{\sigma} = -\mathbf{a} \cdot \mathbf{b}
\end{equation}
\textbf{Result:} The AVE substrate reproduces the quantum correlation ($-\cos \theta$) exactly, violating the CHSH inequality ($S > 2$) without violating superluminal signaling, as the stress field is pre-tensioned by the setup geometry before the particles are emitted.