\section{Quantization as Bandwidth: The Nyquist Limit}

Standard Quantum Mechanics posits that energy is quantized in discrete packets ($E=h\nu$). In the AVE framework, this is not a magical property of light, but a strict \textbf{Bandwidth Constraint} of the discrete receiver.

\subsection{The Discrete Sampling Theorem}
Since the vacuum is a discrete graph with pitch $l_P$, it behaves as a digital sampling system. The Shannon-Nyquist theorem states that a discrete grid cannot support a frequency higher than half its sampling rate: $\nu_{max} = c / (2l_P)$.

\begin{equation}
    \nu_{max} = \frac{c}{2 l_P} \approx \frac{1}{2 t_{tick}}
\end{equation}

\subsection{Re-Deriving Heisenberg Uncertainty}
The Heisenberg Uncertainty Principle ($\Delta x \Delta p \ge \hbar/2$) is often interpreted as a fundamental limit on knowledge. In Signal Dynamics, it is re-derived as **Aliasing Noise**.
\begin{itemize}
    \item \textbf{Position ($\Delta x$):} Limited by the pixel size of the universe ($l_P$).
    \item \textbf{Momentum ($\Delta p$):} Limited by the maximum slew rate of the node ($mc$).
\end{itemize}
Trying to measure a particle's position with precision $\Delta x < l_P$ is physically impossible because there are no nodes between the lattice points to store that information. "Uncertainty" is simply the quantization error of the vacuum hardware.

