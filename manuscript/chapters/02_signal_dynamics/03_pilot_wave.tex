\section{The Pilot Wave: Deterministic Memory}
\label{sec:pilot_wave}

If the vacuum is a physical substance, then a moving particle must create a wake. We model "Quantum Probability" not as a dice roll, but as the deterministic interaction of a particle with its own **Lattice Wake** (Pilot Wave).

\subsection{Lattice Memory}
As a topological defect (mass) moves through the lattice, it displaces the nodes, creating a localized oscillation.
\begin{equation}
    \Psi_{wake}(r, t) = A \cdot e^{i(k r - \omega t)} \cdot e^{-r/L_{decay}}
\end{equation}
This wake persists for a finite relaxation time $\tau$. If the particle loops back (or passes through a slit), it encounters its own wake.

\subsection{Interference Without Magic}
In the Double Slit Experiment, the particle does not pass through both slits.
\begin{enumerate}
    \item The particle passes through **Slit A**.
    \item The Pilot Wave (pressure wave) passes through **both Slit A and Slit B**.
    \item The wave interferes with itself on the other side.
    \item The particle is "surfed" by this interference pattern to a deterministic location on the screen.
\end{enumerate}

This reproduces the statistical distribution of Quantum Mechanics ($\psi^*\psi$) purely via classical fluid dynamics on the substrate, removing the need for "Superposition" of the particle itself.

