\section{The Pilot Wave: Lattice Memory and Non-Locality}
\label{sec:pilot_wave}

If the vacuum is a physically connected substance, then a moving particle must create a wake. We model "Quantum Probability" not as a metaphysical dice roll, but as the deterministic interaction of a particle with the \textbf{Lattice Memory} of the manifold.

\subsection{Lattice Memory}
As a topological defect (mass) moves through the lattice, it displaces the nodes, creating a localized oscillation that propagates through the graph.

\begin{equation}
    \Psi_{wake}(r,t) = A \cdot e^{i(kr - \omega t)} \cdot e^{-r/L_{decay}}
\end{equation}

This wake represents the state vector of the $M_A$ manifold itself. Because the lattice is a globally connected graph, stress at one node is integrated into the global tension field. While dynamic updates propagate at $c$, the static constraint topology of the graph is pre-solved by the boundary conditions. The non-locality arises because the particle traverses a lattice that is \textit{already} globally tensioned, not because signals travel instantly.

\subsection{Interference Without Magic}
In the Double Slit Experiment, the particle does not pass through both slits.
\begin{enumerate}
    \item The particle passes through Slit A.
    \item The Lattice Memory (pressure wave) passes through both Slit A and Slit B.
    \item The wave interferes with itself on the other side.
    \item The particle is "surfed" by this interference pattern to a deterministic location on the screen.
\end{enumerate}
This reproduces the statistical distribution of Quantum Mechanics ($\psi^*\psi$) purely via classical fluid dynamics on the substrate, removing the need for "Superposition" of the particle itself.

\subsection{The Non-Local Stress Tensor: Resolving Bell's Inequality}
A standard critique of "Hidden Variable" theories is their violation of Bell's Inequalities. However, Bell's Theorem only rules out Local Hidden Variables. It does not rule out \textbf{Non-Local Realism}.

In AVE, the "Hidden Variable" is the instantaneous stress tensor $\sigma_{ij}$ of the entire $M_A$ manifold. Because the lattice is a globally connected graph, a change in impedance (measurement setting) at Detector A instantly alters the global boundary conditions of the vacuum solution.

\begin{equation}
    \nabla \cdot \sigma_{global} = 0
\end{equation}

The pilot wave does not need to transmit a signal faster than light to "tell" the particle what spin to have. The particle is traversing a lattice that is already pre-tensioned by the configuration of both detectors.

\subsubsection{Design Note 2.1: The Superdeterministic Defense}
Critics often argue that this violates "Measurement Independence" (the assumption that detector settings are independent of the particle's state). AVE explicitly accepts this as the \textbf{Superdeterministic Loophole}.

In a continuous fluid or solid mechanics model, the stress field at the source is \textit{never} independent of the boundary conditions at the detector. If one changes the impedance (setting) of a detector, the global solution to the elliptic Poisson equation updates across the entire domain.

\begin{quote}
    \textbf{The Holism Postulate:} The "decision" of the particle spin and the "decision" of the detector setting are physically linked by the pre-existing stress tensor of the vacuum substrate connecting them. Independence is an artifact of the point-particle approximation; in a connected lattice, no two events are truly independent.
\end{quote}

This does not imply "cosmic conspiracy"; it implies \textbf{Continuum Mechanics}. The universe solves the boundary value problem for the entire experimental setup as a single coherent system. Bell's inequality is violated not because the physics is magic, but because the "Independence Assumption" of the theorem is false for a solid substrate.