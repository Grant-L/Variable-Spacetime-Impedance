\section{The Pilot Wave: Lattice Memory and Non-Locality}
If the vacuum is a physically connected substance, then a moving particle must create a wake. We model "Quantum Probability" not as a metaphysical dice roll, but as the deterministic interaction of a particle with the \textbf{Lattice Memory} of the manifold.

\subsection{Lattice Memory}
As a topological defect (mass) moves through the lattice, it displaces the nodes, creating a localized oscillation that propagates through the graph.
\begin{equation}
    \Psi_{wake}(r,t) = A \cdot e^{i(kr - \omega t)} \cdot e^{-r/L_{decay}}
\end{equation}
This wake represents the state vector of the $M_A$ manifold itself. Because the lattice is a globally connected graph, stress at one node instantly contributes to the global tension of the mesh (governed by the Graph Laplacian). This naturally supports the non-local correlations observed in Bell tests without requiring "spooky action at a distance"—the action is mediated by the pre-existing tension of the substrate.

\textbf{Bell Assumption Drop:} AVE reproduces Bell-type correlations by rejecting measurement independence (the substrate state and measurement settings are not statistically independent in the full microphysical state), while preserving no-signaling at the emergent level.

\subsection{Interference Without Magic}
In the Double Slit Experiment, the particle does not pass through both slits.
\begin{enumerate}
    \item The particle passes through \textbf{Slit A}.
    \item The Lattice Memory (pressure wave) passes through \textbf{both Slit A and Slit B}.
    \item The wave interferes with itself on the other side.
    \item The particle is "surfed" by this interference pattern to a deterministic location on the screen.
\end{enumerate}
This reproduces the statistical distribution of Quantum Mechanics ($\psi^*\psi$) purely via classical fluid dynamics on the substrate, removing the need for "Superposition" of the particle itself.