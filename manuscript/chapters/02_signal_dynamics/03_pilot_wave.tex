\section{The Pilot Wave: Lattice Memory and Non-Locality}
If the vacuum is a physically connected substance, then a moving particle must create a wake. We model "Quantum Probability" not as a metaphysical dice roll, but as the deterministic interaction of a particle with the \textbf{Lattice Memory} of the manifold.

\subsection{Lattice Memory}
As a topological defect (mass) moves through the lattice, it displaces the nodes, creating a localized oscillation that propagates through the graph.
\begin{equation}
    \Psi_{wake}(r,t) = A \cdot e^{i(kr - \omega t)} \cdot e^{-r/L_{decay}}
\end{equation}
This wake represents the state vector of the $M_A$ manifold itself. Because the lattice is a globally connected graph, stress at one node is integrated into the global tension field. While dynamic updates propagate at $c$, the \textbf{static constraint topology} of the graph is pre-solved by the boundary conditions. The non-locality arises because the particle traverses a lattice that is *already* globally tensioned, not because signals travel instantly.

\subsection{Interference Without Magic}
In the Double Slit Experiment, the particle does not pass through both slits.
\begin{enumerate}
    \item The particle passes through \textbf{Slit A}.
    \item The Lattice Memory (pressure wave) passes through \textbf{both Slit A and Slit B}.
    \item The wave interferes with itself on the other side.
    \item The particle is "surfed" by this interference pattern to a deterministic location on the screen.
\end{enumerate}
This reproduces the statistical distribution of Quantum Mechanics ($\psi^*\psi$) purely via classical fluid dynamics on the substrate, removing the need for "Superposition" of the particle itself.

\subsection{The Non-Local Stress Tensor: Resolving Bell's Inequality}
A standard critique of "Hidden Variable" theories is their violation of Bell's Inequalities. However, Bell's Theorem only rules out \textit{Local} Hidden Variables. It does not rule out Non-Local Realism.

In AVE, the "Hidden Variable" is the instantaneous stress tensor $\sigma_{ij}$ of the entire $M_A$ manifold.
Because the lattice is a globally connected graph, a change in impedance (measurement setting) at Detector A instantly alters the global boundary conditions of the vacuum solution.
\begin{equation}
    \nabla \cdot \sigma_{global} = 0
\end{equation}
The pilot wave does not need to transmit a signal faster than light to "tell" the particle what spin to have. The particle is traversing a lattice that is \textit{already} pre-tensioned by the configuration of both detectors.
\textbf{Conclusion:} AVE is a Non-Local Realist framework. It reproduces quantum correlations not via "spooky action," but via the macroscopic stiffness of the connecting medium.