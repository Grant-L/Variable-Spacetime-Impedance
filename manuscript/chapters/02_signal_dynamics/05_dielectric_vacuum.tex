\subsection{Non-Linear Signal Dynamics: Dielectric Saturation Effects}
The linear wave equation derived earlier in this chapter (see §2.1) assumes constant moduli L and C per unit length in the transmission line analog of the lattice. At high displacement fields, capacitive nodes saturate (Ch. 1, §1.5), introducing voltage-dependent capacitance and non-linear propagation.

Consider a 1D lattice line (or axial direction in a waveguide/shaft). The telegrapher equations are:
\begin{equation}
    \frac{\partial V}{\partial z} = -L\frac{\partial I}{\partial t}
\end{equation}
\begin{equation}
    \frac{\partial I}{\partial z} = -C(V)\frac{\partial V}{\partial t}
\end{equation}
Differentiate the first with respect to z and substitute:
\begin{equation}
    \frac{\partial^{2}V}{\partial z^{2}} = -L\frac{\partial}{\partial t}\left(\frac{\partial I}{\partial z}\right) = L\frac{\partial}{\partial t}\left(C(V)\frac{\partial V}{\partial t}\right)
\end{equation}
Expanding the time derivative yields the full non-linear wave equation:
\begin{equation}
    \frac{\partial^{2}V}{\partial z^{2}} = LC(V)\frac{\partial^{2}V}{\partial t^{2}} + L\frac{dC}{dV}\left(\frac{\partial V}{\partial t}\right)^{2}
\end{equation}
Model saturation phenomenologically (Born-Infeld inspired):
\begin{equation}
    C(V) = \frac{C_{0}}{\sqrt{1+(\frac{V}{V_{s}})^{2}}}
\end{equation}
where $V_{s}$ scales with the local Schwinger threshold ($V_{s} \sim E_{s} l_{0}$). The derivative is:
\begin{equation}
    \frac{dC}{dV} = -C_{0}\frac{V/V_{s}^{2}}{(1+(V/V_{s})^{2})^{3/2}} = -\frac{C(V)V}{V_{s}^{2}(1+(V/V_{s})^{2})}
\end{equation}

The first term in Eq. (2.12) gives field-dependent wave speed $c(V)=1/\sqrt{LC(V)}$ (slows near saturation). The second term drives \textbf{Wave Steepening and Spectral Cascade}. Mathematically, this is not Ohmic dissipation (heat), but a nonlinear reactance that pumps energy from the carrier frequency into higher harmonic modes (Blue Shifting). As the wavefront steepens into a shock, the energy accumulates at the leading edge until it exceeds the breakdown voltage $V_{0}$, at which point true thermodynamic dissipation (pair production) occurs.