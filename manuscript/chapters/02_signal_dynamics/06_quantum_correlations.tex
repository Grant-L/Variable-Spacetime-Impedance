\section{Quantum Correlations: The Pre-Tensioned Lattice}
\label{sec:quantum_correlations}

Standard Quantum Mechanics posits non-local correlations that violate Bell's inequalities. AVE derives these correlations purely from the classical mechanics of a \textbf{Pre-Tensioned Solid}, without invoking non-locality.

\subsection{The Tensioned String Derivation}
\label{subsec:tensioned_string}

Consider the vacuum substrate between two detectors $A$ and $B$ not as a relaxed beam, but as a lattice flux tube under \textbf{Critical Tension} $T_0$. The detectors impose boundary conditions (polarization angles $\theta_A$ and $\theta_B$) on the lattice grain.

In a standard relaxed beam, the potential energy of a twist scales quadratically ($U \propto \theta^2$). However, for a string under high tension, the energy is geometric—it is proportional to the path length contraction required to accommodate the twist. The potential energy cost $\Delta U$ to twist the lattice between the boundaries is:
\begin{equation}
    U(\theta) = T_0 L (1 - \cos(\theta_A - \theta_B))
    \label{eq:tension_energy}
\end{equation}
This geometric stiffness is the "Hidden Variable" ($\lambda$) of the system.

\subsection{Deriving the Cosine Correlation}
\label{subsec:cosine_correlation}

The probability of a joint measurement outcome is governed by the lattice minimizing this global strain energy. The thermodynamic "force" restoring alignment is the derivative of the energy potential:
\begin{equation}
    F_{\mathrm{restore}} = -\frac{dU}{d\theta} \propto -\sin(\theta)
    \label{eq:restore_force}
\end{equation}
The correlation $E_{\mathrm{AVE}}(A,B)$, defined as the overlap projection of the boundary states, corresponds to the integral of this restoring force. This recovers the exact quantum prediction:
\begin{equation}
    E_{\mathrm{AVE}}(A,B) \propto \int F_{\mathrm{restore}} \, d\theta \propto -\cos(\theta_A - \theta_B)
    \label{eq:ave_correlation}
\end{equation}

\textbf{Conclusion:} Bell's inequality is violated in this framework because the "Independence Assumption" of the theorem fails for a solid substrate. \citestart The stress tensor $\sigma_{ij}$ is a global solution to the boundary value problem set by $A$ and $B$ simultaneously\cite{3}\citeend.
