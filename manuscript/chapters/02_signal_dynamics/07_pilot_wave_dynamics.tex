\subsection{Simulation IV: Lattice Memory (The Double Slit)}
\label{sec:pilot_wave_sim}

The most persistent "mystery" of quantum mechanics is the Double Slit Experiment: how can a single particle create an interference pattern? Standard interpretation invokes acausal "superposition," claiming the particle physically passes through both slits simultaneously.

Vacuum Engineering offers a strictly causal, hydrodynamic resolution: **The Particle goes through one slit; the Vacuum goes through both.**

\subsubsection{The Hydrodynamic Wake Hypothesis}
As established in Section 2.3, a particle moving through the $M_A$ lattice is not an isolated point but a topological defect coupled to the substrate. As it moves, it generates a "Wake" of lattice stress ($\phi_{wake}$) that propagates at the speed of light ($c$), faster than the particle ($v < c$).

We simulated this "Pilot Wave" dynamics using a coupled walker-wave model on the discrete lattice. Figure \ref{fig:pilot_wave_mech} visualizes the distinct paths of the particle and its wake.

\begin{figure}[h]
    \centering
    \includegraphics[width=\textwidth]{chapters/02_signal_dynamics/simulations/pilot_wave_comparison.png}
    \caption{Mechanism of Lattice Memory. \textbf{Left:} The Vacuum Wave (Wake). Because the vacuum is a connected solid, the pressure wave generated by the particle passes through \textit{both} slits, creating a global interference pattern. \textbf{Right:} The Particle Trajectory. The particle (Yellow Line) is topologically constrained to pass through a single slit. However, upon exiting, it encounters the pressure ripples originating from the \textit{other} slit. These ripples exert a gradient force (Red Arrow), steering the particle into a quantized path.}
    \label{fig:pilot_wave_mech}
\end{figure}

The process follows four deterministic steps:
\begin{enumerate}
    \item \textbf{The Source:} A single particle (walker) is fired at the barrier.
    \item \textbf{The Wake:} The particle's motion excites a pressure wave in the vacuum ($P_{vac}$). This wave front expands hemispherically and passes through \textit{both} slits.
    \item \textbf{Interference:} On the far side of the barrier, the two wavefronts (from Slit A and Slit B) interfere, creating a landscape of constructive and destructive pressure gradients.
    \item \textbf{The Path:} The particle, passing through only Slit A, encounters this interference field. The pressure gradients "surf" the particle into the constructive fringes.
\end{enumerate}

\begin{figure}[h]
    \centering
    \includegraphics[width=0.9\textwidth]{chapters/02_signal_dynamics/simulations/pilot_wave_interference.png}
    \caption{The Result: Deterministic Quantization. A high-fidelity simulation of the vacuum pressure field ($P_{vac}$). The particle (Cyan Dot) follows a "wobbly" trajectory (White Line) as it navigates the interference ridges. It lands in a constructive fringe not by chance, but by hydrodynamic necessity. The "Wave Function" is revealed to be the real-time pressure map of the vacuum hardware.}
    \label{fig:pilot_wave_result}
\end{figure}

\subsubsection{Resolution of the Measurement Problem}
This simulation demystifies the "Collapse of the Wavefunction." The "Wavefunction" is simply the real, physical stress map of the vacuum hardware.
\begin{itemize}
    \item \textbf{Before Measurement:} The vacuum stress is delocalized (Wave-like).
    \item \textbf{During Measurement:} The particle is a localized knot (Particle-like).
\end{itemize}
The particle does not "know" about the other slit; the \textit{vacuum} knows. The particle simply follows the local pressure gradient of the substrate, which retains a "Lattice Memory" of the global geometry. This reproduces the statistical predictions of Quantum Mechanics without abandoning Local Realism.