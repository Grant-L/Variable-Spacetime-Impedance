\section{The Dielectric Lagrangian: Hardware Mechanics}
\label{sec:dielectric_lagrangian}

Standard Quantum Field Theory (QFT) begins with an abstract Lagrangian density $\mathcal{L}$ that describes fields as mathematical operators. In Vacuum Engineering, we derive the Lagrangian directly from the Lumped Element Model of the substrate. The vacuum is not a continuous probability field; it is a discrete transmission network.

\subsection{Energy Storage in the Node}
The total energy density of the manifold is the sum of the energy stored in the capacitive edges (Dielectric Strain) and the inductive nodes (Flux Flow).
\begin{equation}
    \mathcal{H} = \frac{1}{2}\epsilon_{0}|\mathbf{E}|^{2} + \frac{1}{2\mu_{0}}|\mathbf{B}|^{2}
\end{equation}
This Hamiltonian $\mathcal{H}$ represents the total hardware cost of maintaining a signal.
\begin{itemize}
    \item \textbf{Kinetic Energy ($\mathcal{T}$):} Stored in the lattice compliance $\epsilon_{0}$ (Electric Field / Time-Rate of Flux).
    \item \textbf{Potential Energy ($\mathcal{U}$):} Stored in the nodal inertia $\mu_{0}$ (Magnetic Field / Spatial Flux Gradient).
\end{itemize}
\textit{Note: Because we formulate this continuous Lagrangian using the Vector Potential ($\mathbf{A}$) as the canonical coordinate, the generalized velocity is the Electric Field ($\mathbf{E} = -\partial_t \mathbf{A}$). Thus, by strict Legendre duality, the capacitive energy takes the role of Kinetic Energy, and the inductive energy takes the role of Potential Energy.}

\subsection{The Dimensionally Exact Action Principle}
In classical field theory, the Lagrangian density $\mathcal{L}$ must rigorously evaluate to energy density, measured in Joules per cubic meter $[\text{J}/\text{m}^3]$. To map the discrete LC properties of the $M_A$ manifold to a continuous field theory without dimensional violations, the canonical field variable cannot be the scalar voltage ($\phi$). 

The canonical variable must be the \textbf{Magnetic Vector Potential} ($\mathbf{A}$), defined physically as the magnetic flux linkage per unit length, measured in Webers per meter ($[\text{Wb/m}] = [\text{V} \cdot \text{s} / \text{m}]$).

The continuous Lagrangian density $\mathcal{L}_{AVE}$ for the vacuum substrate is the exact difference between the capacitive kinetic energy density and the inductive potential energy density:
\begin{equation}
    \mathcal{L}_{AVE} = \frac{1}{2} \epsilon_0 \left| \frac{\partial \mathbf{A}}{\partial t} \right|^2 - \frac{1}{2\mu_0} |\nabla \times \mathbf{A}|^2
\end{equation}

\subsection{Strict Dimensional Proof and The Ansatz Reduction}
We rigorously evaluate the SI dimensions of this functional:
\begin{itemize}
    \item \textbf{Kinetic Term:} $[\partial_t \mathbf{A}] = [\text{V/m}]$. Therefore, $\epsilon_0 |\partial_t \mathbf{A}|^2$ yields $[\text{F/m}] \cdot [\text{V}^2/\text{m}^2] = [\text{F} \cdot \text{V}^2 / \text{m}^3]$. Because $1 \text{ J} = 1 \text{ F} \cdot 1 \text{ V}^2$, this evaluates exactly to $\mathbf{[\text{J} / \text{m}^3]}$.
    \item \textbf{Potential Term:} $[\nabla \times \mathbf{A}] = [\text{Wb/m}^2] = [\text{T}]$ (Magnetic Field $\mathbf{B}$). Therefore, $\mu_0^{-1} |\nabla \times \mathbf{A}|^2$ yields $[\text{m/H}] \cdot [\text{Wb}^2/\text{m}^4] = [\text{Wb}^2 / (\text{H} \cdot \text{m}^3)]$. Because $1 \text{ H} = 1 \text{ Wb/A}$, we get $[\text{Wb} \cdot \text{A} / \text{m}^3] = [\text{V} \cdot \text{s} \cdot \text{A} / \text{m}^3] = \mathbf{[\text{J} / \text{m}^3]}$.
\end{itemize}

Dimensional homogeneity is perfectly maintained. However, the true elegance of this functional is revealed under the \textbf{Geometrodynamic Ansatz} ($1\text{ C} \equiv 1\text{ m}$). Applying this topological reduction to the Energy Density:
\begin{equation}
    \left[\frac{\text{J}}{\text{m}^3}\right] = \left[\frac{\text{N} \cdot \text{m}}{\text{m}^3}\right] = \mathbf{\left[\frac{\text{N}}{\text{m}^2}\right]} \equiv \text{Pressure (Pascals)}
\end{equation}
This mathematically proves that the Quantum Lagrangian is not an abstract energy accounting trick; it is identically the \textbf{mechanical stress tensor} of the physical vacuum substrate. Minimizing the action is strictly equivalent to minimizing structural strain in the $M_A$ manifold.