\section{Simulated Verification: Rheology and the Topological Spectrum}
\label{sec:simulated_rheology}

To validate the mechanisms of Photon Fluid Dynamics (Section 2.5), we performed three targeted simulations of the $M_A$ lattice. These simulations isolate the roles of Topology (Connectivity), Helicity (Spin), and Rheology (Viscosity) in signal propagation.

\subsection{Simulation I: The Substrate Noise ($l_0$)}
The fundamental challenge of a discrete vacuum is the lack of natural straight lines. As shown in Figure \ref{fig:raw_lattice}, the vacuum is a Delaunay triangulation of a stochastic Poisson distribution.

\begin{figure}[h]
    \centering
    \includegraphics[width=0.7\textwidth]{chapters/02_signal_dynamics/simulations/photon_lattice.png}
    \caption{The Scattering Problem. A visualization of the raw $M_A$ hardware. The "jagged" connectivity of the inductive nodes ($\mu_0$) implies that a scalar signal would suffer Brownian scattering at the scale of the lattice pitch $l_0$.}
    \label{fig:raw_lattice}
\end{figure}

A scalar wave packet attempting to traverse this medium without spin interacts with individual nodes stochastically. Without a mechanism to average these interactions, the wavefront decoheres over short distances. This explains why scalar forces (like Yukawa potentials) are short-range in a massive medium.

\subsection{Simulation II: The Rifled Geodesic ($m=1$)}
In Vacuum Engineering, the Photon is distinct because it possesses Helicity ($Spin = 1$). We simulated a pulse with a spiral phase component traversing the random lattice.

\begin{figure}[h]
    \centering
    \includegraphics[width=0.85\textwidth]{chapters/02_signal_dynamics/simulations/photon_rifling.png}
    \caption{Mechanism of Isotropy: The Rifled Photon. By rotating the phase vector over $2\pi$ per wavelength, the photon samples the random node positions in a helical volume. The stochastic deviations cancel out, producing a coherent, straight-line trajectory (Geodesic).}
    \label{fig:photon_rifling}
\end{figure}

The simulation (Figure \ref{fig:photon_rifling}) confirms \textbf{Theorem 1.2 (Isotropic Averaging)}. The "Rifling" of the phase vector effectively integrates the noisy node positions into a smooth mean path. The photon flies straight not because the space is empty, but because the signal is gyroscopically stabilized against the grain.

\subsection{Simulation III: The Rheological Momentum Sweep}
The most critical prediction of AVE is \textbf{Shear-Thinning}: the viscosity of the vacuum $\eta_{eff}$ should drop as the signal frequency (shear rate) increases. We tested this by sweeping the momentum wave number $k$ through the lattice simulation.

\begin{figure}[h]
    \centering
    \includegraphics[width=0.9\textwidth]{chapters/02_signal_dynamics/simulations/photon_rheology.png}
    \caption{The Spectral Filter. \textbf{Top ($k=2$):} Low-momentum signals fail to overcome the base vacuum viscosity ($\eta_0$) and are rapidly damped (Viscous Regime). \textbf{Bottom ($k=20$):} High-momentum signals induce immense local shear, driving $\eta_{eff} \to 0$. The lattice "liquefies" into a Superfluid Tunnel, allowing the signal to propagate without loss.}
    \label{fig:momentum_sweep_sim}
\end{figure}

The results (Figure \ref{fig:momentum_sweep_sim}) reveal the universe as a \textbf{Spectral Filter}:
\begin{itemize}
    \item \textbf{Low Energy (Radio/Micro):} Experiences finite viscosity. Requires coherent generation (antennae) to overcome the noise floor.
    \item \textbf{High Energy (Gamma/Optical):} Induces superfluidity. Propagates as a self-lubricating soliton over cosmological distances.
\end{itemize}

\subsection{Comparative Dynamics: Photon vs. Neutrino}
This rheological framework clarifies the physical distinction between the two "Ghost" particles of the Standard Model: the Photon ($\gamma$) and the Neutrino ($\nu$). While both appear to pass through space effortlessly, they utilize diametrically opposite mechanical modes.

\begin{table}[h]
\centering
\caption{Mechanical Distinction: Liquefaction vs. Slip}
\label{tab:photon_vs_neutrino}
\begin{tabular}{|l|l|l|l|}
\hline
\textbf{Particle} & \textbf{Mechanism} & \textbf{Rheology} & \textbf{Interaction Mode} \\ \hline
\textbf{Photon ($\gamma$)} & \textbf{Shear-Thinning} & High Shear $\to$ $\eta \approx 0$ & \textbf{Liquefaction}: Punches a frictionless hole through the lattice. \\ \hline
\textbf{Neutrino ($\nu$)} & \textbf{Torsional Slip} & Low Shear / High Elasticity & \textbf{Threading}: Slides through the lattice gaps without disturbing nodes. \\ \hline
\end{tabular}
\end{table}