\section{Simulated Verification I: Lattice Memory (The Double Slit)}
\label{sec:pilot_wave_sim}

The most persistent mystery of quantum mechanics is the Double Slit Experiment: how can a single particle create an interference pattern? Vacuum Engineering offers a strictly causal, hydrodynamic resolution: \textbf{The Particle goes through one slit; the Vacuum Wake goes through both.}

\subsection{The FDTD Hydrodynamic Proof}
We simulated this "Pilot Wave" dynamic using a continuous Finite-Difference Time-Domain (FDTD) solver strictly operating on the discrete hardware Lagrangian. Because the vacuum is a connected solid, the pressure wave generated by the particle passes through \textit{both} slits, creating a global interference pattern. The particle is topologically constrained to pass through a single slit. However, upon exiting, it encounters the transverse gradient of these pressure ridges, which exerts a ponderomotive force ($\mathbf{F} \propto \nabla |\Psi|^2$), "surfing" the particle deterministically into a quantized path.

\subsection{Measurement as Impedance Damping}
We simulated the "Measurement Effect" by placing a damping load at one of the slits. The detector acts as an Ohmic resistor ($R_{load}$), absorbing the energy of the vacuum wave at that specific location. This thermodynamic extraction removes the source of the interference pattern. Without the "Kick" from the second slit, the particle exiting the first slit travels ballistically. \textbf{Conclusion:} "Collapse" is simply hydrodynamic damping.

\begin{figure}[h]
    \centering
    \includegraphics[width=\textwidth]{chapters/02_signal_dynamics/simulations/outputs/double_slit_comparison.png}
    \caption{Discrete FDTD Simulation of Lattice Memory and Impedance Loading. \textbf{Top Row (Coherent):} The vacuum wake passes through both slits, creating a stable interference pressure field. Discrete particles, launched exclusively from Slit A, are deterministically "surfed" by the spatial gradients into quantized fringes. \textbf{Bottom Row (Measured):} A detector is introduced at Slit B, functioning strictly as an Ohmic impedance load ($R_{load}$). This physically dissipates the local pilot wave energy, eliminating the interference ridges. Bereft of the transverse steering gradients, the particles from Slit A travel strictly ballistically.}
    \label{fig:double_slit_comparison}
\end{figure}