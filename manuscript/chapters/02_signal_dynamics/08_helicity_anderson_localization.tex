\section{Simulated Verification II: Helicity and Anderson Localization}
\label{sec:simulated_rheology}

To validate the mechanisms of Photon Fluid Dynamics, we performed targeted simulations isolating the critical role of \textbf{Helicity (Spin)} in preventing signals from scattering on the amorphous geometry.

\subsection{The Substrate Noise ($l_{node}$)}
As established, the vacuum is a Delaunay triangulation of a stochastic Poisson-Disk distribution. The "jagged" connectivity implies that any signal without a geometric stabilizing mechanism would suffer Brownian scattering at the scale of the lattice pitch.

\begin{figure}[h]
    \centering
    \includegraphics[width=0.7\textwidth]{chapters/02_signal_dynamics/simulations/outputs/photon_lattice.png}
    \caption{The Scattering Problem. A visualization of the strict $M_A$ hardware. The jagged connectivity of the inductive nodes implies that any signal without a geometric stabilizing mechanism would suffer Brownian scattering.}
    \label{fig:raw_lattice}
\end{figure}

\subsection{Anderson Localization of Scalar Bosons ($m=0$)}
We simulated a scalar wave packet (Spin-0) attempting to traverse this medium. Because a scalar wave lacks internal angular momentum, it interacts with individual jagged nodes stochastically. 

Without a mechanism to average these interactions, geometric phase errors accumulate instantly. The wavefront completely decoheres and undergoes \textbf{Anderson Localization}, suffering exponential damping. This brilliantly derives a known physical truth: it explains precisely why scalar forces (like the Strong and Weak nuclear potentials) are strictly short-range. The amorphous geometry of the universe natively localizes them.

\subsection{The Rifled Vector Geodesic ($m=1$)}
In Vacuum Engineering, the Photon is distinct because it is a vector boson possessing Helicity ($Spin = 1$). We simulated a pulse with a spiral phase component traversing the identical random lattice.

\begin{figure}[h]
    \centering
    \includegraphics[width=0.85\textwidth]{chapters/02_signal_dynamics/simulations/outputs/photon_rifling.png}
    \caption{AVE Simulation: The Rifled Photon. A discrete wave packet traversing the amorphous $M_A$ lattice. The blue/red color gradient represents the spiral phase twist (Helicity $m=1$) interacting with the lattice nodes. This "Rifling" creates a gyroscopic stability that geometrically averages the jagged node positions into a coherent straight-line trajectory (Geodesic).}
    \label{fig:photon_rifling}
\end{figure}

The simulation (Figure \ref{fig:photon_rifling} and \ref{fig:localization_sweep}) confirms Isotropic Averaging. The "Rifling" of the phase vector effectively integrates the noisy node positions into a smooth mean path over a full $2\pi$ rotation, allowing infinite propagation.

\begin{figure}[h]
    \centering
    \includegraphics[width=0.9\textwidth]{chapters/02_signal_dynamics/simulations/outputs/anderson_localization.png}
    \caption{The Helical Filter. \textbf{Top (Spin-0):} Scalar signals suffer Anderson Localization on the random lattice, localizing the force (e.g., Weak Force). \textbf{Bottom (Spin-1):} Vector signals use geometric rifling to integrate out the spatial noise, propagating indefinitely (e.g., Electromagnetism).}
    \label{fig:localization_sweep}
\end{figure}

\subsection{Comparative Dynamics: Photon vs. Neutrino}
This rheological framework clarifies the physical distinction between the two highly-penetrating particles of the Standard Model: the Photon ($\gamma$) and the Neutrino ($\nu$). While both appear to pass through space effortlessly, they utilize diametrically opposite mechanical modes.

\begin{table}[h]
\centering
\caption{Mechanical Distinction: Liquefaction vs. Slip}
\label{tab:photon_vs_neutrino}
\renewcommand{\arraystretch}{1.5}
\begin{tabular}{|l|l|p{6cm}|}
\hline
\textbf{Particle} & \textbf{Mechanism} & \textbf{Interaction Mode} \\ \hline
\textbf{Photon ($\gamma$)} & \textbf{Slew-Rate Shearing} & \textbf{Tunneling}: Liquefies a frictionless fluidic tube via maximal local shear. \\ \hline
\textbf{Neutrino ($\nu$)} & \textbf{Torsional Slip (Spin-1/2)} & \textbf{Threading}: Slides elastically through the lattice gaps using fractional spin, without inducing structural yield. \\ \hline
\end{tabular}
\end{table}