\subsection{Deriving the Refractive Gradient via Green's Function}
A skeletal critique of emergent gravity models is the origin of the $1/r$ potential. In AVE, we derive this not from energy density, but from **Linear Elasticity of a Point Defect**.

We model a mass $M$ as a **Point of Dilatation** (a localized volume expansion) in the substrate. The scalar lattice strain $\chi(\mathbf{r})$ is governed by the Poisson equation for an elastic solid:
\begin{equation}
    \nabla^2 \chi(\mathbf{r}) = -\frac{\rho_{mass}(\mathbf{r})}{K_{vac}}
\end{equation}
Where $K_{vac} \approx c^4/G$ is the Bulk Modulus of the vacuum.

\subsubsection{The Elastic Green's Function}
For a point source $M\delta(\mathbf{r})$, the solution is given by the Green's Function of the 3D Laplacian:
\begin{equation}
    G(\mathbf{r}, \mathbf{r}') = -\frac{1}{4\pi |\mathbf{r} - \mathbf{r}'|}
\end{equation}
Convolving the source with the Green's function yields the scalar strain field:
\begin{equation}
    \chi(r) = \frac{GM}{c^2} \int \frac{\delta(\mathbf{r}')}{|\mathbf{r} - \mathbf{r}'|} d^3x' = \frac{GM}{c^2 r}
\end{equation}
\textbf{Result:} The $1/r$ falloff is not an assumption; it is the fundamental geometric response of any 3D elastic medium to a point source. The Refractive Index $n(r) \approx 1 + 2\chi(r)$ thus naturally recovers the Schwarzschild metric profile.