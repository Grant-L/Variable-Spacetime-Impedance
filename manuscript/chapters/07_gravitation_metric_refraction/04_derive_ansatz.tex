\section{Deriving the Einstein Field Equations from Elastodynamics}
\label{sec:derive_einstein_tensor}

While the Gordon Optical Metric (Section \ref{sec:gravity_as_refractive_index}) demonstrates that a variable-density dielectric reproduces the \textit{kinematics} of curved spacetime (lensing, Shapiro delay), we must rigorously derive the \textit{dynamics}. To formally supersede General Relativity, we must prove that the elastic strain of the $\mathcal{M}_A$ lattice mathematically yields the Einstein Field Equations: $G_{\mu\nu} = \frac{8\pi G}{c^4} T_{\mu\nu}$.

\subsection{The Strain-Metric Isomorphism}
In the continuum limit, the deformation of the $\mathcal{M}_A$ lattice nodes is described by a continuous displacement vector field $\xi_\mu(x)$. In the weak-field limit, the symmetric linear elastic strain tensor $\varepsilon_{\mu\nu}$ of the manifold is the Lie derivative of the displacement:
\begin{equation}
    \varepsilon_{\mu\nu} = \frac{1}{2}(\partial_\mu \xi_\nu + \partial_\nu \xi_\mu)
\end{equation}

In General Relativity, gravity is modeled as a metric perturbation $h_{\mu\nu}$ on the flat Minkowski background ($\eta_{\mu\nu}$), such that $g_{\mu\nu} = \eta_{\mu\nu} + h_{\mu\nu}$. We formally map this macroscopic metric perturbation directly to the microscopic physical strain of the lattice:
\begin{equation}
    h_{\mu\nu} \equiv -2\varepsilon_{\mu\nu}
\end{equation}
This establishes that the ``curvature of spacetime'' is identically a physical elastic deformation (compression/shear) of the discrete substrate.

\subsection{The MacCullagh-Aether Condition}
To evaluate the dynamics, we must define the elastic potential energy of the lattice. For an isotropic elastic solid, the potential energy density $\mathcal{U}_{elastic}$ is governed by the Lam\'e parameters: the shear modulus ($\mu_{vac}$) and the first Lam\'e parameter ($\lambda_{vac}$):
\begin{equation}
    \mathcal{U}_{elastic} = \frac{1}{2}\lambda_{vac} (\varepsilon^\mu_\mu)^2 + \mu_{vac} (\varepsilon_{\mu\nu}\varepsilon^{\mu\nu})
\end{equation}

Physical observation of Gravitational Waves confirms they are purely transverse, spin-2 waves propagating at the global slew rate $c$. For the $\mathcal{M}_A$ vacuum to support transverse waves without propagating unphysical longitudinal scalar pressure waves, the longitudinal modes must be non-dynamical. As historically established by the MacCullagh elastic ether theory, this enforces a strict constraint on the Lam\'e parameters:
\begin{equation}
    \lambda_{vac} = -\mu_{vac}
\end{equation}
This negative Lam\'e parameter is the unique mechanical signature of a substrate that perfectly mimics the transverse-traceless (TT) nature of gravitational waves.

\subsection{The Elastic Stress Tensor and Trace-Reversal}
Applying Hooke's Law, the internal elastic stress tensor $\sigma_{\mu\nu}$ of the vacuum reacting against an applied deformation is the derivative of the energy density with respect to strain:
\begin{equation}
    \sigma_{\mu\nu} = \frac{\partial \mathcal{U}_{elastic}}{\partial \varepsilon^{\mu\nu}} = 2\mu_{vac} \varepsilon_{\mu\nu} + \lambda_{vac} \eta_{\mu\nu} \varepsilon^\rho_\rho
\end{equation}
Substituting the MacCullagh condition $\lambda_{vac} = -\mu_{vac}$ and factoring:
\begin{equation}
    \sigma_{\mu\nu} = 2\mu_{vac} \left( \varepsilon_{\mu\nu} - \frac{1}{2}\eta_{\mu\nu} \varepsilon^\rho_\rho \right)
\end{equation}

Now, we apply our metric isomorphism ($h_{\mu\nu} = -2\varepsilon_{\mu\nu}$, which implies the trace is $h = -2\varepsilon^\rho_\rho$):
\begin{equation}
    \sigma_{\mu\nu} = -\mu_{vac} \left( h_{\mu\nu} - \frac{1}{2}\eta_{\mu\nu} h \right) \equiv -\mu_{vac} \bar{h}_{\mu\nu}
\end{equation}
Here, $\bar{h}_{\mu\nu} \equiv h_{\mu\nu} - \frac{1}{2}\eta_{\mu\nu} h$ is the exact definition of the \textbf{Trace-Reversed Metric Perturbation} used in standard General Relativity. In the AVE framework, trace-reversal is not an abstract mathematical gauge trick; it is the direct physical consequence of the $\lambda = -\mu$ Lam\'e parameter relationship of the hardware substrate.

\subsection{Recovering the Einstein Tensor ($G_{\mu\nu}$)}
By d'Alembert's principle applied to a continuum, the wave equation for the lattice reacting to an external stress-energy source $T_{\mu\nu}$ (matter) is:
\begin{equation}
    \Box \sigma_{\mu\nu} = T_{\mu\nu}
\end{equation}
Where $\Box = \nabla^2 - \frac{1}{c^2}\partial_t^2$ is the standard d'Alembertian elastic wave operator. 

To evaluate the proportionality, we use the mechanical stiffness scale of the vacuum. In Section \ref{sec:gravity_as_refractive_index}, we anchored the vacuum bulk modulus via the Planck force limit. The corresponding shear modulus for the transverse gravitational wave modes is mathematically bound to:
\begin{equation}
    \mu_{vac} = \frac{c^4}{16\pi G}
\end{equation}

Substitute our derived vacuum stress tensor $\sigma_{\mu\nu} = -\mu_{vac} \bar{h}_{\mu\nu}$ into the wave equation:
\begin{equation}
    \Box \left( -\frac{c^4}{16\pi G} \bar{h}_{\mu\nu} \right) = T_{\mu\nu}
\end{equation}
Rearranging for the metric perturbation:
\begin{equation}
    -\frac{1}{2} \Box \bar{h}_{\mu\nu} = \frac{8\pi G}{c^4} T_{\mu\nu}
\end{equation}

In the harmonic gauge ($\partial^\mu \bar{h}_{\mu\nu} = 0$, representing divergence-free fluid flow of the vacuum), the left side of this equation ($-\frac{1}{2} \Box \bar{h}_{\mu\nu}$) is exactly the linearized Einstein Tensor ($G_{\mu\nu}^{(1)}$). 
\begin{equation}
    G_{\mu\nu}^{(1)} = \frac{8\pi G}{c^4} T_{\mu\nu}
\end{equation}

\textbf{The Non-Linear Extension (Deser-Feynman Theorem):} A standard critique is that recovering linearized GR is insufficient because General Relativity is inherently non-linear. However, as proven independently by Feynman and Deser, any consistent linear spin-2 field theory that couples to its own stress-energy automatically and uniquely generates the full non-linear Einstein-Hilbert action upon infinite iteration. Because the $\mathcal{M}_A$ substrate possesses an intrinsic energy density governed by its own strain, the elastic field gravitates. The non-linearities of the full Einstein Tensor are the necessary higher-order elastic self-interactions of the vacuum hardware.

\subsection{Stress-Energy Conservation ($\nabla_\mu T^{\mu\nu} = 0$)}
In General Relativity, $\nabla_\mu T^{\mu\nu} = 0$ is enforced geometrically by the contracted Bianchi identity ($\nabla_\mu G^{\mu\nu} = 0$). In Vacuum Engineering, this is derived strictly from \textbf{Newton's Third Law} applied to continuum mechanics. 

For the vacuum lattice to remain in mechanical equilibrium, the divergence of the internal elastic stress ($\sigma^{\mu\nu}_{vac}$) must exactly balance the applied external body forces ($f^\nu_{matter}$) exerted by the topological defects:
\begin{equation}
    \partial_\mu \sigma^{\mu\nu}_{vac} + f^\nu_{matter} = 0
\end{equation}
Because the vacuum action is translationally invariant (the substrate possesses no preferred origin), Noether's Theorem guarantees that the total stress-energy of the coupled system is conserved. Therefore, the divergence of the matter stress tensor must vanish:
\begin{equation}
    \partial_\mu T^{\mu\nu} = 0
\end{equation}
Mass-energy is conserved simply because the lattice cannot exert a net force on itself without a topological defect to push against.