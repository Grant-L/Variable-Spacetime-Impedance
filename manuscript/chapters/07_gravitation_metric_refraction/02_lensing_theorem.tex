\section{The Lensing Theorem: Deriving Einstein}

With the refractive profile $n(r)$ rigorously derived from lattice elasticity, we now calculate the bending of light purely via Snell's Law and optical transit mechanics.

\subsection{Deflection of Light}

Consider a photon passing a mass $M$ with impact parameter $b$. In AVE, light curves not because "space is bent," but because the wavefront velocity is physically slower in the denser compressed lattice near the mass ($v = c/n$), causing the ray to refract inward according to Huygens' Principle.

The trajectory is governed by the gradient of the refractive index perpendicular to the path ($\nabla_\perp n$). Substituting our rigorously derived index $n(r) = 1 + \frac{2GM}{r c^2}$:

\begin{equation}
    \delta = \int_{-\infty}^{\infty} \nabla_\perp n \, dz = \int_{-\infty}^{\infty} \frac{2GM}{c^2} \frac{b}{(b^2 + z^2)^{3/2}} \, dz
\end{equation}

Evaluating this standard geometrical integral yields exactly:

\begin{equation}
    \delta = \frac{4GM}{bc^2}
\end{equation}

\textbf{Result:} This perfectly recovers the exact Einstein deflection angle solely through fluidic refraction.

\subsection{Shapiro Delay (The Refractive Delay)}

The ``slowing'' of light near a massive body is measured as the Shapiro time delay $\Delta t$. In AVE, this is simply the physical transit time integral of a wave traversing a denser dielectric fluid medium:

\begin{equation}
    \Delta t = \int_{path} \left(\frac{1}{v(r)} - \frac{1}{c}\right) dl = \frac{1}{c} \int_{path} (n(r) - 1) dl
\end{equation}

Substituting $n(r) = 1 + \frac{2GM}{rc^2}$ recovers the exact empirical Shapiro Delay:
\begin{equation}
    \Delta t \approx \frac{4GM}{c^3} \ln\left(\frac{4 x_e x_p}{b^2}\right)
\end{equation}
This confirms that the Shapiro Delay is a Dielectric Delay. The vacuum near the sun is physically "thicker," increasing the node-to-node signal processing time.