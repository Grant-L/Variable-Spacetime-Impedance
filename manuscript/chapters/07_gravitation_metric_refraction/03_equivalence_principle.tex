\section{The Equivalence Principle: $\mu$ vs $\epsilon$}
Why do all objects fall at the same rate? Standard physics invokes the Weak Equivalence Principle as an axiom. AVE derives it from \textbf{Constitutive Scaling}.

\subsection{Constitutive Law: Impedance Invariance}
We postulate that the vacuum substrate maintains a constant Characteristic Impedance ($Z_0$) even under elastic strain.
\begin{equation}
    Z(r) = \sqrt{\frac{\mu(r)}{\epsilon(r)}} \equiv Z_0 \text{ (Constant)}
\end{equation}
This implies that any local strain $\chi$ must scale the Inductance ($\mu$) and Capacitance ($\epsilon$) identically:
\begin{equation}
    \mu(r) = \mu_0 \chi, \quad \epsilon(r) = \epsilon_0 \chi
\end{equation}

\subsection{The Identity Proof}
\begin{itemize}
    \item \textbf{Inertial Mass ($m_i$):} Resistance to acceleration (Back-EMF). Proportional to Lattice Inductance ($\mu$).
    \item \textbf{Gravitational Mass ($m_g$):} Coupling to the refractive gradient. Proportional to Lattice Capacitance ($\epsilon$).
\end{itemize}
The ratio of gravitational pull to inertial resistance is:
\begin{equation}
    \frac{m_g}{m_i} = \frac{\epsilon}{\mu} = \frac{\epsilon_0 \chi}{\mu_0 \chi} = \text{Constant}
\end{equation}
\textbf{Conclusion:} Objects fall at the same rate because the property that pulls them (Capacitance) is mechanically linked to the property that slows them (Inductance) by the fixed impedance of the substrate itself. The Equivalence Principle is an \textbf{Impedance Matching} condition.