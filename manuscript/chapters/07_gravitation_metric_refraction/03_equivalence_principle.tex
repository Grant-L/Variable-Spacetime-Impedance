\section{The Equivalence Principle: $\mu$ vs $\epsilon$}
\label{sec:equivalence_principle}

Why do all objects fall at the same rate? Standard physics invokes the Weak Equivalence Principle. AVE derives it from \textbf{Constitutive Scaling}.

\subsection{Inertial Mass ($m_i$)}
Inertia is the resistance to acceleration. In AVE, this is \textbf{Back-EMF} caused by the lattice inductance $\mu$.
\begin{equation}
    m_i \propto \mu_{eff}
\end{equation}

\subsection{Gravitational Mass ($m_g$)}
Gravity is the coupling to the refractive gradient. In AVE, this is determined by the \textbf{Dielectric Saturation} energy density $\epsilon$.
\begin{equation}
    m_g \propto \epsilon_{eff}
\end{equation}

\subsection{The Identity Proof}
Because the vacuum maintains constant impedance $Z_0 = \sqrt{\mu/\epsilon}$, any local strain $\chi$ must scale $\mu$ and $\epsilon$ identically:
\begin{equation}
    \mu(r) = \mu_0 \chi, \quad \epsilon(r) = \epsilon_0 \chi
\end{equation}
Therefore:
\begin{equation}
    \frac{m_g}{m_i} = \frac{\epsilon}{\mu} = \text{Constant}
\end{equation}
\textbf{Conclusion:} Objects fall at the same rate because the property that pulls them (Capacitance) is mechanically linked to the property that slows them (Inductance) by the impedance of the substrate itself. The Equivalence Principle is an \textbf{Impedance Matching} condition.