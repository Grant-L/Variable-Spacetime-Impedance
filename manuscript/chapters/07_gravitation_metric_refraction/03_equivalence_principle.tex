\section{The Equivalence Principle: Ponderomotive Force}

Why do all objects, regardless of mass, fall at the same rate? Standard physics invokes the Weak Equivalence Principle ($m_i = m_g$) as an unexplained axiom. AVE derives it strictly from \textbf{Macroscopic Wave Mechanics} and Impedance Matching.

In Chapters 3 and 4, we mathematically proved that fermions and baryons are not solid point particles; they are localized topological standing waves resonating within the $\mathcal{M}_A$ substrate. 

\subsection{Impedance Invariance}

We postulate that the vacuum substrate maintains a strictly constant Characteristic Impedance ($Z_0$) even under elastic strain to prevent wave scattering:
\begin{equation}
    Z_{local}(r) = \sqrt{\frac{\mu(r)}{\epsilon(r)}} \equiv Z_0 \text{ (Constant)}
\end{equation}

To maintain this invariant ratio while simultaneously altering the local wave speed ($v = c/n = 1/\sqrt{\mu\epsilon}$), both the physical Inductance ($\mu$) and Capacitance ($\epsilon$) must scale identically and proportionally to the refractive index $n(r)$:
\begin{equation}
    \mu(r) = \mu_0 \cdot n(r), \quad \epsilon(r) = \epsilon_0 \cdot n(r)
\end{equation}
As $r \to \infty$, $n(r) \to 1$, completely recovering the zero-density vacuum baseline.

\subsection{The Ponderomotive Force}

When any bounded wave packet enters a medium with a variable refractive index $n(r)$, it experiences a macroscopic kinematic drift toward the denser medium to minimize its energy. This is a purely classical phenomenon known as the \textbf{Ponderomotive Force}:

\begin{equation}
    \mathbf{F}_{grav} = -\nabla U_{wave}
\end{equation}

The localized energy of the trapped topological knot is its rest mass ($m_i c^2$) scaled inversely by the refractive density of the local environment:
\begin{equation}
    U_{wave}(\mathbf{r}) = \frac{m_i c^2}{n(\mathbf{r})}
\end{equation}

Taking the spatial gradient of this energy functional directly yields the gravitational force:
\begin{equation}
    \mathbf{F}_{grav} = -\nabla \left( \frac{m_i c^2}{n(\mathbf{r})} \right) = m_i c^2 \left( \frac{\nabla n}{n^2} \right)
\end{equation}

\textbf{Conclusion:} Notice that the gravitational force $\mathbf{F}_{grav}$ is identically and algebraically proportional to the particle's internal inductive inertia $m_i$. There is no separate "gravitational charge" ($m_g$). The Equivalence Principle is mechanically guaranteed by the refraction of a localized wave packet seeking the lowest energy state in a dielectric gradient.

\begin{figure}[htbp]
    \centering
    \includegraphics[width=0.85\textwidth]{chapters/07_gravitation_metric_refraction/simulations/outputs/ponderomotive_equivalence.png}
    \caption{\textbf{The Equivalence Principle via Ponderomotive Force.} When a wave packet enters a refractive density gradient, its stored energy scales inversely with the local index $n(x)$. The spatial derivative of this energy drives acceleration. Because the energy is fundamentally defined by the particle's inductive mass $m_i$, the resulting acceleration is independent of the mass magnitude, strictly deriving $m_i \equiv m_g$.}
    \label{fig:ponderomotive_equivalence}
\end{figure}