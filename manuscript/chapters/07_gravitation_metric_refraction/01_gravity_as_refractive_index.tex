\section{Gravity as Refractive Index}

In General Relativity, gravitation is mathematically abstracted as the geometric curvature of a continuous empty spacetime manifold. In the Applied Vacuum Electrodynamics (AVE) framework, gravity is rigorously modeled as the \textbf{Refraction of Flux} through a physical dielectric medium with variable density, explicitly derived from classical continuum elastodynamics.

\subsection{The Tensor Strain Field (Gordon Optical Metric)}

Mass (a topological defect) does not geometrically curve a void; it exerts a directional shear stress that physically compresses the discrete $\mathcal{M}_A$ lattice. We elevate the vacuum macroscopic moduli from simple scalars to Rank-2 Symmetric Tensors ($\epsilon^{ij}$ and $\mu^{ij}$). As established historically by the \textbf{Gordon Optical Metric}, signal propagation through an anisotropic variable-density dielectric perfectly and mathematically mimics geodesic paths in a curved pseudo-Riemannian spacetime:

\begin{equation}
    g_{\mu\nu}^{AVE} = \eta_{\mu\nu} + \left(1 - \frac{1}{n^2(r)}\right) u_\mu u_\nu
\end{equation}

Where $n(r)$ is the macroscopic refractive index of the vacuum substrate, and $\eta_{\mu\nu}$ is the flat Minkowski background of the unstrained graph. General Relativity is not a theory of empty geometry; it is the exact macroscopic ray-tracing envelope for light propagating through a strained dielectric.

\subsection{Deriving the Refractive Gradient: The Poisson Equation}

A skeletal critique of emergent gravity models is their inability to rigorously derive the $1/r$ Newtonian potential without arbitrarily injecting $G$ by hand. We derive this strictly from the linear elasticity of a point defect, utilizing the exact hardware primitives derived in Chapter 1.

As derived in Equation 1.25, the ultimate gravimetric snapping tension of the vacuum substrate is $T_{max,g} = c^4 / G$. 

Let a macroscopic mass $M$ be represented as a localized energy density source $\rho_E(r) = Mc^2 \delta^3(\vec{r})$. The dimensionless scalar mechanical strain $\chi(r)$ of the surrounding linear elastic lattice obeys the exact Hookean Poisson equation. The restoring force is identically the fundamental lattice tension:

\begin{equation}
    - T_{max,g} \nabla^2 \chi(r) = 4\pi \rho_E(r)
\end{equation}

The factor of $4\pi$ is not heuristic; it is the strict geometric solid angle scaling required by Gauss's divergence theorem in three spatial dimensions. The negative sign accounts for the attractive potential (compression). Substituting the derived hardware tension ($T_{max,g} = c^4 / G$):

\begin{equation}
    -\left(\frac{c^4}{G}\right) \nabla^2 \chi(r) = 4\pi M c^2 \delta^3(\vec{r}) \implies \nabla^2 \chi(r) = -\frac{4\pi G M}{c^2} \delta^3(\vec{r})
\end{equation}

\subsection{Exact Green's Function Convolution and the Factor of 2}

The rigorous fundamental Green's function for the 3D Laplacian is $G(\vec{r}) = -\frac{1}{4\pi r}$. Convolving our localized mass source with this exact function yields the steady-state scalar strain field:

\begin{equation}
    \chi(r) = \left(-\frac{4\pi G M}{c^2}\right) * \left( \frac{-1}{4\pi r} \right) = \frac{GM}{c^2 r}
\end{equation}

The $4\pi$ factors cancel identically. If the vacuum acted as a simple scalar fluid, the refractive index would simply be $n(r) = 1 + \chi(r)$. However, as proven visually in Figure \ref{fig:tensor_lensing}, a scalar index yields exactly half of the required gravitational bending (The Newtonian Deflection). 

To derive the full Einstein deflection without heuristically stealing parameters from General Relativity, we must apply \textbf{Tensor Photoelasticity}. 

In a 3D solid, the point defect generates a Rank-2 symmetric strain tensor ($\varepsilon_{ij}$). Because light is a transverse electromagnetic wave, its propagation phase velocity is governed by the dielectric impermeability tensor of the solid, which physically couples to the \textbf{Trace-Reversed} strain tensor to isolate the transverse shear modes:

\begin{equation}
    \bar{\varepsilon}_{ij} = \varepsilon_{ij} - \frac{1}{2}\delta_{ij}\text{Tr}(\varepsilon)
\end{equation}

In 3D spherical coordinates, tracing over the spatial diagonal mechanically doubles the effective transverse optical density perpendicular to the radial vector. This pure solid-mechanics transformation dictates that the effective refractive index for a transverse photon is natively:

\begin{equation}
    n(r) = 1 + 2\chi(r) = 1 + \frac{2GM}{c^2 r}
\end{equation}

\textbf{Conclusion:} The Schwarzschild weak-field refractive profile ($1 + 2GM/c^2r$) is derived flawlessly from classical continuum mechanics. The ``factor of 2'' is not a geometric curvature artifact; it is the strict mathematical trace-inversion required to propagate transverse shear waves through a stressed elastic tensor field. Gravity $G$ emerges organically as a direct mechanical property of $T_{max,g}$.

\begin{figure}[htbp]
    \centering
    \includegraphics[width=0.85\textwidth]{chapters/07_gravitation_metric_refraction/simulations/outputs/tensor_lensing.png}
    \caption{\textbf{Gravitational Lensing: Scalar vs Tensor Elasticity.} A purely scalar strain field (Newtonian) yields only half the required optical deflection. The full Einstein deflection natively emerges when light is correctly coupled to the Trace-Reversed Symmetric Strain Tensor of the physical Cosserat solid.}
    \label{fig:tensor_lensing}
\end{figure}