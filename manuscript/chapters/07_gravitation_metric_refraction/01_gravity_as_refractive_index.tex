\section{Gravity as Macroscopic Metric Refraction}

In standard General Relativity, gravitation is mathematically abstracted as the geometric curvature of an otherwise continuous, empty spacetime manifold. In the Applied Vacuum Engineering (AVE) framework, gravity is rigorously defined as the \textbf{Optical Refraction} of continuous phase-flux through a physical dielectric continuum possessing variable density, explicitly derived from classical solid-state elastodynamics.

\subsection{The Tensor Strain Field (Gordon Optical Metric)}
Mass (a trapped topological defect) does not geometrically curve a void; its immense internal inductive pressure exerts a directional shear stress that physically compresses the discrete $\mathcal{M}_A$ Cosserat lattice. 

We elevate the macroscopic vacuum moduli derived in Chapter 1 from simple scalars to Rank-2 Symmetric Tensors ($\epsilon^{ij}$ and $\mu^{ij}$). As established historically by the \textbf{Gordon Optical Metric}, signal propagation through an anisotropic, variable-density continuous dielectric perfectly and mathematically mimics geodesic paths natively drawn in a curved pseudo-Riemannian spacetime:
\begin{equation}
    g_{\mu\nu}^{AVE} = \eta_{\mu\nu} + \left(1 - \frac{1}{n^2(\mathbf{r})}\right) u_\mu u_\nu
\end{equation}
Where $n(\mathbf{r})$ is the macroscopic continuous refractive index of the vacuum substrate, and $\eta_{\mu\nu}$ is the flat Minkowski background of the unstrained graph. General Relativity is not a theory of empty geometric curvature; it is the exact macroscopic ray-tracing envelope for light propagating through a strained dielectric solid.

\subsection{Deriving the Refractive Gradient from Lattice Tension}
A fundamental critique of emergent gravity models is their inability to rigorously derive the $1/r$ Newtonian potential without arbitrarily injecting the empirical constant $G$ by hand. We derive this strictly from the linear elasticity of a point defect, utilizing the exact hardware boundaries derived in Chapter 1.

As derived in Equation 1.25, the ultimate gravimetric snapping tension of the vacuum substrate is evaluated as $G_{calc} = 7G$, resulting in a macroscopic continuous tension of $T_{max,g} = c^4 / 7G$. 

Let a macroscopic mass $M$ be represented as a localized rest-energy density source $\rho_E(\mathbf{r}) = Mc^2 \delta^3(\mathbf{r})$. The dimensionless raw 3D volumetric mechanical strain $\chi_{vol}(\mathbf{r})$ of the surrounding linear elastic lattice obeys the exact Hookean Poisson equation. The structural restoring force mapping this strain is identically the fundamental lattice tension:
\begin{equation}
    - T_{max,g} \nabla^2 \chi_{vol}(\mathbf{r}) = 4\pi \rho_E(\mathbf{r})
\end{equation}

The factor of $4\pi$ is not heuristic; it is the strict geometric solid angle scaling required by Gauss's divergence theorem in three spatial dimensions. The negative sign properly accounts for the attractive (compressive) potential. Substituting the derived hardware tension ($T_{max,g} = c^4 / 7G$):
\begin{equation}
    -\left(\frac{c^4}{7G}\right) \nabla^2 \chi_{vol}(\mathbf{r}) = 4\pi M c^2 \delta^3(\mathbf{r}) \implies \nabla^2 \chi_{vol}(\mathbf{r}) = -\frac{28\pi G M}{c^4} \delta^3(\mathbf{r})
\end{equation}

\subsection{Exact Green's Function Convolution}
The rigorous fundamental Green's function for the 3D Laplacian is $\Gamma(\mathbf{r}) = -1 / (4\pi r)$. Convolving our localized mass source with this exact function yields the steady-state raw 3D volumetric strain field of the spatial lattice:
\begin{equation}
    \chi_{vol}(r) = \left(-\frac{28\pi G M}{c^4}\right) * \left( \frac{-1}{4\pi r} \right) = \mathbf{\frac{7GM}{c^2 r}}
\end{equation}
The physical vacuum lattice around a massive body is volumetrically compressed by exactly $7GM / c^2 r$.