\section{Gravity as Refractive Index}
\label{sec:gravity_refractive_index}

In General Relativity, gravity is the curvature of spacetime geometry. In Vacuum Engineering, it is the Refraction of Flux through a medium with variable density. A massive object saturates the local vacuum nodes, increasing their Inductive Inertia ($\mu$) and Dielectric Stiffness ($\epsilon$).

\subsection{The Tensor Strain Field (Gordon Optical Metric)}
If gravity were a simple scalar refractive index $n(r)$, the vacuum could only support longitudinal compression waves. This is strictly falsified by the detection of transverse, spin-2 quadrupole Gravitational Waves (e.g., by LIGO).

Mass does not compress the $M_A$ lattice isotropically; it exerts a directional \textit{shear stress}. We elevate the vacuum moduli from scalars to Rank-2 Symmetric Tensors ($\epsilon^{ij}$ and $\mu^{ij}$). As established by the Gordon Optical Metric formulation, an anisotropic dielectric perfectly mimics a curved spacetime geometry:
\begin{equation}
g_{\mu\nu}^{VSI} = \eta_{\mu\nu} + \left( 1 - \frac{1}{n^2(r)} \right) u_\mu u_\nu
\end{equation}
By upgrading the moduli to tensors, the ``Hardware Vacuum'' recovers all tensor mathematics of General Relativity, including transverse gravitational waves. Gravity remains a refractive phenomenon, but it operates via \textbf{Tensor Refraction}.