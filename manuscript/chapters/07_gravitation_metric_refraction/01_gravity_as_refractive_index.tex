\section{Gravity as Refractive Index}

In General Relativity, gravity is the curvature of spacetime geometry. In AVE, it is the Refraction of Flux through a medium with variable density. 

\subsection{The Tensor Strain Field (Gordon Optical Metric)}
If gravity were a simple scalar refractive index $n(r)$, the vacuum could only support longitudinal waves. This is falsified by the detection of transverse Gravitational Waves (LIGO).

Mass does not compress the $M_A$ lattice isotropically; it exerts a directional \textit{shear stress}. We elevate the vacuum moduli from scalars to Rank-2 Symmetric Tensors ($\epsilon^{ij}$ and $\mu^{ij}$). As established by the Gordon Optical Metric, an anisotropic dielectric perfectly mimics a curved spacetime geometry:
\begin{equation}
g_{\mu\nu}^{AVE} = \eta_{\mu\nu} + \left( 1 - \frac{1}{n^2(r)} \right) u_\mu u_\nu
\end{equation}
By upgrading the moduli to tensors, the AVE ``Hardware Vacuum'' recovers all tensor mathematics of General Relativity. Gravity operates via \textbf{Tensor Refraction}.