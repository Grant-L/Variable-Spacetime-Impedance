\chapter{Gravitation: Metric Refraction}
\label{ch:gravitation}

\section{Gravity as Refractive Index}
\label{sec:refractive_index}

In General Relativity, gravity is the curvature of spacetime geometry. In Vacuum Engineering, it is the \textbf{Refraction of Flux} through a medium with variable density.

We posit that a massive object saturates the local vacuum nodes, increasing their Inductive Inertia ($\mu$) and Dielectric Stiffness ($\epsilon$). This creates a local gradient in the Refractive Index ($n$).

\subsection{The Metric Strain Field ($\chi$)}
We define the metric deformation $\chi(r)$ not by geometry, but by \textbf{Flux Conservation}.
\begin{equation}
    n(r) = \sqrt{\frac{\mu(r)\epsilon(r)}{\mu_0\epsilon_0}} = 1 + \frac{2GM}{rc^2}
\end{equation}
\textbf{Derivation:}
\begin{enumerate}
\item \textbf{Energy Equipartition:} A mass $M$ represents a stored energy $E = Mc^2$.
\item \textbf{Strain Distribution:} This energy strains the surrounding lattice. To maintain impedance matching ($Z = \sqrt{\mu/\epsilon} = Z_0$), the strain must be distributed equally between Inductance ($\mu$) and Capacitance ($\epsilon$).
\item \textbf{Potential:} The gravitational potential is $\Phi = -GM/r$.
\item \textbf{Refractive Index:} The effective index is $n \approx 1 - 2\Phi/c^2$. The factor of 2 arises strictly from the equipartition of strain energy (half in $\mu$, half in $\epsilon$).
\end{enumerate}