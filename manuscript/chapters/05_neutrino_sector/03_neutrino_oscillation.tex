\section{Neutrino Oscillation: Dispersive Beat Frequencies}

A complete physical model of the Neutrino sector must mathematically account for Flavor Mixing (Neutrino Oscillation)—the verified phenomenon where a neutrino deterministically shifts between Electron ($\nu_e$), Muon ($\nu_\mu$), and Tau ($\nu_\tau$) detection profiles as it traverses the vacuum.

\subsection{Torsional Harmonics}
If massive Leptons are structurally defined by an integer topological crossing resonance (the $3_1, 7_1, 11_1$ knots), the Neutrinos are identically defined by \textbf{Torsional Harmonics} loaded onto the zero-crossing unknot ($0_1$). The three discrete flavors correspond directly to the quantized number of full internal twists ($T$) physically pumped into the continuous loop during the high-energy topological snap of the Weak Interaction:
\begin{itemize}
    \item \textbf{Electron Neutrino ($\nu_e$):} Fundamental Torsion ($T=1$).
    \item \textbf{Muon Neutrino ($\nu_\mu$):} First Overtone ($T=2$).
    \item \textbf{Tau Neutrino ($\nu_\tau$):} Second Overtone ($T=3$).
\end{itemize}

\subsection{Mechanical Derivation of the PMNS Matrix}
When a neutrino is emitted, it is topologically synthesized as a specific, discrete mechanical superposition of these torsional harmonics. In a perfectly continuous, mathematically idealized vacuum, all spatial frequencies would propagate at exactly the identical speed of light ($c$), their relative phases would perfectly lock, and the composite state would never alter.

However, we must address an apparent paradox. In Chapter 1, we proved that massless gauge bosons (photons) completely evade the non-linear dispersion relation of the discrete lattice because they are purely transverse continuous link-variables. If so, why do neutrinos oscillate?

The resolution physically unifies the framework: \textbf{Neutrinos are not massless gauge bosons; they are massive topological defects.} Because they possess inductive rest mass, they are strictly constrained to travel below the speed of light ($v < c$). Because their matter-waves actively interact with the discrete geometric grid, they are natively subjected to the explicit frequency-dependent \textbf{Dispersion Relation} for all massive modes derived in Chapter 1:
\begin{equation}
    v_g(k) = c \cos\left(\frac{k \cdot l_{node}}{2}\right)
\end{equation}

Because the $T=1, 2$, and $3$ torsional overtones inherently possess different spatial wavenumbers ($k_i$), they physically propagate through the discrete Cosserat grid at fractionally different group velocities ($v_g$). As the composite wave packet travels macroscopic distances, the distinct mass harmonics systematically and geometrically drift out of phase relative to each other ($\Delta \Phi_i = k_i (c/v_{g,i} - 1) L$).

\begin{figure}[htbp]
    \centering
    \includegraphics[width=0.95\textwidth]{chapters/05_neutrino_sector/simulations/outputs/neutrino_oscillation_beat.png}
    \caption{\textbf{Neutrino Oscillation via Exact Lattice Dispersion.} The probability profile of detecting a specific topological flavor oscillates periodically as an exact function of propagation distance. This is mathematically identically to a macroscopic fluidic \textbf{Beat Frequency}. It is the direct, un-tuned mechanical consequence of multi-harmonic massive states propagating at slightly different phase velocities across a physical spatial grid possessing a non-zero hardware pitch ($l_{node}$).}
    \label{fig:neutrino_oscillation}
\end{figure}

Neutrino oscillation is not abstract quantum state-vector magic; it is literally the classical, acoustic \textbf{Beat Frequency} of a multi-harmonic torsional wave packet undergoing microscopic structural dispersion across the fundamental hardware grid of the universe. The empirical PMNS (Pontecorvo-Maki-Nakagawa-Sakata) mixing matrix is strictly mathematically isomorphic to the classical coupled-oscillator phase transition matrix for these dispersing mechanical overtones.