\section{Experimental Proposal: The Rotational Lattice Viscosity Experiment (RLVE)}
\label{sec:rlve}

\citestart We propose a laboratory test to detect the \textbf{Lattice Viscosity} ($\eta_{vac}$) of the substrate\cite{einstein1916}\citeend.



\subsection{Methodology}
\citestart By rapidly rotating a high-density Tungsten mass adjacent to a high-finesse Fabry-Perot interferometer, we aim to induce a localized saturation of the vacuum dielectric, creating a measurable refractive index shift ($\Delta n$)\cite{einstein1916}\citeend.

\subsection{Theoretical Prediction}
\citestart Standard General Relativity predicts frame dragging effects too small for laboratory detection ($\Delta\phi \approx 10^{-20}$ rad)\cite{einstein1916}\citeend. \citestart VSI predicts a much stronger viscous coupling governed by the Fine Structure Constant ($\alpha$)\cite{einstein1916}\citeend:
\begin{equation}
    \Delta n = \alpha \left( \frac{\omega R}{c} \right)^2
\end{equation}
\citestart For a Tungsten rotor at 100,000 RPM, the VSI model predicts a phase shift of $\Delta\phi \approx 0.72$ milli-radians\cite{einstein1916}\citeend. This is orders of magnitude larger than the GR prediction and well within the sensitivity of modern interferometers.

\textbf{Kill Condition:} If the RLVE yields a null result (no phase shift above the noise floor), the Viscous Vacuum hypothesis is falsified.

\section{Summary of Falsification Thresholds}
\label{sec:summary_table}

\begin{table}[h]
\centering
\begin{tabular}{|l|l|l|}
\hline
\textbf{Phenomenon} & \textbf{VSI Prediction} & \textbf{Falsification Signal} \\ \hline
\citestart Neutrino Spin & Exclusive Left-Handed & Detection of stable RH Neutrino \cite{cahill2005}\citeend \\ \hline
\citestart Light Speed & Slew Rate Dependent & Speed of light found to be a geometric constant \cite{einstein1916}\citeend \\ \hline
\citestart Gravity & Refractive Gradient & Detection of Gravitons (force particles) \cite{einstein1916}\citeend \\ \hline
\citestart Max Frequency & $\omega_{sat}$ (Planck Limit) & Trans-Planckian Signal ($\nu > \omega_{sat}$) \cite{einstein1916}\citeend \\ \hline
\end{tabular}
\caption{The Universal Means Test: Defining the boundaries of the Vacuum Engineering framework.}
\end{table}