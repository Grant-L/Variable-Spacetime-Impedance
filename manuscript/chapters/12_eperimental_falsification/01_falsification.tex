\chapter{Falsifiability: The Universal Means Test}
\label{ch:falsification}

\section{The Universal Means Test}
\label{sec:means_test}

The Vacuum Engineering framework is a vulnerable theory. \citestart Unlike string theory, which often operates at energy scales inaccessible to experimentation, the Discrete Amorphous Manifold ($M_A$) makes specific, testable predictions about the hardware limits of the vacuum\cite{einstein1916}\citeend.

Its validity rests on the following falsification thresholds:

\begin{enumerate}
    \citestart \item \textbf{The Neutrino Parity Test:} Detection of a stable Right-Handed Neutrino falsifies the Chiral Bias postulate\cite{einstein1916}\citeend.
    \citestart \item \textbf{The Nyquist Limit:} Detection of any signal with $\nu > \omega_{sat}$ (Trans-Planckian) proves the vacuum is a continuum, killing the discrete manifold model\cite{einstein1916}\citeend.
    \citestart \item \textbf{The Metric Null-Result:} If local impedance modification fails to produce refractive delays (Shapiro delay) in the lab, the Engineering Layer is falsified\cite{einstein1916}\citeend.
\end{enumerate}

