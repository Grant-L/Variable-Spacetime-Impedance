\section{The Generative Vacuum Hypothesis}

Standard cosmology relies on the assumption of Metric Expansion---that space ``stretches'' due to a geometric scale factor. The AVE framework proposes a hardware-based alternative: \textbf{Lattice Genesis}. We model the vacuum not as a continuum that stretches, but as a discrete lattice that multiplies.

\subsection{The Growth Equation}
Let $N(t)$ be the total number of nodes along a line of sight. The Lattice Tension induces a proliferation of nodes proportional to the existing population (geometric growth):
\begin{equation}
    \frac{dN}{dt} = R_{g} N(t)
\end{equation}
Where $R_g$ is the \textbf{Node Genesis Rate} (Hz). Solving for $N(t)$:
\begin{equation}
    N(t) = N_0 e^{R_g t}
\end{equation}

\subsection{Recovering Hubble's Law}
The physical distance $D$ is the node count $N$ times the Lattice Pitch $l_P$. The recession velocity $v$ is the rate of growth:
\begin{equation}
    v = \frac{dD}{dt} = l_P \frac{dN}{dt} = l_P (R_g N) = R_g D
\end{equation}
Comparing this to Hubble's Law ($v = H_0 D$), we identify the Hubble Constant mechanically:
\begin{equation}
    H_0 \equiv R_{genesis} \approx 2.3 \times 10^{-18} \text{ Hz}
\end{equation}
\textbf{Conclusion:} The "Expansion of the Universe" is simply the real-time refresh rate of the vacuum hardware. Every second, the lattice creates $2.3 \times 10^{-18}$ new nodes for every existing node.