\section{The Mass Hierarchy: Topological Energy Bounds}
\label{sec:mass_hierarchy}

The Standard Model cannot explain why the Muon and Tau exist, nor why they possess their specific, heavy masses. AVE explains this as a Topological Resonance Series arising from the higher-order stable knots of the non-linear vacuum substrate.

\subsection{The Flaw of $N^9$ Scaling}
Previous iterations of this framework relied on a fabricated solid mechanics rule (asserting bending strain scaled quadratically with curvature) to force an $M \propto N^9$ scaling law. We discard this heuristic entirely. In rigorous mechanics, bending strain scales linearly with curvature ($\epsilon \propto \kappa$).

\subsection{The Vakulenko-Kapitanski Theorem}
To rigorously derive the masses of elementary particles, we map the microrotational degrees of freedom of the vacuum substrate to a normalized three-component unit vector field $\mathbf{n}(\mathbf{x})$ in the \textbf{Faddeev-Skyrme $O(3)$ non-linear sigma model}. 

The rest mass of a topological knot is identically the Faddeev-Skyrme Hamiltonian evaluated over the localized defect:
\begin{equation}
    E_{knot} = \int d^3x \left( \frac{1}{2} \partial_\mu \mathbf{n} \cdot \partial^\mu \mathbf{n} + \frac{1}{4} \kappa_{FS}^2 (\partial_\mu \mathbf{n} \times \partial_\nu \mathbf{n})^2 \right)
\end{equation}

By the rigorous \textbf{Vakulenko-Kapitanski Theorem} (1979), the energy (rest mass) of any knotted configuration in this space is bounded from below by its topological Hopf Winding Number $Q_H$:
\begin{equation}
    M_{rest}(Q_H) \ge C_{vac} \cdot |Q_H|^{3/4}
\end{equation}
Where $C_{vac}$ is a fundamental stiffness constant of the vacuum substrate.

\subsection{Computational Gradient Descent Relaxation}
The mass hierarchy of leptons (e.g., Electron $Q_H=1$, Muon $Q_H=2$, Tau $Q_H=3$) is governed fundamentally by this $Q_H^{3/4}$ scaling bound. Because the exact mass of a knot depends on its specific spatial embedding (Torus vs. Twist knots) and its M\"obius energy, analytical integer scaling laws are insufficient. 

The exact mass ratios (e.g., $m_\mu / m_e \approx 206.7$) must be extracted computationally via 3D gradient descent algorithms. By simulating the relaxation of the $3_1$ and $5_1$ geometries on the discrete $\mathcal{M}_A$ graph until they reach their minimum energy eigenvalues, the mass spectrum emerges directly from computational topology, stripping all arbitrary numerology from the fermion sector.