\section{The Fundamental Theorem of Knots}
\label{sec:fundamental_theorem_of_knots}

In the DCVE framework, ``Matter'' is not a substance distinct from the vacuum; it is a localized, self-sustaining topological knot in the vacuum's flux field. We posit that every stable elementary particle corresponds to a discrete graph topology. The physical properties of the particle must be derived strictly from the non-linear topology of this knot.

\subsection{Fine Structure ($\alpha$) via Magnetic Helicity}
The Fine Structure Constant ($\alpha$) cannot be derived by heuristically summing scalar geometries of differing SI dimensions. It must be defined rigorously as the dimensionless topological self-impedance of the minimal ground-state knot (the Electron, modeled as a $3_1$ Trefoil). 

Because the canonical variable of the discrete manifold is the Magnetic Vector Potential $\mathbf{A}$, the energy coupling of the knot to the linear lattice is dictated by its \textbf{Magnetic Helicity} ($\int \mathbf{A} \cdot \mathbf{B} \, d^3x$). To yield a purely dimensionless scalar, $\alpha$ is derived by computing the exact \textbf{Neumann Self-Inductance Integral} over the minimal $Q_H=1$ knot geometry $\gamma$, normalized by the fundamental flux quantum $\Phi_0$:
\begin{equation}
    \alpha \propto \frac{1}{\Phi_0^2} \oint_{\gamma} \oint_{\gamma} \frac{d\mathbf{l}_1 \cdot d\mathbf{l}_2}{4\pi |\mathbf{r}_1 - \mathbf{r}_2|}
\end{equation}
This mathematically guarantees a dimensionless scalar output based strictly on the geometric self-inductance of the topological knot, anchoring $\alpha$ to computational topology rather than numeric coincidence.

\subsection{Mass as Inductive Energy}
We have defined the vacuum edges as possessing distributed inductance $\mu_0$. Therefore, any closed loop of topological flux stores energy in the localized magnetic field:
\begin{equation}
    E_{mass} = \frac{1}{2}L_{eff} |\mathbf{A}|^2
\end{equation}
Where $L_{eff}$ is the Effective Inductance of the knotted manifold. Mass is simply the Stored Inductive Energy required to maintain the topological integrity of the knot against the elastic pressure of the vacuum. 

\textbf{Circuit Analogy: The Inductive Flywheel.}
Why does mass resist acceleration? In DCVE, we replace the concept of ``Mass'' with the electrical concept of Inductive Inertia. A heavy flywheel resists changes in rotation; when you try to spin it up, it fights you (Back-EMF). An elementary particle is a knot of flux spinning so fast it acts as a Gyroscopic Flywheel. It resists acceleration not because it has ``stuff'' inside it, but because the magnetic field possesses Lenz's Law Inertia.