\chapter{Topological Crystallography: The Fermion Sector}
\label{ch:fermion_sector}

\section{The Fundamental Theorem of Knots}
\label{sec:knot_theorem}

In the Vacuum Engineering framework, "Matter" is not a substance distinct from the vacuum; it is a localized, self-sustaining knot in the vacuum's flux field.

We posit that every stable elementary particle corresponds to a **Prime Knot** topology. The physical properties of the particle are derived strictly from the geometry of this knot.

\subsection{Mass as Inductive Energy}
We have defined the vacuum node as having inductance $\mu_0$ (Section 1.2). Therefore, any loop of flux $I_\phi$ stores energy in the magnetic field.
\begin{equation}
    E_{mass} = \frac{1}{2} L_{eff} I_\phi^2
\end{equation}
Where $L_{eff}$ is the Effective Inductance of the knot.
\begin{itemize}
    \item **Standard Loop ($N=1$):** Low inductance. (Neutrino).
    \item **Knotted Loop ($N > 1$):** High inductance due to mutual coupling between the crossings. (Electron/Proton).
\end{itemize}
**Conclusion:** Mass is simply the **Stored Inductive Energy** required to maintain the topological integrity of the knot against the elastic pressure of the vacuum.

