\section{The Electron: The Trefoil Soliton ($3_1$)}
\label{sec:electron_trefoil}

We identify the Electron ($e^-$) as the simplest non-trivial knot: the **Trefoil Knot** ($3_1$).

\subsection{Deriving Alpha: The Geometric Impedance}
The Fine Structure Constant ($\alpha \approx 1/137$) is one of the greatest mysteries of physics. In VSI, it is the **Geometric Impedance Ratio** of the Trefoil.

The impedance of a free photon (linear flux) is $Z_0 \approx 377 \Omega$. The impedance of a knotted soliton ($Z_{knot}$) is higher due to the self-induction of its crossings. We propose that $\alpha$ represents the coupling efficiency between the linear lattice and the knotted soliton:
\begin{equation}
    \alpha \equiv \frac{Z_{vac}}{Z_{knot}} \approx \frac{1}{137.036}
\end{equation}
This suggests that the value $137$ is not arbitrary; it is the **Inductive Q-Factor** of a minimal Trefoil knot. The electron couples weakly to the vacuum (low charge) because most of its flux is trapped in self-reinforcing loops.

