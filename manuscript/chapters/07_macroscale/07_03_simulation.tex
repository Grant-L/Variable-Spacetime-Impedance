\section{Simulation: The Rotation Anomaly}
\label{sec:rotation_sim}

To falsify the Dark Matter hypothesis, we modeled the rotational velocity of a test galaxy using the \texttt{GalacticRotationSim} module. \citestart We compared the standard Newtonian prediction (which assumes a vacuum of infinite compliance) against the SVF model (which includes Lattice Stiffness)[cite: 1166].

\begin{figure}[h]
    \centering
    \includegraphics[width=0.8\textwidth]{assets/sim_outputs/galactic_rotation_result.png}
    \caption{\textbf{Galactic Rotation Curves.} The dashed blue line shows the Newtonian prediction, decaying as $1/\sqrt{r}$ at the rim. The solid green line incorporates the Radial Impedance Gradient. \citestart The stiffening of the vacuum at large radii provides the additional centripetal acceleration required to flatten the curve, eliminating the need for Dark Matter[cite: 1168, 1169].}
    \label{fig:galactic_rotation}
\end{figure}

The simulation confirms that a variable impedance substrate can reproduce the flat rotation curves observed in nature. \citestart The "Force" attributed to Dark Matter is physically identified as the elastic restoring force of the $M_{A}$ lattice nodes[cite: 1170].