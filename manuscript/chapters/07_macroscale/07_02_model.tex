\section{The Tully-Fisher Relation as an Impedance Law}
\label{sec:tully_fisher}

The observed relationship between a galaxy's luminosity and its rotation velocity is derived here as a consequence of the manifold’s saturation limit. \citestart In SVF, the \textbf{Tully-Fisher Relation} is the macroscale equivalent of the **Saturation Threshold** ($\omega_{sat}$) established in Chapter 2, where the total mass-load dictates the maximum rotational velocity the local lattice can support[cite: 417].

\section{Falsification: The Bullet Cluster and Lattice Memory}
\citestart A primary "Means Test" for the SVF framework is the observation of the \textbf{Bullet Cluster}, where gravitational lensing appears offset from visible gas[cite: 411].

\begin{itemize}
    \citestart \item \textbf{SVF Prediction:} The "Impedance Wake" left behind by colliding galaxies persists in the $M_A$ manifold as \textbf{Lattice Memory} (Phase Lag)[cite: 412]. \citestart Because the nodes possess a finite update frequency, the metric strain $\epsilon$ does not vanish instantly when the baryonic mass moves[cite: 413].
    \citestart \item \textbf{Failure Condition:} If the gravitational lensing signal is found to be instantaneous or perfectly coupled to the center of mass of the gas during the collision, the SVF "Impedance Wake" theory is false[cite: 414].
\end{itemize}