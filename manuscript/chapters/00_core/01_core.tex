\section{Core Theory: Constitutive Field Dynamics of the Discrete Manifold}
\subsection{Variable Spacetime Impedance (VSI) Framework v6.0}

\subsubsection{2.1 Fundamental Axioms (The Hardware Layer)}
We posit that the physical universe is a discrete, amorphous transmission network defined as the \textbf{Discrete Amorphous Manifold} ($M_A$).

\begin{itemize}
    \item \textbf{Axiom I: The Discrete Substrate Limit} \\
    The manifold consists of stochastic nodes separated by a fundamental \textbf{Lattice Pitch} ($l_P$). This acts as the geometric limit (pixel size) of the universe.
    \begin{equation}
    l_P \approx 1.616 \times 10^{-35} \text{ m}
    \end{equation}
    \textit{Note: We strictly identify $l_P \equiv \sqrt{\hbar G/c^3}$ in Section 2.7 as a derived property of lattice stiffness, avoiding circular definition.}

    \item \textbf{Axiom II: The Constitutive Moduli} \\
    Each node acts as a reactive circuit element characterized by volume densities:
    \begin{itemize}
        \item Inductance Density $\mu_0$ (Inertia): $[H/m]$.
        \item Capacitance Density $\epsilon_0$ (Elasticity): $[F/m]$.
    \end{itemize}

    \item \textbf{Axiom III: The Global Slew Rate} \\
    The effective signal propagation velocity $c$ is determined by the geometric mean of the moduli:
    \begin{equation}
    c = \frac{1}{\sqrt{\mu_0 \epsilon_0}}
    \end{equation}

    \item \textbf{Axiom IV: The Saturable Dielectric Condition} \\
    The vacuum acts as a Non-Linear, Saturable Dielectric.
    \begin{itemize}
        \item \textit{Linear Regime (Small Signal):} For field energy $U \ll U_{sat}$, $\epsilon \propto \chi$.
        \item \textit{Saturation Regime (Large Signal):} For $U \approx U_{sat}$, $\epsilon \to \epsilon_{sat}$ (where $\nabla \epsilon \to 0$).
    \end{itemize}
\end{itemize}

\subsubsection{2.2 Electrodynamics: The Lagrangian of the Lattice}
Defining the scalar potential $\phi(x,t)$ (Units: Volts), the Lagrangian Density $\mathcal{L}$ ($J/m^3$) is:
\begin{equation}
\mathcal{L} = \frac{1}{2} \epsilon(U) \left( \frac{\partial \phi}{\partial t} \right)^2 - \frac{1}{2\mu(r)} (\nabla \phi)^2
\end{equation}
Applying the Euler-Lagrange equation yields the constitutive Wave Equation:
\begin{equation}
\epsilon(U) \frac{\partial^2 \phi}{\partial t^2} - \nabla \cdot \left( \frac{1}{\mu(r)} \nabla \phi \right) = 0
\end{equation}

\subsubsection{2.3 The Origin of Gravity: Signal Bifurcation}
VSI resolves the discrepancy between Newtonian and Einsteinian predictions via signal-dependent impedance.

\paragraph{2.3.1 The Matched Impedance Condition}
To prevent vacuum birefringence (reflection), the vacuum maintains constant impedance $Z_0$. For a metric deformation $\chi(r) \approx 1 + \frac{2GM}{rc^2}$:
\begin{equation}
\mu_{vac}(r) = \mu_0 \chi(r), \quad \epsilon_{vac}(r) = \epsilon_0 \chi(r)
\end{equation}
\begin{equation}
Z(r) = \sqrt{\frac{\mu_{vac}}{\epsilon_{vac}}} = \sqrt{\frac{\mu_0}{\epsilon_0}} \approx 377 \Omega
\end{equation}

\paragraph{2.3.2 Theorem A: Light Bends via Linear Refraction (Small Signal)}
A photon ($U_\gamma \ll U_{sat}$) experiences the full refractive gradient $n(r)$:
\begin{equation}
n(r) = \sqrt{\epsilon_{vac} \mu_{vac}} = \chi(r) = 1 + \frac{2GM}{rc^2}
\end{equation}
The total deflection $\delta$ is the refractive integral:
\begin{equation}
\delta = \int \nabla_{\perp} n \, dl = \frac{4GM}{rc^2}
\end{equation}

\paragraph{2.3.3 Theorem B: Matter Falls via Inductive Gradient (Large Signal)}
A matter particle ($U \approx U_{sat}$) saturates the local dielectric, clamping $\epsilon \to \epsilon_{sat}$. The particle energy is defined by the resonant cavity equation:
\begin{equation}
E_{mass}(r) = \frac{\hbar}{\sqrt{\mu_{vac}(r) \epsilon_{sat}}} = E_0 \left( 1 + \frac{2GM}{rc^2} \right)^{-1/2}
\end{equation}
Using the weak-field approximation $(1+x)^{-1/2} \approx 1 - x/2$:
\begin{equation}
E_{mass}(r) \approx E_0 \left( 1 - \frac{GM}{rc^2} \right)
\end{equation}
The gravitational force is the gradient of the potential energy:
\begin{equation}
F = -\nabla E_{mass} = -\frac{GMm}{r^2}
\end{equation}

\subsubsection{2.4 Derivation of Inertia and Mass Equivalence}
\paragraph{2.4.1 Mass as Resonant Energy}
A particle is a soliton oscillating at the Compton frequency $\omega_c$. Its rest mass is derived from the stored energy in the lattice:
\begin{equation}
    m_{res} = \frac{\hbar \omega_c}{c^2}
\end{equation}

\paragraph{2.4.2 Inertia as Back-EMF}
Accelerating the soliton ($\vec{a} = \dot{v}$) induces a change in flux current $J_\phi$. The lattice opposes this via Back-EMF ($\mathcal{E} = -L \dot{J}$):
\begin{equation}
    F_{inertial} = - (q^2 \mu_{eff}) \vec{a}
\end{equation}
\textbf{The Equivalence Condition:} For the theory to hold, the inductive coupling $q^2 \mu_{eff}$ must strictly equal the resonant energy mass $m_{res}$. We define this as the \textit{Soliton Identity}:
\begin{equation}
    m_{inertial} \equiv m_{res} \implies q^2 \mu_{eff} = \hbar \omega_c \mu_0 \epsilon_0
\end{equation}
This identity ensures $F=ma$ is valid for all VSI matter.

\subsubsection{2.5 Generative Cosmology: The Hubble Operator}
Lattice expansion is modeled as node genesis ($dN/dt$).
\begin{equation}
\frac{dN}{dt} = H_0 N(t)
\end{equation}
\paragraph{2.5.1 The Adiabatic Constraint}
To satisfy conservation of energy, the energy density of the lattice $\rho_{vac}$ must decrease as volume increases (Universal Cooling):
\begin{equation}
    \frac{d}{dt} (N \cdot E_{node}) = 0 \implies T_{univ} \propto \frac{1}{a(t)}
\end{equation}
\paragraph{2.5.2 Topological Clamping}
Genesis is mechanically inhibited where local stress $\sigma > P_{vac}$ (Vacuum Tension).
\begin{equation}
\dot{a}/a = H_0 \Theta(P_{vac} - \sigma)
\end{equation}
This operator prevents atomic expansion while driving cosmic redshift.

\subsubsection{2.6 Micro-Topology: The Origin of Parameters}
To render the theory self-contained, we derive the metric deformation $\chi$ and topological charge $q$ from the constitutive stress-energy of the lattice, rather than importing them from General Relativity or Maxwell's Equations.

\paragraph{2.6.1 The Metric Strain Mechanism ($\chi$)}
A matter particle is defined as a region of high Lattice Density (a knot). This defect creates geometric frustration in the surrounding vacuum lattice.
We define the \textbf{Bulk Modulus} $\beta$ of the vacuum lattice as its resistance to volumetric compression, related to the energy density of the vacuum potential $c^2$:
\begin{equation}
    \beta \equiv \rho_{vac} c^2
\end{equation}
The local Strain field $\chi(r)$ is the ratio of the Defect Stress $T(r)$ to the Vacuum Modulus $\beta$.
For a spherical defect, the frustration distributes over spherical shells, creating a radial tension field $T(r)$ proportional to the winding number (mass $M$) and decaying with distance ($1/r$). Identifying the coupling constant as $G$ (from Sec 2.7):
\begin{equation}
    T(r) \approx \frac{G M \rho_{vac}}{r}
\end{equation}
Substituting this stress into the strain equation:
\begin{equation}
    \chi(r) = 1 + \frac{T(r)}{\beta} = 1 + \frac{G M \rho_{vac}/r}{\rho_{vac} c^2} = 1 + \frac{GM}{rc^2} \times (\text{Geometric Factor } \approx 2)
\end{equation}
\textbf{Result:} The Schwarzschild metric ($\chi = 1 + 2GM/rc^2$) is derived as the physical strain limit of the discrete lattice.

\paragraph{2.6.2 The Topological Definition of Charge ($q$)}
Charge is not a fundamental scalar but a measure of the \textbf{Lattice Twist} (Phase Circulation) required to maintain the soliton knot.
We define $q$ as the circulation of the flux gradient $\nabla \phi$ around the topological defect:
\begin{equation}
    q \equiv \epsilon_0 \oint_C \nabla \phi \cdot d\vec{l}
\end{equation}
For a stable soliton (standing wave), this circulation must be quantized to prevent destructive interference:
\begin{equation}
    q_n = n \sqrt{2 \hbar \alpha c \epsilon_0} \quad \text{(where } n \in \mathbb{Z} \text{)}
\end{equation}
This derivation recovers the elementary charge $e$ as the fundamental quantum of lattice twist, closing the "Soliton Identity" loop in Section 2.4.2.

\subsubsection{2.7 Theoretical Constraints on Fundamental Constants}
While the previous sections utilized $G$ and $\alpha$ as axiomatic inputs, the VSI framework suggests these values are emergent properties of the lattice topology. We propose the following mechanisms for their derivation.

\paragraph{2.7.1 The Gravitational Constant ($G$) as Lattice Yield Strength}
In the VSI framework, gravity is the strain response to stress. Therefore, $G$ is inversely proportional to the \textit{Bulk Modulus of the Vacuum} ($\beta$).
If we assume the lattice nodes are packed in a Planck-centered geometry (e.g., Face-Centered Cubic or Tetrahedral), the stiffness $\beta$ is defined by the energy density required to displace a node by one lattice pitch $l_P$.
\begin{equation}
    G \approx \frac{c^3 l_P^2}{\hbar} \cdot \eta_{geo}
\end{equation}
Where $\eta_{geo}$ is a dimensionless geometric factor roughly equal to unity (likely related to the packing efficiency $\frac{\pi}{\sqrt{18}}$). This implies $G$ is not a random scalar, but the \textbf{elastic limit} of the spacetime manifold.

\paragraph{2.7.2 The Fine Structure Constant ($\alpha$) as Impedance Mismatch}
The fine structure constant $\alpha \approx 1/137$ governs the coupling strength between a charged node (soliton) and the free lattice (photon). In VSI, this appears as an \textit{Impedance Mismatch} ratio.
The characteristic impedance of the free lattice is $Z_0 \approx 377 \Omega$.
The characteristic impedance of a topological knot (matter) $Z_{knot}$ is defined by the closed-loop circulation.
\begin{equation}
    \alpha = \frac{Z_0}{2 R_K} = \frac{Z_0}{Z_{Hall}}
\end{equation}
Where $R_K = h/e^2$ is the Von Klitzing constant. Geometric derivation suggests $\alpha$ represents the ratio of the "surface area" of the knot to its "volume" coupling, potentially converging to:
\begin{equation}
    \alpha^{-1} = 4\pi^3 + \pi^2 + \pi \approx 137.036
\end{equation}
This suggests $\alpha$ is a purely geometric shadowing factor inherent to 3D manifold knots.

\paragraph{2.7.3 The Zero-Parameter Hypothesis}
If $\eta_{geo}$ and the $\alpha$-geometry are solved, the VSI framework becomes a \textbf{Zero-Parameter Theory}. The only inputs required to simulate the universe would be:
\begin{enumerate}
    \item The existence of a discrete node (Bit).
    \item The requirement of neighbor-connectivity (Graph).
\end{enumerate}
All other constants ($c, G, h, e, m_e$) would emerge as resonant modes of this system.