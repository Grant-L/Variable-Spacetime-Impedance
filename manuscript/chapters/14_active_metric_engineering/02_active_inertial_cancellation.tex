\section{Active Inertial Cancellation}
In classical engineering, extreme acceleration maneuvers are limited entirely by the structural shear limits of internal delicate instrumentation. Under the Topo-Kinematic Identity, macroscopic inertial G-forces are not an abstract consequence of coordinate geometry; they are literally and mathematically equivalent to \textbf{Inductive Voltage Spikes} within the lattice. 

When a physical test mass ($m$) accelerates rapidly ($a$), the discrete inductive nodes of the substrate resist the displacement, generating a massive back-electromotive force ($V_{spike} = -L \frac{di}{dt} \propto -ma$). Because inertial resistance is an electrical transient, it can be damped electrically.

By utilizing high-temperature superconducting (HTS) coils integrated directly into the outer casing of a test vehicle, a control system can actively monitor the inductive wake. During a severe acceleration shock, the boundary coils dynamically inject an opposing Vector Potential ($-\partial_t \mathbf{A}$) into the interior cavity. This acts as a \textbf{Transient Metric Snubber}, generating an exact Counter-Electromotive Force (CEMF) that electrically shunts the inductive spike. The effective G-force experienced by the internal payload is safely reduced to near-zero, actively decoupled from the external macroscopic acceleration of the hull.