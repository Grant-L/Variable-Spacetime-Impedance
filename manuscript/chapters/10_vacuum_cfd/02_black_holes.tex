\section{Black Holes: The Trans-Sonic Sink}

General Relativity describes a Black Hole as a geometric mathematical singularity. Vacuum Computational Fluid Dynamics (VCFD) describes it mechanically as a \textbf{Trans-Sonic Fluid Sink}.

By adopting the Gullstrand-Painlevé coordinate transformation, gravity can be formally represented as the flow of the vacuum fluid itself. Space flows radially inward toward the mass like a river falling into a sink ($v_{flow}(r) = -\sqrt{2GM/r}$).

In this hydrodynamic continuum, the invariant speed of light ($c$) acts exactly as the \textbf{Speed of Sound} ($c_s$) of the vacuum fluid. Consequently, the ``Event Horizon'' ($R_s$) is physically and mechanically identified as the \textbf{Sonic Point (Mach 1)}. The inward river moves exactly at the speed of sound ($|v_{flow}| = c$). Light trying to propagate outward is swept backward at the exact speed it travels forward, freezing it in place as a trapped standing wave.