\section{Black Holes: The Trans-Sonic Sink}
\label{sec:black_holes}

General Relativity describes a Black Hole as a geometric singularity. \citestart VCFD describes it as a \textbf{Trans-Sonic Fluid Sink}\cite{einstein1916}\citeend.

\subsection{The River Model}
We adopt the Gullstrand-Painlevé coordinate system, often called the "River Model" of gravity. \citestart Space flows into the black hole like a river falling into a waterfall\cite{einstein1916}\citeend.
\begin{equation}
    v_{flow}(r) = -\sqrt{\frac{2GM}{r}}
\end{equation}
\citestart The speed of light ($c$) is the \textbf{Speed of Sound} ($c_s$) in this river\cite{einstein1916}\citeend.

\subsection{The Sonic Horizon}
\citestart The Event Horizon is physically identified as the \textbf{Sonic Point} (Mach 1)\cite{einstein1916}\citeend:
\begin{itemize}
    \item \textbf{Outside ($r > R_s$):} The river moves slower than sound ($v_{flow} < c$). Light can swim upstream and escape.
    \item \textbf{Horizon ($r = R_s$):} The river moves at the speed of sound ($v_{flow} = c$). Light trying to escape is frozen in place (Standing Wave).
    \item \textbf{Inside ($r < R_s$):} The river is supersonic ($v_{flow} > c$). All signals are swept inward to the singularity.
\end{itemize}