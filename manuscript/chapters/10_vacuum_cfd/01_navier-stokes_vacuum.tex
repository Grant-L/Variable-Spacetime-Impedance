\section{Continuum Mechanics of the Amorphous Manifold}

If the vacuum is a physical graph ($\mathcal{M}_A$) supporting momentum and wave propagation, its macroscopic low-energy effective field theory (EFT) must flawlessly map to continuum fluid dynamics. We propose that the macroscopic kinematics of the universe are governed exactly by the generalized Navier-Stokes Equations applied to the structural density and non-Newtonian rheology of the substrate.

\subsection{The Dimensionally Exact Density and Momentum Equation}

Previous classical aether models failed because they incorrectly mapped vacuum density to magnetic permeability ($\mu_0$); however, this violates SI dimensional analysis, as $[H/m] \neq [kg/m^3]$. Furthermore, tying density strictly to localized transient electromagnetic fields results in a divide-by-zero singularity in empty space, causing fluid acceleration to diverge to infinity.

To resolve this, we strictly define the baseline macroscopic bulk mass density ($\rho_{bulk}$) of the vacuum fluid using the exact hardware invariants derived in Chapter 1. By the Geometrodynamic Ansatz, the inductive inertia of a single node is $L_{node} = \mu_0 l_{node}$. Dividing this mass by the derived Voronoi volume of a node ($\kappa_V l_{node}^3$) seamlessly yields a constant, massive substrate density:

\begin{equation}
    \rho_{bulk} = \frac{\mu_0 l_{node}}{\kappa_V l_{node}^3} = \frac{\mu_0}{\kappa_V l_{node}^2} \quad \left[\frac{kg}{m^3}\right]
\end{equation}

With a rigorously defined, invariant background density, the flow of the vacuum substrate ($\mathbf{u}$) is governed by the dimensionally exact Cauchy momentum equation. Integrating the Shear-Thinning Bingham rheology ($\eta(\dot{\gamma})$) derived in Chapter 9, the governing equation is:

\begin{equation}
    \rho_{bulk} \left( \frac{\partial \mathbf{u}}{\partial t} + \mathbf{u} \cdot \nabla \mathbf{u} \right) = -\nabla P + \nabla \cdot \left[ \eta(\dot{\gamma}) \left( \nabla \mathbf{u} + (\nabla \mathbf{u})^T \right) \right] + \mathbf{f}_{ext}
\end{equation}

Where $P$ is the scalar dielectric strain potential (Pressure). In the limit where viscosity is dominant and flow is steady, the spatial pressure gradient in the fluid maps exactly to the Newtonian gravitational potential, mathematically confirming that General Relativity operates as the macroscopic hydrodynamics of this substrate.

\subsection{Deriving Kinematic Viscosity ($\nu_{vac}$)}

In classical kinetic theory, the Kinematic Viscosity ($\nu$) of a fluid is the product of its signal velocity and its mean free path, modulated by a dissipation factor. 

For the $\mathcal{M}_A$ lattice, the absolute signal velocity is $c$, and the mean free path is the fundamental lattice pitch $l_{node}$. As rigorously derived in Chapter 3, the inverse of the Fine Structure Constant ($\alpha^{-1} \approx 137$) is the exact geometric Q-Factor of the lattice. Therefore, $\alpha$ itself represents the dimensionless \textbf{Structural Dissipation Factor} of the network.

Multiplying these mechanical hardware primitives together yields the exact Kinematic Viscosity of the vacuum, perfectly satisfying SI units $[m^2/s]$ without any heuristic tuning:
\begin{equation}
    \nu_{vac} = \alpha \cdot c \cdot l_{node}
\end{equation}