\section{Navier-Stokes for the Vacuum}
\label{sec:navier_stokes_vacuum}

If the vacuum is a physical fluid (the Amorphous Manifold), it must obey continuum fluid dynamics. We propose that the macroscopic kinematics of the universe are governed by the Navier-Stokes Equations applied to the effective kinematic density ($\rho_{vac}$) and structural viscosity ($\eta_{vac}$) of the substrate.

\subsection{The Dimensionally Exact Momentum Equation}
To apply classical fluid dynamics to the electromagnetic vacuum, we must rigorously define the effective mass density of the substrate. Previous heuristic models incorrectly mapped density to magnetic permeability ($\mu_0$); however, this violates SI dimensional analysis, as $[\text{H}/\text{m}] \neq [\text{kg}/\text{m}^3]$. 

We define the effective kinematic vacuum density ($\rho_{vac}$) via mass-energy equivalence applied to the local electromagnetic energy density $u_{local}$:
\begin{equation}
    \rho_{vac} = \frac{u_{local}}{c^2} = \frac{\frac{1}{2}\epsilon_0 |\mathbf{E}|^2 + \frac{1}{2\mu_0}|\mathbf{B}|^2}{c^2} \quad \left[\frac{\text{kg}}{\text{m}^3}\right]
\end{equation}

The flow of the vacuum substrate ($\mathbf{u}$) is governed by the dimensionally exact momentum equation:
\begin{equation}
    \rho_{vac} \left( \frac{\partial \mathbf{u}}{\partial t} + \mathbf{u} \cdot \nabla \mathbf{u} \right) = -\nabla u_{local} + \eta_{vac} \nabla^2 \mathbf{u} + \mathbf{f}_{ext}
\end{equation}
Where:
\begin{itemize}
    \item $\rho_{vac}$: Equivalent Kinematic Mass Density $[\text{kg}/\text{m}^3]$.
    \item $u_{local}$: The scalar energy potential (Pressure) $[\text{J}/\text{m}^3] = [\text{Pa}]$.
    \item $\eta_{vac}$: The dynamic Lattice Viscosity (Dark Matter coupling) $[\text{Pa} \cdot \text{s}]$.
\end{itemize}

\subsection{Recovering Gravity}
In the limit where viscosity is dominant ($\eta \gg 0$) and flow is steady, the spatial pressure gradient in the fluid maps exactly to the Newtonian gravitational potential, confirming that General Relativity operates as the macroscopic hydrodynamics of the substrate.