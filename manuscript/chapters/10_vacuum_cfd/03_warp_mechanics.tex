\section{Warp Mechanics: Supersonic Pressure Vessels}
\label{sec:warp_mechanics}

The Alcubierre Warp Drive is often described geometrically. \citestart In VCFD, it is a \textbf{Supersonic Pressure Vessel}\cite{alcubierre1994}\citeend.

\subsection{The Moving Pressure Gradient}
\citestart A warp drive functions by creating a localized pressure gradient: High Pressure (Compression) in the front, Low Pressure (Rarefaction) in the rear\cite{einstein1916}\citeend.
\begin{equation}
    v_{bubble} \propto \Delta P = P_{rear} - P_{front}
\end{equation}

\subsection{The Vacuum Sonic Boom (Cherenkov Radiation)}
\citestart When the bubble velocity $v_b$ exceeds the vacuum sound speed $c$ (Mach $> 1$), a conical \textbf{Bow Shock} forms at the leading edge\cite{einstein1916}\citeend.
\begin{itemize}
    \item \textbf{Hazard:} This shockwave continuously accumulates high-energy vacuum fluctuations (Hawking Radiation).
    \item \textbf{Doppler Piling:} At the shock front, the lattice is stressed faster than it can relax ($\tau \approx l_{node}/c$). \citestart This forces the generated flux waves into the highest possible frequency modes (Gamma/Blue spectrum)\cite{einstein1916}\citeend.
\end{itemize}
\textbf{Engineering Implication:} Upon deceleration, this accumulated "Blue Flash" is released forward, potentially sterilizing the destination. \citestart A practical warp drive requires active \textbf{Flow Control} (Streamlining) to mitigate this shock\cite{einstein1916}\citeend.