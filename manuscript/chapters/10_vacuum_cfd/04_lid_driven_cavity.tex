\section{VCFD Benchmark: Discrete Graph Calculus}

To computationally validate the VCFD model, we evaluate the classical ``Lid-Driven Cavity'' benchmark utilizing the exact topological discrete operators of the $\mathcal{M}_A$ graph.

Rather than relying on continuous partial differential equations, the true physics of the vacuum must be evaluated via finite-difference operations across adjacent nodes. The graph divergence ($\mathbf{D}$) and gradient ($\mathbf{G}$) matrices map potentials from nodes to edges, strictly conserving local flux. 

The discrete Laplacian operator ($\mathbf{L} = \mathbf{D} \mathbf{G}$) allows us to solve the Pressure-Poisson equation exactly on the $\mathcal{M}_A$ hardware:
\begin{equation}
    \mathbf{L} P^{n+1} = \frac{\rho_{bulk}}{\Delta t} \mathbf{D} \mathbf{u}^*
\end{equation}

Where $\mathbf{u}^*$ is the intermediate velocity field. Evaluating this purely algebraic matrix equation under constant shear from a moving boundary flawlessly generates a stable central vortex. 

\begin{figure}[htbp]
    \centering
    \includegraphics[width=0.85\textwidth]{chapters/10_vacuum_cfd/simulations/outputs/lid_driven_cavity.png}
    \caption{\textbf{VCFD Lid-Driven Cavity Result.} By applying the rigorously derived kinematic viscosity ($\nu_{vac} = \alpha c l_{node}$), the Navier-Stokes momentum equations force the formation of a stable central vortex. In AVE theory, this macroscopic rotational stability is the hydrodynamic precursor to Topological Matter generation.}
    \label{fig:lid_cavity}
\end{figure}