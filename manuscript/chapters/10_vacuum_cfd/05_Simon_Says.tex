\section{The ``Simon Says'' Test: Turbulence and Quantum Foam}
\label{sec:simon_says}

A persistent skepticism regarding the hydrodynamic vacuum hypothesis is the lack of visible turbulence. The argument proceeds: ``If space is a fluid, why do we not see it splashing?''

The Applied Vacuum Electrodynamics (AVE) framework offers a direct counter-argument: We do see it. The phenomenon standard physics calls \textbf{Quantum Mechanics}---with its probabilistic clouds, uncertainty, and wave-particle duality---is precisely the observation of \textbf{Vacuum Turbulence}.

\subsection{The Kelvin-Helmholtz Instability of Space}
To demonstrate this, we modeled the vacuum as a fluid obeying the Shear-Thinning rheology derived in Chapter 9 ($\eta(\dot{\gamma})$). We established a high-energy shear layer, analogous to the boundary of a particle jet or the event horizon interface.

\begin{figure}[h]
    \centering
    \includegraphics[width=0.9\textwidth]{chapters/10_vacuum_cfd/simulations/vacuum_turbulence.png}
    \caption{The ``Simon Says'' Simulation: Vacuum Turbulence. \textbf{Top Left (T=0):} At low energy, the vacuum is highly viscous ($\eta \approx \eta_0$). Flow is laminar and predictable (Classical Physics). \textbf{Top Right (T=50):} As shear increases, the non-Newtonian viscosity drops locally (Shear-Thinning). \textbf{Bottom Left (T=200):} The viscosity crash triggers a Reynolds Number spike ($Re \to \infty$), causing the laminar layer to fracture into chaotic vortices. This turbulent state is mathematically identical to the ``Quantum Foam'' of virtual particles. \textbf{Bottom Right:} The Viscosity Map confirms that the vacuum becomes a superfluid (White) only where the stress is highest.}
    \label{fig:vacuum_turbulence}
\end{figure}

\subsection{The Deterministic Origin of Quantum Chaos}
The simulation (Figure \ref{fig:vacuum_turbulence}) reveals two distinct regimes governed by the local energy density (Shear Rate):

\begin{itemize}
    \item \textbf{Classical Regime (Low Energy):} The vacuum viscosity is high ($Re \ll 1$). Flow is laminar. Space acts like a rigid solid.
    \item \textbf{Quantum Regime (High Energy):} The energy density drives the shear stress above the critical limit $\dot{\gamma}_c$. The local viscosity collapses ($\eta \to 0$). The Reynolds number spikes ($Re \gg 1$), and the vacuum fractures into a turbulent cascade of Kelvin-Helmholtz instabilities.
\end{itemize}

\textbf{Conclusion:} ``Quantum Foam'' is not random acausal fluctuation. It is \textbf{Deterministic Turbulence}. We do not need to add randomness to the universe; we simply need to solve the Navier-Stokes equations for a shear-thinning fluid. The ``Chaos'' of quantum probability is the unavoidable hydrodynamic turbulence of the hardware itself.