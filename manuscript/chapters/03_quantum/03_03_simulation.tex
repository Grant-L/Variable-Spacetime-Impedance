\section{Simulation: Determinism and the Pilot Wave}
\label{sec:pilot_wave_sim}

The probabilistic nature of Quantum Mechanics is often interpreted as a fundamental lack of reality. \citestart SVF restores determinism through \textbf{Lattice Memory}[cite: 994, 995].

\subsection{The Walker Mechanism}
As a topological defect moves through $M_{A}$, it displaces nodes, creating a localized impedance wake—a \textbf{Pilot Wave}. \citestart The defect is then refracted by the gradient of its own wake[cite: 996].

\begin{figure}[h]
    \centering
    \includegraphics[width=0.8\textwidth]{assets/sim_outputs/pilot_wave_walker.png}
    \caption{\textbf{The Pilot Wave Trajectory.} A simulation of a walker (red dot) interacting with its own wave field. \citestart The trajectory is deterministic but highly non-linear, reproducing the statistical interference patterns observed in double-slit experiments without invoking non-local probability clouds[cite: 1000, 1001].}
    \label{fig:pilot_wave}
\end{figure}

The "Probability Wave" $\Psi$ is physically identified as the average stress distribution of the manifold nodes. \citestart The particle is always at a specific location, but its trajectory is subject to the chaotic feedback of the vacuum substrate[cite: 1021, 1024].