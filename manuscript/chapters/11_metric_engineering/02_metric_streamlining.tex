\section{Metric Streamlining: Reducing Inertial Mass}
\label{sec:streamlining}

Standard physics treats inertia ($m$) as an immutable scalar. \citestart VCFD reveals it as a drag force dependent on geometry ($C_d$)\cite{einstein1916}\citeend. To reach relativistic speeds without infinite energy cost, we must apply the principles of Vacuum Aerodynamics.

\subsection{The Inductive Drag Coefficient ($C_d$)}
A moving object creates a turbulent wake in the lattice (Back-EMF). The force required to push it is:
\begin{equation}
    F_{drag} = \frac{1}{2} \rho_{vac} v^2 C_d A_{cross}
\end{equation}
\citestart Where $C_d$ is the \textbf{Metric Drag Coefficient}\cite{einstein1916}\citeend.
\begin{itemize}
    \item \textbf{Blunt Body ($C_d \approx 1$):} A standard mass (proton/sphere) creates a large turbulent wake. \citestart High Inertia\cite{einstein1916}\citeend.
    \citestart \item \textbf{Streamlined Body ($C_d \ll 1$):} A hull shaped to guide vacuum flux around it laminarly can reduce its effective mass\cite{einstein1916}\citeend.
\end{itemize}



[Image of streamline flow vs turbulent flow]


\subsection{Active Flow Control: The Metric "Dimple"}
\citestart Just as golf balls use dimples to energize the boundary layer and reduce drag, a relativistic vessel can use \textbf{Metric Actuators}\cite{einstein1916}\citeend.
\begin{itemize}
    \citestart \item \textbf{Mechanism:} High-frequency toroidal emitters ($\omega \gg \omega_{plasma}$) placed at the leading edge can "pre-stress" the vacuum, lowering the local viscosity\cite{einstein1916}\citeend.
    \item \textbf{Result:} The vacuum fluid adheres to the hull surface (Laminar Flow) rather than separating into a turbulent wake. \citestart This effectively "lubricates" the spacetime trajectory, reducing the inertial mass of the vessel\cite{einstein1916}\citeend.
\end{itemize}

\subsubsection{Naval Analogy: Supercavitation}
How do we reduce the inertial mass of a spacecraft? We apply the principles of Supercavitating Torpedoes to the vacuum.
\begin{itemize}
    \item \textbf{Standard Flight (Viscous Drag):} A ship moving through the vacuum is like a boat hull moving through water. It drags a massive wake of lattice distortion ($m_i$).
    \item \textbf{Metric Streamlining (The Gas Bubble):} A supercavitating torpedo ejects gas from its nose to envelop itself in a bubble of low-density air. The hull never touches the water, reducing drag by 99\%.
\end{itemize}
\textbf{AVE Application:} By emitting a high-frequency metric field ($\sigma < 1$) ahead of the ship, we create a "Vacuum Bubble." The ship slips through this rarefied pocket, effectively decoupling from the viscous inertia of the bulk universe.