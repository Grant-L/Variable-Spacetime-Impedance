\section{Neutron Decay: The Threading Instability}

The free Neutron is slightly heavier than the bare Proton (939.5 MeV vs 938.3 MeV) and decays into a Proton via Beta Decay ($n \rightarrow p^+ + e^- + \overline{\nu}_e$). The Standard Model treats this as a magical transmutation mediated by the Weak Force. We model this macroscopically and deterministically as a \textbf{Topological Snap}.

\subsection{The Neutron Topology ($6^3_2 \cup 3_1$)}
We identify the Neutron not as a distinct, novel knot, but as a composite architecture: a Proton ($6^3_2$) with an Electron ($3_1$ Trefoil) \textbf{Topologically Trapped} within its central structural void.

\begin{itemize}
    \item \textbf{The Threading:} The $3_1$ Trefoil physically passes through the topological void created by the Borromean triad.
    \item \textbf{Topological Link:} Crucially, this is a Topological Link ($\cup$), not a Connected Sum ($\#$). If the electron were physically fused to the proton, releasing it would require violently severing the flux tubes—a catastrophic threshold exceeding the pair-production Schwinger Limit. Because it is a trapped link, the electron remains a separate continuous sub-manifold, restrained solely by the extreme pressure gradient of the Borromean core.
    \item \textbf{The Instability (Mass Excess):} This geometric state is highly metastable. Because Axiom 1 dictates that no flux tube can shrink below a thickness of $1~l_{node}$, forcing an electron tube into the proton's core requires the Borromean rings to physically stretch outward. This immense expansion tension natively and mechanically yields the exact $+1.3$ MeV mass surplus the Neutron possesses relative to the bare Proton.
\end{itemize}

\begin{figure}[htbp]
    \centering
    \includegraphics[width=0.9\textwidth]{chapters/04_baryon_sector/simulations/outputs/neutron_threaded.png}
    \caption{\textbf{The Threaded Neutron ($6^3_2 \cup 3_1$).} The Neutron is modeled precisely as a compound topological defect. A Golden Torus ($3_1$ electron soliton) resides highly tensioned inside the central void of the Borromean Proton core. Because the electron cannot artificially shrink to fit ($d \equiv 1$), the proton's loops must physically stretch, generating the extreme outward elastic tension that dictates Beta Decay.}
    \label{fig:neutron_threaded}
\end{figure}

\subsection{The Snap (Beta Decay)}
The Beta decay event is a literal topological phase transition:
\begin{equation}
    6^3_2 \cup 3_1 \xrightarrow{\text{Dielectric Tunneling}} 6^3_2 + 3_1 + \overline{\nu}_e
\end{equation}

\begin{enumerate}
    \item \textbf{Tunneling:} Driven by stochastic background lattice perturbations (CMB noise), the highly tensioned threaded electron eventually slips its topological lock.
    \item \textbf{Ejection:} Deprived of the holding pressure, the electron ($e^-$) is violently ejected at high velocity (Inductive Release).
    \item \textbf{Relaxation:} The expanded Proton core abruptly snaps back, elastically relaxing inward to its lower-energy, un-threaded ground state.
    \item \textbf{Conservation of Spin:} To explicitly conserve angular momentum during this rapid structural snap, the local lattice sheds a pure transverse ``Twist Defect'' (The Antineutrino, $\overline{\nu}_e$). The neutrino is simply a massless spatial torsional shockwave propagating through the discrete lattice.
\end{enumerate}