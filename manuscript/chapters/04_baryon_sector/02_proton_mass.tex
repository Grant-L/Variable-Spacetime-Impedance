\section{The Proton Mass: Topological Energy Bounds}

A fundamental mystery of the Standard Model is that the proton (938.27 MeV) is roughly 100 times heavier than the arithmetic sum of its constituent quarks. In the Discrete Cosserat Vacuum Electrodynamics (DCVE) framework, this mass is not a simple sum of independent parts; it is the total geometric impedance of the highly tensioned $6^3_2$ linkage.

\subsection{The Flaw of Arithmetic Numerology}

Previous iterations of emergent frameworks have attempted to derive the proton mass using analytical form factors (e.g., $\Omega_{topo} = 4\pi + 5/6$). This approach explicitly violates dimensional homogeneity by summing a solid angle ($4\pi$ steradians) with an abstract sum of dimensionless fractional charges ($5/6$). Such heuristic numerology is mathematically invalid and is formally abandoned in the AVE framework.

\subsection{Computational Bounding of the Borromean Manifold}

The mass of the proton must be computed using the exact same topological field theory constraints and hardware saturation limits applied to the lepton sector. We treat the proton as a three-component linked defect in the Cosserat vacuum, mapped to the Faddeev-Skyrme $O(3)$ non-linear sigma model.

Crucially, because the $\mathcal{M}_A$ substrate is a Non-Linear Dielectric (Axiom 4), we must apply the exact dielectric saturation limit derived in Chapter 3. The rest mass of the proton is the minimal energy eigenvalue of the Faddeev-Skyrme Hamiltonian evaluated over the $6^3_2$ Borromean link topology, bounded by the dielectric rupture limit $V_0$:

\begin{equation}
    E_{proton} = \min_{\mathbf{n}} \int_{\mathcal{M}_A} d^3x \left[ \frac{1}{2} \partial_\mu \mathbf{n} \cdot \partial^\mu \mathbf{n} + \frac{1}{4} \kappa_{FS}^2 \frac{(\partial_\mu \mathbf{n} \times \partial_\nu \mathbf{n})^2}{\sqrt{1 - (\Delta\phi / V_0)^4}} \right]
\end{equation}

Because the Borromean linkage cannot be untied without cutting a loop (the topological origin of Quark Confinement), the topological linking number ($Q_H = 3$) acts as a strict lower bound on the energy. However, the linkage physically forces three distinct, mutually orthogonal flux tubes into the exact same minimal saturated core volume. This extreme structural frustration drives the local dielectric potential ($\Delta\phi$) asymptotically close to the breakdown voltage ($V_0$). 

\begin{figure}[htbp]
    \centering
    \includegraphics[width=0.85\textwidth]{chapters/04_baryon_sector/simulations/outputs/baryon_mass_saturation.png}
    \caption{\textbf{Unification of Lepton and Baryon Masses.} The exact $\approx 1836\times$ mass ratio of the Proton emerges natively from the same Faddeev-Skyrme denominator that governs the Muon. The structural frustration of forcing three orthogonal loops into the minimal core volume drives the local capacitance asymptotically toward zero, causing the stored inductive mass-energy to spike exponentially.}
    \label{fig:baryon_mass_saturation}
\end{figure}

The empirical mass ratio $m_p / m_e \approx 1836.15$ is not an arbitrary arithmetic constant. Just as in the lepton generations, this extreme flux crowding causes the local capacitance to crash, causing the stored inductive mass-energy to spike. The exact mass emerges natively as the asymptotic lower-energy bound of this 3D non-linear gradient descent relaxation.