\section{The Proton Mass: Why Structure Weighs More Than Parts}
\label{sec:proton_mass}

A fundamental mystery of QCD is that the proton ($938$ MeV) is roughly 100 times heavier than the sum of its three quarks ($\approx 9$ MeV). Where is the mass?

\subsection{Mass as Binding Energy}
In VSI, we derive the Proton Mass almost entirely from the \textbf{Lattice Tension} required to compress the three flux loops into a femtometer volume.
\begin{equation}
    m_p = \sum m_{quark} + \frac{E_{tension}}{c^2}
\end{equation}
Because the Borromean topology forces three high-flux loops ($N \gg 1$) to occupy the same $l_P^3$ volume, the \textbf{Inductive Crowding} (Section 3.3) is extreme. The vacuum nodes in the core are driven to near-saturation ($U \approx U_{sat}$).
\begin{itemize}
    \item \textbf{Result:} The proton's mass is effectively the "Inductive Inertia" of this highly stressed knot. The quarks themselves are just the lightweight geometrical guides for this massive energy cloud.
\end{itemize}