\section{Borromean Confinement: Deriving the Strong Force}

In the Standard Model, the Strong Force is mediated by the exchange of gluons between quarks carrying abstract ``Color Charge.'' In Vacuum Engineering, we replace this abstract symmetry with \textbf{Topological Geometry}.

We identify the Proton not as a bag of independent point particles, but as a \textbf{Borromean Linkage} of three continuous flux loops ($6^3_2$).

\subsection{The Borromean Topology}

The Borromean Rings consist of three loops interlinked such that no two individual loops are linked to each other directly, but the three together are topologically inseparable.

\begin{itemize}
    \item \textbf{Quark ($q$):} A single flux loop. Unstable on its own (cannot exist in isolation without shedding its energy).
    \item \textbf{Confinement:} If any single loop is cut or removed, the other two immediately fall apart.
\end{itemize}

This geometry intrinsically enforces \textbf{Quark Confinement}. It is topologically impossible to isolate a single quark because the Borromean linkage requires the complete triad to maintain its structural integrity.

\subsection{The Gluon Field as 1D Lattice Tension}

In standard Quantum Chromodynamics (QCD), the strong force does not drop off with distance like electromagnetism ($1/r^2$); it remains constant, forming a ``flux tube'' that binds quarks together with a force of roughly 10,000 Newtons. The Standard Model inserts this linear potential phenomenologically. AVE derives it strictly from the hardware limits of the $\mathcal{M}_A$ substrate.

Because the vacuum is a non-linear dielectric, extreme field separation causes the flux lines connecting the Borromean loops to collimate into a 1D cylindrical tube rather than spreading out into 3D space. The force required to stretch this flux tube is exactly the absolute tensile breaking strength of the discrete edges.

As derived in Chapter 1, the maximum force a discrete electromagnetic flux tube can sustain before the lattice ruptures is the \textbf{EM Tension Limit ($T_{EM}$)}:
\begin{equation}
    F_{confinement} = T_{EM} = u_{sat} \kappa_V l_{node}^2
\end{equation}

``Gluons'' are not discrete particles flying between quarks. They are the mathematical representation of the extreme \textbf{Elastic Stress} of the vacuum lattice trapped between the separating loops. As the loops are pulled apart, the force remains absolutely constant ($T_{EM}$). The flux tube does not break until the stored elastic energy exceeds the pair-production threshold ($E > 2mc^2$), at which point the lattice snaps and re-triangulates, creating a new meson rather than releasing a free quark.

\textbf{Structural Analogy: The Tripod Stool.} Consider a three-legged stool where the legs are not screwed in, but held together by mutual tension (Tensegrity). The three loops (legs) lock each other into a rigid volume. If you remove one leg, the other two act as loose cables and collapse instantly. You cannot isolate a ``leg'' (Quark) because the leg defines the structural integrity of the whole. The Proton is not a bag of parts; it is a Topological Truss.

\begin{figure}[htbp]
    \centering
    \includegraphics[width=0.85\textwidth]{chapters/04_baryon_sector/simulations/outputs/proton_borromean.png}
    \caption{\textbf{AVE Simulation: The Borromean Proton ($6^3_2$).} The discrete physical representation of Quark Confinement. The three distinct topological loops are mutually entangled. The ``Gluon Field'' is mathematically identical to the mechanical strain exerted on the $\mathcal{M}_A$ lattice nodes occupying the interstitial space. The $\mathbb{Z}_3$ symmetry naturally dictates the SU(3) color rules (Cyan, Magenta, Yellow).}
    \label{fig:proton_borromean}
\end{figure}