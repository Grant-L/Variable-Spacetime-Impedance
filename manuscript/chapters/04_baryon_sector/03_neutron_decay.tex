\section{Neutron Decay: The Threading Instability}
\label{sec:neutron_decay}

The Neutron is slightly heavier than the Proton and decays into a Proton via Beta Decay ($n \to p + e^- + \bar{\nu}_e$). We model this as a **Topological Snap**.

\subsection{The Neutron Topology ($6_2^3 \# 3_1$)}
We identify the Neutron not as a distinct knot, but as a Proton ($6_2^3$) with an Electron ($3_1$) **Threaded** through its center.
\begin{itemize}
    \item **The Threading:** The electron loop passes through the void of the Borromean triad.
    \item **The Instability:** This state is metastable. The threaded electron exerts a torsional strain on the proton core.
\end{itemize}

\subsection{The Snap (Beta Decay)}
The decay event is a topological transition:
\begin{equation}
    6_2^3 \# 3_1 \xrightarrow{\text{Tunneling}} 6_2^3 + 3_1 + 0_1
\end{equation}
1.  **Tunneling:** The threaded electron slips its topological lock.
2.  **Ejection:** The electron ($e^-$) is ejected at high velocity (Inductive Release).
3.  **Relaxation:** The Proton core relaxes to its ground state.
4.  **Conservation:** To conserve angular momentum during the snap, the lattice sheds a "Twist Defect" (Antineutrino, $\bar{\nu}_e$).

\textbf{Prediction:} The lifetime of the neutron ($\approx 880$ s) is mathematically determined by the tunneling probability of the electron knot through the impedance barrier of the proton core.

\subsubsection{Mechanical Analogy: The Snapped Guitar String}
The decay of a Neutron into a Proton, Electron, and Antineutrino ($n \rightarrow p + e^- + \bar{\nu}_e$) is modeled as a sudden release of Lattice Tension.

Consider a guitar string pulled tight by a tuning peg:
\begin{enumerate}
    \item \textbf{The Tension (Mass):} The potential energy is stored in the elastic stretch of the string (the Vacuum Lattice), not inside the peg itself. This tension is the "Mass" of the Neutron.
    \item \textbf{The Tunneling (Slip):} The threaded electron knot is the "peg" holding this tension. When it tunnels through the potential barrier, the peg slips.
    \item \textbf{The Snap (Neutrino):} The electron flies off, but the energy stored in the string doesn't vanish. It snaps back, creating a transverse vibration wave that propagates down the string.
\end{enumerate}

\textbf{Conclusion:} The Antineutrino is not a particle in the traditional sense; it is the \textbf{Lattice Shockwave}—the "sound" of the vacuum snapping back to its ground state after the tension is released.