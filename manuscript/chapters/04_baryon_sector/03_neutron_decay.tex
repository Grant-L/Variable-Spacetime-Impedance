\section{Neutron Decay: The Threading Instability}

The Neutron is slightly heavier than the Proton and decays into a Proton via Beta Decay ($n \rightarrow p^+ + e^- + \overline{\nu}_e$). We model this macroscopically as a \textbf{Topological Snap}.

\subsection{The Neutron Topology ($6^3_2 \cup 3_1$)}

We identify the Neutron not as a distinct, isolated knot, but as a Proton ($6^3_2$) with an Electron ($3_1$) \textbf{Trapped} within its center.

\begin{itemize}
    \item \textbf{The Threading:} The electron loop physically passes through the void of the Borromean triad.
    \item \textbf{Topological Link:} Crucially, this is a Topological Link ($6^3_2 \cup 3_1$), not a Connected Sum ($\#$). If the electron were physically fused to the proton via a connected sum, releasing it would require severing the flux tubes---a catastrophic event requiring immense energy (exceeding the Schwinger Limit $V_0$). Because it is a trapped link, the electron remains a separate sub-manifold restrained solely by the extreme pressure gradient of the Borromean core.
    \item \textbf{The Instability:} This state is metastable. The threaded electron exerts immense outward torsional strain on the proton core, driving the local impedance even closer to the yield limit and accounting for the neutron's slightly higher mass relative to the bare proton.
\end{itemize}

\begin{figure}[htbp]
    \centering
    \includegraphics[width=0.85\textwidth]{chapters/04_baryon_sector/simulations/outputs/neutron_threaded.png}
    \caption{\textbf{AVE Simulation: The Threaded Neutron ($6^3_2 \cup 3_1$).} The Neutron is modeled as a compound topological defect. A Golden Torus ($3_1$ electron soliton) resides inside the central void of the Borromean Proton core. Beta decay occurs when the highly-tensioned electron probabilistically tunnels out of this topological lock.}
    \label{fig:neutron_threaded}
\end{figure}

\subsection{The Snap (Beta Decay)}

The decay event is a topological phase transition:

\begin{equation}
    6^3_2 \cup 3_1 \xrightarrow{\text{Tunneling}} 6^3_2 + 3_1 + 0_1
\end{equation}

\begin{enumerate}
    \item \textbf{Tunneling:} The threaded electron slips its topological lock due to background lattice perturbations.
    \item \textbf{Ejection:} The electron ($e^-$) is ejected at high velocity (Inductive Release).
    \item \textbf{Relaxation:} The Proton core snaps back, relaxing to its lower-energy ground state.
    \item \textbf{Conservation:} To conserve angular momentum during the rapid snap, the lattice sheds a ``Twist Defect'' (Antineutrino, $\overline{\nu}_e$).
\end{enumerate}

\textbf{Mechanical Analogy: The Snapped Guitar String.} The decay of a Neutron is modeled as a sudden release of Lattice Tension. Consider a guitar string pulled tight by a tuning peg. The potential energy (Mass) is stored in the elastic stretch of the string (Vacuum Lattice). The threaded electron knot is the ``peg'' holding this tension. When the peg slips, the electron flies off, but the energy stored in the string snaps back, creating a transverse vibration wave. The Antineutrino is simply this Lattice Shockwave---the ``sound'' of the vacuum snapping back to its ground state.

\textbf{Prediction:} The lifetime of the free neutron ($\approx 880$~s) is mathematically determined by the quantum tunneling probability of the electron knot escaping through the dielectric impedance barrier of the proton core.