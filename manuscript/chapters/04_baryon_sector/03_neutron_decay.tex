\section{Neutron Decay: The Threading Instability}
\label{sec:neutron_decay}

The Neutron is slightly heavier than the Proton and decays into a Proton via Beta Decay ($n \to p + e^- + \bar{\nu}_e$). We model this as a **Topological Snap**.

\subsection{The Neutron Topology ($6_2^3 \# 3_1$)}
We identify the Neutron not as a distinct knot, but as a Proton ($6_2^3$) with an Electron ($3_1$) **Threaded** through its center.
\begin{itemize}
    \item **The Threading:** The electron loop passes through the void of the Borromean triad.
    \item **The Instability:** This state is metastable. The threaded electron exerts a torsional strain on the proton core.
\end{itemize}

\subsection{The Snap (Beta Decay)}
The decay event is a topological transition:
\begin{equation}
    6_2^3 \# 3_1 \xrightarrow{\text{Tunneling}} 6_2^3 + 3_1 + 0_1
\end{equation}
1.  **Tunneling:** The threaded electron slips its topological lock.
2.  **Ejection:** The electron ($e^-$) is ejected at high velocity (Inductive Release).
3.  **Relaxation:** The Proton core relaxes to its ground state.
4.  **Conservation:** To conserve angular momentum during the snap, the lattice sheds a "Twist Defect" (Antineutrino, $\bar{\nu}_e$).

\textbf{Prediction:} The lifetime of the neutron ($\approx 880$ s) is mathematically determined by the tunneling probability of the electron knot through the impedance barrier of the proton core.