\section{Spatial Flux Partitioning: The Origin of Fractional Charge}
\label{sec:spatial_flux_partitioning}

A fundamental requirement for any topological model of the Proton ($6^3_2$ Borromean linkage) is the derivation of fractional electric charges ($+2/3, -1/3, -1/3$). In VSI, where charge is defined strictly as an integer topological winding number, true fractional twists are mechanically forbidden as they would tear the $M_A$ manifold.

One might initially hypothesize that the integer charge of the proton ($+1e$) is a single topological twist that rapidly ``shuttles'' or time-averages across the three identical flux loops. 

\textbf{Falsification via Deep Inelastic Scattering (DIS):} At relativistic scattering speeds ($\approx 10^{-24}$ s), an electron probe measures the instantaneous cross-section, which scales with the \textit{square} of the target's charge ($\sigma \propto q^2$). If $+1e$ spent $2/3$ of its time in one loop, the expectation value of the cross-section would be $E[q^2] = (1^2 \times 2/3) + (0^2 \times 1/3) = 2/3$. However, accelerator data definitively shows the cross-section is proportional to the square of the fraction: $(2/3)^2 = 4/9$. Because $2/3 \neq 4/9$, time-averaging is physically falsified.

\subsection{Topological Solid Angle Division}
We resolve this paradox via \textbf{Spatial Flux Partitioning}. In the $M_A$ manifold, Electric Charge is the Gaussian flux of the phase twist radiating outward through a spherical boundary (solid angle $4\pi$). 

The fundamental integer charge ($q_{nat} = +1e$) belongs to the \textit{entire} Borromean linkage manifold, trapped in the central topological void where the loops intersect. To minimize the M\"obius energy of the linkage, the three rigid loops partition the Gaussian sphere asymmetrically. 

The crossing geometry acts as a physical stencil. One topological boundary (the Up quark lobe) bounds exactly $2/3$ of the effective flux solid angle. The other two boundaries (the Down quarks) are inverted and compressed into the topological interior, each bounding $1/3$ of the solid angle. Quarks are not independent sub-particles; they are the geometrically constrained lobes of a single, integer-charged Borromean flux manifold.