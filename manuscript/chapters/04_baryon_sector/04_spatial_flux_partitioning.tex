\section{Spatial Flux Partitioning: The Origin of Fractional Charge}

A fundamental requirement for any topological model of the Proton ($6^3_2$ Borromean linkage) is the derivation of fractional electric charges. In AVE, where charge is strictly an integer topological winding number, true fractional twists are mechanically forbidden as they would tear the $M_A$ manifold.

\textbf{Falsification of Time-Averaging:} One might hypothesize the integer charge rapidly ``shuttles'' across the three loops. However, Deep Inelastic Scattering (DIS) data falsifies this. The measured cross-section scales as the square of the fraction ($(2/3)^2 = 4/9$), not the average of the squares ($(1^2 \times 2/3) = 2/3$). 

\subsection{Topological Solid Angle Division}
We resolve this paradox via \textbf{Spatial Flux Partitioning}. The fundamental integer charge ($q_{nat} = +1e$) belongs to the \textit{entire} Borromean linkage manifold. To minimize M\"obius energy, the rigid loops partition the Gaussian sphere asymmetrically. 

The crossing geometry acts as a physical stencil. One topological boundary bounds exactly $2/3$ of the effective flux solid angle. The other two boundaries are inverted and compressed into the interior, each bounding $1/3$ of the solid angle. Quarks are not independent sub-particles; they are the geometrically constrained lobes of a single, integer-charged Borromean flux manifold.