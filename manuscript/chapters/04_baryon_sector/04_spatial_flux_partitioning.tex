\section{Spatial Flux Partitioning: The Origin of Fractional Charge}
\label{sec:spatial_flux_partitioning}

A fundamental requirement for any topological model of the Proton ($6^3_2$ Borromean linkage) is the derivation of fractional electric charges for its constituent quarks ($+2/3$, $+2/3$, $-1/3$). In the Applied Vacuum Electrodynamics (AVE) framework, where charge is defined strictly as an integer topological winding number, true fractional twists are mechanically forbidden as they would tear the $M_A$ manifold.

How, then, does an integer-winding framework produce fractional charges?

\subsection{Falsification of the Time-Averaging Hypothesis}
One might initially hypothesize that the integer charge of the proton ($q_{nat} = +1e$) is a single topological twist that rapidly ``shuttles'' or time-averages across the three identical flux loops. 

We rigorously falsify this using the mathematics of Deep Inelastic Scattering (DIS). At relativistic scattering speeds ($\approx 10^{-24}$ seconds), an electron probe acts as an ultra-fast camera shutter, measuring the \textit{instantaneous} scattering cross-section ($\sigma$), which scales with the \textit{square} of the target's charge ($q^2$). 

If a $+1e$ charge spent $2/3$ of its time in one loop and $1/3$ of its time as neutral, the expectation value of the cross-section would be the average of the squares:
\begin{equation}
E[q^2]_{Up} = (1^2 \times 2/3) + (0^2 \times 1/3) = 2/3
\end{equation}

However, particle accelerator data definitively shows that the cross-section is proportional to the square of the fraction:
\begin{equation}
q_{Up}^2 = (+2/3)^2 = 4/9
\end{equation}

Because $2/3 \neq 4/9$, the time-averaging hypothesis is physically falsified. The fractional charges must be simultaneously and spatially static.

\subsection{Topological Solid Angle Division}
We resolve this paradox via \textbf{Spatial Flux Partitioning}. In the $M_A$ manifold, Electric Charge is the Gaussian flux of the phase twist radiating outward through a spherical boundary (a solid angle of $4\pi$). 

In a perfectly symmetric prime knot (like the Trefoil electron), the flux radiates isotropically, yielding an integer charge $N = \pm 1$. However, the Borromean linkage is a \textit{composite} knot. The fundamental integer charge ($+1e$) belongs to the \textit{entire} linkage manifold, trapped in the central topological void where the three loops intersect and mutually compress the dielectric. 

To minimize the M\"obius energy of the highly tensioned linkage, the three rigid loops partition the Gaussian sphere asymmetrically. The crossing geometry of the $6^3_2$ knot acts as a physical stencil blocking and shaping the flux emission:

\begin{itemize}
    \item \textbf{The Up Quarks ($+2/3$):} Two of the topological boundaries are forced outward by mutual repulsion, each stenciling exactly $2/3$ of the effective outward flux solid angle.
    \item \textbf{The Down Quark ($-1/3$):} To mechanically close the linkage, the third boundary loop is inverted and compressed into the topological interior. This inversion reverses its relative helicity (negative sign) and restricts its bounded solid angle to $1/3$ of the total flux.
\end{itemize}

\textbf{Summation:} $(+2/3) + (+2/3) + (-1/3) = +1e$.

\textbf{Conclusion:} Quarks are not independent sub-particles possessing magical fractional charges. They are the geometrically constrained lobes of a single, integer-charged Borromean flux manifold. The fractions observed in particle accelerators are the strict geometric ratios of the solid angle statically partitioned by the tightened linkage topology.