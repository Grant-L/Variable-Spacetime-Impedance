\section{Topological Fractionalization: The Origin of Quarks}
\label{sec:spatial_flux_partitioning}

A fundamental requirement for any topological model of the Proton is the derivation of fractional electric charges for its constituent quarks ($+2/3$, $+2/3$, $-1/3$). In the DCVE framework, where charge is defined strictly as an integer topological Winding Number ($N \in \mathbb{Z}$), true continuous fractional twists are mechanically forbidden as they would tear the $\mathcal{M}_A$ manifold.

\subsection{Falsification of Geometric "Stenciling"}
Earlier hypotheses suggested these fractions arose because the loops physically "stenciled" or blocked $1/3$ or $2/3$ of the geometric solid angle. This macroscopic classical analogy fails at the quantum lattice level, where charge must be governed by the discrete Aharonov-Bohm phase, not shadow-casting.

\subsection{Rigorous Derivation: The Witten Effect and $\mathbb{Z}_3$ Symmetry}
We resolve the fractional charge paradox via the exact mathematics of \textbf{Topological Fractionalization} on a frustrated discrete graph.

The proton possesses a total, strictly integer topological winding number of $Q = +1$. However, this integer flux is trapped within the tri-partite symmetry of the $6^3_2$ Borromean linkage. Because the three loops are topologically entangled such that the removal of any one loop unlinks the others, the total phase twist is distributed across a degenerate structural ground state.

In a non-linear dielectric substrate, a composite topological defect with internal permutation symmetry natively generates a discrete CP-violating $\theta$-vacuum phase. By the \textbf{Witten Effect}, a topological defect embedded in a $\theta$-vacuum mathematically acquires a fractionalized electric charge shift:
\begin{equation}
    q_{eff} = n + \frac{\theta}{2\pi} e
\end{equation}

For the $6^3_2$ Borromean linkage, the strict three-fold permutation symmetry ($\mathbb{Z}_3$) restricts the allowed topological phase angles of the vacuum strictly to thirds: $\theta \in \{0, \pm \frac{2\pi}{3}, \pm \frac{4\pi}{3}\}$. 

Substituting these discrete topological angles into the Witten charge equation rigorously yields the exact effective fractional charges:
\begin{equation}
    q_{eff} = \pm \frac{1}{3}e, \quad \pm \frac{2}{3}e
\end{equation}

\textbf{Conclusion:} Quarks are not independent fundamental particles possessing intrinsically fractional charges. They are \textit{deconfined quasiparticles} emerging from a frustrated topology. The global integer charge of the proton ($+1e$) is mathematically partitioned by the fundamental group $\pi_1$ of the Borromean knot complement.