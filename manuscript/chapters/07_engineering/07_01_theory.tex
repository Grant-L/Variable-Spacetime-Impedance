\section{The Principle of Local Refractive Control}
\label{sec:refractive_control}

In the VSI v4.0 framework, vacuum engineering is defined as the active modification of the local $M_A$ lattice Refractive Index ($n$). We do not "curve space" geometry; instead, we induce physical \textbf{Lattice Density Shifts} via external high-frequency toroidal flux to tune the local Group Velocity ($v_g$).

Crucially, to maintain causal connectivity and prevent Cherenkov-like radiation losses, the engineering process must satisfy the \textbf{Impedance Matching Condition}:
\begin{equation}
    Z_{eng} = \sqrt{\frac{L'_{node}}{C'_{node}}} \approx Z_0
\end{equation}

By scaling Node Inductance and Capacitance proportionally ($L \downarrow, C \downarrow$), the vacuum becomes a "Faster-Than-Light" medium ($\chi < 1$) without altering its characteristic impedance ($Z_0$). This allows for superluminal translation without the catastrophic back-scatter reflections predicted by scalar theories.