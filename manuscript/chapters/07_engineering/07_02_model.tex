\section{Metric Refraction: The Non-Geometric Warp}
\label{sec:refraction_warp}

\citestart SVF replaces the abstract "warping" of spacetime with the mechanical \textbf{Refraction of Flux}\cite{1157}\citeend. \citestart A region of modified node density relative to the background creates a local Refractive Index ($\chi$)\cite{1158}\citeend:

\begin{equation}
    \chi = \frac{n_{local}}{n_0} = \sqrt{\frac{L'_{node} C'_{node}}{L_{node} C_{node}}}
\end{equation}

When $\chi < 1$, the local group velocity $v_g = c/\chi$ exceeds the background speed of light. \citestart This creates a \textbf{Lattice Slip} zone\cite{1159}\citeend. \citestart Because the impedance remains matched ($Z' = Z_0$), the vessel does not encounter a "light barrier" or shockwave; it simply traverses a medium with a higher local slew rate limit\cite{1160}\citeend.

\subsection{The Lattice Stress Coefficient ($\sigma$)}
\citestart The magnitude of the modification is governed by the \textbf{Lattice Stress Coefficient} ($\sigma$)\cite{1161}\citeend:
\begin{itemize}
    \item \textbf{Compression ($\sigma > 1$):} Increases node density ($L \uparrow, C \uparrow$). \citestart This slows light (Gravity)\cite{1162}\citeend.
    \item \textbf{Rarefaction ($\sigma < 1$):} Decreases node density ($L \downarrow, C \downarrow$). \citestart This speeds light (Warp)\cite{1163}\citeend.
\end{itemize}

\citestart This unified definition links Gravity and Warp Drive as opposite poles of the same mechanical stress function\cite{1164}\citeend.

\begin{figure}[h]
    \centering
    \includegraphics[width=1.0\textwidth]{assets/sim_outputs/metric_stress_profile.png}
    \caption{\textbf{The Unified Stress Field.} A simulation of effective light speed vs. Lattice Stress ($\sigma$).
    \textbf{Right ($\sigma > 1$):} Gravitational slowing (Black Hole regime).
    \textbf{Left ($\sigma < 1$):} Metric Rarefaction (Warp Drive regime). The vertical asymptote represents the background vacuum state ($c$).}
    \label{fig:metric_stress}
\end{figure}