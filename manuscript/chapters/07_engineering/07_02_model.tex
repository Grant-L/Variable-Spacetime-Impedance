\section{Metric Refraction: The Non-Geometric Warp}
\label{sec:refraction_warp}

SVF replaces the abstract "warping" of spacetime with the mechanical \textbf{Refraction of Flux}. A region of modified node density relative to the background creates a local Refractive Index ($\chi$):

\begin{equation}
    \chi = \frac{n_{local}}{n_0} = \sqrt{\frac{L'_{node} C'_{node}}{L_{node} C_{node}}}
    \label{eq:engineered_index}
\end{equation}

When $\chi < 1$, the local group velocity $v_g = c/\chi$ exceeds the background speed of light. This creates a \textbf{Lattice Slip} zone. Because the impedance remains matched ($Z' = Z_0$), the vessel does not encounter a "light barrier" or shockwave; it simply traverses a medium with a higher local slew rate limit.

\subsection{The Lattice Stress Coefficient ($\sigma$)}
The magnitude of the modification is governed by the \textbf{Lattice Stress Coefficient} ($\sigma$).
\begin{itemize}
    \item \textbf{Compression ($\sigma > 1$):} Increases node density ($L \uparrow, C \uparrow$). This slows light (Gravity).
    \item \textbf{Rarefaction ($\sigma < 1$):} Decreases node density ($L \downarrow, C \downarrow$). This speeds light (Warp).
\end{itemize}

This unified definition links Gravity and Warp Drive as opposite poles of the same mechanical stress function.