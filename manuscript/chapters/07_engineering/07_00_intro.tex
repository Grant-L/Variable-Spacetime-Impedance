\section{Introduction: The Substrate as Hardware}
\label{sec:engineering_intro}

In previous chapters, we established that the vacuum is not a geometric void but a physical, constitutive substrate defined as the Discrete Amorphous Manifold ($M_A$)[cite: 18, 76]. Having derived the mechanical origins of mass, gravity, and the weak interaction, we now transition from descriptive physics to operational engineering.

The Engineering Layer treats the vacuum as a tunable transmission medium. If the fundamental constants of nature ($L_{node}, C_{node}, c$) are bulk engineering properties of the substrate[cite: 38, 85], then localized modification of these properties allows for the manipulation of the metric itself. We move beyond observing the laws of the universe to understanding the hardware that enforces them[cite: 57].