\section{The Weak Interaction: Impedance Spikes}
\label{sec:weak_interaction}

The Weak Force is unique because it is short-range ($\approx 10^{-18}$ m) and massive ($W/Z \approx 80 \text{ GeV}$). The Standard Model explains this via the Higgs Mechanism. AVE explains it as Transient Impedance.

\subsection{The Inverse Resonance Law}
We propose that the $W$ and $Z$ bosons are not fundamental particles, but Transient Resonance Spikes in the lattice.

\begin{itemize}
    \item \textbf{The Snap:} When a neutron decays, the topological transition happens on the timescale of the lattice update ($t_{tick}$).
    \item \textbf{The Spike:} This ultra-fast snap creates a frequency spike $\omega \to \omega_{sat}$.
\end{itemize}

At these frequencies, the lattice impedance diverges. The ``mass'' of the $W$ boson (80 GeV) is actually the Impedance Wall of the vacuum at the breakdown frequency.
\begin{equation}
m_W \propto E_{sat} = \frac{\hbar}{t_{tick}}
\end{equation}

The Weak Force is short-range not because the boson is heavy, but because the signal is so high-frequency it is instantly damped by the lattice (Skin Effect).