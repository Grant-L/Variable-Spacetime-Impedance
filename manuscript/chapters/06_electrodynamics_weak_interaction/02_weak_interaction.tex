\section{The Weak Interaction: The Impedance Bridge}
\label{sec:weak_interaction}

The Weak Force is unique because it is short-range ($\approx 10^{-18}$ m) and massive ($W/Z \approx 80-91$ GeV). The Standard Model explains this via the Higgs Mechanism. AVE explains it as **Impedance Coupling** between the Baryon sector and the Vacuum.

\subsection{The Base Impedance Scale ($S$)}
In Chapter 4, we established the Proton as a geometric linkage with mass $m_p$. In Chapter 3, we defined the vacuum impedance $\alpha^{-1}$.
We define the **Base Impedance Scale ($S$)** as the energy required to stress a proton-sized topological defect to the full impedance limit of the vacuum:

\begin{equation}
    S \equiv m_p \cdot \alpha^{-1}_{AVE} \approx 938.27 \text{ MeV} \times 137.036 \approx 128.58 \text{ GeV}
\end{equation}

This scale represents the dielectric yield point of the "Strong" topology against the "Electromagnetic" vacuum.

\subsection{Deriving the W Boson ($5/8$ Resonance)}
The W boson mediates the transmutation of quarks. We derived in Chapter 4 that the Proton's charge flux is partitioned by a factor of $5/6$.
We propose that the W boson corresponds to the **5/8 Harmonic** of the Base Impedance Scale.

\begin{equation}
    m_W = S \times \frac{5}{8} = (m_p \cdot \alpha^{-1}) \cdot 0.625
\end{equation}

\textbf{Result:}
\begin{itemize}
    \item \textbf{Prediction:} $128.58 \text{ GeV} \times 0.625 \approx \mathbf{80.36 \text{ GeV}}$
    \item \textbf{Experiment:} $80.379 \text{ GeV}$
    \item \textbf{Error:} $\mathbf{0.02\%}$
\end{itemize}

\subsection{Deriving the Z Boson (Geometric Mixing)}
The Z boson is heavier than the W due to the Weak Mixing Angle ($\theta_W$). In the Standard Model, $m_W = m_Z \cos \theta_W$.
In AVE, the mixing angle is a fixed geometric property of the lattice. We derive it as the projection of the 3D spatial manifold onto the $\sqrt{7}$ diagonal of the 7-node interaction cell (or the 7-crossing Tau knot).

\begin{equation}
    \cos \theta_W = \frac{\sqrt{7}}{3} \approx 0.8819
\end{equation}

\begin{equation}
    m_Z = \frac{m_W}{\cos \theta_W} = m_W \cdot \frac{3}{\sqrt{7}}
\end{equation}

\textbf{Result:}
\begin{itemize}
    \item \textbf{Prediction:} $80.36 \text{ GeV} \times 1.1339 \approx \mathbf{91.12 \text{ GeV}}$
    \item \textbf{Experiment:} $91.187 \text{ GeV}$
    \item \textbf{Error:} $\mathbf{0.07\%}$
\end{itemize}

\begin{figure}[ht]
\centering
\includegraphics[width=0.8\textwidth]{chapters/06_electrodynamics_weak_interaction/simulations/weak_force_derivation.png}
\caption{\textbf{Derivation of the Weak Force.} The masses of the W and Z bosons are derived strictly from the Proton Mass and Fine Structure Constant using simple geometric ratios ($5/8$ and $\sqrt{7}/3$). The sub-0.1\% accuracy suggests the Weak Force is a geometric resonance of the Proton-Vacuum coupling.}
\label{fig:weak_derivation}
\end{figure}