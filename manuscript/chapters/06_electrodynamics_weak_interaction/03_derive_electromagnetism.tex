\section{The Gauge Layer: From Scalars to Symmetry}
\label{sec:gauge_layer}

While the vacuum acts fundamentally as a reactive scalar medium ($\epsilon_0, \mu_0$), the Standard Model forces require vector gauge symmetries ($U(1), SU(3)$). We derive these symmetries directly from the stochastic connectivity of the $M_A$ manifold.

\subsection{The Stochastic Link Variable ($U_{ij}$)}
The physical connection between node $i$ and node $j$ is a \textbf{Flux Tube} described by a unitary link variable $U_{ij}$ that parallel-transports the internal phase state: $\psi_j = U_{ij} \psi_i$. To minimize energy, flux must flow smoothly ($U_{ij} \approx 1$). The simplest gauge-invariant quantity is the Plaquette (closed loop) product $U_P = U_{ij}U_{jk}U_{kl}U_{li}$.

\subsection{Derivation of Electromagnetism ($U(1)$)}
Assuming a single complex phase ($N=1$), we expand the link variable $U_{ij} \approx e^{ig l_P A_\mu}$ in the continuum limit ($l_P \to 0$). Evaluating the real part of the trace of the Plaquette yields:
\begin{equation}
Re(U_P) \approx 1 - \frac{1}{2}g^2 l_P^4 F_{\mu\nu}^2
\end{equation}
This perfectly recovers the Maxwell Lagrangian ($-\frac{1}{4}F_{\mu\nu}^2$) purely from the stochastic requirement that local node phases must be parallel-transported across the $M_A$ lattice.

\subsection{Derivation of Color ($SU(3)$)}
The Borromean proton introduces a 3-component internal state vector to the node, representing the three topologically indistinguishable flux loops. The link variable becomes a $3 \times 3$ unitary matrix. The non-commutative Plaquette product generates the self-interaction tensor $F_{\mu\nu}^a$, naturally yielding the $SU(3)$ gluon field as the mathematical permutation of the lattice connections.