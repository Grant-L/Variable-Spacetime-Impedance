\section{The Gauge Layer: From Scalars to Symmetry}
\label{sec:gauge_layer}

While the vacuum acts fundamentally as a reactive scalar medium ($\epsilon_0, \mu_0$), the Standard Model forces require vector gauge symmetries ($U(1), SU(3)$). We derive these symmetries directly from the stochastic connectivity of the $M_A$ manifold.
\begin{tcolorbox}[colback=gray!10!white,colframe=black!75!black,title=Design Note 6.1: Gauge Architecture and Network Conservation]
    To resolve the ambiguity between physical observables and mathematical redundancy, the AVE framework strictly separates the \textbf{Longitudinal} (Pressure) and \textbf{Transverse} (Shear) degrees of freedom on the $M_A$ lattice.
    
    \textbf{1. The Node Scalar ($\phi_n$): Longitudinal Pressure}\\
    The scalar potential $\phi$ defined at each node $n$ is a physical state variable representing the local \textbf{Dielectric Compression} (Voltage) of the vacuum substrate.
    \[ \phi_n \in \mathbb{R} \quad \text{(Observable: Local Vacuum Potential)} \]
    \textit{Role:} Governs electrostatic attraction and gravitational refraction via modulation of the refractive index $n(\phi)$.
    
    \textbf{2. The Link Variable ($U_{nm}$): Transverse Flux}\\
    The connection between nodes $n$ and $m$ is defined by a unitary link variable $U_{nm}$, representing the \textbf{Phase Transport} (Magnetic Flux) along the edge.
    \[ U_{nm} = e^{i \theta_{nm}} \in U(1) \quad \text{(Gauge Variable: Phase Twist)} \]
    \textit{Role:} Carries the magnetic helicity and transverse wave components. The physics is invariant under local rotation $\phi_n \to \phi_n'$ provided links update as $U_{nm} \to \Omega_n U_{nm} \Omega_m^\dagger$.
    
    \textbf{3. Recovering Maxwell and Gauss}\\
    \begin{itemize}
        \item \textbf{Maxwell's Lagrangian} arises from the "Plaquette" sum (closed loop product) of link variables: $S_{plaq} \approx -\frac{1}{4} F_{\mu\nu}F^{\mu\nu}$.
        \item \textbf{Gauss's Law} emerges strictly from \textbf{Kirchhoff's Current Law (KCL)}: the sum of flux entering a node equals the rate of change of the node's potential (charge accumulation).
    \end{itemize}
    In AVE, "Gauge Symmetry" is simply the \textbf{Network Conservation Law} of the hardware.
    \end{tcolorbox}
\subsection{The Stochastic Link Variable ($U_{ij}$)}
The physical connection between node $i$ and node $j$ is a \textbf{Flux Tube} described by a unitary link variable $U_{ij}$ that parallel-transports the internal phase state. To minimize energy, flux must flow smoothly ($U_{ij} \approx 1$). The simplest gauge-invariant quantity is the Plaquette (closed loop) product $U_P = U_{ij}U_{jk}U_{kl}U_{li}$.

\subsection{Derivation of Electromagnetism ($U(1)$)}
Assuming a single complex phase ($N=1$), we expand the link variable $U_{ij} \approx e^{ig l_P A_\mu}$ in the continuum limit ($l_P \to 0$). Evaluating the real part of the trace of the Plaquette yields:
\begin{equation}
Re(U_P) \approx 1 - \frac{1}{2}g^2 l_P^4 F_{\mu\nu}^2
\end{equation}
This perfectly recovers the Maxwell Lagrangian ($-\frac{1}{4}F_{\mu\nu}^2$) purely from the stochastic requirement that local node phases must be parallel-transported across the $M_A$ lattice.

\subsection{Conjectural Mapping of Color ($SU(3)$)}
The Standard Model relies on $SU(3)$ to describe the strong force. In the AVE framework, we map this programmatically to the Borromean proton ($6^3_2$). The 3-component internal state vector represents the three topologically indistinguishable flux loops. 

The link variable becomes a $3 \times 3$ unitary matrix, and the non-commutative Plaquette product generates the self-interaction tensor $F_{\mu\nu}^a$. We posit that the $SU(3)$ gluon field is the macroscopic mathematical representation of the physical permutation of these lattice connections. While this mapping is currently programmatic and conjectural, it provides a strictly physical mechanism for topological confinement and baryon number emergence, establishing a quantitative target for future lattice QCD simulations to address anomaly cancellation and correct chiral structures.