\section{Electrodynamics: The Gradient of Stress}

In standard physics, the Electric Field ($\mathbf{E}$) and Magnetic Field ($\mathbf{B}$) are treated as fundamental, irreducible vectors occupying an empty void. In Vacuum Engineering, they are explicitly derived as the macroscopic \textbf{Elastic Stress Gradients} and \textbf{Fluidic Vorticities} of the discrete $\mathcal{M}_A$ substrate.

\subsection{Deriving Coulomb's Law from the Laplace Equation}

Consider a topological defect (a charged node) with winding number $N$. This localized defect exerts a continuous rotational twist on the surrounding dielectric lattice. 

Instead of relying on heuristic geometric spreading, we rigorously derive the electrostatic force via continuum linear elasticity. Because the vacuum substrate acts as a linear elastic medium in the far-field (Axiom 2), the static rotational strain ($\theta$) of the lattice must strictly obey the 3D \textbf{Laplace Equation} to minimize the stored elastic energy:

\begin{equation}
    \nabla^2 \theta = 0
\end{equation}

The unique spherically symmetric solution to the Laplace equation dictates that the twist amplitude decays inversely with distance: $\theta(r) \propto 1/r$. 

The physical Electric Flux Density ($\mathbf{D}$) is precisely the spatial gradient of this structural twist ($\mathbf{D} = \nabla\theta$). Differentiating the Laplace solution naturally yields the exact inverse-square field:

\begin{equation}
    \mathbf{D} \propto -\frac{1}{r^2}\mathbf{\hat{r}}
\end{equation}

Because the vacuum resists this twist with an intrinsic capacitive compliance ($\epsilon_0$), the mechanical restoring force between two topological defects $q_1$ and $q_2$ evaluates perfectly to Coulomb's Law:

\begin{equation}
    F_{coulomb} = \frac{1}{4\pi\epsilon_0} \frac{q_1 q_2}{r^2}
\end{equation}

\textbf{Physical Insight:} ``Charge'' is not a magical, independent substance. It is the geometric measure of how severely a topological knot twists the local vacuum graph. ``Electrostatic Attraction'' is simply the physical vacuum substrate attempting to mechanically untwist to its lowest energy state.

\subsection{Magnetism as Convective Twist (Kinematic Vorticity)}

If ``Electricity'' is the static elastic twist of the lattice, ``Magnetism'' is its dynamic fluidic flow. 

As established in Chapter 2, the canonical momentum of the discrete lattice is the Magnetic Vector Potential ($\mathbf{A}$). When a twisted node translates through the lattice at constant velocity $\mathbf{v}$, it physically displaces the background nodes, inducing a convective shear flow. 

In fluid dynamics, the time evolution of a translating steady-state strain field $\mathbf{D}(\mathbf{r} - \mathbf{v}t)$ is governed identically by the convective derivative:
\begin{equation}
    \partial_t \mathbf{D} = -(\mathbf{v} \cdot \nabla)\mathbf{D}
\end{equation}

Using standard vector calculus identities for a uniform velocity and a source-free displacement field ($\nabla \cdot \mathbf{D} = 0$), this rigorously resolves to:
\begin{equation}
    -(\mathbf{v} \cdot \nabla)\mathbf{D} = \nabla \times (\mathbf{v} \times \mathbf{D})
\end{equation}

By equating this to the Maxwell-Ampere law for the substrate ($\nabla \times \mathbf{H} = \partial_t \mathbf{D}$), we flawlessly derive the macroscopic magnetic field without asserting it as an axiom:
\begin{equation}
    \mathbf{H} = \mathbf{v} \times \mathbf{D} \implies \mathbf{B} = \mu_0 (\mathbf{v} \times \mathbf{D})
\end{equation}

Magnetism is not a separate fundamental force. It is the exact \textbf{Kinematic Vorticity} generated when a static lattice twist is dragged through the inertial medium ($\mu_0$).