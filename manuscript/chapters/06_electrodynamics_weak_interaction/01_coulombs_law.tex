\section{Electrodynamics: The Gradient of Stress}
\label{sec:coulomb_stress}

In standard physics, the Electric Field ($\mathbf{E}$) is treated as a fundamental vector field. In Vacuum Engineering, we derive it as the \textbf{Elastic Stress Gradient} of the lattice.

\subsection{Deriving Coulomb's Law}
Consider a charged node (Section 3.4) with winding number $N$. This topological defect twists the surrounding lattice, creating a rotational strain field.
\begin{itemize}
    \item \textbf{Flux Density ($\mathbf{D}$):} The twist density drops off as $1/r^2$ due to geometric spreading in 3D space.
    \item \textbf{Lattice Elasticity ($\epsilon_0$):} The vacuum resists this twist with stiffness $\epsilon_0^{-1}$.
\end{itemize}
The force between two defects $q_1$ and $q_2$ is simply the mechanical restoration force of the intervening lattice nodes trying to untwist.
\begin{equation}
    F_{coulomb} = \frac{1}{4\pi \epsilon_0} \frac{q_1 q_2}{r^2}
\end{equation}
\textbf{Physical Insight:} "Charge" is not a magical fluid. It is the measure of how much a particle twists the vacuum. "Attraction" is simply the vacuum trying to relax to a lower energy state (Untwisting).

\subsection{Magnetism as Coriolis Force}
If "Electricity" is static twist, "Magnetism" is dynamic flow. When a twisted node moves, it drags the surrounding lattice (Pilot Wave).
\begin{equation}
    \mathbf{B} = \mu_0 (\mathbf{v} \times \mathbf{D})
\end{equation}
This derivation identifies the Magnetic Field ($\mathbf{B}$) as the \textbf{Coriolis Force} of the vacuum fluid. It is not a separate force; it is the inertial reaction of the lattice ($\mu_0$) to the movement of twist.