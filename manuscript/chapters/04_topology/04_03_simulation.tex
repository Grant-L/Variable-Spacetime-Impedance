\section{Simulation: The Trefoil Geometry}
\label{sec:trefoil_sim}

To visualize the stability of the proton, we modeled the 3D phase structure of a $3_1$ Trefoil Knot using the \texttt{ProtonTopology} module.

\begin{figure}[h]
    \centering
    \includegraphics[width=0.8\textwidth]{assets/sim_outputs/proton_trefoil_knot.png}
    \caption{\textbf{The Proton Topology.} The red tube represents the region of saturated vacuum flux (Mass). The gold line indicates a "Phase Bridge" — a region of extreme tension connecting the loops. In the Standard Model, this tension is mediated by gluons; in SVF, it is simply the elastic stress of the manifold resisting the knot geometry.}
    \label{fig:proton_knot}
\end{figure}

The simulation highlights the \textbf{Confinement} mechanism naturally. The loops of the knot are pulled together by the tension of the manifold nodes trying to return to the ground state ($Z_0$). Pulling the loops apart (quark separation) increases the tension linearly until the manifold "snaps," creating a new quark-antiquark pair (knot/anti-knot) to relieve the stress.