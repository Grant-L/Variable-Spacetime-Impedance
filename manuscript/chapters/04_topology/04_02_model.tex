\section{Modeling the Electron and Proton}
\label{sec:particle_models}

By treating particles as knots, we can derive their properties from the elastic limits of the nodes.

\subsection{The Electron: The Simple Vortex}
The electron is modeled as the simplest possible stable defect—a single $h=-1$ vortex. Its "point-like" nature is an illusion of the $l_P$ scale; it is actually a localized region of \textbf{Metric Strain} ($\sigma$) where the manifold nodes are driven into the non-linear regime.

\subsection{The Proton: The Trefoil Knot}
The proton is a complex topological defect modeled as a \textbf{Trefoil Knot} ($3_1$ knot). It consists of three entangled phase-twists. This explains why the proton is significantly more massive than the electron: the complex knot structure creates a much higher degree of local strain ($\sigma$), loading a larger number of manifold nodes into the saturation regime ($\omega_{spin} \to \omega_{sat}$).

\subsection{Topological Stability}
The stability of the proton is guaranteed by the \textbf{Conservation of Helicity}. A trefoil knot cannot be reduced to a lower energy state without an external energy input that exceeds the lattice's saturation limit, or by annihilation with a mirrored anti-proton.