\section{Modeling the Electron and Proton}
\label{sec:particle_models}

By treating particles as knots, we derive their properties from the topology of their flux loops.

\subsection{The Electron: The Trefoil Soliton ($3_1$)}
\citestart The electron is identified as the simplest non-trivial knot: the \textbf{Trefoil ($3_1$)}\cite{1076}\citeend.
\begin{itemize}
    \item \textbf{Mass Origin:} The knot crossings increase the self-inductance of the loop, trapping energy.
    \item \textbf{Chirality:} The Trefoil is chiral. The Left-Handed version is the Electron; the Right-Handed version is the Positron.
\end{itemize}

\subsection{The Proton: Borromean Confinement ($6^3_2$)}
\citestart The proton is a composite system of three linked flux loops (Quarks), modeled as \textbf{Borromean Rings}\cite{1077}\citeend.
\begin{itemize}
    \item \textbf{Confinement:} The Borromean topology consists of three loops interlinked such that no two are linked, but the three together are inseparable. If one loop is cut, the others fall apart. \citestart This geometrically enforces \textbf{Quark Confinement}\cite{1085}\citeend.
    \item \textbf{Gluon Tension:} The mass of the proton comes from the extreme lattice tension required to compress these three loops into a shared volume.
\end{itemize}