\section{The Mechanical Weinberg Angle}
\label{sec:weinberg_angle}

The Standard Model defines the Weinberg Angle ($\theta_{W}$) as a mixing parameter between force fields. In SVF, it is redefined as the mechanical orientation of the lattice's chiral bias relative to the axis of flux propagation.

\begin{equation}
    \cos(\theta_{W}) = \frac{Z_{0}}{Z_{total}}
    \label{eq:weinberg_angle}
\end{equation}

This ratio describes the "mixing" of the baseline electromagnetic impedance ($Z_{0}$) and the additional chiral impedance introduced by the biased substrate. Parity violation is naturally explained as a directional filter: the hardware has a preferred grain, and signals propagating against this grain encounter higher resistance.

\section{Beta Decay as Hardware Discharge}
\label{sec:beta_decay}

Beta decay ($n \rightarrow p + e^{-} + \overline{\nu}_{e}$) is modeled as the mechanical relaxation of a saturated node structure:
\begin{enumerate}
    \item \textbf{Transition:} The complex knot structure (Neutron) reconfigures into a lower-energy stable trefoil (Proton).
    \item \textbf{Discharge:} The excess flux density is ejected as a high-frequency pulse ($e^{-}$).
    \item \textbf{Neutrino Emission:} The "Neutrino" is the characteristic ringing of the lattice's elastic recovery. Because the discharge follows the path of least resistance in a biased manifold, the emission is exclusively Left-Handed ($Z \approx Z_0$). A Right-Handed emission would face infinite impedance ($Z \to \infty$) and is therefore mechanically forbidden.
\end{enumerate}