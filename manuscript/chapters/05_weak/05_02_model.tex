\section{The Mechanical Weinberg Angle}
\label{sec:weinberg_angle}

The Standard Model defines the Weinberg Angle ($\theta_{W}$) as a mixing parameter. In SVF, it is the mechanical orientation of the lattice's chiral bias relative to the axis of flux propagation:
\begin{equation}
    \cos(\theta_{W}) = \frac{Z_{0}}{Z_{total}}
\end{equation}
This ratio describes the "mixing" of the baseline electromagnetic impedance ($Z_{0}$) and the additional chiral impedance introduced by the biased substrate.

\section{Beta Decay as Hardware Discharge}
\label{sec:beta_decay}

Beta decay ($n \rightarrow p + e^{-} + \overline{\nu}_{e}$) is modeled as the mechanical relaxation of a saturated node structure:
\begin{enumerate}
    \item \textbf{Transition:} The threaded electron (Neutron: $6^3_2 \cup 3_1$) slips its topological lock and is ejected.
    \item \textbf{Discharge:} The lattice snaps back to the stable Borromean configuration (Proton: $6^3_2$).
    \item \textbf{Neutrino Emission:} To conserve angular momentum during the snap, the lattice sheds a "Twist Defect" (Antineutrino). Because the discharge follows the path of least resistance, the emission is exclusively \textbf{Left-Handed}. A Right-Handed emission would face infinite impedance and is mechanically forbidden.
\end{enumerate}