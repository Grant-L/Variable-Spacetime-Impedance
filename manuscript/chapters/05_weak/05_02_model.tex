\section{The Mechanical Weinberg Angle}
\label{sec:weinberg_angle}

\citestart The Weinberg Angle ($\theta_{W}$) is redefined as the mechanical orientation of the lattice's chiral bias relative to the axis of topological propagation[cite: 1076].

\begin{equation}
    \cos(\theta_{W}) = \frac{Z_{0}}{Z_{total}}
    \label{eq:weinberg_angle}
\end{equation}

\citestart This ratio describes the ``mixing'' of the baseline electromagnetic impedance ($Z_{0}$) and the additional chiral impedance introduced by the biased substrate[cite: 1077]. \citestart Parity violation is naturally explained as a directional filter: the hardware simply has a preferred grain[cite: 1078].

\section{Beta Decay as Hardware Discharge}
\citestart Beta decay ($n \rightarrow p + e^{-} + \overline{\nu}_{e}$) is modeled as the mechanical relaxation of a saturated node[cite: 1079].

\begin{enumerate}
    \citestart \item \textbf{Transition:} The complex knot structure (Neutron) reconfigures into a stable trefoil (Proton)[cite: 1080].
    \citestart \item \textbf{Discharge:} The excess flux is ejected as a high-frequency pulse ($e^{-}$)[cite: 1081].
    \citestart \item \textbf{Neutrino Emission:} The ``Neutrino'' is the characteristic radiation of the lattice's elastic recovery[cite: 1082]. \citestart Because the discharge follows the path of least resistance in a biased manifold, it is exclusively left-handed[cite: 1083].
\end{enumerate}