\section{The Inverse Resonance Scaling Law}
\label{sec:inverse_resonance}

We define the interaction range ($D$) of a topological defect not by an arbitrary mass term, but as a function of its characteristic resonance frequency ($\nu$) relative to the substrate's saturation limit.

The interaction range is given by the \textbf{Inverse Resonance Scaling Law}:
\begin{equation}
    D(\nu) = \frac{\zeta}{Z_{metric}(\nu) \cdot \nu}
\end{equation}
Where $\zeta$ is the Lattice Flux Constant.

As the signal frequency $\nu$ approaches the hardware Saturation Threshold ($\omega_{sat}$), or as the chiral impedance $Z_{metric}$ spikes due to parity violation, the denominator grows non-linearly. This forces the energy into a localized \textbf{Topological Short}, restricting the interaction range to the immediate nodal neighborhood ($\approx 10^{-18}$ m).

\textbf{Conclusion:} The large "mass" of the W/Z bosons ($80-90$ GeV) is simply the manifestation of this extreme lattice stiffness resisting the topological snap.