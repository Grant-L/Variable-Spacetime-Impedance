\section{Experimental Falsification: The RLVE}
\label{sec:rlve}

If the AVE viscous vacuum hypothesis is correct, this macroscopic fluid dynamics effect must be measurable in a controlled laboratory environment. We propose the \textbf{Rotational Lattice Viscosity Experiment (RLVE)}.

\subsection{Methodology and Theoretical Prediction}
As proven dimensionally, the Vacuum Viscosity ($\eta_{vac}$) possesses the exact units of dynamic viscosity $[\text{Pa} \cdot \text{s}]$. By rapidly rotating a mass adjacent to a high-finesse Fabry-Perot interferometer, we induce a localized viscous ``drag'' in the vacuum dielectric, creating a measurable refractive index shift ($\Delta n$). The effect scales with the tangential velocity ($v_{tan}$) and the material mass density relative to a reference saturation ($\rho_{rotor}/\rho_{ref}$):
\begin{equation}
\Delta n = \alpha \left(\frac{v_{tan}}{c}\right)^2 \left(\frac{\rho_{rotor}}{\rho_{ref}}\right)
\end{equation}

\subsection{Simulation and Falsification Condition}
Using the \texttt{run\_rlve\_prediction.py} simulation module, we model a 0.1 m radius Tungsten rotor spun to 100,000 RPM, adjacent to a 0.2 m optical cavity with a finesse of 10,000. 

\begin{figure}[ht]
\centering
\includegraphics[width=0.85\textwidth]{chapters/12_experimental_falsification/simulations/rlve_prediction.png}
\caption{\textbf{RLVE Viscous Drag Prediction.} The simulation contrasts the strong 0.72 mrad signal produced by a high-density Tungsten rotor against an Aluminum control. General Relativity predicts a near-zero frame-dragging effect ($\sim 10^{-20}$ rad) at this scale.}
\label{fig:rlve_prediction}
\end{figure}

The simulation predicts a phase shift of $\Delta\phi \approx 0.72$ milli-radians for Tungsten, which is orders of magnitude larger than General Relativity predictions and well above the noise floor of modern interferometry ($10^{-6}$ rad). An Aluminum control rotor yields a heavily suppressed signal due to its lower density, successfully isolating the AVE metric viscosity from purely geometric aerodynamic turbulence.

\textbf{The Metric Null-Result Kill-Switch:} If the RLVE is constructed and yields a null result (no density-dependent phase shift above the noise floor), the macroscopic fluid dynamics of the AVE framework, including the Hubble-MOND unification and the viscosity of space, are decisively falsified.

\section{Summary of Falsification Thresholds}
\label{sec:summary_table}

\begin{table}[h]
\centering
\begin{tabular}{|l|l|l|}
\hline
\textbf{Phenomenon} & \textbf{AVE Prediction} & \textbf{Falsification Signal} \\ \hline
\citestart Neutrino Spin & Exclusive Left-Handed & Detection of stable RH Neutrino \cite{cahill2005}\citeend \\ \hline
\citestart Light Speed & Slew Rate Dependent & Speed of light found to be a geometric constant \cite{einstein1916}\citeend \\ \hline
\citestart Gravity & Refractive Gradient & Detection of Gravitons (force particles) \cite{einstein1916}\citeend \\ \hline
\citestart Max Frequency & $\omega_{sat}$ (Planck Limit) & Trans-Planckian Signal ($\nu > \omega_{sat}$) \cite{einstein1916}\citeend \\ \hline
\end{tabular}
\caption{The Universal Means Test: Defining the boundaries of the Applied Vacuum Electrodynamics framework.}
\end{table}

\subsubsection{Discriminative Signature: The Metric Viscosity Ratio}
To rigorously distinguish AVE from General Relativity (GR), we define the \textbf{Metric Viscosity Ratio} ($\Psi$). While GR predicts a Frame-Dragging effect (Lense-Thirring) that is purely geometric and independent of the rotor's material density ($\rho$), AVE predicts that the refractive index shift ($\Delta n$) is a \textbf{constitutive response} of the substrate.

\begin{equation}
    \Psi = \frac{\Delta n_{Tungsten}}{\Delta n_{Aluminum}}
\end{equation}

\begin{itemize}
    \item \textbf{GR Prediction:} $\Psi \approx 1.0$. The effect depends only on geometry and angular momentum (Frame Dragging).
    \item \textbf{AVE Prediction:} $\Psi \approx \frac{\rho_{W}}{\rho_{Al}} \approx 7.1$. The effect scales with the inductive density of the rotor material.
\end{itemize}

\textbf{Kill Condition:} A measured value of $\Psi > 5$ would falsify the "frictionless void" model of General Relativity and provide the first direct laboratory measurement of the vacuum's kinematic viscosity ($\nu_{vac}$). Conversely, a result of $\Psi \approx 1$ would decisively falsify the AVE hydrodynamic framework.