\section{Experimental Falsification: The RLVE}
\label{sec:rlve}

If the AVE viscous vacuum hypothesis is correct, this macroscopic fluid dynamics effect must be measurable in a controlled laboratory environment. We propose the \textbf{Rotational Lattice Viscosity Experiment (RLVE)}.

\subsection{Methodology and Theoretical Prediction}
As proven dimensionally, the Vacuum Viscosity ($\eta_{vac}$) possesses the exact units of dynamic viscosity $[\text{Pa} \cdot \text{s}]$. By rapidly rotating a mass adjacent to a high-finesse Fabry-Perot interferometer, we induce a localized viscous ``drag'' in the vacuum dielectric, creating a measurable refractive index shift ($\Delta n$). The effect scales with the tangential velocity ($v_{tan}$) and the material mass density relative to a reference saturation ($\rho_{rotor}/\rho_{ref}$):
\begin{equation}
    \Delta n = \alpha \left(\frac{v_{tan}}{c}\right)^2 \left(\frac{\rho_{rotor}}{\rho_{ref}}\right)
\end{equation}

Here, $\rho_{ref} \equiv \rho_{nuc} \approx 2.3 \times 10^{17} \text{ kg/m}^3$ represents the \textbf{Nuclear Saturation Density}—the maximum matter density the lattice can support before dielectric breakdown (the event horizon limit). The ratio $(\rho_{rotor}/\rho_{ref})$ quantifies the degree to which the material stresses the vacuum substrate toward its elastic limit.

\subsection{Simulation and Falsification Condition}
Using the \texttt{run\_rlve\_prediction.py} simulation module, we model a 0.1 m radius Tungsten rotor spun to 100,000 RPM, adjacent to a 0.2 m optical cavity with a finesse of 10,000. 

\begin{figure}[ht]
\centering
\includegraphics[width=0.85\textwidth]{chapters/12_experimental_falsification/simulations/rlve_prediction.png}
\caption{\textbf{RLVE Viscous Drag Prediction.} The simulation contrasts the strong 0.72 mrad signal produced by a high-density Tungsten rotor against an Aluminum control. General Relativity predicts a near-zero frame-dragging effect ($\sim 10^{-20}$ rad) at this scale.}
\label{fig:rlve_prediction}
\end{figure}

The simulation predicts a phase shift of $\Delta\phi \approx 0.72$ milli-radians for Tungsten, which is orders of magnitude larger than General Relativity predictions and well above the noise floor of modern interferometry ($10^{-6}$ rad). An Aluminum control rotor yields a heavily suppressed signal due to its lower density, successfully isolating the AVE metric viscosity from purely geometric aerodynamic turbulence.

\textbf{The Metric Null-Result Kill-Switch:} If the RLVE is constructed and yields a null result (no density-dependent phase shift above the noise floor), the macroscopic fluid dynamics of the AVE framework, including the Hubble-MOND unification and the viscosity of space, are decisively falsified.

\section{Summary of Falsification Thresholds}
\label{sec:summary_table}

\begin{table}[h]
\centering
\begin{tabular}{|l|l|l|}
\hline
\textbf{Phenomenon} & \textbf{AVE Prediction} & \textbf{Falsification Signal} \\ \hline
\citestart Neutrino Spin & Exclusive Left-Handed & Detection of stable RH Neutrino \cite{cahill2005}\citeend \\ \hline
\citestart Light Speed & Slew Rate Dependent & Speed of light found to be a geometric constant \cite{einstein1916}\citeend \\ \hline
\citestart Gravity & Refractive Gradient & Detection of Gravitons (force particles) \cite{einstein1916}\citeend \\ \hline
\citestart Max Frequency & $\omega_{sat}$ (Planck Limit) & Trans-Planckian Signal ($\nu > \omega_{sat}$) \cite{einstein1916}\citeend \\ \hline
\end{tabular}
\caption{The Universal Means Test: Defining the boundaries of the Applied Vacuum Electrodynamics framework.}
\end{table}

\subsubsection{Discriminative Signature: The Metric Viscosity Ratio}
To rigorously distinguish AVE from General Relativity (GR), we define the \textbf{Metric Viscosity Ratio} ($\Psi$). While GR predicts a Frame-Dragging effect (Lense-Thirring) that is purely geometric and independent of the rotor's material density ($\rho$), AVE predicts that the refractive index shift ($\Delta n$) is a \textbf{constitutive response} of the substrate.

\begin{equation}
    \Psi = \frac{\Delta n_{Tungsten}}{\Delta n_{Aluminum}}
\end{equation}

\begin{itemize}
    \item \textbf{GR Prediction:} $\Psi \approx 1.0$. The effect depends only on geometry and angular momentum (Frame Dragging).
    \item \textbf{AVE Prediction:} $\Psi \approx \frac{\rho_{W}}{\rho_{Al}} \approx 7.1$. The effect scales with the inductive density of the rotor material.
\end{itemize}

\textbf{Kill Condition:} A measured value of $\Psi > 5$ would falsify the "frictionless void" model of General Relativity and provide the first direct laboratory measurement of the vacuum's kinematic viscosity ($\nu_{vac}$). Conversely, a result of $\Psi \approx 1$ would decisively falsify the AVE hydrodynamic framework.

\subsubsection{RLVE Systematics and Error Budget}
To confirm the signal $\Psi > 5$, we must isolate the constitutive density effect from mundane mechanical noise.
The primary systematic threats and their suppression strategies are defined below.

\begin{table}[h]
\centering
\begin{tabular}{|l|l|l|}
\hline
\textbf{Noise Source} & \textbf{Magnitude} & \textbf{Suppression Strategy} \\ \hline
Aerodynamic Drag & $\sim 10^{-4}$ rad & **High Vacuum** ($<10^{-7}$ Torr) enclosure. \\ \hline
Rotor Vibration & $\sim 10^{-5}$ rad & **Common-Mode Rejection**: Differential interferometer measures relative phase between Tungsten and Aluminum sectors on the \textit{same} rotor. \\ \hline
Thermal Gradient & $\sim 10^{-6}$ rad & **Chopping**: Signal is modulated at rotor frequency ($f_{rot} = 1.6$ kHz), rejecting DC thermal drift. \\ \hline
Magnetic Coupling & $\sim 10^{-8}$ rad & **Shielding**: Non-magnetic Tungsten alloy + Mu-Metal shielding. \\ \hline
\textbf{Target Signal} & $\textbf{7.2} \times \textbf{10}^{-4}$ \textbf{rad} & **SNR > 100** (using Lock-in Amplification) \\ \hline
\end{tabular}
\caption{RLVE Error Budget. The density-dependent signal is isolatable via differential measurement and synchronous detection.}
\end{table}

\subsubsection{Experimental Protocols and Orthogonal Controls}
To decisively isolate the Vacuum Viscosity signal from mundane environmental noise, the RLVE employs a \textbf{Tri-Phasic Control Protocol}.

\textbf{Phase I: The Density Swap (The Signal)}
We compare a Tungsten Rotor ($\rho \approx 19.3$ g/cc) against an Aluminum Rotor ($\rho \approx 2.7$ g/cc) of identical geometry.
\begin{itemize}
    \item \textbf{Prediction:} The Tungsten phase shift $\Delta \phi_W$ must be $\approx 7.1\times$ larger than $\Delta \phi_{Al}$.
    \item \textbf{Control:} If $\Delta \phi_W \approx \Delta \phi_{Al}$, the signal is aerodynamic/mechanical (Null Result).
\end{itemize}

\textbf{Phase II: The Vacuum Sweep (The Drag)}
We measure the signal as a function of chamber pressure from $10^{-3}$ Torr to $10^{-8}$ Torr.
\begin{itemize}
    \item \textbf{Prediction:} The AVE signal is pressure-independent below $10^{-6}$ Torr.
    \item \textbf{Control:} If the signal scales linearly with chamber pressure, it is residual gas drag.
\end{itemize}

\textbf{Phase III: The Retrograde Reversal (The Symmetry)}
We reverse the rotation direction of the rotor ($\omega \rightarrow -\omega$).
\begin{itemize}
    \item \textbf{Prediction:} The phase shift sign must invert ($\Delta \phi \rightarrow -\Delta \phi$).
    \item \textbf{Control:} If the signal polarity does not track rotation direction, it is thermal drift or vibration.
\end{itemize}

\subsubsection{Derivation of the Density-Viscosity Coupling}
The RLVE predicts that the vacuum viscosity shift $\Delta n$ scales with the material density of the rotor. We derive this constitutive relationship from the definition of Mass as Stored Flux.

\textbf{Step 1: Mass as Flux Density}
In AVE, atomic mass is not "solid matter" but a count of confined topological flux loops (protons/neutrons). The local flux density $\Phi_{local}$ inside a material of density $\rho_{mat}$ is:
\begin{equation}
    \Phi_{local} \propto \frac{\rho_{mat}}{m_p} \cdot \Phi_{proton}
\end{equation}

\textbf{Step 2: Viscosity as Flux Drag}
The vacuum viscosity $\eta$ arises from the node-to-node coupling. The presence of stored flux loops (matter) tightens the local lattice, increasing the effective coupling coefficient (Impedance Stiffening).
\begin{equation}
    \eta_{eff} = \eta_0 (1 + \chi_{mag} \Phi_{local})
\end{equation}
Since the magnetic susceptibility of the vacuum $\chi_{mag}$ is linear in the weak-field limit, the increase in viscosity is directly proportional to the density of flux loops.

\textbf{Step 3: The Constitutive Equation}
Combining these, we obtain the fundamental scaling law for the Rotational Lattice Viscosity:
\begin{equation}
    \Delta n_{viscous} = \alpha \left( \frac{v_{tan}}{c} \right)^2 \left( \frac{\rho_{rotor}}{\rho_{sat}} \right)
\end{equation}
Where $\rho_{sat} \approx 2.3 \times 10^{17}$ kg/m$^3$ is the nuclear saturation density (the maximum flux density of the lattice).

\textbf{Conclusion:} A Tungsten rotor ($\rho \approx 19.3$) creates a viscous drag 7.1x stronger than Aluminum ($\rho \approx 2.7$) because it contains 7.1x more topological flux loops per unit volume to drag against the vacuum substrate. This density dependence is the "Smoking Gun" that distinguishes AVE from the purely geometric Frame Dragging of General Relativity.