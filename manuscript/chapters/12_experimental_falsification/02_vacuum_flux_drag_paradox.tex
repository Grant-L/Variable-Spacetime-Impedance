\section{The Vacuum-Flux Drag Paradox: A Lesson in Stability}

To directly test the Topo-Kinematic Isomorphism (Axiom 1: $\mathbf{A} \equiv \mathbf{p}_{vac}$), one might propose a \textbf{Vacuum-Flux Drag Test (VFDT)}. If a powerful toroidal magnetic field is identically a continuous fluidic flywheel of physical vacuum momentum, shouldn't firing a massive 50 kA EMP pulse mechanically drag a laser beam passing through its core and induce a measurable Fresnel-Fizeau optical phase shift?

In Chapter 2, we established the exact dimensional homogeneity of Magnetic Flux ($\Phi$) and Mechanical Momentum ($p_{vac}$). 
\begin{align}
    1\,\text{Weber} &= 1\,\text{V}\cdot\text{s} = 1\,\frac{\text{J}}{\text{C}}\cdot\text{s} \nonumber \\
    &\text{Substitute exactly } 1\,\text{C} \equiv \xi_{topo}\,\text{m}: \nonumber \\
    1\,\text{Wb} &= \frac{1\,\text{J}\cdot\text{s}}{\xi_{topo}\,\text{m}} = \frac{1}{\xi_{topo}} \left( \frac{\text{N}\cdot\text{m}\cdot\text{s}}{\text{m}} \right) = \mathbf{\frac{1}{\xi_{topo}} \left[ \text{kg} \cdot \frac{\text{m}}{\text{s}} \right]}
\end{align}
Thus, the total macroscopic mechanical momentum stored in a magnetic flux loop is exactly $p_{vac} = \Phi \cdot \xi_{topo}$. 

For a colossal laboratory-scale toroidal pulse ($R=0.25$m, $r=0.05$m, $I = 50,000$ A), it generates a $\sim 4.0$ Tesla field, enclosing a total flux of $\Phi \approx 0.0314$ Webers. The total vacuum momentum evaluates exactly to $p_{vac} \approx 1.30 \times 10^{-8} \text{ kg m/s}$.

\subsection{Dimensional Exactness of the Drift Velocity}
To find the physical fluidic drift velocity of the continuous space ($v_{vac} = p_{vac} / M_{vac}$), we must strictly divide this momentum by the \textit{true bulk 3D mass} of the vacuum fluid occupying the torus core. 

A common heuristic error is to define this vacuum mass linearly ($M_{linear} = \xi_{topo}^2 \mu_0 \cdot L_{path}$), inadvertently treating the magnetic field as a 1D string. This violently violates the volumetric dimensional bounds of continuum hydrodynamics. 

In Chapter 10, we proved the physical bulk mass density of the 3D spatial vacuum is $\rho_{bulk} \approx 7.9 \times 10^6 \text{ kg/m}^3$. The physical mass of the vacuum fluid inside a small $0.012$ m$^3$ torus is an astronomical \textbf{$97,450 \text{ kg}$}.
\begin{equation}
    v_{vac} = \frac{1.30 \times 10^{-8} \text{ kg m/s}}{97,450 \text{ kg}} \approx \mathbf{1.33 \times 10^{-13} \text{ m/s}}
\end{equation}

For a standard 200m Fiber-Optic Sagnac interferometer, a fluid drift velocity of $10^{-13}$ m/s yields a Fresnel-Fizeau optical phase shift of $\Delta \phi \approx 10^{-14}$ radians. The electromagnetic drag signal is entirely undetectable by standard optical bench hardware.

\subsection{The Triumph of the Null Result}
While this nullifies electromagnetic dragging as a viable tabletop experiment, the mathematical result is a \textbf{triumphant theoretical proof of the framework's macroscopic optical stability}. 

If a simple 50 kA magnet could easily drag the vacuum fluid at macroscopic speeds (e.g., mm/s), the spatial metric inside a standard hospital MRI machine would aggressively and visibly warp the path of ambient light, violently violating standard Lorentz invariance to the naked eye. The fact that AVE mathematically suppresses the kinematic drift to $10^{-13}$ m/s directly because of the immense, White Dwarf-level bulk density of the vacuum ($\rho_{bulk}$) proves exactly and mechanically why light flawlessly propagates in straight, stable lines through intense classical magnetic fields (see Figure \ref{fig:vfdt_null_prediction}).

\begin{figure}[htbp]
    \centering
    \includegraphics[width=0.9\textwidth]{chapters/12_experimental_falsification/simulations/outputs/vfdt_dimensional_correction.png}
    \caption{\textbf{The VFDT and Optical Stability.} Due to the immense macroscopic 3D mass density of the $\mathcal{M}_A$ substrate ($\rho_{bulk} \approx 7.9 \times 10^6$ kg/m$^3$), massive electromagnetic momentum fields generate absolutely negligible fluid advection ($v_{vac} \sim 10^{-13}$ m/s). This rigidly protects macroscopic Lorentz invariance, proving structurally why intense laboratory magnetic fields do not visibly bend ambient light via fluidic frame-dragging.}
    \label{fig:vfdt_null_prediction}
\end{figure}