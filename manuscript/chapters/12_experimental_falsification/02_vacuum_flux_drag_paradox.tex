\section{The Tabletop Graveyard: Why Intuitive Tests Fail}

To effectively falsify the AVE framework, one must understand why intuitive, classical tabletop tests fail to detect the continuous macroscopic vacuum substrate. The failure of these tests is not a flaw in the AVE framework; rather, these "failures" are mathematically required by the framework to rigorously preserve macroscopic Lorentz Invariance.

\subsection{The Vacuum-Flux Drag Test (VFDT) and Magnetic Stability}
To directly test the Topo-Kinematic Isomorphism (Axiom 1: $\mathbf{A} \equiv \mathbf{p}_{vac}$), one might intuitively propose a \textbf{Vacuum-Flux Drag Test (VFDT)}. If a toroidal magnetic field is identically a continuous fluidic flywheel of physical vacuum momentum, shouldn't firing a massive 50 kA EMP pulse mechanically drag a laser beam passing through its core via the Fresnel-Fizeau effect?

By equating Magnetic Flux to mechanical momentum ($p_{vac} = \Phi \cdot \xi_{topo}$), a massive 4.0 Tesla toroidal field generates exactly $p_{vac} \approx 1.30 \times 10^{-8} \text{ kg m/s}$ of continuous vacuum momentum.

To find the physical fluidic drift velocity ($v_{vac} = p_{vac} / M_{vac}$), we must strictly divide this momentum by the \textit{true bulk 3D mass} of the vacuum fluid occupying the torus core. In Chapter 10, we proved the physical bulk mass density of the spatial vacuum is $\rho_{bulk} \approx 7.9 \times 10^6 \text{ kg/m}^3$. The physical mass of the vacuum fluid inside a small $0.012$ m$^3$ tabletop torus is an astronomical \textbf{$97,450 \text{ kg}$}.
\begin{equation}
    v_{vac} = \frac{1.30 \times 10^{-8} \text{ kg m/s}}{97,450 \text{ kg}} \approx \mathbf{1.33 \times 10^{-13} \text{ m/s}}
\end{equation}
This microscopic drift velocity yields an optical phase shift of $\sim 10^{-14}$ radians, which is entirely undetectable. 

\textbf{Theoretical Triumph:} This null result is an absolute requirement for stable physics. If a 50 kA magnet could drag the vacuum fluid at 1 cm/s, the spatial metric inside a standard hospital MRI machine would aggressively and visibly warp the path of ambient light, violently violating standard Lorentz invariance to the naked eye. The hyper-density of the AVE vacuum acts as a massive inertial anchor, perfectly explaining why light propagates in straight lines through intense classical magnetic fields.

\subsection{The Regenerative Vacuum Receiver (RVR) and the Scalar Gap}
A second intuitive approach is to utilize high-gain electronics. Because vacuum density ($\rho$) dictates the local scalar refractive index ($n_{scalar}$), and magnetic permeability scales identically with $n$, one could build a \textbf{Regenerative Vacuum Receiver (RVR)}. By rapidly spinning a lobed Tungsten rotor next to an LC tank circuit, one could theoretically modulate the Kinetic Inductance of the circuit ($\Delta L$) and use a negative-resistance regenerative amplifier to catch the parametric ripple.

However, the change in scalar density induced by a moving mass is governed strictly by the volumetric strain: $\chi_{vol} = \frac{7GM}{c^2 r}$. For a $1$ kg Tungsten lobe passing $1$ cm away from the coil, the resulting modulation depth ($\delta_L \approx \frac{1}{7}\chi_{vol}$) is astronomically small:
\begin{equation}
    \delta_L \approx \frac{G \cdot (1\text{ kg})}{c^2 \cdot (0.01\text{ m})} \approx \mathbf{7.4 \times 10^{-26}}
\end{equation}

For a parametric amplifier to achieve spontaneous regenerative oscillation, the product of the circuit's Quality Factor ($Q$) and the modulation depth must exceed unity ($Q \cdot \delta_L \ge 2$). Therefore, the RVR would require an LC tank circuit with a $Q \ge 2.7 \times 10^{25}$. Because the highest Q-factors ever achieved in cryogenic superconducting SRF cavities max out at $\sim 10^{11}$, the RVR falls short of the absolute thermal noise limit by 15 orders of magnitude.

Scalar gravity tests fail on a tabletop because they are fatally suppressed by the $G/c^2$ scalar gap.