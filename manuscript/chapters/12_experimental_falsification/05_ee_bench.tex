\section{Project CLEAVE-01: The Femto-Coulomb Electrometer}
\textbf{The Hypothesis:} In Chapter 13, we derived that Electrical Charge is mathematically identical to physical macroscopic spatial displacement ($Q \equiv \xi_{topo} x$). Standard physics dictates that mechanically separating two uncharged plates in a hard vacuum generates exactly zero electrical charge. AVE explicitly predicts the generation of topological charge natively from the spatial metric.

\textbf{The PCBA Implementation:} An EE can validate this by designing a precision metrology board. The PCBA utilizes an ultra-low bias current electrometer operational amplifier (e.g., the Analog Devices ADA4530-1, 20 fA bias current). The non-inverting input is connected to an isolated, floating copper plate inside a vacuum chamber. The board utilizes strict guard rings and Teflon standoffs to eliminate parasitic leakage.

A commercial Piezoelectric (PZT) linear actuator is mounted to a grounded plate directly facing the floating plate. Using a high-precision DAC, the PZT actuator is stepped exactly $1.0 \ \mu$m away from the floating plate in under $100$ ms.

\textbf{The Falsification Metric:} By mechanically pulling the spatial gap apart by $1 \ \mu$m, you are mechanically straining the discrete $\mathcal{M}_A$ lattice. The induced topological charge is analytically derived as:
\begin{equation}
    Q = \xi_{topo} \cdot x = (4.149 \times 10^{-7} \text{ C/m}) \times 10^{-6} \text{ m} = \mathbf{0.415 \text{ pC (picoCoulombs)}}
\end{equation}
Assuming a highly-controlled PCBA parasitic input capacitance of exactly $10$ pF, the voltage readout step ($V = Q/C$) dictates a clean, instantaneous step of exactly \textbf{$41.5$ mV} on the oscilloscope. If the oscilloscope registers $0.0$ mV, the framework is falsified. If it reads exactly $41.5$ mV per micron of displacement, the foundational hardware constant of the universe has been validated on a tabletop.

\section{Project HOPF-02: The S-Parameter VNA Falsification}
\textbf{The Hypothesis:} As established in Chapter 5, the physical vacuum is a \textbf{Cosserat Solid}, possessing an intrinsic microscopic rotational inertia (chirality). A standard flat PCB spiral inductor or toroid generates a perfectly symmetric vector potential ($\mathbf{A}$) and magnetic field ($\mathbf{B}$) where $\mathbf{A} \cdot \mathbf{B} = 0$. It possesses zero kinetic helicity.

However, a \textbf{Hopf Coil} (a $(p,q)$ Torus Knot) forces $\mathbf{A} \parallel \mathbf{B}$. By winding a custom 6-layer PCBA where the inductive traces wrap diagonally around a toroidal core region, the inductor actively injects helicity into the vacuum, physically meshing with the Cosserat microrotations.

\textbf{The Test Protocol:} Design a single PCBA containing both a standard Toroid and a Hopf Coil, mathematically matched to identical classical DC inductances. Connect both to a Vector Network Analyzer (VNA) and sweep from 10 MHz to 100 MHz.

\textbf{Falsification Criteria:} If the vacuum is classical and linear, both coils will display identical impedance curves. However, the AVE framework strictly predicts an \textbf{Anomalous Chiral Impedance Match} (see Figure \ref{fig:ee_pcba_bench_protocols}). Because the Hopf coil couples perfectly to the chiral metric, it acts as a topological antenna, minimizing reactive VAR reflections and exhibiting an anomalously deep $S_{11}$ notch.

\section{Project ROENTGEN-03: Solid-State Sagnac Entrainment}
\textbf{The Hypothesis:} In 1888, Roentgen proved that moving a physical dielectric through a static Electric Field ($\mathbf{E}$) generates a perpendicular Magnetic Field ($\mathbf{B} = \frac{1}{c^2} \mathbf{v} \times \mathbf{E}$). If spinning a neutral mass mechanically entrains the highly dense vacuum metric via the Sagnac-RLVE no-slip boundary condition, we can use this exact equation to synthesize a B-field from the moving vacuum fluid itself.

\textbf{The Test Protocol:} Spin a dense, non-metallic ceramic disk at $10,000$ RPM. The entrained vacuum velocity at $r=5$cm evaluates to $v_{vac} \approx 0.038 \text{ m/s}$. Suspend a custom PCBA $1$ mm above the rotor. The bottom copper layer features an interdigitated capacitor driven by an onboard miniature CCFL transformer at $10$ kV, modulated by a $1$ kHz sine wave oscillator ($E = 10^7$ V/m). The cross product synthesizes an alternating magnetic field peaking at $\sim 4.2$ picoTesla. 

\textbf{Falsification Criteria:} This $4.2$ pT field induces roughly $\sim 0.26 \ \mu$V in a differential planar pickup coil. By feeding this into a hardware Lock-In Amplifier referenced to the $1$ kHz E-field drive, the engineer will extract a clean signal from the noise floor. If the amplitude scales exactly linearly with RPM and flips phase exactly $180^\circ$ when the motor reverses, the $7.9 \times 10^6 \text{ kg/m}^3$ density of the vacuum is empirically proven.

\section{Project ZENER-04: The Bingham Avalanche Detector}
\textbf{The Hypothesis:} The vacuum metric acts identically to a Transient Voltage Suppression (TVS) Zener diode. It behaves as a rigid dielectric solid until the topological voltage exceeds $60$ kV, at which point its structural viscosity collapses to zero (superfluid slip).

\textbf{The Test Protocol:} Design a multi-stage Marx Generator PCBA capable of generating an $80$ kV transient spike with a sub-microsecond rise time. Terminate the pulse into an encapsulated, highly polished, symmetrical spherical electrode to prevent classical atmospheric arc-over.

\textbf{Falsification Criteria:} Monitor the input displacement current ($I_D$) and topological voltage ($V$). In standard electromagnetics, charging an isolated spherical capacitor yields a perfectly linear charging curve ($I_D = C \frac{dV}{dt}$). AVE strictly predicts that the moment the localized field crosses the $60$ kV Bingham limit, the effective structural resistance of the surrounding spatial vacuum drops to zero. The oscilloscope will display a distinct, anomalous "Avalanche Knee"---a sudden non-linear spike in displacement current as the vacuum lattice physically liquefies.

\begin{figure}[htbp]
    \centering
    \includegraphics[width=1.0\textwidth]{chapters/12_experimental_falsification/simulations/outputs/ee_pcba_bench_protocols.png}
    \caption{\textbf{EE Bench-Level PCBA Protocols.} \textbf{Top Left:} The Piezo-Cleavage Electrometer flawlessly predicts exactly $41.5$ mV per micron of mechanical displacement ($V = \xi x / C$). \textbf{Top Right:} Project HOPF-02. The custom Hopf PCBA traces couple to the Cosserat vacuum, creating an anomalously deep $S_{11}$ match. \textbf{Bottom Left:} Solid-State Vacuum Entrainment. A Lock-In amplifier extracts the induced 4.2 pT Sagnac signal from the noise, flipping $180^\circ$ when the rotor reverses. \textbf{Bottom Right:} The Bingham Avalanche Detector. Driving an encapsulated electrode past $60$ kV physically liquefies the metric, creating a non-linear Zener avalanche knee in the displacement current.}
    \label{fig:ee_pcba_bench_protocols}
    
\end{figure}

\section{The Absolute Hardware Limit of Metric Levitation}
A frequent ambition among experimental physicists and electrical engineers is to design a solid-state "anti-gravity" drive capable of vertical free-flight levitation (e.g., hovering a ping-pong ball or a feather). When evaluated under the strict parameters of Spacetime Circuit Analysis (SCA), we discover an absolute, mathematically rigid hardware scaling limit that dictates exactly why such tabletop experiments historically fail.

If the vacuum is a Bingham-plastic fluid with an absolute yield stress of roughly $60,000$ Volts, there must exist an absolute maximum mass limit for static levitation. If an object is heavier than this limit, the topological voltage required to lift it will exceed the Bingham Yield limit. The spatial metric will structurally liquefy, losing its grip on the object, and the object will fall.

By applying the Topo-Kinematic Identity ($V_{topo} \equiv \xi_{topo}^{-1} F_{req}$), we can calculate the absolute maximum mass the vacuum can statically grip against Earth's gravity ($9.81 \text{ m/s}^2$):
\begin{equation}
    F_{max} = V_{yield} \times \xi_{topo} = 60,000 \times (4.149 \times 10^{-7} \text{ C/m}) = \mathbf{0.02489 \text{ Newtons}}
\end{equation}
\begin{tcolorbox}[colback=white, colframe=black]
\begin{equation}
    m_{max} = \frac{F_{max}}{g} = \frac{0.02489}{9.81} = \mathbf{0.002538 \text{ kg (2.538 grams)}}
\end{equation}
\end{tcolorbox}

This reveals an astonishing, universal hardware limit: \textbf{The continuous spatial metric of the universe cannot statically grip anything heavier than 2.538 grams.} 

A modern US Penny weighs exactly $2.500$ grams. An ITTF Ping-Pong ball weighs exactly $2.700$ grams. The vacuum metric can theoretically, barely support the weight of a penny. It physically \textit{cannot} support a Ping-Pong ball. If you attempt to hover a $2.7$g Ping-Pong ball, the required topological voltage is $63.8$ kV. Because $63.8 \text{ kV} > 60.0 \text{ kV}$, the spatial vacuum structurally shears apart during the upward power stroke, and the object drops.

\subsection{The Dielectric Death Spiral}
To lower the voltage requirement, one must reduce the payload mass. A $0.01$-gram feather requires only a $236$ V topological grip. However, to actively generate upward lift, a Transient Asymmetric Metric Drive (TAMD) must slowly charge at $236$ V (gripping the solid vacuum), and then violently discharge via an inductive flyback transient exceeding $-60,000$ V to trigger localized superfluid slip and reset the inductor without generating downward recoil.

If you construct a micro-inductor attached to a feather, the copper winding must be insulated to survive a $60,000$ Volt internal transient. Standard magnet wire enamel breaks down at roughly $600$ V. Adding enough high-voltage Kapton tape and potting epoxy to insulate against 60 kV increases the mass of the payload from $0.01$ grams to over $5$ grams, which natively exceeds the 2.538g absolute limit. 

This is the Topological Rocket Equation. Classical copper wire and chemical insulators mathematically cannot scale to vertical 1G levitation.

\section{Project TORSION-05: Horizontal Metric Rectification}
\textbf{The Hypothesis:} We can circumvent the Dielectric Death Spiral by eliminating the 1G vertical payload requirement. By mounting a heavy, heavily-potted TAMD PCBA on a delicately balanced Cavendish Torsion Pendulum suspended inside a hard vacuum chamber, the downward force of gravity is entirely neutralized by the suspension wire. The lateral resistance is effectively $0G$, allowing an EE to measure continuous micro-Newtons of pure metric thrust.

\textbf{The PCBA Implementation \& Falsification:} The EE designs a High-Voltage Flyback PCBA. An ultra-fast Silicon Carbide (SiC) MOSFET drives a heavy ferrite-core ignition coil with a specifically timed asymmetric sawtooth wave. 

\begin{enumerate}
    \item \textbf{The Slow Edge (Solid Grip):} The MOSFET charges the coil slowly. The inductive voltage ($L \frac{di}{dt}$) is $+500$ Volts. Because $500 \text{ V} \ll 60 \text{ kV}$, the vacuum acts as a rigid Cosserat solid. The coil physically grips the spatial lattice, generating an induced forward lateral thrust of exactly $+0.207$ mN.
    \item \textbf{The Fast Edge (Superfluid Slip):} The SiC MOSFET snaps off in $<10$ ns. The inductive kickback violently spikes to $-75,000$ Volts. Because $|-75 \text{ kV}| > 60 \text{ kV}$, the vacuum instantly liquefies. The metric melts into a frictionless superfluid, producing exactly $0.0$ mN of backward reaction force.
\end{enumerate}

If the AVE framework is correct, the torsion balance will slowly but continuously accelerate in a perfectly circular path inside the $10^{-6}$ Torr vacuum chamber, generating a time-averaged DC thrust of roughly $\sim \mathbf{100 \ \mu N}$. If the pendulum remains perfectly stationary, the Bingham-plastic fluid-dynamic geometry of the universe is permanently falsified.

\begin{figure}[htbp]
    \centering
    \includegraphics[width=1.0\textwidth]{chapters/12_experimental_falsification/simulations/outputs/levitation_and_torsion_protocol.png}
    \caption{\textbf{Benchtop Metric Thrust Protocols.} \textbf{Left:} The $60$ kV Bingham Yield limit mathematically restricts static levitation to objects under 2.538 grams. A US penny barely survives; a Ping-Pong ball liquefies the vacuum. \textbf{Center:} The Dielectric Death Spiral. Attempting to add insulation to survive the required 60 kV flyback transient exponentially increases payload mass, mathematically forbidding continuous vertical 1G levitation with classical copper wire. \textbf{Right:} Project TORSION-05. By operating on a horizontal torsion balance ($0G$), a manageable 75 kV snap safely liquefies the vacuum, while the 500 V slow charge generates roughly $100 \ \mu$N of continuous, measurable macroscopic DC thrust.}
    \label{fig:levitation_and_torsion_protocol}
\end{figure}