\section{Existing Experimental Proof: Anomalies as Signatures}

While the RLVE is a prospective test, the AVE framework is already supported by major experimental discrepancies that the Standard Model fails to explain. In AVE, these are not errors; they are the expected mechanical signatures of the discrete substrate.

\subsection{Electro-Optic Metric Compression}

We correct the standard interpretation of the Proton Radius Puzzle. The observed shrinkage ($r_p \to 0.84$ fm) is not gravitational, but Electro-Optic.

The Muon orbits 200x closer than the electron, creating an electric field intensity $E_\mu$ that is $200^2 = 40,000\times$ stronger. This intense field activates the Vacuum Kerr Effect, locally increasing the refractive index $n$ of the space between the muon and proton:
\begin{equation}
    n(r) = n_0 + n_2 E_\mu^2(r)
\end{equation}
The 4\% discrepancy arises directly from the integration of the Kerr index $n(E_\mu)$ over the muon's orbital volume, confirming the dielectric nonlinearity of the substrate. The proton has not shrunk; the ``ruler'' (the vacuum wavelength) has been compressed by the massive muon's inductive wake.

\subsection{The Neutron Lifetime Anomaly: Topological Stability}

Neutrons appear to die 9 seconds faster when trapped in a bottle than when flying in a beam.

As defined in Chapter 4, the Neutron is a metastable ``threaded'' knot ($6^3_2 \cup 3_1$). Its decay is a Topological Snap caused by the tunneling of the central thread. In the Bottle Method, the neutrons interact with the containment walls (atomic lattices). In AVE, matter-matter proximity induces Phonon Coupling between the neutron's knot topology and the wall's lattice. This external vibrational noise lowers the tunneling barrier for the threaded electron, statistically accelerating the ``snap'' event. 

\subsection{The Hubble Tension: Lattice Crystallization}

The expansion rate of the universe ($H_0$) appears faster now than predicted by its initial conditions.

This tension is the exact definition of Generative Cosmology (Chapter 8). In the Early Universe (Pre-Geometric Melt), crystallization was thermodynamically limited by the release of Latent Heat (CMB), governing the rate at $\approx 67$ km/s/Mpc. In the Late Universe (Cold Vacuum), crystallization is unconstrained, allowing the Genesis Rate ($R_g$) to settle at its hardware equilibrium of $\approx 73$ km/s/Mpc. The Hubble Tension is simply the cooling curve of the vacuum phase transition.