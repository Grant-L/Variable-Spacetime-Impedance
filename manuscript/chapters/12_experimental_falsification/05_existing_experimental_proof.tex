\section{Existing Experimental Proof: Anomalies as Signatures}
While the Rotational Lattice Viscosity Experiment (RLVE) proposed above is a prospective test, the Applied Vacuum Electrodynamics (AVE) framework is already supported by three major experimental discrepancies that the Standard Model fails to explain. In AVE, these are not errors; they are the expected mechanical signatures of the discrete substrate.

\subsubsection{Electro-Optic Metric Compression}
We correct the standard interpretation of the Proton Radius Puzzle. The observed shrinkage ($r_p \to 0.84$ fm) is not gravitational, but \textbf{Electro-Optic}.

The Muon orbits 200x closer than the electron, creating an electric field intensity $E_\mu$ that is $200^2 = 40,000\times$ stronger. This intense field activates the \textbf{Vacuum Kerr Effect}, locally increasing the refractive index $n$ of the space between the muon and proton:
\begin{equation}
    n(r) = n_0 + n_2 E_\mu^2(r)
\end{equation}
Where $n_2$ is the second-order nonlinear refractive coefficient of the vacuum. The "shrunken" radius is simply the optical path length compression:
\begin{equation}
    r_{observed} = \int_{0}^{r_{physical}} \frac{1}{n(r)} dr < r_{physical}
\end{equation}
The 4\% discrepancy arises directly from the integration of the Kerr index $n(E_\mu)$ over the muon's orbital volume, confirming the dielectric nonlinearity of the substrate.

\textbf{AVE Resolution:} In Vacuum Engineering, the Muon is a higher-order topological knot ($N=5$) with significantly higher Inductive Mass than the Electron ($N=3$).
Because the muon has a smaller orbital radius and higher mass, it exerts immense \textbf{Dielectric Stress} on the vacuum lattice separating it from the proton. According to the Lattice Stress Coefficient ($\sigma > 1$), this local compression increases the refractive index of the intervening space.
The proton has not shrunk; the "ruler" (the vacuum wavelength) has been compressed by the massive muon's inductive wake.

\subsection{The Neutron Lifetime Anomaly: Topological Stability}
\textbf{The Anomaly:} There are two methods to measure how long a neutron lives before decaying ($n \rightarrow p + e^- + \bar{\nu}_e$), and they yield contradictory results.
\begin{itemize}
    \item \textbf{Beam Method:} Counts the decay products (protons) emitted by a beam of neutrons. Result: $\tau_n \approx 888$ s.
    \item \textbf{Bottle Method:} Traps ultracold neutrons in a magnetic or material jar and counts the survivors. Result: $\tau_n \approx 879$ s.
\end{itemize}
Neutrons appear to die \textbf{9 seconds faster} when trapped in a bottle than when flying in a beam.

\textbf{AVE Resolution:} As defined in Chapter 4, the Neutron is a metastable "threaded" knot ($6_2^3 \# 3_1$). Its decay is a \textbf{Topological Snap} caused by the tunneling of the central thread.
In the Bottle Method, the neutrons interact with the containment walls (atomic lattices). In AVE, matter-matter proximity induces \textbf{Phonon Coupling} between the neutron's knot topology and the wall's lattice. This external vibrational noise lowers the tunneling barrier for the threaded electron, statistically accelerating the "snap" event.
The Beam Method measures the "free space" lifetime; the Bottle Method measures the "coupled" lifetime. The discrepancy is a direct measure of the \textbf{Topological Sensitivity} of the neutron to environmental noise.

\subsection{The Hubble Tension: Lattice Crystallization}
\textbf{The Anomaly:} The expansion rate of the universe ($H_0$) depends on when you measure it.
\begin{itemize}
    \item \textbf{Early Universe (CMB):} $H_0 \approx 67.4$ km/s/Mpc (Planck Data).
    \item \textbf{Late Universe (Supernovae):} $H_0 \approx 73.0$ km/s/Mpc (SH0ES/Riess et al.).
\end{itemize}
This $5\sigma$ discrepancy suggests the universe is expanding faster now than predicted by its initial conditions.

\textbf{AVE Resolution:} This tension is the definition of \textbf{Generative Cosmology} (Chapter 8).
\begin{enumerate}
    \item The "Expansion" is actually \textbf{Node Genesis} (Lattice Crystallization).
    \item In the Early Universe (Pre-Geometric Melt), the crystallization was thermodynamically limited by the release of Latent Heat (CMB), governing the rate at $67$ km/s/Mpc.
    \item In the Late Universe (Cold Vacuum), the crystallization is unconstrained, allowing the Genesis Rate ($R_g$) to settle at its hardware equilibrium of $\approx 73$ km/s/Mpc.
\end{enumerate}
The Hubble Tension is not a crisis; it is the cooling curve of the vacuum phase transition.