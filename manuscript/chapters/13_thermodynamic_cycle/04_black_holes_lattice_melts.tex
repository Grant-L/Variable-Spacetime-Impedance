\section{State 3: Black Holes and the Death of the Rubber Sheet}
\label{sec:black_holes_death_of_rubber_sheet}

For over a century, General Relativity has illustrated gravitation via the ``Rubber Sheet'' metaphor: a massive object rests on a continuous geometric fabric, curving it into a deep funnel. In the extreme case of a Black Hole, the mathematics dictate that this sheet stretches infinitely downward to a singular point of infinite density.

A mathematical singularity of infinite density signals the absolute breakdown of a physical theory. In engineering, no material stretches infinitely; every physical substrate possesses an ultimate tensile strength. The DCVE framework applies rigorous material science directly to the fabric of reality.

\subsection{The Dielectric Snap}
In DCVE, the ``rubber sheet'' is not a continuous geometry; it is the discrete, triangulated $\mathcal{M}_A$ lattice. As matter aggregates, the inductive and capacitive strain on the local nodes increases, pulling them closer together and manifesting as gravity (Tensor Refraction). However, the discrete edges cannot stretch to infinity.

As established in Chapter 1, the hardware is strictly bounded by the \textbf{Schwinger Yield Energy Density} ($u_{sat} \approx 10^{25} \text{ J/m}^3$). As we approach the Event Horizon of a black hole, the tensor strain on the discrete edges reaches this absolute thermodynamic limit.

At the exact radius of the Event Horizon, the rubber sheet physically snaps. 

The compressive stress shatters the Delaunay triangulation of the graph. The discrete nodes undergo a sudden thermodynamic phase transition, melting back into the unstructured Pre-Geometric continuous fluid. There is no infinite funnel; there is a flat thermodynamic plasma floor.

\subsection{Resolution of the Information Paradox}
The phase transition from an organized graph to an unstructured melt provides the mechanical resolution to the Black Hole Information Paradox.

In DCVE, fermions and baryons are stable topological knots tied out of the discrete lattice edges. Because the melted interior of the event horizon lacks a discrete graphical structure, it physically cannot support phase transport or topological defects. When knotted matter crosses the Event Horizon, the underlying lattice supporting the knot ceases to exist.

The knot is not crushed into a singularity; it is instantly unraveled. The energy of the knot is perfectly conserved and added to the heat of the melt, but the geometric information (the topology) is physically erased. The paradox is resolved because the structural canvas upon which the quantum information was encoded is thermodynamically destroyed. Black holes are cosmic recycling vats, melting exhausted discrete space back into the quantum continuum.