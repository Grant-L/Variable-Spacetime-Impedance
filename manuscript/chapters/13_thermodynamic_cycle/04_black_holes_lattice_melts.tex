\section{State 3: Black Holes and the Death of the Rubber Sheet}
\label{sec:black_holes_melts}

For over a century, General Relativity has illustrated gravitation via the ``Rubber Sheet'' metaphor: a massive object rests on a continuous, infinitely stretchable geometric fabric, curving it into a deep funnel. In the extreme case of a Black Hole, the mathematics dictate that this sheet stretches infinitely downward to a singular point of infinite density---a \textbf{Singularity}.

A mathematical singularity of infinite density and infinite depth signals the absolute breakdown of a physical theory. In engineering, \textit{no material stretches infinitely}. Every physical substrate possesses an ultimate tensile strength. The Applied Vacuum Electrodynamics (AVE) framework applies material science directly to the fabric of reality.

\subsection{The Dielectric Snap}
In AVE, the ``rubber sheet'' is not a continuous geometry; it is the discrete, triangulated $M_A$ lattice. As matter aggregates, the Inductive Tension ($\mu_0$) and Capacitive Strain ($\epsilon_0$) of the local nodes increase, pulling them closer together and manifesting as gravity (Tensor Refraction). However, the lattice cannot stretch to infinity. 

As derived in Axiom VI, the hardware is strictly bounded by the \textbf{Vacuum Breakdown Voltage} ($V_{break} \approx 1.04 \times 10^{27}$ V). As we approach the Event Horizon of a black hole, the tensor strain on the discrete edges reaches this absolute hardware limit. 

\textbf{At the exact radius of the Event Horizon, the rubber sheet snaps.} 

The compressive stress shatters the Delaunay triangulation of the graph. The discrete nodes undergo a sudden thermodynamic phase transition, \textbf{melting} back into the unstructured Pre-Geometric continuous fluid. There is no infinite funnel; there is a flat thermodynamic floor.

\begin{figure}[ht]
\centering
\includegraphics[width=1.0\textwidth]{chapters/13_thermodynamic_cycle/simulations/rubber_sheet_melt.png}
\caption{\textbf{AVE Simulation: The Death of the Rubber Sheet.} Unlike the continuous rubber sheet of General Relativity, the discrete AVE manifold ($M_A$) physically yields when tensor strain exceeds $V_{break}$. The intact Delaunay wireframe (Gravity) abruptly severs at the Event Horizon (Red Ring), transitioning into the chaotic, un-triangulated plasma of the Pre-Geometric Melt (Magma floor). A topological knot of matter (Cyan Trefoil) is shown approaching the boundary where it will inevitably untie, bypassing the singularity completely.}
\label{fig:rubber_sheet_melt}
\end{figure}

\subsection{Resolution of the Information Paradox}
The visual transition from an organized graph to an unstructured melt provides the mechanical resolution to the Black Hole Information Paradox. In standard quantum mechanics, information cannot be destroyed, leading to paradoxes when matter falls into a singularity and evaporates via Hawking radiation. 

In AVE, fermions and baryons are stable topological knots tied \textit{out of} the discrete lattice edges (e.g., the Trefoil Soliton). Because the melted interior of the event horizon lacks a discrete graphical structure, it physically cannot support phase transport or topological defects. When knotted matter crosses the Event Horizon, the underlying lattice supporting the knot ceases to exist. 

The knot is not crushed into a singularity; it is instantly unraveled. The \textit{energy} of the knot is perfectly conserved and added to the heat of the melt, but the geometric \textit{information} (the topology) is physically erased. The paradox is resolved because the canvas upon which the quantum information was painted is thermodynamically destroyed. Black holes are not infinitely deep trash cans; they are cosmic recycling vats, melting exhausted discrete space back into the quantum continuum to fuel ongoing cosmic genesis.