
\section{The Topo-Kinematic Circuit Identity}
To map continuum mechanics to electrical networks, we rely on the Topological Conversion Constant ($\xi_{topo} \equiv e/l_{node}$), which defines the fundamental dimensional isomorphism between spatial dislocation and electrical charge. 

In standard SI units, electrical charge ($Q$) is the time integral of current ($Q = \int I \, dt$). By substituting our kinematic mapping for current ($I \equiv \xi_{topo} v$), we derive the absolute mechanical identity of charge within the condensate:

\begin{equation}
Q = \int (\xi_{topo} v) \, dt = \xi_{topo} \int v \, dt = \xi_{topo} x
\end{equation}

Electrical charge is physically isomorphic to \textbf{Macroscopic Spatial Displacement} ($x$). We can rigorously verify this through the Work-Energy Theorem. The physical work done to charge a capacitor is $W = \int V \, dQ$. By substituting our topological identities for Voltage ($V \equiv \xi_{topo}^{-1} F$) and Charge ($dQ \equiv \xi_{topo} \, dx$):

\begin{equation}
W = \int (\xi_{topo}^{-1} F)(\xi_{topo} \, dx) = \int F \, dx
\end{equation}

The scaling constants flawlessly cancel. A capacitor storing electrical charge is mathematically identical to a mechanical lattice storing localized elastic spatial strain. Dielectric breakdown occurs precisely when the continuous spatial lattice is dynamically displaced beyond its absolute physical yield limit.