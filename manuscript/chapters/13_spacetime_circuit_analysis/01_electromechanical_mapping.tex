\section{Constitutive Circuit Models for Vacuum Non-Linearities}

Standard circuit simulators rely on ideal linear RLC components. However, physical materials and fluid dynamics exhibit highly non-linear behaviors under extreme mechanical stress. By rigorously mapping mechanical properties to electrical properties via the Topo-Kinematic Identity ($V \equiv \xi_{topo}^{-1} F$), we can mathematically construct the exact non-linear components of the universe.

\subsection{1. The Metric Varactor (Modeling Dielectric Yield)}
As discussed in Chapter 3 regarding dielectric saturation, the vacuum lattice exhibits an absolute upper bound on allowable topological stress ($V_{crit} \equiv \alpha$). As the local topological potential approaches this limit, the effective compliance (capacitance) increases non-linearly. This models the threshold for spontaneous pair production and behaves exactly as an idealized \textbf{Voltage-Dependent Varactor Diode}:
\begin{equation}
    C_{vac}(V) = \frac{C_0}{\sqrt{1 - (V/V_{crit})^4}}
\end{equation}

\subsection{2. The Vacuum TVS Zener Diode (Voltage-Driven Yield)}
A critical correction to classical continuum models is the proper orientation of causal fluid dynamics. In a Bingham Plastic fluid (Chapter 9), viscosity does not yield due to high velocity (Current); it physically yields strictly when subjected to extreme \textit{Shear Stress} ($\tau > \tau_{yield}$). Because macroscopic shear stress is strictly proportional to mechanical Force, the Topo-Kinematic mapping dictates that vacuum liquefaction must be a \textbf{Voltage-Driven Breakdown}.

Therefore, the vacuum substrate acts electrically as a \textbf{Transient Voltage Suppression (TVS) Zener Diode}. 
\begin{itemize}
    \item \textbf{Solid Regime ($|V| < V_{yield}$):} The lattice bonds hold. The vacuum acts as a highly resistive solid ($R_{eff} = R_{solid}$), kinematically gripping accelerating matter.
    \item \textbf{Fluid Regime ($|V| \ge V_{yield}$):} The topological voltage breaks the lattice. The diode enters avalanche breakdown. The local resistance collapses ($R_{eff} \to R_{fluid}$), allowing frictionless superfluid slip.
\end{itemize}

\subsection{3. The Relativistic Inductor (Lorentz Saturation)}
Because inertia maps to inductance and velocity to current, Special Relativity is identically modeled in SCA as a non-linear inductor that saturates as current approaches the fundamental wave propagation limit ($I_{max} = \xi_{topo} c$):
\begin{equation}
    L_{vac}(I) = \frac{L_0}{\sqrt{1 - (I / I_{max})^2}}
\end{equation}
This mathematically enforces the universal speed limit natively in software, proving exactly why SPICE simulators physically cannot push vacuum current (matter) past $c$.

\subsection{4. The Vacuum Memristor (Thixotropic Hysteresis)}
In 1971, Leon Chua mathematically postulated the existence of a fourth fundamental circuit element: the \textbf{Memristor} (Memory Resistor), an element whose instantaneous resistance depends intrinsically on the historical integral of the current or voltage passing through it.

Because the $\mathcal{M}_A$ vacuum is a Bingham Plastic, it possesses \textit{thixotropy}---a finite geometric relaxation time ($\tau_{macro} \approx L/c$) required for the structural lattice edges to physically break down and liquefy. The vacuum cannot change its fluidic resistance instantaneously. Its current state of viscosity is rigidly dependent on the history of the stress applied to it over the preceding $\tau$ window.

Therefore, the physical vacuum substrate is formally and mathematically defined as a \textbf{Macroscopic Memristor}. 
If we drive a localized vacuum gap with a continuous sinusoidal AC topological voltage, the finite relaxation time delays the onset of superfluidity. As simulated in the AVE-SPICE solver, mapping the topological Voltage against the Kinematic Current produces the universally recognized fingerprint of memristance: \textbf{The Pinched Hysteresis Loop} (see Figure \ref{fig:memristor_and_skineffect}). This perfectly completes the physical spacetime realization of all four fundamental electronic circuit elements (R, L, C, M).

\subsection{The Charge-Displacement Identity}
While the mapping of Voltage and Current is vital, we must also derive the absolute mechanical identity of Electrical Charge ($Q$). In SI units, charge is the time integral of current ($Q = \int I dt$). By substituting our Topo-Kinematic Identity ($I \equiv \xi_{topo} v$):
\begin{equation}
    Q = \int (\xi_{topo} v) dt = \xi_{topo} \int v dt = \mathbf{\xi_{topo} x}
\end{equation}

Electrical Charge is physically and mathematically identical to \textbf{Macroscopic Spatial Displacement} ($x$). We can rigorously prove this using the Work-Energy Theorem. The work done to charge a capacitor is $W = \int V dQ$. Substituting our identities ($V \equiv \xi_{topo}^{-1} F$ and $dQ \equiv \xi_{topo} dx$):
\begin{equation}
    W = \int (\xi_{topo}^{-1} F) (\xi_{topo} dx) = \mathbf{\int F dx}
\end{equation}
The topological constants flawlessly cancel. A capacitor storing electrical charge is literally the mechanical lattice storing spatial displacement (elastic strain). Dielectric breakdown (Axiom 4) occurs precisely when the spatial lattice is displaced beyond its absolute physical elastic limit.