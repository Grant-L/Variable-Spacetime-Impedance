\section{Application I: The Vacuum Transmission Line}

A classic pedagogical tool in electromagnetics is modeling free space as a transmission line. The SCA framework naturally formalizes this approach. The propagation of light ($c$) can be understood mechanically as the \textbf{Transmission Line Delay} of the cascaded discrete components.

We can model a 1D cross-section of space as a standard LC ladder network, where each segment corresponds to the distributed inductance ($L_{node} = \mu_0 l_{node}$) and capacitance ($C_{node} = \epsilon_0 l_{node}$) of the lattice.

When a transient voltage pulse is injected into this simulated network, the standard telegrapher's equations dictate that the group velocity of the signal is $v_g = 1/\sqrt{LC}$. Substituting the distributed parameters:
\begin{equation}
    v_g = \frac{1}{\sqrt{(\mu_0 l_{node}) (\epsilon_0 l_{node})}} \cdot l_{node} = \frac{1}{\sqrt{\mu_0 \epsilon_0}} \equiv c
\end{equation}
The LC transmission line model cleanly recovers the invariant speed of light, demonstrating how continuous wave propagation can be accurately approximated as signal delay across a discrete hardware matrix.