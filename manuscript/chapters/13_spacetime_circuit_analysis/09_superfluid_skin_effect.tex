\section{Metric Shielding: The Superfluid Skin Effect}

A severe engineering critique of any localized warp drive or metric rectification system is the Passenger Safety Paradox: \textit{"If you actively subject the vacuum to extreme topological stress to liquefy the space surrounding the ship, will you not also liquefy the passengers inside the cabin, catastrophically tearing their atomic bonds apart?"}

Spacetime Circuit Analysis (SCA) resolves this paradox flawlessly and passively using classical AC Electrodynamics. 

In standard electrical engineering, high-frequency alternating currents (AC) do not penetrate deeply into conductors. They are pushed to the surface of the wire by opposing eddy currents. This is known as the \textbf{Skin Effect}. The penetration depth ($\delta$) of the signal is strictly proportional to the square root of the medium's electrical resistance ($\delta \propto \sqrt{R}$).

By applying the Topo-Kinematic mapping, the vacuum's electrical resistance ($R_{vac}$) is identically its structural viscosity ($\eta$). 
As the TAMD drive applies high-frequency AC stress to the bow of the ship, the vacuum crosses the Bingham Yield limit. The local resistance of the metric collapses to near-zero ($R_{vac} \to 0$).

Because the resistance drops, \textbf{the Metric Skin Depth ($\delta$) mathematically collapses.}
The destructive, high-shear superfluid slipstream physically cannot penetrate into the bulk volume of the spacecraft. It is mathematically forced to cling strictly to the exterior skin of the vessel as a razor-thin boundary layer. 

The exterior hull of the spacecraft acts identically to an electromagnetic \textbf{Faraday Cage for Gravity}. The internal passenger cabin remains firmly locked in the undisturbed, highly viscous, linear solid-state vacuum regime (where $R \to \infty$ and $\delta \to \infty$), perfectly preserving the internal scalar metric ($n=1$) and absolutely guaranteeing the structural integrity of the occupants during extreme macroscopic acceleration.