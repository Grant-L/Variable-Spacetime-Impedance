\section{Real vs. Reactive Power: The Orbital Friction Paradox}

A historical and persistent critique of analog fluidic spacetime models is the "Friction Paradox": \textit{If a planet is physically moving through a dense spatial condensate, why doesn't fluidic drag drain its kinetic energy, causing its orbit to decay over cosmological timescales?} [cite: 2142]

Within the SCA framework, this paradox is resolved flawlessly utilizing classical AC Power Analysis[cite: 2142]. In electrical engineering, total apparent power ($S$) is divided into two distinct components depending on the phase angle ($\theta$) between Voltage ($V$) and Current ($I$)[cite: 2143]:
\begin{enumerate}
\item \textbf{Real Power ($P$):} Measured in Watts[cite: 2144]. $P = VI \cos(\theta)$[cite: 2144]. This represents energy physically dissipated from the system (e.g., heat, mechanical friction)[cite: 2144].
\item \textbf{Reactive Power ($Q$):} Measured in Volt-Amperes Reactive (VARs)[cite: 2145]. $Q = VI \sin(\theta)$[cite: 2145]. This represents energy conservatively exchanged back and forth without permanent dissipation[cite: 2146].
\end{enumerate}

By applying the Topo-Kinematic Identity, the continuous Gravitational Force vector acts identically as the AC Voltage ($V_{condensate} \propto F_g$), and the Orbital Velocity vector acts as the AC Current ($I_{condensate} \propto v_{orb}$)[cite: 2147]. In a stable, circular planetary orbit, the radial gravitational force vector is perfectly and mathematically orthogonal ($90^\circ$) to the tangential velocity vector[cite: 2148]. Therefore, the phase angle between the vacuum Voltage and Current is exactly $\theta = 90^\circ$[cite: 2149]. Evaluating the Real Power physically dissipated by the planetary body into the vacuum fluid yields[cite: 2150]:
\begin{equation}
P_{real} = F \cdot v \cdot \cos(90^\circ) \equiv 0\ \text{Watts}
\end{equation}

The orbiting body experiences absolutely zero macroscopic fluidic friction[cite: 2150]. A stable planetary orbit is the macroscopic mechanical equivalent of a \textbf{Lossless LC Tank Circuit} operating purely in the reactive power domain (VARs), continuously conserving its stored energy without thermodynamically heating the ambient vacuum fluid[cite: 2151].

\begin{figure}[htbp]
    \centering
    \includegraphics[width=\textwidth]{chapters/13_spacetime_circuit_analysis/simulations/outputs/orbital_reactive_power.png}
\caption{\textbf{Orbital Mechanics as Reactive AC Power.} Because the topological voltage (gravitational force) is perfectly 90-degrees out of phase with the spatial current (orbital velocity), the Real Power (Watts) dissipated by the planetary body evaluates identically to zero[cite: 2152]. The orbit operates as a pure LC reactive circuit, elegantly resolving the classical fluid friction paradox in condensed-matter models of the vacuum[cite: 2153].}
    \label{fig:orbital_power}
\end{figure}