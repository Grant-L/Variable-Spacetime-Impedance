\section{Condensate IMD Spectroscopy: The Harmonic Fingerprint}
By modeling the universe as a non-linear network, we can extract the exact theoretical signature of the AVE framework using standard RF analysis techniques. In electrical engineering, when a non-linear component (such as a varactor diode) is excited simultaneously by two pure frequencies ($f_1$ and $f_2$), the non-linearity acts as an RF mixer, generating highly predictable harmonic sidebands known as Intermodulation Distortion (IMD).

Standard physical materials (e.g., piezoelectric crystals or optical glass) typically possess 2nd-order or 3rd-order non-linearities, generating standard sidebands at $2f_1 \pm f_2$. However, as rigorously demanded by the EFT boundary conditions, the effective capacitance of the discrete vacuum lattice is governed exactly by a 4th-order geometric polynomial bound ($1 - (V/V_{crit})^4$). 

This absolute 4th-order constraint makes the physical condensate a highly unique RF mixer. If a massive, dual-tone mechanical stress field is injected into the vacuum using opposed acoustic transducers, the non-linear integration of the $V^4$ vacuum varactor mathematically forces the generation of distinct \textbf{5th-Order Intermodulation Products} (such as $3f_1 - 2f_2$) in the localized metric strain field. 

Because these specific harmonic signatures are mathematically suppressed in linear QED and standard continuous General Relativity, the detection of this exact EFT Harmonic Fingerprint provides an absolute, unique empirical signature completely isolated from standard material noise.

\begin{figure}[htbp]
    \centering
    \includegraphics[width=\textwidth]{chapters/13_spacetime_circuit_analysis/simulations/outputs/condensate_imd_spectroscopy.png}
    \caption{\textbf{Condensate IMD Spectroscopy: The 4th-Order Harmonic Fingerprint.} Simulated via the AVE-SPICE ODE solver. By driving the local spatial metric with two pure frequencies ($f_1$, $f_2$), the exact 4th-order non-linear saturation bound of the condensate varactor ($1 - V^4$) mathematically forces the generation of highly specific 5th-order intermodulation sidebands (e.g., $3f_1 - 2f_2$). This provides an absolute, unique harmonic signature completely isolated from standard 3rd-order material noise.}
    \label{fig:imd_spectroscopy}
\end{figure}

\subsection{Time-Domain Wavelength Compression}
To visualize the mechanical reality of this perfect impedance matching, we evaluated the transient wave equation through a localized Gaussian gravity well ($n(x) \propto \rho_{bulk}$).

As the time-domain wave enters the optical density gradient, the localized speed of light mechanically drops ($c_{local} = c/n$). Because the leading edge of the wave slows down before the trailing edge, the physical wavelength compresses dynamically (analogous to the relativistic blue-shift of an infalling photon). Crucially, because $Z_0$ remains perfectly flat throughout the well, the simulation confirms absolutely zero backward-propagating reflections are generated by the intense density gradient.

\begin{figure}[htbp]
    \centering
    \includegraphics[width=\textwidth]{chapters/13_spacetime_circuit_analysis/simulations/outputs/impedance_gravity_well_time_domain.png}
    \caption{\textbf{Time-Domain Wavelength Compression.} A linear spatial snapshot of a high-frequency wave propagating through a gravitational metric. Because analog gravity scales $\mu$ and $\epsilon$ proportionately, $Z_0$ remains invariant. The light wave physically compresses its wavelength inside the well without suffering reflections, perfectly modeling General Relativistic frequency shifts via solid-state optics.}
    \label{fig:time_domain_gravity}
\end{figure}