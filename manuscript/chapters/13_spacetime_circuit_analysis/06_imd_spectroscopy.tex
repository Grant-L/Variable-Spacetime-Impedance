\section{Condensate IMD Spectroscopy: The Harmonic Fingerprint}

By modeling the universe as a non-linear network, we can extract the exact theoretical signature of the AVE framework using standard RF analysis techniques.

\paragraph{The 4th-Order Falsification Test:}
Standard optical materials possess 2nd-order or 3rd-order non-linearities, generating standard intermodulation sidebands ($2f_1 - f_2$). However, Axiom 4 mandates a strict 4th-order geometric saturation limit ($1 - V^4$) for the vacuum condensate.

\begin{equation}
    C_{vac}(V) = \frac{C_0}{\sqrt{1 - (V/V_{crit})^4}}
\end{equation}

\paragraph{Predicted Signal:}
This unique constraint forces the vacuum to act as a quintic RF mixer. Simulations using the AVE-SPICE solver demonstrate that when driven by a dual-tone signal ($f_1, f_2$) at 80\% of the breakdown voltage, the vacuum generates distinct \textbf{5th-Order Intermodulation Products} (specifically $3f_1 - 2f_2$) with a power magnitude of approximately \textbf{-56 dBc}.
The detection of this specific harmonic signature at high field strengths would constitute definitive experimental proof of the AVE hardware, as it is mathematically suppressed in standard linear QED.

\begin{figure}[htbp]
    \centering
    \includegraphics[width=\textwidth]{chapters/13_spacetime_circuit_analysis/simulations/outputs/condensate_imd_spectroscopy.png}
    \caption{\textbf{Condensate IMD Spectroscopy: The 4th-Order Harmonic Fingerprint.} Simulated via the AVE-SPICE ODE solver. By driving the local spatial metric with two pure frequencies ($f_1$, $f_2$), the exact 4th-order non-linear saturation bound of the condensate varactor ($1 - V^4$) mathematically forces the generation of highly specific 5th-order intermodulation sidebands (e.g., $3f_1 - 2f_2$). This provides an absolute, unique harmonic signature completely isolated from standard 3rd-order material noise.}
    \label{fig:imd_spectroscopy}
\end{figure}

\subsection{Time-Domain Wavelength Compression}
To visualize the mechanical reality of this perfect impedance matching, we evaluated the transient wave equation through a localized Gaussian gravity well ($n(x) \propto \rho_{bulk}$).

As the time-domain wave enters the optical density gradient, the localized speed of light mechanically drops ($c_{local} = c/n$). Because the leading edge of the wave slows down before the trailing edge, the physical wavelength compresses dynamically (analogous to the relativistic blue-shift of an infalling photon). Crucially, because $Z_0$ remains perfectly flat throughout the well, the simulation confirms absolutely zero backward-propagating reflections are generated by the intense density gradient.

\begin{figure}[htbp]
    \centering
    \includegraphics[width=\textwidth]{chapters/13_spacetime_circuit_analysis/simulations/outputs/impedance_gravity_well_time_domain.png}
    \caption{\textbf{Time-Domain Wavelength Compression.} A linear spatial snapshot of a high-frequency wave propagating through a gravitational metric. Because analog gravity scales $\mu$ and $\epsilon$ proportionately, $Z_0$ remains invariant. The light wave physically compresses its wavelength inside the well without suffering reflections, perfectly modeling General Relativistic frequency shifts via solid-state optics.}
    \label{fig:time_domain_gravity}
\end{figure}