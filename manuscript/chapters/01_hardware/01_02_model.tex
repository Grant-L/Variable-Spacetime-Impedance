\section{Node Geometry and Topological Helicity}
\label{sec:node_geometry}

Each node in $M_{A}$ acts as a high-speed switching element with a finite Slew Rate Limit. The fundamental unit of interaction and substance within this substrate is the \textbf{Topological Helicity ($h$)}—a quantized, self-reinforcing phase twist in the local flux field.

\subsection{The Chiral Bias Equation (CBE)}
The manifold $M_A$ is not perfectly symmetric; it possesses an intrinsic orientation vector $\mathbf{\Omega}_{vac}$. We define the \textbf{Dynamic Metric Impedance} ($Z_{metric}$) as a function of the signal’s angular momentum vector $\mathbf{J}$ relative to this vacuum orientation.

The impedance of a signal propagating through the manifold is given by the \textbf{Chiral Bias Equation}:

\begin{equation}
    Z_{metric} = Z_{0} \left( 1 + \eta \frac{\mathbf{J} \cdot \mathbf{\Omega}_{vac}}{|\mathbf{J}| |\mathbf{\Omega}_{vac}|} \right)
    \label{eq:cbe}
\end{equation}

Where:
\begin{itemize}
    \item $Z_{0} = \sqrt{L_{node}/C_{node}}$ is the baseline Characteristic Impedance ($\approx 376.73 \Omega$).
    \item $\eta$ is the \textbf{Asymmetry Coefficient}, representing the magnitude of the vacuum's chiral bias.
\end{itemize}

This equation provides the mechanical basis for \textbf{Parity Violation}. Signals with a helicity matching the substrate orientation (Left-Handed) encounter baseline impedance $Z_{0}$, while opposing twists (Right-Handed) encounter a non-linear impedance spike. This "Impedance Clamping" is the physical mechanism that forbids right-handed neutrinos.