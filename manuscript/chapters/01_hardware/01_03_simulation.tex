\section{Simulation: The Amorphous Substrate}
\label{sec:simulation_substrate}

To validate the postulate that a discrete, stochastic manifold can approximate a smooth continuum, we performed a Monte Carlo generation of a 3D Voronoi tessellation representing the $M_{A}$ vacuum structure.

\subsection{Connectivity Analysis}
Unlike a crystalline lattice, where the coordination number (neighbor count) is fixed (e.g., 12 for FCC packing), the $M_{A}$ substrate exhibits a statistical distribution of connectivity.

Running the simulation script \texttt{run\_lattice\_gen.py} with $N=10,000$ nodes yields a mean connectivity of:

\begin{equation}
    \langle k \rangle \approx 15.54 \pm 1.3
\end{equation}

Figure \ref{fig:connectivity} illustrates this distribution. The Gaussian profile confirms that while individual nodes have varying local geometries, the \textbf{bulk average} is highly consistent. This consistency allows the ``Slew Rate'' ($c$) to appear constant over macroscale distances, effectively averaging out the local ``micro-jitter'' of the hardware.

\begin{figure}[h]
    \centering
    \includegraphics[width=0.8\textwidth]{assets/sim_outputs/connectivity_histogram.png}
    \caption{\textbf{Stochastic Node Connectivity.} The distribution of neighbors in the generated Voronoi vacuum. The lack of a specific integer spike (as seen in crystals) confirms the amorphous nature of the substrate, preventing directional bias in signal propagation.}
    \label{fig:connectivity}
\end{figure}

\subsection{Implications for Isotropy}
Standard lattice theories often fail because they predict a ``Manhattan Distance'' effect where light travels faster along the grid axes. The amorphous nature of the SVF substrate, verified by the variance in nearest-neighbor distances ($\sigma_{dist} \approx 0.1 l_P$), destroys these preferred axes. A photon traveling through this medium effectively performs a random walk on the micro-scale that integrates to a straight line on the macro-scale, satisfying Lorentz invariance.