\section{The Lattice Hardware: Micro-Geometry}
\label{sec:lattice_hardware}

To validate the postulate that a discrete, stochastic manifold can approximate a smooth continuum, we formally define the vacuum graph structure.

\subsection{The Discrete Amorphous Manifold ($M_A$)}
We define the physical vacuum as a stochastic graph $G = (V, E)$ embedded in a topological volume.
\begin{definition}[The Voronoi Vacuum]
The manifold $M_A$ is the dual graph of a \textbf{Poisson Point Process} in 3D space.
\begin{itemize}
    \item \textbf{Nodes ($V$):} Stochastic points distributed with mean density $\rho_{node} \approx l_P^{-3}$.
    \item \textbf{Edges ($E$):} The Delaunay Triangulation connecting nearest neighbors.
\end{itemize}
\end{definition}

\subsection{Connectivity Analysis}
Unlike a crystalline lattice, where the coordination number is fixed (e.g., 12 for FCC), the $M_{A}$ substrate exhibits a statistical distribution of connectivity.
Running the simulation ($N=10,000$) yields a mean connectivity of $\langle k \rangle \approx 15.54$.

\begin{figure}[h]
    \centering
    \includegraphics[width=0.85\textwidth]{assets/sim_outputs/connectivity_histogram.png}
    \caption{\textbf{Stochastic Node Connectivity.} The distribution of neighbors in the VSI vacuum follows a Gaussian-like profile. The lack of a specific integer spike (as seen in crystals) confirms the amorphous nature of the substrate.}
    \label{fig:connectivity}
\end{figure}

\subsection{Isotropy and the Graph Laplacian}
A critical requirement for VSI is that this discrete graph must behave like smooth spacetime at macroscopic scales.
\begin{theorem}[Isotropic Averaging]
For a Delaunay graph generated from a Poisson distribution, the Graph Laplacian converges to the Laplace-Beltrami operator $\nabla^2$ in the limit of large node count $N$.
\end{theorem}

\begin{figure}[h]
    \centering
    \includegraphics[width=0.65\textwidth]{assets/sim_outputs/lattice_slice.png}
    \caption{\textbf{The Amorphous Graph (2D Slice).} A cross-section of the generated hardware. The randomized triangulation ensures that a photon performs a random walk on the micro-scale that integrates to a straight line on the macro-scale, preventing "Grid Artifacts" and preserving Lorentz Invariance.}
    \label{fig:isotropy}
\end{figure}

**Physical Implication:** The randomness destroys the "Manhattan Distance" effect of regular grids. Light travels at the same average speed in every direction, satisfying Lorentz Invariance without requiring a continuum.