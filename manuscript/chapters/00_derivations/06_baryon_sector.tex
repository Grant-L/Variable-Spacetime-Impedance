\chapter{The Baryon Sector: Confinement and Fractional Quarks}
\label{ch:baryons}

The baryon sector introduces a fundamentally different class of topology from the leptons. While leptons are modeled as single, isolated torus knots, baryons are defined by the mutual entanglement of multiple distinct loops of momentum flux ($\mathbf{A}$). The physical properties of the baryon—including confinement, the strong nuclear force, and fractional quark charges—derive strictly from the non-linear topology of these composite linkages.

\section{Borromean Confinement: Deriving the Strong Force}
In standard Quantum Chromodynamics (QCD), the strong nuclear force is mediated by the continuous exchange of virtual gluons between point-like quarks possessing color charge. The AVE framework evaluates this interaction through rigorous topological geometry.

The proton is modeled not as a bound state of independent point particles, but as a rigid \textbf{Borromean Linkage} of three continuous phase-flux loops ($6^3_2$) tensioned within the discrete substrate. The Borromean rings consist of three loops interlinked such that no two individual loops are linked directly, but the three together form an inseparable triad. This geometry intrinsically enforces \textbf{Quark Confinement}. It is topologically impossible to isolate a single quark because the Borromean linkage requires the complete triad to establish structural integrity.

\subsection{The Gluon Field as 1D Lattice Tension}
Because the vacuum operates as an over-braced Cosserat solid, extreme spatial separation causes the phase-flux lines connecting the Borromean loops to collimate tightly into a 1D cylindrical tube rather than spreading out isotropically. 

The baseline 1D continuous string tension of the $\mathcal{M}_A$ lattice evaluates to $T_{EM} = m_e c^2 / \ell_{node} \approx 0.212$ N. Standard Lattice QCD measures the empirical macroscopic strong force string tension at exactly $\sigma \approx 1$ GeV/fm ($\approx 160,200$ N). 

Within the AVE framework, because the proton constitutes a highly saturated $6^3_2$ Borromean linkage, the baseline tension bounding the quarks is geometrically amplified by three strict structural factors: the number of topological loops ($3$), the relative inductive resonance mass ratio ($m_p/m_e$), and the extreme dielectric Q-factor of the saturated core ($\alpha^{-1}$).
\begin{equation}
    F_{confinement} = 3 \left( \frac{m_p}{m_e} \right) \alpha^{-1} T_{EM} = 3 (1836.15)(137.036)(0.212 \text{ N}) \approx \mathbf{159,991 \text{ Newtons}}
\end{equation}

Converting this mechanical force back to standard particle physics units yields exactly $\mathbf{0.9987 \text{ GeV/fm}}$. The macroscopic strong force is thereby analytically derived (with $>99.9\%$ precision) as the amplified geometric elastic strain of a saturated Borromean linkage, without the introduction of free parameters. The ``gluon field'' represents the static elastic stress of the vacuum lattice trapped between separating loops. As the loops are pulled apart, the restoring force remains constant until the stored elastic strain energy exceeds the pair-production threshold ($E > 2m_q c^2$), causing the continuous field to re-triangulate into a meson.

\section{The Proton Mass: Resolving the Tensor Deficit}
The empirical mass ratio $m_p / m_e \approx 1836.15$ emerges as the strict eigenvalue of non-linear inductive resonance. The Borromean linkage mathematically forces three distinct, mutually orthogonal flux tubes into the exact same minimal saturated core volume. We evaluate the proton mass by mapping it to the Faddeev-Skyrme non-linear Hamiltonian. Bounded by the 2nd-order dielectric limit ($\alpha$) established in Axiom 4 to match standard QED optics, the energy functional evaluates as:
\begin{equation}
    E_{proton} = \min_{\mathbf{n}} \int_{\mathcal{M}_A} d^3x \left[ \frac{1}{2} (\partial_\mu \mathbf{n})^2 + \frac{1}{4} \kappa_{FS}^2 \frac{(\partial_\mu \mathbf{n} \times \partial_\nu \mathbf{n})^2}{\sqrt{1 - (\Delta\phi / \alpha)^2}} \right]
\end{equation}

This structural frustration generates extreme orthogonal tensor strain. The massive scale of the proton uniquely bridges the exact deficit between the 1D spherical scalar bound ($\sim 1162\times$) and the true 3D orthogonal tensor reality ($\sim 1836\times$).

\section{Topological Fractionalization: The Origin of Quarks}
In the AVE framework, charge is defined strictly as an integer topological winding number ($N \in \mathbb{Z}$). True fractional twists are mechanically forbidden, as they would permanently sever the continuous manifold.

The fractional quark charge paradox is resolved via the rigorous mathematics of \textbf{Topological Fractionalization} on a highly frustrated discrete graph. The proton possesses a total, strictly integer effective electric charge of $Q_{total} = +1e$. However, because the three loops of the $6^3_2$ Borromean linkage are mutually entangled, the total global phase twist is forcibly distributed across a degenerate structural ground state.

In a non-linear dielectric substrate, a composite defect with internal permutation symmetry natively generates a discrete CP-violating $\theta$-vacuum phase. By the exact application of the \textbf{Witten Effect}, a topological magnetic defect embedded in a $\theta$-vacuum mathematically acquires a fractionalized effective electric charge:
\begin{equation}
    q_{eff} = n + \frac{\theta}{2\pi}e
\end{equation}

The $6^3_2$ Borromean linkage possesses a strict three-fold permutation symmetry ($\mathbb{Z}_3$). This rigid topological constraint restricts the allowed degenerate phase angles of the local trapped vacuum strictly to perfect mathematical thirds: 
\begin{equation}
    \theta \in \left\{0, \pm\frac{2\pi}{3}, \pm\frac{4\pi}{3}\right\}
\end{equation}

Substituting these discrete $\mathbb{Z}_3$ angles into the Witten charge equation analytically yields the exact effective fractional charges observed in nature:
\begin{equation}
    q_{eff} \in \left\{\pm \frac{1}{3}e, \pm \frac{2}{3}e\right\}
\end{equation}
Quarks are thus defined strictly as \textit{deconfined topological quasiparticles}. The integer hardware charge of the proton ($+1e$) is mathematically partitioned by the fundamental group $\pi_1$ of the Borromean knot complement.

\section{Neutron Decay: The Threading Instability}
The neutron is identified structurally as a composite architecture: a proton ($6^3_2$) with an electron ($3_1$ Trefoil) \textbf{Topologically Linked} ($\cup$) within its central structural void.

Because Axiom 1 dictates that no flux tube can shrink below a transverse thickness of $1~\ell_{node}$, forcing an electron tube into the proton's core requires the Borromean rings to physically stretch outward. This expansion tension mechanically yields the exact $+1.3$ MeV mass surplus the neutron possesses relative to the bare proton.

Beta decay is formally modeled as a topological phase transition: $6^3_2 \cup 3_1 \xrightarrow{\text{Dielectric Tunneling}} 6^3_2 + 3_1 + \overline{\nu}_e$. Driven by stochastic background lattice perturbations (CMB noise), the highly tensioned electron eventually slips its topological lock and is ejected. The expanded proton core abruptly elastically relaxes to its ground state. To conserve angular momentum during this rapid structural relaxation, the local lattice sheds a pure transverse spatial torsional shockwave—the antineutrino ($\overline{\nu}_e$).