\section{Deriving the Gravitational Coupling ($G$)}

To derive gravity, we must define the breaking point of the $M_A$ lattice strictly from its nodal parameters ($l_{node}, C_{node}, L_{node}$).

\subsection{The Lattice Tension Limit ($T_{max}$)}
Using the Geometrodynamic Ansatz ($1F \equiv 1 m/N$), the Vacuum Capacitance $C_{node}$ represents the manifold's compliance. The maximum tension the lattice can sustain is the ratio of its characteristic length to its compliance:
\begin{equation}
T_{max} \equiv \frac{l_{node}}{C_{node}}
\end{equation}

\subsection{The Geometric Emergence of G (First-Principles Derivation)}
Instead of assuming General Relativity's field equations apriori or arbitrarily defining $G$, we can derive the Gravitational Coupling strictly by evaluating Newton's classical law of gravitation ($G = F \cdot r^2 / M^2$) at the fundamental geometric limit of two adjacent $M_A$ nodes.

Consider two nodes fully saturated into localized trapped-flux masses. The absolute geometric limits of their interaction are dictated by the manifold's inductive and capacitive parameters:
\begin{enumerate}
    \item \textbf{Minimum Separation ($r$):} The absolute minimum discrete distance between their centers is exactly one characteristic nodal length ($l_{node}$).
    \item \textbf{Maximum Mass ($M$):} By the Geometrodynamic Ansatz ($1\text{ C} \equiv 1\text{ m}$), the effective mass of a fully saturated node in the gravimetric domain is identically its localized inductance ($L_g$).
    \item \textbf{Maximum Force ($F$):} The absolute maximum gravitational pull they can exert on each other before the manifold suffers topological failure is the fundamental tension limit ($T_{max} = l_{node}/C_g$).
\end{enumerate}

Substituting these strict $LC$ network primitives into Newton's classical formulation unspools the precise underlying lattice mechanics:
\begin{equation}
G = \frac{F_{max} \cdot r_{min}^2}{M_{max}^2} = \frac{(l_{node}/C_g) \cdot (l_{node})^2}{(L_g)^2} = \frac{l_{node}^3}{L_g^2 C_g}
\end{equation}

By subsequently substituting our rigorously derived definitions for the invariant wave speed ($c = l_{node}/\sqrt{L_g C_g}$) and Lattice Tension ($T_{max} = l_{node}/C_g$) into this resultant expression, we find:
\begin{equation}
G = \frac{l_{node}^3}{L_g^2 C_g} = \left( \frac{l_{node}^4}{L_g^2 C_g^2} \right) \cdot \left( \frac{C_g}{l_{node}} \right) = \frac{c^4}{T_{max}}
\end{equation}
This mathematically proves that Newton's $G$ and Einstein's $c^4/F_P$ are not empirical continuous primitives, but exact geometric composites of the $M_A$ lattice's discrete $LC$ network acting at its fundamental failure limit.