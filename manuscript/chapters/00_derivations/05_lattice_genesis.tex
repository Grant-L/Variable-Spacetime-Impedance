\section{The Thermodynamics of Lattice Genesis}

\subsection{Stable Phantom Dark Energy and the Big Rip Resolution}
During lattice genesis, the mechanical pressure required to supply both the internal energy of newly created vacuum volume ($dU_{vac} = \rho_{vac} dV$) and the exothermic latent heat released into the universe ($dQ_{latent} = \rho_{latent} dV$) leads to a clear thermodynamic balance:
\begin{equation}
w_{vac} = \frac{P_{tot}}{\rho_{vac}} = \frac{-(\rho_{vac} + \rho_{latent})}{\rho_{vac}} = -1 - \frac{\rho_{latent}}{\rho_{vac}} < -1
\end{equation}

In standard cosmology, phantom energy ($w < -1$) leads to a runaway Big Rip. In AVE, the vacuum density ($\rho_{vac}$) is geometrically fixed by the hardware packing fraction ($\kappa_V = 8\pi\alpha$), so the lattice has no structural capacity to store excess internal work. The excess is fully ejected as latent heat ($\rho_{latent}$), naturally preventing a Big Rip.

Because the CMB follows an adiabatic expansion cooling curve ($T \propto 1+z$), today's radiation density is dominated by primordial heat from the Big Bang. In the asymptotic limit ($a \to \infty$), adiabatic cooling approaches absolute zero, and the constant latent heat establishes a permanent asymptotic thermal attractor floor ($u_{rad, \infty} \to \frac{3}{4} \rho_{latent}$).

Given that the universe is cooling toward absolute zero today, this latent heat floor is bounded by the Unruh-Hawking horizon temperature of the expanding causal boundary ($T_U = \hbar H_0 / 2\pi k_B \approx 10^{-30}$ K). Thus, $\rho_{latent}$ is physically infinitesimal.

By using the known current transient photon density ($\Omega_{rad, today} \approx 5.38 \times 10^{-5}$) as an upper bound, we obtain a hard analytical limit on the Dark Energy equation of state:
\begin{equation}
w_{vac} = -1 - \frac{4 u_{rad, \infty}}{3 \rho_{vac}} > -1 - \frac{4 \Omega_{rad, today}}{3 \Omega_\Lambda}
\end{equation}
\begin{equation}
w_{vac} > -1 - \frac{4 (5.38 \times 10^{-5})}{3 (0.68)} \approx \mathbf{-1.0001}
\end{equation}
AVE offers an analytical argument that Dark Energy is bounded phantom energy ($-1.0001 < w_{vac} < -1$). This is consistent with recent DESI 2024 measurements ($w = -1.04 \pm 0.09$) but places a strict upper limit below the current central value. The framework provides a falsifiable prediction that future high-precision geometric surveys will find $w$ bounded just beneath $-1.000000$.