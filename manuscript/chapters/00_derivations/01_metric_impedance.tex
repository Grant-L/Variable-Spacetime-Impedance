% Section: Foundations of Metric Impedance
\section{The Impedance of the Discrete Amorphous Manifold}

To establish a model free of ad-hoc constants, we define the vacuum not as an empty void, but as a Discrete Amorphous Manifold ($M_A$) with inherent electromagnetic and gravimetric properties derived from its characteristic impedance, $Z_0$.

\subsection{Characteristic Moduli}
The manifold is composed of nodes with a characteristic length $l_P$ (Planck length). Each node possesses a discrete Inductance ($L_{node}$) and Capacitance ($C_{node}$), representing the vacuum's resistance to change in flux and its ability to store potential energy, respectively.

The local speed of light $c$ and the characteristic impedance $Z_0$ are defined by:
\begin{equation}
c = \frac{1}{\sqrt{L_{node}C_{node}}} \quad \text{and} \quad Z_0 = \sqrt{\frac{L_{node}}{C_{node}}}
\end{equation}

\subsection{Mass as Trapped Flux (The Saturation Threshold)}
Mass emergence occurs when the local frequency $\nu$ exceeds the node's Saturation Threshold $\nu_{sat}$:
\begin{equation}
\nu_{sat} = \frac{1}{2\pi\sqrt{L_{node}C_{node}}}
\end{equation}
At this threshold, the wave packet "curls," creating a standing wave of trapped flux. This localized energy density performs "Dual-Modulus Loading" on the surrounding $M_A$ lattice, increasing the local $L$ and $C$.

\subsection{Deconstructing G: Gravity as Metric Refraction}
Instead of invoking Newton's constant $G$ as a fundamental primitive, we define the Gravitational Coupling as the sensitivity of the manifold's impedance to local **Vacuum Tension** (Force). The refractive index $\chi(r)$ of spacetime is given by the ratio of the modified impedance to the vacuum impedance:
\begin{equation}
\chi(r) = \frac{Z_{metric}(r)}{Z_0} = 1 + \sigma_Z \cdot F_{strain}
\end{equation}

By equating this to the Schwarzschild solution, we find that the traditional constant $G$ is a macro-scale manifestation of the manifold's "stiffness" gradient:
\begin{equation}
G \equiv c^4 \cdot \sigma_Z
\end{equation}

\subsection{Inertia as Back-EMF}
An object with mass $M$ resists acceleration because any change in velocity requires a change in flux linkage. By defining **Metric Current** ($I_m$) simply as the velocity of the flux packet ($I_m \equiv v$), the force is exactly the Back-Electromotive Force:
\begin{equation}
\mathbf{F} = \frac{d(L_{eff} I_m)}{dt} = L_{eff} \cdot \frac{dv}{dt} = M \mathbf{a}
\end{equation}
Here, Mass ($M$) acts strictly as the inductive coefficient ($L_{eff}$) of the $M_A$ lattice, and acceleration is the time-derivative of the current.