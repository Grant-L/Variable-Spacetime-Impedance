\section{The Impedance of the Discrete Amorphous Manifold}

\subsection{The Geometrodynamic Isomorphism (Ohm's Equivalence)}
To mathematically bridge electrical and mechanical phenomena without ad-hoc assumptions, we formally adopt the \textbf{Geometrodynamic Ansatz}:
\begin{quote}
\textit{In the $M_A$ topology, electric charge $q$ is geometrically equivalent to spatial displacement $x$ ($1 \text{ Coulomb} \equiv 1 \text{ Meter}$).}
\end{quote}
Under this topological mapping, electrical Impedance (Ohms) rigorously reduces to exact SI Mechanical Impedance ($kg/s$):
\begin{equation}
1 \, \Omega = 1 \frac{\text{V}}{\text{A}} = 1 \frac{\text{J/C}}{\text{C/s}} = 1 \frac{\text{J}\cdot\text{s}}{\text{C}^2} \xrightarrow{1\text{ C} \equiv 1\text{ m}} 1 \frac{\text{J}\cdot\text{s}}{\text{m}^2} = 1 \frac{\text{N}\cdot\text{m}\cdot\text{s}}{\text{m}^2} = 1 \frac{\text{N}}{\text{m/s}} = 1 \text{ kg/s}
\end{equation}
This dimensional proof demonstrates that the vacuum's characteristic electrical resistance and its mechanical inertial drag are identically the same physical phenomenon.

\subsection{The Dual-Impedance Hierarchy (Derivation)}
The $M_A$ lattice supports two distinct impedance domains: Electromagnetic ($Z_{EM}$) and Gravimetric ($Z_g$). However, both domains exist on the same lattice and must propagate information at the invariant speed of light $c$.

We define the Hierarchy Coupling $\xi$ as the dimensionless topological multiplier between the gravimetric and electromagnetic impedances. To satisfy the wave-speed constraint:
\begin{equation}
c = \frac{l_{node}}{\sqrt{L_{EM}C_{EM}}} = \frac{l_{node}}{\sqrt{L_{g}C_{g}}}
\end{equation}
Given that $Z_g = \xi Z_{EM}$, we solve the system of equations to derive the exact topological scaling of the nodal parameters:
\begin{equation}
L_g = \xi \cdot L_{EM} \quad \text{and} \quad C_g = \frac{C_{EM}}{\xi}
\end{equation}
This derivation proves that to support a higher impedance (stiffness) while maintaining constant velocity, the vacuum's inductive inertia must increase by $\xi$ while its capacitive compliance decreases by $1/\xi$.

\subsection{The Chiral Bias Equation (CBE)}
The local metric impedance is modified by two factors: the local energy density ($\rho_E$) which strains the lattice, and the geometric alignment of the particle's spin ($\mathbf{S}$) with the vacuum vorticity ($\mathbf{\Omega}_{vac}$).

To quantify the structural overlap without recursively double-counting the directional phase or suffering macroscopic divergent scaling, we define the Chiral Coefficient $\alpha$ strictly as a positive scalar representing the \textbf{nodal saturation fraction}—the volume of the topological defect's core ($V_{core}$) exclusively contained within a single nodal volume ($V_{node}$):
\begin{equation}
    \alpha_c = \frac{V_{core}}{V_{node}} < 1
\end{equation}

To maintain strict dimensional homogeneity, the continuous energy density field ($\rho_E$, with units of pressure $[N/m^2]$) must be converted into a localized discrete force by multiplying it by the nodal cross-sectional area ($l_{node}^2$). This perfectly cancels the metric compliance ($[N^{-1}]$). The total Local Metric Impedance is thus derived without circularity or dimensional collapse:
\begin{equation}
    Z_{metric} = Z_{g} \left( 1 + \sigma_Z (\rho_E \cdot l_{node}^2) + \alpha_c \frac{\mathbf{S} \cdot \mathbf{\Omega}_{vac}}{|\mathbf{S}| |\mathbf{\Omega}_{vac}|} \right)
\end{equation}
Here, $\sigma_Z (\rho_E \cdot l_{node}^2)$ correctly resolves to a strictly dimensionless scalar representing the "stiffness" increase due to localized stress (Gravity), while the single explicit unit-vector dot product exactly handles the geometric projection ($\pm 1$) for spin alignment (Chirality).