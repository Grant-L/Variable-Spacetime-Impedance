\section{The Thermodynamics of Lattice Genesis}
To rigorously derive the Dark Energy equation of state ($w=-1$) and the Cosmic Microwave Background (CMB) without thermodynamic contradictions, we must model the expanding universe strictly as an \textbf{Open Thermodynamic System undergoing a Phase Transition}.

The First Law of Thermodynamics for an open system expanding via the genesis (crystallization) of new lattice nodes is:
\begin{equation}
dU = dQ_{latent} - P dV + \mu dN
\end{equation}
Where $dQ_{latent}$ is the latent heat exchanged, $P dV$ is the mechanical work of expansion, and $\mu dN$ is the chemical work of adding new nodes. 

\textbf{1. The Dark Energy Pressure ($w = -1$):}
Because Lattice Genesis creates new volumetric space with a constant baseline structural energy density ($\rho_{vac}$), the total internal energy scales strictly with volume ($dU = \rho_{vac} dV$). For the mechanical expansion equation of state, the chemical work of node creation ($\mu dN$) perfectly supplies the energy of the new volume. Therefore, the mechanical pressure strictly enforces the constant density: 
\begin{equation}
\rho_{vac} dV = -P dV \implies P = -\rho_{vac}
\end{equation} 
This yields the exact Dark Energy parameter without relying on closed-system adiabatic assumptions:
\begin{equation}
w = \frac{P}{\rho_{vac}} = -1
\end{equation}

\textbf{2. The CMB Latent Heat ($dQ \neq 0$):}
Simultaneously, the physical transition of the unstructured pre-geometric fluid into the discrete $\mathcal{M}_A$ lattice is an exothermic phase transition. The creation of each node releases a discrete quantum of latent heat ($\epsilon_f$) into the ambient photon gas. Therefore, $dQ_{latent} = \epsilon_f \dot{N} dt$.
The continuous volumetric injection of this heat exactly balances the adiabatic cooling of the expanding photon gas:
\begin{equation}
\dot{u}_{rad} = -4H_0 u_{rad} + \frac{\epsilon_f \dot{N}}{V} = 0 \implies T_{CMB} \approx 2.7 \text{ K}
\end{equation}
This perfectly unifies Dark Energy (mechanical structural pressure) and the CMB (thermodynamic latent heat) under a single, rigorous phase transition framework.