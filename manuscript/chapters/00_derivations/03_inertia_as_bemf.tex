\section{Inertia as Back-Electromotive Force (B-EMF)}

\subsection{The Metric Flux Density Field}
To connect continuum mechanics to a discrete lattice in SI units, we use the Topological Charge-to-Length Constant ($\xi_{topo} = e/\ell_{node}$). Under this topology, Inductance maps to Mass ($[L] \equiv [M]$) and Metric Current maps to Velocity ($\mathbf{I} \equiv \mathbf{v}$). 

Consequently, discrete Macroscopic Inductive Flux ($\mathbf{\Phi}_Z = L \cdot \mathbf{I}$) is isomorphic to discrete mechanical momentum ($\mathbf{p} = M \mathbf{v}$). We show this equivalence by evaluating the SI unit of magnetic flux (the Weber) with the $\xi_{topo}$ conversion factor:
\begin{equation}
    1 \text{ Wb} = 1 \text{ V}\cdot\text{s} = 1 \frac{\text{J}}{\text{C}}\cdot\text{s} \equiv 1 \frac{\text{J}}{\xi_{topo} \text{ m}}\cdot\text{s} = \frac{1}{\xi_{topo}} \left( \frac{\text{N}\cdot\text{m}}{\text{m}}\cdot\text{s} \right) = \mathbf{\frac{1}{\xi_{topo}} \left[ \text{kg}\cdot\frac{\text{m}}{\text{s}} \right]}
\end{equation}
Thus, mechanical momentum is mapped to magnetic flux by the equivalence $\mathbf{p} = \xi_{topo} \mathbf{\Phi}_Z$. Transitioning to a continuous fluidic model, we define the Metric Flux Density Field $\mathbf{\phi}_Z$ by substituting discrete mass with continuous mass density ($\rho_{mass}$):
\begin{equation}
\mathbf{\phi}_Z(\mathbf{x},t) \equiv \rho_{mass} \mathbf{v}
\end{equation}

\subsection{Inertial Force as the Eulerian Momentum Rate}
The Metric Flux Density $\mathbf{\phi}_Z$ has mechanical units of $[kg \cdot m^{-2} \cdot s^{-1}]$, so its time rate of change as it convects through the manifold gives an Inertial Force Density ($\mathbf{f}_{inertial}$) with units of $[N/m^3]$. To conserve momentum per the Reynolds Transport Theorem, we use the Eulerian conservative form with the divergence of the flux tensor:
\begin{equation}
\mathbf{f}_{inertial} = -\left( \frac{\partial \mathbf{\phi}_Z}{\partial t} + \nabla \cdot (\mathbf{\phi}_Z \otimes \mathbf{v}) \right)
\end{equation}

To recover Newton's discrete Macroscopic Inertial Force ($\mathbf{F}_{inertial}$) acting on a localized particle, we integrate this continuum force density field over the spatial volume of the particle ($V_p$):
\begin{equation}
\mathbf{F}_{inertial} = \int_{V_p} \mathbf{f}_{inertial} \, dV
\end{equation}
This connects Newton's Second Law to the continuous fluid dynamics of the $\mathcal{M}_A$ lattice, showing inertia as the Back-EMF of the vacuum.