\section{Inertia as Back-Electromotive Force (B-EMF)}

\subsection{The Metric Flux Density Field}
To apply continuum mechanics to a discrete lattice, we define the Metric Flux $\mathbf{\Phi}_Z$ not as a single particle vector, but as a continuous **Flux Density Field** permeating the manifold.
\begin{equation}
\mathbf{\Phi}_Z(\mathbf{x},t) \equiv \rho_{mass} \mathbf{v}
\end{equation}
This field represents the momentum density of the trapped flux knots.

\subsection{Inertial Force as the Material Derivative}
Because the Metric Flux $\mathbf{\Phi}_Z$ is mathematically defined as a continuous momentum density field (units of $[kg \cdot m^{-2} \cdot s^{-1}]$), its total time rate of change as it convects through the manifold yields an Inertial Force Density ($\mathbf{f}_{inertial}$), resolving to units of $[N/m^3]$. We apply the Eulerian Material (Convective) Derivative to track this continuum response:
\begin{equation}
\mathbf{f}_{inertial} = -\frac{D\mathbf{\Phi}_Z}{Dt} = -\left( \frac{\partial \mathbf{\Phi}_Z}{\partial t} + (\mathbf{v} \cdot \nabla)\mathbf{\Phi}_Z \right)
\end{equation}

To strictly recover Newton's discrete Macroscopic Inertial Force ($\mathbf{F}_{inertial}$) acting on a localized particle in Newtons, we must integrate this continuum force density field over the spatial volume of the particle ($V_p$):
\begin{equation}
\mathbf{F}_{inertial} = \int_{V_p} \mathbf{f}_{inertial} \, dV
\end{equation}
This derivation strictly bridges the gap between the localized discrete mechanics of Newton's Second Law and the continuous fluid dynamics of the $M_A$ lattice, flawlessly avoiding any calculus category errors.