\section{Inertia as Back-Electromotive Force (B-EMF)}

In this section, we derive the identity of mass ($M$) as a measure of the total inductance of a trapped flux packet within the $M_A$ lattice. 

\subsection{The Inductive Resistance to Acceleration}
In classical mechanics, $F = M \frac{dv}{dt}$. In the VSI framework, we propose that the force required to accelerate a mass is identical to the voltage (potential) required to change the current in an inductive system:
\begin{equation}
V = L \frac{dI}{dt}
\end{equation}

By mapping the velocity of a particle $v$ to an equivalent "Metric Current" $I_{m}$ and the force $F$ to a "Metric Voltage" $V_{m}$, we can define Mass ($M$) as the **Geometric Inductance** ($L_g$) of the particle-lattice interaction.

\subsection{Unit Transformation: Mass as Inductance}
To verify this without ad-hoc assumptions, we align the mechanical and electrical definitions of Energy:
1. Mechanical Kinetic Energy: $E_k = \frac{1}{2} M v^2$
2. Inductive Magnetic Energy: $E_L = \frac{1}{2} L I^2$

If we establish the identity that **Metric Current is Velocity** ($I \equiv v$), it logically follows that **Mass is Inductance** ($M \equiv L$).

\begin{equation}
1 \text{ kg} \equiv 1 \text{ Henry (in Metric Units)}
\end{equation}

This transformation allows us to treat Inertia as a reactive power component. The "weight" of an object is the measure of the vacuum's inductive reactance ($X_L$) to the object's movement through the time domain.

\subsection{The Lenz's Law of Gravitation}
Just as Lenz's Law states that an induced current flows in a direction that opposes the change in flux, **Inertia** is the manifestation of the $M_A$ lattice opposing the translation of a localized flux gradient. 
\begin{equation}
\mathbf{F}_{inertial} = -\oint_{node} \frac{\partial \Phi_Z}{\partial t} d\mathbf{l}
\end{equation}
Where $\Phi_Z$ is the "Impedance Flux" tied to the particle.