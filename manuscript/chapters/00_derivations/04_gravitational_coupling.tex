\section{Deriving the Gravitational Coupling ($G$)}
Because $[C] \equiv [m]$, electrical Capacitance ($[C/V]$) strictly maps to mechanical compliance ($[m/N]$). The maximum gravimetric tension the lattice can sustain before topological failure is the ratio of its characteristic length to its gravimetric compliance ($C_g$):
\begin{equation}
T_{max,g} \equiv \frac{l_{node}}{C_g}
\end{equation}
By evaluating Newton's classical law of gravitation ($G = F \cdot r^2 / M^2$) at the absolute fundamental geometric limit of two adjacent saturated $\mathcal{M}_A$ nodes ($r = l_{node}$, $M_{max} = L_g$, $F_{max} = T_{max,g}$), we unspool the precise underlying lattice mechanics:
\begin{equation}
G = \frac{F_{max} \cdot r_{min}^2}{M_{max}^2} = \frac{(l_{node}/C_g) \cdot l_{node}^2}{L_g^2}
\end{equation}
Substituting the invariant wave speed ($c = l_{node}/\sqrt{L_g C_g}$) into this expression mathematically resolves to:
\begin{equation}
G = \frac{c^4}{T_{max,g}}
\end{equation}
This proves that $G$ is not a fundamental scalar, but the geometric breaking tension of the vacuum continuum. We define the \textbf{Hierarchy Coupling ($\xi$)} strictly as the dimensionless topological stiffness multiplier representing the exact ratio between this gravimetric bulk tension limit ($T_{max,g}$) and the baseline 1D electromagnetic tension limit ($T_{EM} = E_{sat}/l_{node}$). The extreme weakness of gravity is thus mechanically resolved by the vast structural density required to distort the 3D bulk lattice relative to simply displacing a single 1D flux edge.