\section{AQUAL Fluid Dynamics and the Flat Rotation Curve}
The flat galactic rotation curve emerges naturally from the Bingham Plastic Navier-Stokes formulation, without additional constant insertions.

The empirical MOND acceleration boundary ($a_0$) arises from the fundamental acceleration floor of the expanding universe, which corresponds exactly to the Unruh-Hawking acceleration of the cosmic causal horizon:
\begin{equation}
a_{genesis} = \frac{c \cdot H_0}{2\pi} \approx 1.1 \times 10^{-10} \text{ m/s}^2
\end{equation}

Since the universe crystallizes exactly $H_0$ new nodes per unit time, the background lattice exerts a continuous macroscopic kinematic drift on all trapped topological defects, establishing a rigid, invariant acceleration floor $a_{genesis}$.

The Bingham Plastic non-Newtonian rheology of the substrate modifies the continuous Gauss-Poisson gravitational permeability according to the ratio of the localized Keplerian shear ($|\nabla \Phi|$) to this fundamental drift rate: $\mu_{g} \approx |\nabla \Phi|/a_{genesis}$. Integrating the stress equation $\nabla \cdot (\mu_g \nabla \Phi) = 4\pi G \rho_{mass}$ over a galactic mass $M$ recovers the AQUAL limit:
\begin{equation}
\frac{|\nabla \Phi|^2}{a_{genesis}} = \frac{GM}{r^2} \implies |\nabla \Phi| = \frac{\sqrt{GM a_{genesis}}}{r}
\end{equation}
Equating this to the centripetal acceleration ($v^2/r = |\nabla \Phi|$) yields the asymptotic flat velocity curve:
\begin{equation}
v_{flat} = (GM a_{genesis})^{1/4}
\end{equation}
This provides a concrete solid-state mechanical substrate—the Bingham Plastic fluid transition—as the physical origin of entropic force behavior, offering an alternative to mathematical dark matter halos.