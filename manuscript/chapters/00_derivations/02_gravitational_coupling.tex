\section{Deriving the Gravitational Coupling ($G$)}

\subsection{The Lattice Tension Limit ($T_{max,g}$)}
As a direct dimensional corollary of Axiom 1 ($1\text{ C} \equiv 1\text{ m}$), electrical Capacitance strictly maps to mechanical compliance ($1\text{ F} = 1\text{ C}^2/\text{J} \rightarrow 1\text{ m}^2/(\text{N}\cdot\text{m}) = 1\text{ m/N}$). The gravimetric Vacuum Capacitance $C_g$ represents the manifold's ultimate compliance. The absolute maximum tension the lattice can sustain before topological failure is:
\begin{equation}
T_{max,g} \equiv \frac{l_{node}}{C_g} \quad \text{[Newtons]}
\end{equation}

\subsection{The Geometric Emergence of G (Laplacian Reduction)}
To derive $G$ without circularly assuming Newton's macroscopic inverse-square law, we must examine the continuum limits of the discrete graph. In any 3D interconnected elastic matrix, the static stress field $\Phi$ around a localized defect strictly obeys the 3D Graph Laplacian ($\nabla^2 \Phi = 0$). The fundamental Green's function solution to this purely geometric operator mandates that the resultant force field (the gradient) decays asymptotically as $1/r^2$ at all scales down to the fundamental discrete boundary.

The macroscopic coupling constant $G$ is simply the scale factor of this Laplacian solution. We define it by evaluating the continuous Green's function strictly at its boundary condition: the minimum discrete cutoff limit of a fully saturated node pair ($r = l_{node}$, $M_{max} = L_g$, $F_{max} = T_{max,g}$).

Evaluating the Laplacian scale factor at this geometric boundary yields:
\begin{equation}
G = \frac{F_{max} \cdot r_{min}^2}{M_{max}^2} = \frac{(l_{node}/C_g) \cdot (l_{node})^2}{(L_g)^2} = \frac{l_{node}^3}{L_g^2 C_g}
\end{equation}

By subsequently substituting our rigorously derived definitions for the invariant wave speed squared ($c^2 = l_{node}^2 / (L_g C_g)$) and Lattice Tension ($T_{max,g} = l_{node}/C_g$), we find the exact algebraic reduction:
\begin{equation}
\frac{c^4}{T_{max,g}} = \left( \frac{l_{node}^4}{L_g^2 C_g^2} \right) \cdot \left( \frac{C_g}{l_{node}} \right) = \frac{l_{node}^3}{L_g^2 C_g} = G
\end{equation}
This mathematically proves that Newton's $G$ is not an empirical continuous primitive, but the exact geometric Laplacian boundary constant of the $\mathcal{M}_A$ discrete $LC$ network acting at its fundamental failure limit.