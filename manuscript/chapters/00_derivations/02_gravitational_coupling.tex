\section{Deriving the Gravitational Coupling ($G$)}

\subsection{The Lattice Tension Limit ($T_{max,g}$)}
To derive macroscopic gravity, we must define the ultimate breaking point of the $\mathcal{M}_A$ lattice strictly from its nodal parameters. As a direct dimensional corollary of the Topo-Kinematic Isomorphism ($1\text{ C} \equiv 1\text{ m}$), electrical Capacitance strictly maps to mechanical compliance ($1\text{ F} = 1\text{ C}^2/\text{J} \rightarrow 1\text{ m}^2/(\text{N}\cdot\text{m}) = 1\text{ m/N}$). The gravimetric Vacuum Capacitance $C_g$ represents the manifold's ultimate compliance. The absolute maximum tension the lattice can sustain before topological failure is:
\begin{equation}
T_{max,g} \equiv \frac{l_{node}}{C_g} \quad \text{[Newtons]}
\end{equation}

\subsection{The Geometric Emergence of G}
Instead of assuming General Relativity's field equations a priori, we derive the Gravitational Coupling strictly by evaluating Newton's classical law of gravitation ($G = F \cdot r^2 / M^2$) at the fundamental geometric limit of two adjacent $\mathcal{M}_A$ nodes.

Consider two nodes fully saturated into localized trapped-flux masses. The absolute geometric limits of their interaction are dictated by the manifold's inductive and capacitive parameters:
\begin{enumerate}
    \item \textbf{Minimum Separation ($r$):} The absolute minimum discrete distance between their centers is exactly one characteristic nodal length ($l_{node}$).
    \item \textbf{Maximum Mass ($M$):} By the Isomorphism ($1\text{ C} \equiv 1\text{ m}$), the effective mass of a fully saturated node in the gravimetric domain is identically its localized inductance ($L_g$).
    \item \textbf{Maximum Force ($F$):} The absolute maximum gravitational pull they can exert before the manifold suffers topological failure is the fundamental tension limit ($T_{max,g} = l_{node}/C_g$).
\end{enumerate}

Substituting these strict $LC$ network primitives into Newton's classical formulation unspools the precise underlying lattice mechanics:
\begin{equation}
G = \frac{F_{max} \cdot r_{min}^2}{M_{max}^2} = \frac{(l_{node}/C_g) \cdot (l_{node})^2}{(L_g)^2} = \frac{l_{node}^3}{L_g^2 C_g}
\end{equation}

By subsequently substituting our rigorously derived definitions for the invariant wave speed squared ($c^2 = l_{node}^2 / (L_g C_g)$) and Lattice Tension ($T_{max,g} = l_{node}/C_g$), we find the exact algebraic reduction:
\begin{equation}
\frac{c^4}{T_{max,g}} = \left( \frac{l_{node}^4}{L_g^2 C_g^2} \right) \cdot \left( \frac{C_g}{l_{node}} \right) = \frac{l_{node}^3}{L_g^2 C_g} = G
\end{equation}
This mathematically proves that Newton's $G$ and Einstein's string tension limit ($c^4/G$) are not empirical continuous primitives, but exact geometric composites of the $\mathcal{M}_A$ lattice's discrete $LC$ network acting at its fundamental failure limit.