\section{Deriving the Gravitational Coupling ($G$)}

\subsection{The Lattice Tension Limit ($T_{max,g}$) and QED Independence}
In emergent gravity models, deriving $G$ from string tension can sometimes involve circular definitions. Here, we derive the baseline tension from independent Quantum Electrodynamic (QED) limits.

The 1D electromagnetic baseline tension of a discrete flux tube ($T_{EM}$) is bounded by the volumetric Schwinger Yield Limit ($u_{sat}$) applied over the geometric packing area of a single node ($\kappa_V \ell_{node}^2$). Substituting the derived packing fraction ($\kappa_V = 8\pi\alpha$) from Section 2.3 yields an algebraic collapse:
\begin{equation}
T_{EM} = u_{sat} \cdot (\kappa_V \ell_{node}^2) = \left( \frac{1}{2} \epsilon_0 \frac{m_e^2 c^4}{e^2 \ell_{node}^2} \right) (8\pi\alpha) \ell_{node}^2
\end{equation}
Using the identity $\alpha = e^2 / 4\pi\epsilon_0\hbar c$, this reduces to the classical rest-mass energy distributed over the edge length:
\begin{equation}
T_{EM} = \frac{m_e c^2}{\hbar / m_e c} = \mathbf{\frac{m_e c^2}{\ell_{node}}} \quad \text{[Newtons]}
\end{equation}
This connects the 3D volumetric saturation limit and the 1D linear rest-mass limit, unifying the geometry. 

Because macroscopic gravitation is a 3D volumetric strain of the Delaunay graph, the Gravimetric Tension Limit ($T_{max,g}$) is the 1D EM tension scaled by the \textbf{Hierarchy Coupling ($\xi$)}.
\begin{equation}
T_{max,g} = \xi \cdot T_{EM}
\end{equation}

\subsection{Eliminating the Hidden Variable: The Machian Topological Coupling}
In some models, $\xi$ serves as an adjustable parameter to match $G$. In AVE, $\xi$ is derived from the boundary conditions of the universe. 

In a connected graph, the structural ratio between the macroscopic 3D bulk and the microscopic 1D edge is bounded by the Information Capacity of the Cosmic Horizon. By applying Mach's Principle to the discrete lattice, the macroscopic impedance is the sum of all microscopic nodes spanning the causal radius of the universe. 

To evaluate this boundary, we use the \textbf{Hubble Radius} ($R_H = c/H_0$)—the apparent causal boundary of the visible universe. The coupling is scaled by the structural porosity of the lattice ($\alpha^{-2}$, derived geometrically in Section 2.2).
\begin{equation}
\xi \equiv 4\pi \left( \frac{R_H}{\ell_{node}} \right) \alpha^{-2} = 4\pi \left( \frac{c/H_0}{\ell_{node}} \right) \alpha^{-2}
\end{equation}
Because $\alpha$ is derived from geometry, the $10^{44}$ hierarchy scale emerges from the ratio of the cosmic horizon to the electron pitch.

\subsection{The Geometric Emergence of $G$ (Laplacian Reduction)}
In any 3D interconnected elastic matrix, the static stress field $\Phi$ around a localized defect obeys the 3D Graph Laplacian ($\nabla^2 \Phi = 0$). The macroscopic coupling constant $G_{calc}$ is the scale factor of this Laplacian solution, evaluated at the boundary condition ($r_{min} = \ell_{node}$, $M_{max} = L_g$, $F_{max} = T_{max,g}$):
\begin{equation}
G_{calc} = \frac{F_{max} \cdot r_{min}^2}{M_{max}^2} = \frac{(\xi T_{EM}) \cdot \ell_{node}^2}{L_g^2}
\end{equation}
Substituting the invariant wave speed squared ($c^2 = \ell_{node}^2 / (L_g C_g)$) yields:
\begin{equation}
G_{calc} = \frac{c^4 C_g}{\ell_{node}} = \frac{c^4}{T_{max,g}} = \frac{c^4}{\xi T_{EM}}
\end{equation}
Substituting the derived $\xi$ and $T_{EM} = m_e c^2 / \ell_{node}$ yields:
\begin{equation}
G_{calc} = \frac{\ell_{node}^2 \alpha^2 H_0 c}{4\pi m_e} = \frac{\hbar^2 \alpha^2 H_0}{4\pi m_e^3 c}
\end{equation}

\subsection{The Lagrangian Derivation of the Cosserat Projection (1/7)}
In General Relativity, the interaction energy density between a metric strain $h_{\mu\nu}$ and a localized stress-energy source $T_{\mu\nu}$ is governed by:
\begin{equation}
\mathcal{L}_{int} = \frac{1}{2} h_{\mu\nu} T^{\mu\nu} \equiv \frac{1}{2} \bar{T}_{\mu\nu} h^{\mu\nu}
\end{equation}
where $\bar{T}_{\mu\nu} = T_{\mu\nu} - \frac{1}{2} \eta_{\mu\nu} T$ is the trace-reversed source.

For a 1D uniaxial string under absolute saturation, the stress tensor is $T_{\mu\nu} = \text{diag}(T_{EM}, 0, 0, -T_{EM})$. 
The trace $T = -2 T_{EM}$. The trace-reversed transverse components are:
\begin{equation}
\bar{T}_{11} = \bar{T}_{22} = T_{EM}
\end{equation}

The transverse Cosserat strain is $h_\perp = \nu_{vac} h_\parallel$. Substituting gives the transverse isotropic interaction energy:
\begin{equation}
\mathcal{L}_\perp = \frac{1}{2} \bar{T}_\perp h_\perp = \frac{1}{2} T_{EM} (\nu_{vac} h_\parallel) = \left( \frac{\nu_{vac}}{2} \right) T_{EM} h_\parallel
\end{equation}
With $\nu_{vac} = 2/7$, the Lagrangian coupling factor is:
\begin{equation}
\text{Lagrangian Projection Factor} = \frac{1}{2} \cdot \frac{2}{7} = \frac{1}{7}
\end{equation}
Applying this projection to $G_{calc}$ yields $G = \hbar^2 \alpha^2 H_0 / (28\pi m_e^3 c)$.

\subsection{Quantitative Resolution of the Hubble Tension}
Rearranging gives:
\begin{equation}
H_0 = \frac{28\pi m_e^3 c G}{\hbar^2 \alpha^2} \approx 69.32\,\text{km/s/Mpc}
\end{equation}
This value falls in the center of the current observational range, offering a geometric perspective on the Hubble tension.