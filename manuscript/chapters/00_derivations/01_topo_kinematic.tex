\section{The Impedance of the Discrete Amorphous Manifold}

\subsection{Fundamental Axiom 1: The Topo-Kinematic Isomorphism}
To connect electrical and mechanical phenomena in a cohesive way, we establish the foundation of the framework with a geometric postulate.

\begin{quote}
\textbf{Axiom 1 (The Topo-Kinematic Isomorphism):} Let the vacuum be a Discrete Amorphous Manifold ($\mathcal{M}_A$) with a mean discrete edge length $\ell_{node}$. Electric charge $q$ is defined identically as the discrete topological Hopf charge (phase vortex) $Q_H \in \mathbb{Z}$ around a 1D closed loop. Because the manifold is a physical finite-difference graph, a continuous fractional spatial phase rotation is physically impossible. A single quantized $2\pi$ phase twist ($Q_H=1$, representing the elementary charge $e$) structurally requires an edge dislocation (a Burgers vector) in the spatial lattice. 

The absolute minimum magnitude of this spatial dislocation is exactly one fundamental edge length ($\ell_{node}$). Therefore, the fundamental dimension of charge is strictly identical to the fundamental dimension of length ($[Q] \equiv [L]$).
\end{quote}

\textit{Contextual Note:} Unlike historical Kaluza-Klein theories, which involve compactified extra dimensions to map charge to geometry, the AVE framework achieves dimensional unification within 3D Euclidean space by identifying charge directly as a structural lattice dislocation.

To translate this dimensional equivalence into macroscopic SI units, we define the \textbf{Topological Charge-to-Length Constant ($\xi_{topo}$)}:
\begin{equation}
\xi_{topo} \equiv \frac{e}{\ell_{node}} \quad \text{[Coulombs / Meter]}
\end{equation}

By substituting the dimensional conversion $1\text{ C} \equiv \xi_{topo} \text{ m}$ into the standard SI definition of electrical impedance, we map Ohms to mechanical kinematic impedance:
\begin{equation}
1 \, \Omega = 1 \frac{\text{V}}{\text{A}} = 1 \frac{\text{J/C}}{\text{C/s}} = 1 \frac{\text{J}\cdot\text{s}}{\text{C}^2} \equiv 1 \frac{\text{J}\cdot\text{s}}{( \xi_{topo} \text{ m} )^2} = \frac{1}{\xi_{topo}^2} \frac{\text{J}\cdot\text{s}}{\text{m}^2} = \frac{1}{\xi_{topo}^2} \left( \frac{\text{N}\cdot\text{m}\cdot\text{s}}{\text{m}^2} \right) = \mathbf{\frac{1}{\xi_{topo}^2} \text{ kg/s}}
\end{equation}
This establishes a dimensional connection showing that Electrical Resistance is isomorphic to the inverse of mechanical inertial drag within the vacuum substrate, scaled by the geometric constant $\xi_{topo}^2$.

\subsection{The Geometric Interpretation of the Fine Structure Constant ($\alpha$)}
To provide a clear foundation for subsequent derivations, we define the geometric role of the Fine Structure Constant ($\alpha$) within the $\mathcal{M}_A$ lattice. 

The discrete vacuum graph is governed by two fundamental geometric scales:
\begin{enumerate}
    \item \textbf{The Kinematic Lattice Pitch ($\ell_{node}$)}: The fundamental center-to-center spacing of the manifold, scaled to the kinematic mass-gap resolution (the electron's reduced Compton limit, $\bar{\lambda}_c = \hbar / m_e c$).
    \item \textbf{The Structural Core Radius ($r_{core}$)}: The physical cross-section of the finite-element node where the dielectric strain energy density reaches classical saturation.
\end{enumerate}

The structural core radius is bounded by the classical limit where the electrostatic potential energy of the topological defect equals its total mass-energy ($U_E = m_e c^2$). Solving for the radius $r_{core}$ at this saturation limit yields:
\begin{equation}
m_e c^2 = \frac{e^2}{4\pi \epsilon_0 r_{core}} \implies r_{core} = \frac{e^2}{4\pi \epsilon_0 m_e c^2}
\end{equation}
We define the \textbf{Vacuum Porosity Ratio} as the geometric ratio of the hard structural core ($r_{core}$) to the effective kinematic lattice spacing ($\ell_{node} = \bar{\lambda}_c$):
\begin{equation}
\text{Porosity Ratio} \equiv \frac{r_{core}}{\ell_{node}} = \frac{e^2}{4\pi \epsilon_0 \hbar c} \equiv \alpha \approx \frac{1}{137.036}
\end{equation}

\subsection{Deriving the Geometric Packing Fraction ($\kappa_V$)}
The effective geometric volume of a single discrete node is $V_{node} = \kappa_V \ell_{node}^3$. By equating the maximum energy of a single topological node to the quantum of action over one clock cycle, we obtain:
\begin{equation}
\kappa_V = 8\pi\alpha \approx 0.1834
\end{equation}

\subsubsection{Cosserat Trace-Reversal and the Longitudinal P-Wave Paradox ($\nu_{vac} = 2/7$)}
The total macroscopic bulk modulus is:
\begin{equation}
K_{vac} = 2 G_{vac}
\end{equation}

The vacuum Poisson's ratio is:
\begin{equation}
\nu_{vac} = \frac{2}{7}
\end{equation}

The weak-boson mass ratio follows:
\begin{equation}
\frac{m_W}{m_Z} = \sqrt{\frac{7}{9}} \approx 0.8819
\end{equation}

\subsection{Deriving the Gravitational Coupling ($G$)}
The 1D electromagnetic tension collapses to $T_{EM} = m_e c^2 / \ell_{node}$. The Machian hierarchy coupling is $\xi = 4\pi (c/H_0 / \ell_{node}) \alpha^{-2}$. The Laplacian reduction gives $G_{calc} = \hbar^2 \alpha^2 H_0 / (4\pi m_e^3 c)$. The trace-reversed Cosserat Lagrangian projection factor is exactly $1/7$, yielding:
\begin{equation}
G = \frac{\hbar^2 \alpha^2 H_0}{28\pi m_e^3 c}
\end{equation}

Inverting gives the derived Hubble constant:
\begin{equation}
H_0 = \frac{28\pi m_e^3 c G}{\hbar^2 \alpha^2} \approx 69.32\,\text{km/s/Mpc}
\end{equation}

\subsection{Inertia, Topological Mass Hierarchies, Thermodynamics, and AQUAL}
All remaining sections (inertia as B-EMF, 1D topological mass solver, stable phantom dark energy bound $w_{vac} > -1.0001$, and the flat rotation curve $v_{flat} = (G M a_{genesis})^{1/4}$ with $a_{genesis} = c H_0 / 2\pi$) follow from the same axioms and are detailed in the provided source.

\subsection{Axiomatic Dependency and Mathematical Closure}
The framework is a closed Directed Acyclic Graph with exactly two axioms and one calibration point. All constants emerge forward without circularity.