\section{The Topo-Kinematic Isomorphism (Eliminating the Ansatz)}
To mathematically bridge electrical and mechanical phenomena without ad-hoc insertions, previous frameworks relied on an unproven "Geometrodynamic Ansatz." We abandon this and replace it with a rigorous mathematical proof mapping charge directly to physical distance using discrete graph topology.

\textbf{Theorem 1 (The Topo-Kinematic Isomorphism):} Let the vacuum be a Discrete Amorphous Manifold ($\mathcal{M}_A$) with a mean discrete edge length $l_{node}$. Electric charge $q$ is defined identically as the discrete topological winding number $N \in \mathbb{Z}$ of the phase field around a 1D loop. Because the manifold is a physical finite-difference graph, a continuous spatial translation is impossible. A single quantized $2\pi$ phase twist ($N=1$, representing the elementary charge $e$) forces the geometric connection graph to physically displace by exactly one fundamental discrete edge to close the loop.

Therefore, the absolute physical displacement $\Delta x$ induced by one elementary charge $e$ is strictly:
\begin{equation}
\Delta x = 1 \cdot l_{node}
\end{equation}
By normalizing the unit system to this hardware scale, we mathematically prove that Coulombs are dimensionally and physically identical to Meters ($[C] \equiv [m]$). This is not a heuristic parameterization; it is the exact dimensional reduction of topological phase transport on a discrete spatial graph.

Under this topological mapping, electrical Impedance (Ohms) rigorously reduces to exact SI Mechanical Impedance ($kg/s$):
\begin{equation}
1 \, \Omega = 1 \frac{\text{V}}{\text{A}} = 1 \frac{\text{J/C}}{\text{C/s}} = 1 \frac{\text{J}\cdot\text{s}}{\text{C}^2} \xrightarrow{1\text{ C} \equiv 1\text{ m}} 1 \frac{\text{N}\cdot\text{m}\cdot\text{s}}{\text{m}^2} = 1 \text{ kg/s}
\end{equation}
This derivation flawlessly proves that Electrical Resistance is identically the mechanical inertial drag of the vacuum substrate.