% 10_generative_cosmology.tex
\chapter{Generative Cosmology and Thermodynamic Attractors}
\label{ch:cosmology}

\section{Lattice Genesis: The Origin of Metric Expansion}
Standard cosmology often models metric expansion as the continuous expansion of an unstructured coordinate geometry. The AVE framework restricts the macroscopic stretching of this fundamental limit. Because a discrete lattice cannot stretch macroscopically without disrupting its Delaunay triangulation, metric expansion is modeled strictly as the discrete, real-time \textbf{crystallization} of new topological nodes.

To preserve the invariant spatial density of the condensate globally ($\partial_t \rho_n = 0$), the Eulerian continuity equation dictates the discrete generative source term must identically match the macroscopic volumetric expansion divergence. We hypothesize that the Hubble Constant ($H_0$) is not a velocity, but the \textbf{Lattice Crystallization Rate} required to maintain the vacuum's structural integrity against the compressive tension of gravity.

As derived in Chapter 4, evaluating the Machian boundary impedance against the quantum mass-gap establishes an absolute geometric relationship for the asymptotic expansion limit:
\begin{equation}
    H_{\infty} = \frac{28\pi m_e^3 c G}{\hbar^2 \alpha^2}
\end{equation}

\subsection{Verification: Resolving the Hubble Tension}
Substituting the fundamental constants ($m_e, c, \hbar, G$) and the derived fine-structure geometry ($\alpha^{-1} \approx 137.036$) into this geometric bound evaluates to:
\begin{equation}
    H_{\infty} \approx 69.32 \text{ km/s/Mpc}
\end{equation}
This baseline relationship lies precisely between the Early Universe measurements (Planck 2018: $67.4 \pm 0.5$) and Late Universe measurements (SHOES: $73.0 \pm 1.4$). This suggests that the "Hubble Tension" is an artifact of measuring effective expansion across different thermodynamic regimes, while the underlying hardware generation rate asymptotes to this exact geometric bound.

\section{Dark Energy: The Stable Phantom Derivation}
During lattice genesis, the phase transition continuously expels a latent heat of fusion ($\rho_{latent} dV$) into the ambient photon gas (CMB). By the first law of thermodynamics, to physically fund the internal energy of the newly created spatial volume ($\rho_{vac}$) while simultaneously expelling this latent heat, the total macroscopic mechanical pressure ($P_{tot}$) of the vacuum must be strictly negative.

Calculating the Equation of State ($w = P/\rho$) for this generative process:
\begin{equation}
    w_{vac} = -1 - \frac{\rho_{latent}}{\rho_{vac}} \approx -1.0001
\end{equation}
The AVE framework identifies "Dark Energy" not as a mysterious scalar field, but as the thermodynamic latent heat of the vacuum's continuous crystallization. It predicts a stable Phantom Energy state ($w < -1$) that fundamentally drives cosmic acceleration without leading to a Big Rip singularity.

\begin{figure}[h]
    \centering
    \includegraphics[width=0.9\textwidth]{bingham_rheology.png}
    \caption{\textbf{The Bingham Plastic Phase Transition.} The vacuum exhibits a dual rheological nature. At low stress (Galactic Halos), it acts as a rigid solid with high viscosity, creating the "Dark Matter" drag effect. At high stress (Planetary Orbits), it yields into a frictionless superfluid, allowing stable orbital mechanics.}
    \label{fig:bingham_rheology}
\end{figure}

\section{The CMB as an Asymptotic Thermal Attractor}
The continuous injection of latent heat into the photon gas (Cosmic Microwave Background) dynamically forms a permanent asymptotic thermal floor. Competing against standard adiabatic expansion cooling ($a^{-4}$), the thermodynamic history of the universe perfectly integrates to:
\begin{equation}
    u_{rad}(a) = U_{hot} \, a^{-4} + \frac{3}{4} \rho_{latent}
\end{equation}
As $a \to \infty$, the universe does not freeze to absolute zero; it smoothly asymptotes to the fundamental Unruh-Hawking temperature limit ($T_U \sim 10^{-30}$ K), structurally resolving the thermodynamic Heat Death paradox.

\section{Black Holes and Dielectric Rupture}

No physical substrate stretches infinitely to a geometric singularity. As matter aggregates into
a hyper-dense core, the macroscopic inductive refractive strain ($n_{\perp}=1+2GM/rc^{2}$) increases.

At the exact mathematical radius of the event horizon, the continuous tensor strain on
the discrete edges reaches the strictly squared (2nd-order) Axiom 4 dielectric saturation limit. At this
threshold, the spatial structure physically ruptures. The discrete nodes undergo a sudden
thermodynamic phase transition, melting back into an unstructured, pre-geometric continuous
plasma. The concept of the geometric singularity is replaced by a flat thermodynamic floor.

Because topological particles (knots) fundamentally require the discrete lattice edges to
maintain their invariants, crossing the event horizon destroys the structural canvas supporting
them. The knots mechanically unravel. The mass-energy is conserved strictly as latent heat,
but the geometric quantum information is physically, mathematically, and permanently erased.

The AVE framework explicitly sides with Hawking's original assessment: the thermody-
namic phase transition of the substrate dictates that quantum unitarity is macroscopically
violated at the event horizon, strictly enforcing information loss.