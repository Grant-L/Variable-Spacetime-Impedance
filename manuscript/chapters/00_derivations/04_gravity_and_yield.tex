\chapter{Trace-Reversal, Gravity, and Macroscopic Yield}
\label{ch:gravity_and_yield}

\section{Cosserat Trace-Reversal ($K=2G$)}
To support strictly transverse waves matching the kinematics of General Relativity, the 3D isotropic stress-strain relationship of the vacuum must natively accommodate the 4D trace-reversal metric signature ($\bar{h}_{\mu\nu} = h_{\mu\nu} - \frac{1}{2}\eta_{\mu\nu} h$). In 3D elasticity, volumetric strain is governed by the bulk modulus ($K$) and deviatoric (trace-free) strain is governed by the shear modulus ($G$). To inherently balance this exact $1/2$ geometric projection factor without suffering thermodynamic Cauchy instability, the elastic moduli must strictly lock in a $2:1$ ratio. 

Because the macroscopic Cosserat solid must be strictly trace-reversed, the bulk modulus is structurally locked to exactly double the shear modulus ($K_{vac} = 2 G_{vac}$). Substituting this exact symmetry requirement into the standard equation for Poisson's ratio geometrically locks the vacuum's mechanics:
\begin{equation}
    \nu_{vac} = \frac{3K_{vac} - 2G_{vac}}{2(3K_{vac} + G_{vac})} = \frac{6G - 2G}{2(6G + G)} = \frac{4}{14} = \mathbf{\frac{2}{7}}
\end{equation}

\section{Macroscopic Gravity and The $1/7$ Projection}
The maximum transmissible mechanical tension across a discrete flux tube is bounded by $T_{EM} = m_e c^2 / \ell_{node}$. Macroscopic Gravity ($G$) evaluates in the 3D trace-reversed bulk domain, structurally shielded by the total Machian causal hierarchy of the universe. 

The Machian coupling factor $\xi$ is strictly derived as the 3D isotropic geometric integration of the structural graph out to the cosmic horizon. It is evaluated as the exact geometric product of the 3D spherical solid angle ($4\pi$ steradians), the 1D radial distance to the horizon ($R_H/\ell_{node}$), and the structural cross-sectional porosity of the graph ($A_{node}/A_{core} = \alpha^{-2}$).

By integrating the 1D structural resistance isotropically across the causal horizon ($R_H = c/H_0$) and scaling by this cross-sectional node porosity, the dimensionless Machian impedance is defined exactly:
\begin{equation}
    \xi = \oint d\Omega \frac{R_H / \ell_{node}}{\alpha^2} = 4\pi \left(\frac{R_H}{\ell_{node}}\right) \alpha^{-2}
\end{equation}

Projecting the localized 1D string into a 3D isotropic bulk metric requires evaluating the Interaction Lagrangian utilizing the trace-reversed stress-energy tensor. This geometry natively yields a transverse spatial projection factor of \textbf{1/7}. Applying this tensor scaling yields $G = c^4 / 7 \xi T_{EM}$. Rearranging strictly isolates the Hubble parameter dynamically:
\begin{equation}
H_0 = \frac{28\pi m_e^3 c G}{\hbar^2 \alpha^2} \approx \mathbf{69.32 \pm 0.05 \text{ km/s/Mpc}}
\end{equation}

\section{The Macroscopic Bingham Yield Stress ($\tau_{yield}$)}
Because macroscopic fluidic shear is a 3D volumetric strain of the trace-reversed bulk continuum, the fundamental 1D node breakdown voltage ($511.0$ kV) must be rigidly scaled by the exact same $1/7$ bulk tensor projection factor:
\begin{equation}
V_{yield} = \frac{V_{snap}}{7} = \mathbf{73.0 \text{ kV}} \implies F_{yield} = V_{yield} \times \xi_{topo} \approx \mathbf{0.03028 \text{ N}}
\end{equation}

Structural yield is strictly governed by macroscopic mechanical stress ($\tau = F/A$), not an intensive 1D force. Applying this topological force limit across the fundamental cross-sectional area of a single spatial node ($A_{node} = \ell_{node}^2 \approx 1.49 \times 10^{-25} \text{ m}^2$) derives the absolute \textbf{Macroscopic Bingham Yield Stress}:
\begin{equation}
    \tau_{yield} = \frac{F_{yield}}{\ell_{node}^2} \approx \mathbf{2.03 \times 10^{23} \text{ Pascals}}
\end{equation}

By converting the 1D topological breakdown force into a 3D macroscopic cross-sectional stress, it is formally proven that macroscopic solids cannot spontaneously melt the vacuum. Because this macroscopic structural yield limit evaluates to roughly 2 quintillion atmospheres of pressure, bulk macroscopic masses resting on a spatial metric drive will not trigger vacuum liquefaction.

\subsection{Microscopic Point-Yield: The 16.50 keV Fusion Limit}
In high-energy particle physics, collisions occur on the scale of a single node. For a head-on collision between two individual ions, the total force is concentrated entirely within the microscopic $A_{node}$ cross-section. The classical turning point Coulomb force relates directly to the square of the kinetic collision energy ($E_k$). Evaluating exactly where this point-force shatters the $0.03028$ N structural yield limit:
\begin{equation}
    F_{yield} = \frac{E_k^2}{\left( \frac{e^2}{4\pi\epsilon_0} \right)} \implies E_k = \sqrt{F_{yield} \left( \frac{e^2}{4\pi\epsilon_0} \right)} \equiv \mathbf{16.50 \text{ keV}}
\end{equation}
This establishes the strict kinematic limit where thermonuclear fusion generates sufficient local nodal pressure to physically melt the spatial containment vessel.