\section{Topological Helicity and the Winding Number}

To account for the discrete nature of particles and their varying masses, we introduce Topological Helicity ($h$). If Mass is Geometric Inductance, then the specific particle type is determined by the "Winding Number" of the flux trapped within the $M_A$ lattice.

\subsection{The Helicity Quotient}
The helicity $h$ represents the number of times a wave-front wraps around a localized manifold defect before closing its loop. This is defined by the integral of the vector potential $\mathbf{A}$ over the volume of the node:
\begin{equation}
h = \int_{V_{node}} \mathbf{A} \cdot (\nabla \times \mathbf{A}) \, dV
\end{equation}

In our circuit-equivalent model, $h$ is the dimensionless "turns ratio" ($N$) of the node. 

\subsection{Quantized Mass-Inductance}
The total effective inductance $L_{eff}$ (and thus the Mass $M$) of a particle is proportional to the square of its winding number, following standard inductive laws ($L \propto N^2$):
\begin{equation}
M = M_{base} \cdot h^2
\end{equation}
Where $M_{base}$ is the minimum mass-energy required to saturate a single $M_A$ node. 

\subsection{The Chiral Bias Equation (CBE)}
The interaction between the particle's internal helicity ($h$) and the vacuum's background vorticity ($\mathbf{\Omega}_{vac}$) creates a geometric bias in the local metric impedance. Since $h$ is a quantized winding number, the coupling is unitary:
\begin{equation}
Z_{metric} = Z_{0} \left( 1 + \frac{h \cdot \mathbf{\Omega}_{vac}}{|\mathbf{\Omega}_{vac}|} \right)
\end{equation}
This implies that impedance increases when the particle's "spin" opposes the vacuum vorticity (friction) and decreases when aligned (super-fluidity), derived purely from topological alignment without arbitrary coupling constants.