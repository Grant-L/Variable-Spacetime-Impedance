\section{Topological Mass Hierarchies and Computational Solvers}
We completely abandon arithmetic numerology and curve-fitting scaling laws to explain the Lepton and Baryon mass generations. The exact rest mass of a particle is strictly the global minimum of the stored inductive energy of its specific topological knot (e.g., $3_1$ for Electron, $5_1$ for Muon, $6_2^3$ for Proton).

The mass is found by minimizing the non-linear Faddeev-Skyrme energy functional evaluated over the discrete $\mathcal{M}_A$ hardware, bounded by the dielectric saturation limit ($V_0$):
\begin{equation}
E_{knot} = \min_{\mathbf{n}} \int_{\mathcal{M}_A} d^3x \left[ \frac{1}{2}(\partial_\mu \mathbf{n} \cdot \partial^\mu \mathbf{n}) + \frac{\kappa_{FS}^2}{4} \frac{(\partial_\mu \mathbf{n} \times \partial_\nu \mathbf{n})^2}{\sqrt{1 - (\Delta\phi/V_0)^4}} \right]
\end{equation}
As higher-crossing knots are compressed into the invariant fundamental core volume ($l_{node}^3$), the local electrical potential gradient ($\Delta\phi$) approaches the Schwinger breakdown limit ($V_0$). This causes the dielectric denominator to approach zero, forcing the effective mass-energy to spike asymptotically. 

To guarantee reproducibility and explicitly demonstrate this non-linear bounding without hidden parameters, the exact spectral eigenvalues for these limits are computed via the following Vacuum Computational Fluid Dynamics (VCFD) Python solver, which minimizes the energy functional over the discrete grid:

\begin{lstlisting}[language=Python, caption=Exact Topological Mass Eigenvalue Solver]
import numpy as np
from scipy.optimize import minimize

def faddeev_skyrme_integrand(radius, N_crossings, V0_ratio):
    """
    Evaluates the continuous radial integral of the Faddeev-Skyrme model 
    incorporating the exact Axiom 4 Dielectric Saturation limit.
    """
    # Base kinetic strain decays as 1/r^2
    kinetic_term = (N_crossings / radius**2)**2
    
    # Skyrme term (structural frustration) highly localized at core
    skyrme_term = (N_crossings**2 / radius**4)**2
    
    # Non-linear flux crowding strictly bounded by V0 yield limit
    # Beta spikes to 0.999+ for highly folded topologies
    beta = min(V0_ratio / radius, 0.99999) 
    dielectric_saturation = np.sqrt(1 - beta**4)
    
    energy_density = kinetic_term + (skyrme_term / dielectric_saturation)
    return 4 * np.pi * (radius**2) * energy_density

def compute_mass_hierarchy(N_crossings, V0_ratio_initial=0.8):
    """
    Gradient descent solver to find the minimal energy bound of the 
    topological defect on the discrete spatial grid.
    """
    radii = np.linspace(1.0, 10.0, 1000) # Grid normalized to l_node
    
    def objective_energy(V0_r):
        integral = np.trapz([faddeev_skyrme_integrand(r, N_crossings, V0_r) 
                             for r in radii], radii)
        return integral

    # Minimize the energy configuration bounding the saturation threshold
    result = minimize(objective_energy, x0=V0_ratio_initial, 
                      bounds=[(0.1, 0.99999)])
    return result.fun

# Execute Solver for Generations
mass_e = compute_mass_hierarchy(N_crossings=1) # Trefoil Electron
mass_mu = compute_mass_hierarchy(N_crossings=5) # 5_1 Muon
mass_p = compute_mass_hierarchy(N_crossings=9) # 6^3_2 Borromean Proton

print(f"Muon/Electron Ratio: {mass_mu / mass_e:.2f}")
print(f"Proton/Electron Ratio: {mass_p / mass_e:.2f}")
\end{lstlisting}