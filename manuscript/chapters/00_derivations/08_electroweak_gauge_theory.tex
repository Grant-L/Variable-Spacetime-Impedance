\chapter{Electroweak Mechanics and Gauge Symmetries}
\label{ch:electroweak}

\section{Electrodynamics: The Gradient of Topological Stress}
A localized charged node permanently exerts a continuous rotational phase twist ($\theta$) on the surrounding dielectric condensate. Because the unsaturated vacuum acts as a tensioned linear elastic solid in the far-field, the static structural strain must strictly obey the 3D \textbf{Laplace Equation} ($\nabla^2 \theta = 0$).

The spherically symmetric geometric solution dictates that the twist amplitude decays exactly inversely with distance ($\theta(r) \propto 1/r$). The continuous electric displacement field ($\mathbf{D}$) is physically identical to the spatial gradient of this structural twist ($\mathbf{D} = \nabla\theta \propto -1/r^2 \mathbf{\hat{r}}$), analytically deriving Coulomb's Law.

\subsection{Magnetism as Convective Vorticity}
When a twisted node translates at a velocity $\mathbf{v}$, it induces a convective shear flow in the momentum field. In classical fluid dynamics, the time evolution of a translating steady-state strain field $\mathbf{D}(\mathbf{r} - \mathbf{v}t)$ is governed by the convective material derivative:
\begin{equation}
    \partial_t \mathbf{D} = -(\mathbf{v} \cdot \nabla)\mathbf{D} \implies \nabla \times (\mathbf{v} \times \mathbf{D})
\end{equation}
Equating this to the Maxwell-Ampere law derives the macroscopic magnetic field strictly from fluid dynamics: $\mathbf{H} = \mathbf{v} \times \mathbf{D}$.

This relationship is rigorously supported by dimensional analysis. Applying the topological conversion constant ($\xi_{topo} \equiv e/\ell_{node}$), the displacement field reduces to $[\mathbf{D}] = \xi_{topo}[1/\text{m}]$. Evaluating the cross product $[\mathbf{v} \times \mathbf{D}]$ yields strictly $\xi_{topo}[1/\text{s}]$. Standard SI units for magnetic field intensity $\mathbf{H}$ ($[\text{A/m}]$) identically reduce to this exact same dimensional basis ($\xi_{topo}[1/\text{s}]$). Magnetism is thereby dimensionally proven to represent the continuous kinematic vorticity of the vacuum condensate.

\section{The Weak Interaction: Micropolar Cutoff Dynamics}
In classical solid mechanics, the ratio of the Cosserat microrotational bending stiffness ($\gamma_c$) to the macroscopic shear modulus ($G_{vac}$) rigidly defines a fundamental \textbf{Characteristic Length Scale} ($l_c = \sqrt{\gamma_c/G_{vac}}$). This length scale is identified as the physical origin of the weak force range ($r_W \approx 10^{-18}$ m). 

Weak interactions lack the kinetic energy required to overcome the ambient Cosserat rotational stiffness. Any physical excitation operating \textit{below} a medium's natural cutoff frequency is mathematically forced to become an \textbf{Evanescent Wave}. The static field equation transforms from the Laplace equation to the massive Helmholtz equation ($\nabla^2 \theta - \frac{1}{l_c^2}\theta = 0$). The solution natively yields the exact \textbf{Yukawa Potential}:
\begin{equation}
    V_{weak}(r) \propto \frac{e^{-r/l_c}}{r}
\end{equation}

\subsection{Deriving the Gauge Bosons ($W^\pm / Z^0$) as Acoustic Modes}
The gauge bosons of the weak interaction represent the fundamental macroscopic \textbf{acoustic cutoff excitations} required to mechanically induce a localized phase twist.
\begin{itemize}
    \item The charged $W^\pm$ bosons correspond to the pure longitudinal-torsional acoustic mode ($k \propto G_{vac}J$).
    \item The neutral $Z^0$ boson corresponds to the transverse-bending acoustic mode ($k \propto E_{vac}I$).
\end{itemize}

For a uniform cylindrical bond ($J=2I$), the exact geometric ratio of their acoustic cutoff rest masses is natively governed by the vacuum Poisson's ratio ($\cos \theta_W = 1/\sqrt{1+\nu_{vac}}$). By substituting the geometric Cosserat trace-reversed limit mathematically proven in Chapter \ref{ch:gravity_and_yield} ($\nu_{vac} \equiv 2/7$), the weak mixing angle emerges as an exact analytical prediction:
\begin{equation}
    \frac{m_W}{m_Z} = \frac{1}{\sqrt{1 + 2/7}} = \frac{1}{\sqrt{9/7}} = \mathbf{\frac{\sqrt{7}}{3}} \approx \mathbf{0.881917}
\end{equation}
This derivation matches the experimental ratio to within 0.05\% error, offering a direct mechanical origin for the mass splitting without invoking symmetry-breaking scalar fields.

\section{The Gauge Layer: From Topology to Symmetry}
The physical continuous connection between nodes is mathematically described by a unitary link variable $U_{ij}$. The simplest gauge-invariant geometric quantity is the 3-node triangular plaquette ($U_P = U_{ij}U_{jk}U_{ki}$). Expanding this topologically continuous loop via Taylor series natively recovers the Maxwell Lagrangian ($-\frac{1}{4}F_{\mu\nu}F^{\mu\nu}$). \textbf{U(1) Electromagnetism} represents the strict enforcement of unitary topological continuity across the discrete graph.

Furthermore, because the Borromean proton ($6^3_2$) consists of three topologically indistinguishable interlocked loops, its discrete mathematical permutation symmetry is exactly $S_3$. The continuous mathematical envelope required to locally parallel-transport the phase smoothly across a tri-partite symmetric graph is exactly the $SU(3)$ Lie group. \textbf{SU(3) Color Charge} is derived as the exact effective field theory limit of a three-loop topological defect traversing a discrete condensate grid.