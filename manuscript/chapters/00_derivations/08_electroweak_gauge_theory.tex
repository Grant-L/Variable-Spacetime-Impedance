% 08_electroweak_gauge_theory.tex
\chapter{Electroweak Mechanics and Gauge Symmetries}
\label{ch:electroweak}

\section{Electrodynamics: The Gradient of Topological Phase}
A localized charged node permanently exerts a continuous rotational phase twist ($\theta$) on the surrounding LC condensate. Because the unsaturated vacuum acts as a linear dielectric in the far-field, the static structural phase strain must strictly obey the 3D \textbf{Laplace Equation} ($\nabla^2 \theta = 0$).

The spherically symmetric geometric solution dictates that the twist amplitude decays exactly inversely with distance ($\theta(r) \propto 1/r$). The continuous electric displacement field ($\mathbf{D}$) is physically identical to the spatial gradient of this structural phase twist ($\mathbf{D} = \nabla\theta \propto -1/r^2 \mathbf{\hat{r}}$), analytically deriving Coulomb's Law.

\subsection{Magnetism as Convective Vorticity}
When a twisted node translates at a velocity $\mathbf{v}$, it induces a convective shear flow in the momentum field. In classical network dynamics, the time evolution of a translating steady-state strain field $\mathbf{D}(\mathbf{r} - \mathbf{v}t)$ is governed by the convective material derivative:
\begin{equation}
    \partial_t \mathbf{D} = -(\mathbf{v} \cdot \nabla)\mathbf{D} \implies \nabla \times (\mathbf{v} \times \mathbf{D})
\end{equation}
Equating this to the Maxwell-Ampere law derives the macroscopic magnetic field strictly from network dynamics: $\mathbf{H} = \mathbf{v} \times \mathbf{D}$.

This relationship is rigorously supported by dimensional analysis. Applying the topological conversion constant ($\xi_{topo} \equiv e/\ell_{node}$), the displacement field reduces to $[\mathbf{D}] = \xi_{topo}[1/\text{m}]$. Evaluating the cross product $[\mathbf{v} \times \mathbf{D}]$ yields strictly $\xi_{topo}[1/\text{s}]$. Standard SI units for magnetic field intensity $\mathbf{H}$ ($[\text{A/m}]$) identically reduce to this exact same dimensional basis ($\xi_{topo}[1/\text{s}]$). Magnetism is thereby dimensionally proven to represent the continuous kinematic vorticity of the vacuum condensate.

\subsection{The Inductive Origin of Gauge Invariance}
Standard Quantum Field Theory mandates that the vector potential is a gauge field, where transformations of the form $\mathbf{A} \to \mathbf{A} + \nabla \Lambda$ leave physical observables ($\mathbf{B}$ and $\mathbf{E}$) unchanged. A common critique of identifying $\mathbf{A}$ as a physical momentum field is that this gauge freedom would imply the unphysical, spontaneous shifting of macroscopic mass, violating Noether's theorem.

This paradox is resolved rigorously via the \textbf{Helmholtz Decomposition Theorem} in classical network dynamics. Any continuous vector field can be decomposed into a solenoidal (divergence-free) component and an irrotational (curl-free) component. Adding the gradient of a scalar field ($\nabla \Lambda$) to the mass flow strictly introduces a uniform, irrotational velocity potential to the background network.

Because the $\mathcal{M}_A$ vacuum is highly incompressible ($K = 2G$), an irrotational flow field generates no localized compression ($-\partial_t \mathbf{A}$), no transverse vorticity ($\nabla \times \mathbf{A}$), and no topological defects. It is physically isomorphic to performing a \textbf{Galilean or Lorentz coordinate boost} of the observer's reference frame. Gauge invariance is not violated; it is strictly revealed to be the classical network-dynamic freedom to shift the irrotational background coordinate velocity without altering the physical transverse observables.

\section{The Weak Interaction: Inductive Cutoff Dynamics}
In classical electrodynamics, the ratio of the LC network's microrotational bending inductance ($\gamma_c$) to the macroscopic optical shear modulus ($G_{vac}$) rigidly defines a fundamental \textbf{Characteristic Length Scale} ($l_c = \sqrt{\gamma_c/G_{vac}}$). This length scale is identified as the physical origin of the weak force range ($r_W \approx 10^{-18}$ m).

Weak interactions lack the kinetic energy required to overcome the ambient LC rotational inductance. Any physical excitation operating \textit{below} a medium's natural cutoff frequency is mathematically forced to become an \textbf{Evanescent Wave}. The static field equation transforms from the Laplace equation to the massive Helmholtz equation ($\nabla^2 \theta - \frac{1}{l_c^2}\theta = 0$). The solution natively yields the exact \textbf{Yukawa Potential}:
\begin{equation}
    V_{weak}(r) \propto \frac{e^{-r/l_c}}{r}
\end{equation}

\subsection{Deriving the Gauge Bosons (\texorpdfstring{$W^{\pm}/Z^{0}$}{W/Z}) as Evanescent Modes}

The gauge bosons of the weak interaction represent the fundamental macroscopic evanescent cutoff excitations required to mechanically induce a localized phase twist.

\begin{itemize}
\item The charged $W^{\pm}$ bosons correspond to the pure longitudinal-torsional evanescent mode ($k\propto G_{vac}J$).
\item The neutral $Z^{0}$ boson corresponds to the transverse-bending evanescent mode ($k\propto E_{vac}I$).
\end{itemize}

Because Axiom 1 strictly bounds the physical diameter of a fundamental flux tube to exactly $d \equiv 1 l_{node}$ (the hard-sphere exclusion limit), these topological connections mechanically act as volume-bearing physical 3D continuous cylinders at the macroscopic limit. Furthermore, because the tube is formed by a radially symmetric dielectric displacement field, the Perpendicular Axis Theorem strictly dictates that its polar moment of inertia evaluates exactly to $J=2I$. This is a geometric absolute for any circular cross-section, not an assumed relationship.

Because the rest mass of an evanescent cutoff mode scales exactly with the square root of its structural stiffness ($m \propto \sqrt{k}$), the mass ratio evaluates to $m_W/m_Z = \sqrt{GJ / EI}$. Substituting the fundamental cylinder geometry ($J=2I$) strictly yields $\sqrt{2G/E}$. Applying the standard isotropic elastic continuous identity ($E = 2G(1+\nu)$) mathematically reduces this stiffness ratio to:

\begin{equation}
\frac{m_W}{m_Z} = \sqrt{\frac{2G}{2G(1+\nu_{vac})}} = \frac{1}{\sqrt{1+\nu_{vac}}}
\end{equation}

By substituting the geometric Chiral LC trace-reversed limit mathematically proven in Chapter 4 ($\nu_{vac} \equiv 2/7$), the weak mixing angle emerges as an exact analytical prediction:

\begin{equation}
\frac{m_W}{m_Z} = \frac{1}{\sqrt{1+2/7}} = \frac{1}{\sqrt{9/7}} = \frac{\sqrt{7}}{3} \approx 0.881917
\end{equation}

This derivation matches the experimental ratio to within 0.05\% error, offering a direct mechanical origin for the mass splitting without invoking symmetry-breaking scalar fields.

\begin{figure}[h]
    \centering
    \includegraphics[width=0.95\textwidth]{electroweak_acoustic_modes.png}
    \caption{\textbf{The $W^{\pm} / Z^0$ Bosons: Acoustic Trace-Reversal.} The weak interaction is derived not as a force, but as the high-energy acoustic cutoff limit required to induce a localized phase twist in the dense vacuum. Because the fundamental topological flux tube operates macroscopically as an elastic physical cylinder, the solid mechanics of the $\mathcal{M}_A$ substrate necessitate two distinct vibrational modes: $W^{\pm}$ pure transverse torsion ($J_{\text{polar}}$) and $Z^0$ lateral bending ($I_{\text{area}}$). The fundamental identity $J=2I$ rigidly defines their rest mass ratio ($m_W / m_Z$), mechanically deriving the empirical Weak Mixing Angle without hypothetical scalar potentials.}
    \label{fig:electroweak_acoustic_modes}
\end{figure}

\section{The Gauge Layer: From Topology to Symmetry}
The physical continuous connection between nodes is mathematically described by a unitary link variable $U_{ij}$. The simplest gauge-invariant geometric quantity is the 3-node triangular plaquette ($U_P = U_{ij}U_{jk}U_{ki}$). Expanding this topologically continuous loop via Taylor series natively recovers the Maxwell Lagrangian ($-\frac{1}{4}F_{\mu\nu}F^{\mu\nu}$). \textbf{U(1) Electromagnetism} represents the strict enforcement of unitary topological continuity across the discrete graph.

Furthermore, because the Borromean proton ($6^3_2$) consists of three topologically indistinguishable interlocked loops, its discrete mathematical permutation symmetry is exactly $S_3$. The continuous mathematical envelope required to locally parallel-transport the phase smoothly across a tri-partite symmetric graph is exactly the $SU(3)$ Lie group. \textbf{SU(3) Color Charge} is derived as the exact effective field theory limit of a three-loop topological defect traversing a discrete condensate grid.