\section{Simulation: The Warp Bubble}
\label{sec:warp_sim}

\citestart To test the feasibility of Metric Refraction, we simulated a "Warp Bubble" where the local refractive index is driven to $\chi = 0.5$ (Figure \ref{fig:warp_bubble})[cite: 1198].

\begin{figure}[h]
    \centering
    \includegraphics[width=0.8\textwidth]{assets/sim_outputs/warp_bubble_result.png}
    \caption{\textbf{Superluminal Translation.} The heatmap shows the propagation of a signal packet. The white dashed line represents the background speed of light ($c$, slope=1). The green line traces the packet trajectory inside the engineered bubble ($v=1.5c$). \citestart Because the local impedance is lower, the signal covers more lattice nodes per update cycle than a background photon, effectively outrunning light without violating local causality[cite: 1200, 1201, 1202].}
    \label{fig:warp_bubble}
\end{figure}

The simulation confirms that $c$ is only a limit for the \textit{ground state} impedance $Z_0$. \citestart If $Z_{local}$ is artificially lowered, the local speed limit increases proportionally[cite: 1203]. \citestart The "ship" never exceeds its local light speed; it simply raises the speed limit of the road it is driving on[cite: 1204].