\section{The Principle of Local Refractive Control}
In previous chapters, we established that gravity and inertia are consequences of the vacuum's variable refractive index $n(r)$. The central thesis of Metric Engineering is that if $n$ is a physical property of the substrate (density), it can be modified locally by external fields.

We define \textbf{Metric Engineering} as the active modulation of the Lattice Stress Coefficient ($\sigma$) to alter the local Group Velocity ($v_{g}$) of the vacuum.

\subsection{The Lattice Stress Coefficient ($\sigma$)}
We define the local state of the vacuum by the stress parameter $\sigma$:
\begin{equation}
    n_{local} = n_{0} \cdot \sigma
\end{equation}

\begin{itemize}
    \item \textbf{Vacuum State ($\sigma=1$):} Standard empty space ($c$).
    \item \textbf{Compression ($\sigma>1$):} Increased node density. Light slows down. This is Artificial Gravity.
    \item \textbf{Rarefaction ($\sigma<1$):} Decreased node density. Light speeds up ($v_{g}>c$). This is the basis of Warp Mechanics.
\end{itemize}

% --- INSERT: DESIGN NOTE 11.1 (CAUSALITY BOX) ---
\begin{tcolorbox}[colback=white,colframe=red!75!black,title=Design Note 11.1: The Causal Limit (Front vs. Group Velocity)]
Crucially, while Metric Engineering permits the local Group Velocity ($v_g$) to exceed $c$ via rarefaction ($\sigma < 1$), this does not violate the fundamental causality of the hardware. We rigorously distinguish between three propagation velocities:

\begin{itemize}
    \item \textbf{Phase Velocity ($v_p$):} The rate at which the carrier wave ripples. Can arbitrarily exceed $c$ (e.g., in waveguides) without carrying information.
    \item \textbf{Group Velocity ($v_g$):} The rate at which the envelope of the wave packet moves. In regions of anomalous dispersion (or engineered vacuum rarefaction), $v_g$ may exceed $c$, appearing as "superluminal" translation of the vessel.
    \item \textbf{Front Velocity ($v_{front}$):} The speed of the leading edge of a signal (the first discontinuity). This is strictly bounded by the hardware update rate of the discrete lattice ($t_{tick}$).
\end{itemize}

\textbf{The Non-Signaling Theorem:}
\[ v_{front} = \lim_{\omega \to \infty} \frac{\omega}{k(\omega)} \equiv c_{asymptotic} \]
Even if a warp bubble translates at effective speed $v_{eff} > c$, the \textit{causal influence} (the "start" command) cannot propagate faster than the asymptotic slew rate of the naked substrate.
\end{tcolorbox}
% ------------------------------------------------