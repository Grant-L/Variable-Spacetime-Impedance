\section{Metric Streamlining: Reducing Inertial Mass}
\label{sec:metric_streamlining}

Standard physics treats inertia ($m$) as an immutable scalar. Vacuum Computational Fluid Dynamics (VCFD) reveals it as a dynamic drag force dependent on hull geometry ($C_d$) and local vacuum density. To reach relativistic speeds without requiring infinite energy, we must apply the principles of \textbf{Vacuum Aerodynamics}.

\subsection{The Dimensionally Exact Drag Coefficient ($C_d$)}
A moving physical object (a complex topological knot) creates a turbulent inductive wake in the lattice (Back-EMF). The dynamic force required to push it through the substrate is governed by the classical drag equation:
\begin{equation}
    F_{drag} = \frac{1}{2} \rho_{vac} v^2 C_d A_{cross} \quad [\text{Newtons}]
\end{equation}
Where:
\begin{itemize}
    \item $\rho_{vac} = u_{local}/c^2$: The effective kinematic mass density of the vacuum $[\text{kg}/\text{m}^3]$.
    \item $C_d$: The dimensionless Metric Drag Coefficient.
    \item $A_{cross}$: The cross-sectional interaction area of the topological defect $[\text{m}^2]$.
\end{itemize}
Because $\rho_{vac}$ is rigorously defined in SI mass density units, this equation evaluates flawlessly to Newtons $[\text{kg} \cdot \text{m} / \text{s}^2]$.

\begin{itemize}
    \item \textbf{Blunt Body ($C_d \approx 1$):} A standard baryonic mass generates a large turbulent wake, manifesting macroscopically as high inertial mass.
    \item \textbf{Streamlined Body ($C_d \ll 1$):} A hull shaped to guide vacuum flux around it laminarly reduces its effective inertial mass.
\end{itemize}

\subsection{Active Flow Control: The Metric "Dimple"}
Just as golf balls use dimples to energize the boundary layer and reduce drag, a relativistic vessel can utilize Metric Actuators. 

By emitting high-frequency toroidal shear fields ($\omega \gg \omega_{plasma}$) at the leading edge, the vessel "pre-stresses" the vacuum, triggering non-Newtonian shear-thinning. The vacuum fluid adheres to the hull surface (Laminar Flow) rather than separating into a turbulent wake. This effectively "lubricates" the spacetime trajectory, mechanically reducing the inertial mass of the vessel without violating conservation laws.