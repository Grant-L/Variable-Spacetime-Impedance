\section{Metric Streamlining: Reducing Inertial Mass}
Standard physics treats inertia ($m$) as an immutable scalar. Vacuum Computational Fluid Dynamics (VCFD) reveals it as a drag force dependent on geometry ($C_{d}$). To reach relativistic speeds without infinite energy cost, we must apply the principles of Vacuum Aerodynamics.

\subsection{The Inductive Drag Coefficient ($C_{d}$)}
A moving object creates a turbulent wake in the lattice (Back-EMF). The force required to push it is:
\begin{equation}
    F_{drag} = \frac{1}{2}\rho_{vac}v^{2}C_{d}A_{cross}
\end{equation}
Where $C_{d}$ is the Metric Drag Coefficient.

\begin{itemize}
    \item \textbf{Blunt Body ($C_{d}\approx1$):} A standard mass (proton/sphere) creates a large turbulent wake. High Inertia.
    \item \textbf{Streamlined Body ($C_{d}\ll1$):} A hull shaped to guide vacuum flux around it laminarly can reduce its effective mass.
\end{itemize}

\subsection{Active Flow Control: The Metric "Dimple"}
Just as golf balls use dimples to energize the boundary layer and reduce drag, a relativistic vessel can use Metric Actuators.

\textbf{Mechanism:} High-frequency toroidal emitters ($\omega \gg \omega_{plasma}$) placed at the leading edge can "pre-stress" the vacuum, lowering the local viscosity.

\textbf{Result:} The vacuum fluid adheres to the hull surface (Laminar Flow) rather than separating into a turbulent wake. This effectively "lubricates" the spacetime trajectory, reducing the inertial mass of the vessel.

\subsubsection{Naval Analogy: Supercavitation}
How do we reduce the inertial mass of a spacecraft? We apply the principles of Supercavitating Torpedoes to the vacuum.

\begin{enumerate}
    \item \textbf{Standard Flight (Viscous Drag):} A ship moving through the vacuum is like a boat hull moving through water. It drags a massive wake of lattice distortion ($m_{i}$).
    \item \textbf{Metric Streamlining (The Gas Bubble):} A supercavitating torpedo ejects gas from its nose to envelop itself in a bubble of low-density air. The hull never touches the water, reducing drag by 99\%.
\end{enumerate}

\textbf{AVE Application:} By emitting a high-frequency metric field ($\sigma < 1$) ahead of the ship, we create a "Vacuum Bubble." The ship slips through this rarefied pocket, effectively decoupling from the viscous inertia of the bulk universe.