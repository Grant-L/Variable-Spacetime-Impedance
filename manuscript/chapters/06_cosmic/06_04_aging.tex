\section{Metric Aging and Radioactive Decay}
\label{sec:metric_aging}

VSI posits that the rate of radioactive decay ($\lambda$) is not an immutable constant, but a frequency-dependent lattice response. The tunneling probability for an alpha particle depends on the "stiffness" (Impedance) of the vacuum barrier it must penetrate.

\begin{equation}
    \lambda(t) \propto \frac{1}{Z_{0}(t)}
    \label{eq:decay_rate}
\end{equation}

This implies that radioactive clocks (e.g., Carbon-14, Uranium-Lead) ran faster in the low-impedance past, as the "tunneling resistance" of the vacuum was lower.

\subsection{Recalibrating Chronometers}
Because $Z_0$ was lower in the past, the "ticks" of atomic clocks were compressed relative to cosmic time. Geological samples dated to billions of years ago may actually be younger when corrected for **Metric Aging**. Calibrating radiometric dates against the \textbf{Impedance Evolution Curve} (Eq 6.1) is a primary requirement for reconciling SVF with geological history.