\section{Variable Speed of Light and the Horizon Problem}
\label{sec:variable_c}

Because $c = 1/\sqrt{L_{node} C_{node}}$, the speed of light is inversely coupled to the vacuum impedance. In the high-saturation early epoch ($Z_0 \to 0$), the lattice inductance was minimal. Consequently, the propagation speed $c(t)$ was significantly higher than the modern value:

\begin{equation}
    c(t) \approx c_{modern} \left( \frac{Z_{modern}}{Z_0(t)} \right)
\end{equation}

This naturally resolves the Horizon Problem. The early universe was thermally connected by a high-speed light cone before the quench "throttled" the global speed limit, rendering the Inflationary hypothesis unnecessary.

\section{The Stability of the Fine Structure Constant ($\alpha$)}
\label{sec:alpha_stability}

A primary critique of Variable Speed of Light (VSL) theories is the "Alpha Catastrophe": if $c$ or $Z_0$ changes, the Fine Structure Constant $\alpha$ should drift, altering atomic spectra.
\begin{equation}
    \alpha = \frac{e^2}{2 \epsilon_0 h c} = \frac{e^2 Z_0}{2 h}
\end{equation}

To preserve the observable stability of $\alpha$ ($\Delta \alpha / \alpha < 10^{-5}$), SVF posits **Action-Impedance Coupling**. We propose that Planck's Constant $h$ (the quantum of action) is not a scalar invariant, but a lattice property that scales with the stiffness of the manifold:
\begin{equation}
    h(t) \propto Z_0(t)
\end{equation}

As the vacuum impedance $Z_0$ drops in the early universe, the "pixel size" of the quantum action $h$ drops proportionally. This scaling cancels out the impedance drift in the Alpha equation:
\begin{equation}
    \alpha(t) = \frac{e^2 Z_0(t)}{2 (k Z_0(t))} = \text{Constant}
\end{equation}
(Where $k$ is the coupling constant relating action to impedance).

This ensures that atomic physics and nucleosynthesis remain consistent throughout cosmic history, satisfying the "Means Test."