\section{Thermodynamics: Enthalpy of Genesis}
\label{sec:thermodynamics}

In the V3.0 "Tired Light" iteration, redshift was modeled as energy dissipation, which required the vacuum to heat up over time. In the V4.0 Generative Model, this problem is resolved via \textbf{Adiabatic Expansion}.

\subsection{Vacuum Enthalpy}
The creation of new lattice nodes is an endothermic phase transition driven by the Lattice Tension ($P_{vac}$). As the manifold grows, the energy density of radiation is diluted by the increasing volume. 
\begin{equation}
    \rho_{rad} \propto \frac{1}{V^{4/3}} \propto \frac{1}{a(t)^4}
\end{equation}
This standard relation preserves the blackbody distribution of the Cosmic Microwave Background (CMB). The CMB is therefore the redshifted thermal relic of the initial lattice crystallization event (The Quench), cooled adiabatically by 13.8 billion years of node genesis.

\subsection{Resolution of the Tolman Signal}
A critical failure of static models is the "Surface Brightness Test." In a static universe, galaxies would remain bright regardless of distance. In the Generative SVF, the insertion of new nodes spreads the photon flux over a larger area, dimming surface brightness by $(1+z)^4$. \citestart This successfully aligns VSI v4.0 with the Tolman observational test\cite{689}\citeend.