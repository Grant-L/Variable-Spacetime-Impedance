\section{Variable Speed of Light and the Horizon Problem}
\label{sec:variable_c}

Because $c = 1/\sqrt{L_{node} C_{node}}$, the speed of light is inversely coupled to the vacuum impedance. In the high-saturation early epoch ($Z_0 \to 0$), the lattice inductance $L_{node}$ was effectively nullified by the saturation current. Consequently, the propagation speed $c(t)$ was orders of magnitude higher than the modern value:

\begin{equation}
    c(t) \approx c_{modern} \left( \frac{Z_{modern}}{Z_0(t)} \right)
\end{equation}

This allows for the universe to establish thermal equilibrium (CMB homogeneity) naturally before the quench "throttled" the global speed limit to its current value. This renders the Inflationary Epoch unnecessary; the horizon was simply larger in the past due to lower impedance.

\section{The Stability of the Fine Structure Constant ($\alpha$)}
\label{sec:alpha_stability}

To pass the "Spectroscopic Audit," SVF must ensure that the Fine Structure Constant $\alpha$ remains relatively stable, even as $c$ changes.
\begin{equation}
    \alpha = \frac{e^2}{2 \epsilon_0 h c}
\end{equation}

In the $M_A$ substrate, $\epsilon_0$ (Capacitance) and $c$ (Slew Rate) shift in a coupled ratio dictated by the node geometry. As the vacuum relaxes, $C_{node}$ decreases while $L_{node}$ increases. This geometric coupling ensures that the product $\epsilon_0 c$ remains largely invariant, preserving atomic transition energies within observable limits.