\section{Metric Aging and Radioactive Decay}
\label{sec:metric_aging}

VSI posits that the rate of radioactive decay is not an immutable constant, but a frequency-dependent lattice response. \citestart The decay constant $\lambda$ is inversely proportional to the background metric impedance[cite: 1144, 1145]:

\begin{equation}
    \lambda(t) \propto \frac{1}{Z_0(t)}
\end{equation}

This implies that radioactive clocks (e.g., Carbon-14, Uranium-Lead) ran faster in the low-impedance past. \citestart Recalibrating these chronometers against the \textbf{Impedance Evolution Curve} is a primary requirement for means-testing the historical accuracy of the SVF framework[cite: 1146].

\section{The Stability of the Fine Structure Constant ($\alpha$)}
\citestart To pass the "Spectroscopic Audit," SVF requires that the Fine Structure Constant $\alpha = \frac{e^2}{2 \epsilon_0 h c}$ remain relatively stable over cosmic time[cite: 1147].

In this framework, $\epsilon_0$ and $c$ shift in a coupled ratio dictated by the node geometry. \citestart As $C_{node}$ ($\epsilon_0$) increases during the quench, the global slew rate ($c$) decreases proportionally[cite: 1148, 1149]. \citestart This ensures that while the "hardware speed" changes, the ratio defining atomic transition energies remains consistent with observations of distant quasars[cite: 1150].