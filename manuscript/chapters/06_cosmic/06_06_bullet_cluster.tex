\subsection{6.7 Simulation: The Bullet Cluster Reconstruction}
The "Smoking Gun" for Dark Matter is the Bullet Cluster (1E 0657-558), where the gravitational potential (blue) is observed to separate from the visible X-ray gas (red) after a collision. The Standard Model attributes this to collisionless Dark Matter particles.

We successfully reconstructed this phenomenon using the \texttt{BulletClusterSim\_v2} engine, relying solely on the **Inductive Inertia** of the VSI substrate.

\textbf{Mechanism:}
\begin{itemize}
    \item \textbf{Gas (Diffuse):} Low topological complexity ($N=1$). The particles interact via direct lattice friction (Viscosity $\eta > 0$), converting kinetic energy into heat (X-rays) and stalling at the impact site.
    \item \textbf{Galaxies (Condensed):} High topological complexity ($N \sim 10^{57}$). Due to the $N^9$ scaling law (Section 4.3.3), the Inductive Mass $M_{ind}$ dominates the interaction. The "Back-EMF" of the galaxies is so immense that the vacuum viscosity is negligible.
\end{itemize}

\textbf{Result:}
As shown in Figure \ref{fig:bullet_cluster}, the galaxies (carrying the majority of the inductive mass/gravity) separate from the gas naturally. This proves that "Dark Matter" is simply the observation of \textbf{High-Q Inductive Inertia} in condensed matter.

\begin{figure}[h]
    \centering
    \includegraphics[width=0.9\linewidth]{assets/sim_outputs/bullet_cluster_v2.png}
    \caption{\textbf{VSI Bullet Cluster Reconstruction.} The red scatter plot shows the X-Ray gas slowing down due to lattice viscosity. The blue contours show the gravitational potential following the galaxies, which plow through the vacuum due to their extreme Inductive Inertia ($N^9$). This separation arises mechanically, without requiring non-baryonic Dark Matter.}
    \label{fig:bullet_cluster}
\end{figure}