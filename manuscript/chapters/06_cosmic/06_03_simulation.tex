\section{Simulation: The Hubble Pulse}
\label{sec:hubble_pulse_sim}

To test the Quench Hypothesis, we modeled the expansion history of the universe using the \texttt{CosmicQuenchSim} module. \citestart The simulation tracks the Hubble Parameter $H(t)$ as the vacuum impedance $Z_0(t)$ transitions from a low-energy primordial state to the high-impedance modern state [cite: 561-563].

\begin{figure}[h]
    \centering
    \includegraphics[width=0.8\textwidth]{assets/sim_outputs/cosmic_quench_result.png}
    \caption{\textbf{The Cosmic Quench.} The red dashed line shows the evolution of vacuum impedance $Z_0$. The blue line tracks the expansion rate $H(t)$. Note the "bump" or pulse in expansion rate at the transition point. This acceleration corresponds to the release of Latent Heat from the manifold, which standard cosmology misidentifies as "Dark Energy."}
    \label{fig:cosmic_quench}
\end{figure}

The result demonstrates that the observed acceleration of the universe is consistent with a global phase transition. \citestart As the manifold relaxes, it sheds energy (Latent Heat), driving the expansion[cite: 83].