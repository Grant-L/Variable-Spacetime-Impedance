\section{Simulation: The Hubble Pulse}
\label{sec:hubble_sim}

To test the Quench Hypothesis, we modeled the expansion history of the universe using the \texttt{CosmicQuenchSim} module. The simulation tracks the Hubble Parameter $H(t)$ driven by the Vacuum Pressure term derived in Eq \ref{eq:vacuum_pressure}.

\begin{figure}[h]
    \centering
    \includegraphics[width=1.0\textwidth]{assets/sim_outputs/cosmic_quench_result.png}
    \caption{\textbf{The Cosmic Quench \& Latent Heat Release.} 
    \textbf{Red Dashed Line:} The evolution of Vacuum Impedance $Z_0$ rising from $0\Omega$ to $377\Omega$.
    \textbf{Blue Line:} The Hubble Parameter $H(t)$. Note the distinct "bump" in expansion rate. This acceleration corresponds to the peak release of Latent Heat ($P_{vac}$) during the phase transition, which standard cosmology misidentifies as Dark Energy.}
    \label{fig:cosmic_quench}
\end{figure}

The result demonstrates that the observed acceleration of the universe is not caused by a new field, but is the thermodynamic signature of the vacuum settling into its ground state.