\section{Simulation: Genesis vs. Dark Energy}
\label{sec:genesis_sim}

To validate the Generative Cosmology model, we simulated the redshift-distance relation predicted by Lattice Genesis and compared it against the standard $\Lambda$CDM (Dark Energy) model.

\subsection{Methodology}
We define the \textbf{Genesis Rate} $R_g$ derived from the local Hubble constant $H_0 = 70$ km/s/Mpc.
\begin{equation}
    R_g = H_0 \approx 2.3 \times 10^{-18} \text{ Hz}
\end{equation}

We calculate the predicted Redshift ($z$) for a source at distance $D$ assuming exponential node proliferation:
\begin{equation}
    z_{VSI} = e^{\frac{R_g D}{c}} - 1
\end{equation}

\subsection{Results: The Illusion of Acceleration}
The simulation results (Figure \ref{fig:viscous_redshift}) reveal a critical insight. While linear metric expansion would follow a straight line, the VSI \textbf{Exponential Growth} function naturally curves upward at high distances.

\begin{figure}[h]
    \centering
    \includegraphics[width=1.0\textwidth]{assets/sim_outputs/viscous_redshift.png}
    \caption{\textbf{Genesis Mimics Dark Energy.} 
    \textbf{Blue Line:} The standard $\Lambda$CDM model requiring 70\% Dark Energy.
    \textbf{Red Dashed Line:} The VSI Generative Vacuum model with zero Dark Energy.
    \textit{Note:} The exponential nature of geometric growth ($e^{R_g t}$) produces an upward curve indistinguishable from "acceleration" for $z < 1.5$. This suggests that Dark Energy is an artifact of fitting a linear expansion model to a non-linear generative process.}
    \label{fig:viscous_redshift}
\end{figure}

\subsection{Conclusion}
The "accelerating expansion" of the universe is identified as the signature of \textbf{Geometric Growth}. The lattice is not merely stretching; it is multiplying.