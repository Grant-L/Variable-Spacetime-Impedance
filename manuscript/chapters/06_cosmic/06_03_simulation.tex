\section{Simulation: Viscosity vs. Dark Energy}
\label{sec:viscosity_sim}

To validate the Dissipative Cosmology model, we simulated the redshift-distance relation predicted by the Vacuum Viscosity hypothesis and compared it against the standard $\Lambda$CDM (Dark Energy) model.

\subsection{Methodology}
We define the Vacuum Viscosity Coefficient $\gamma$ derived from the local Hubble constant $H_0 = 70$ km/s/Mpc.
\begin{equation}
    \gamma = \frac{H_0}{c} \approx 2.33 \times 10^{-18} \text{ m}^{-1}
\end{equation}

We calculate the predicted Redshift ($z$) for a source at distance $D$ using two models:
\begin{itemize}
    \item \textbf{Standard Model ($\Lambda$CDM):} Assumes metric expansion with $\Omega_m = 0.3, \Omega_\Lambda = 0.7$.
    \item \textbf{VSI Model (Viscosity):} Assumes static space with photon energy dissipation:
    \begin{equation}
        z_{VSI} = e^{\frac{\gamma D}{c}} - 1
    \end{equation}
\end{itemize}

\subsection{Results: The Illusion of Acceleration}
The simulation results (Figure \ref{fig:viscous_redshift}) reveal a critical insight. While linear Hubble expansion would follow a straight line ($z \propto D$), the VSI exponential decay function naturally curves upward at high distances.

\begin{figure}[h]
    \centering
    \includegraphics[width=1.0\textwidth]{assets/sim_outputs/viscous_redshift.png}
    \caption{\textbf{Viscosity Mimics Dark Energy.} 
    \textbf{Blue Line:} The standard $\Lambda$CDM model requiring 70\% Dark Energy.
    \textbf{Red Dashed Line:} The VSI Viscous Vacuum model with zero Dark Energy.
    \textit{Note:} The exponential nature of energy loss ($e^{\gamma D}$) produces an upward curve indistinguishable from "acceleration" for $z < 1.5$. This suggests that Dark Energy is an artifact of fitting a linear expansion model to a non-linear dissipative process.}
    \label{fig:viscous_redshift}
\end{figure}

\subsection{Conclusion}
The "accelerating expansion" of the universe is identified as a mathematical artifact. It is the observational signature of the non-linearity of the exponential decay function at cosmological scales.