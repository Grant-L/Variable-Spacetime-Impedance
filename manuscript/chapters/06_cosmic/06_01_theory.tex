\section{Lattice Viscosity \& The Hubble Illusion}
\label{sec:hubble_viscosity}

We abandon the V2.0 hypothesis that atomic spectra or fundamental constants ($c, h$) have evolved over time. Instead, we treat the redshift as a propagation effect.

\subsection{The Viscosity Coefficient ($\gamma$)}
As an electromagnetic wave propagates through the lattice, it performs work on the nodes, experiencing a small, continuous energy loss due to the vacuum's finite viscosity. We define the \textbf{Vacuum Viscosity Coefficient} $\gamma$ ($m^{-1}$).

The energy of a photon is $E = hf$. The power loss over distance $dx$ is proportional to the energy carried:
\begin{equation}
    \frac{dE}{dx} = - \frac{\gamma}{c} E
\end{equation}

Since Planck's constant $h$ is invariant in V3.0, the frequency $f$ must decay:
\begin{equation}
    \frac{df}{dx} = - \frac{\gamma}{c} f \implies f_{obs} = f_{emit} e^{-\frac{\gamma D}{c}}
\end{equation}

\subsection{Deriving the Hubble Constant}
Redshift is defined as $z = \Delta f / f$. For local galaxies (small distance $D$), we Taylor expand the exponential decay:
\begin{equation}
    z \approx \frac{\gamma D}{c}
\end{equation}

Comparing this to the empirical Hubble Law ($z = \frac{H_0 D}{c}$), we identify the Hubble Constant not as an expansion rate, but as the \textbf{Viscosity Rate} of the vacuum:
\begin{equation}
    H_0 \equiv \gamma_{viscosity} \cdot c
\end{equation}

This derivation mathematically reproduces the linear redshift-distance relation observed in local cosmology. The "acceleration" of the universe at high $z$ is simply the non-linearity of the exponential decay function ($e^{-\gamma D/c}$) becoming significant at large distances.