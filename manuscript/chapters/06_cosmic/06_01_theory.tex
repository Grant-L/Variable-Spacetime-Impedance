\section{Generative Cosmology: The Crystallizing Vacuum}
\label{sec:generative_cosmology}

\subsection{The Failure of Static Viscosity}
The "Tired Light" model (VSI v3.0) successfully predicted the redshift-distance relation but failed the \textbf{Supernova Time Dilation Test}. Observations of Type Ia supernovae confirm that distant events are temporally dilated by a factor of $(1+z)$. This is geometrically impossible in a static universe, regardless of viscosity.

\subsection{The Lattice Genesis Hypothesis}
VSI v4.0 asserts that the vacuum manifold $M_A$ is not static, but \textbf{Generative}.
We identified in Chapter 2 that the vacuum possesses a Lattice Tension ($P_{vac}$). We propose that this tension drives a continuous phase transition: the crystallization of new lattice nodes from the underlying substrate.

\subsection{Derivation of the Genesis Rate ($R_g$)}
Let $N(t)$ be the total number of nodes along a line of sight. The Lattice Tension induces a proliferation of nodes proportional to the existing volume (a geometric growth series):
\begin{equation}
\frac{dN}{dt} = R_g N(t)
\end{equation}
Where $R_g$ is the \textbf{Node Genesis Rate} (Hz).

Solving for $N(t)$:
\begin{equation}
N(t) = N_0 e^{R_g t}
\end{equation}

\subsection{Recovering the Hubble Parameter}
The physical distance $D$ is simply the node count $N$ times the Lattice Pitch $l_P$:
\begin{equation}
D(t) = N(t) \cdot l_P
\end{equation}
The recession velocity $v$ is the rate of change of distance:
\begin{equation}
v = \frac{dD}{dt} = l_P \frac{dN}{dt} = l_P (R_g N) = R_g D
\end{equation}
Comparing this to Hubble's Law ($v = H_0 D$), we identify the Hubble Constant as the Genesis Rate:
\begin{equation}
H_0 \equiv R_{genesis}
\end{equation}

\subsection{Solving the Supernova Clock}
Because new nodes are inserted into the path \textit{during} the photon's transit, the wavelength $\lambda$ is mechanically stretched by the ratio of the node count at reception vs. emission:
\begin{equation}
1+z = \frac{N(t_{obs})}{N(t_{emit})}
\end{equation}
This mechanical insertion of space dilates both the wavelength (Redshift) and the wave-train duration (Time Dilation) identically.
\textbf{Result:} VSI v4.0 recovers the $(1+z)$ Supernova Timing signal while retaining the hardware-based derivation of Dark Energy (Lattice Tension driving Genesis).