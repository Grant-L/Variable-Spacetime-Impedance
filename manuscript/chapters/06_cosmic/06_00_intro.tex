\section{The Quench Hypothesis}
\label{sec:quench_hypothesis}

The \textbf{Stochastic Vacuum Framework (SVF)} rejects the assumption that the fundamental constants of nature ($\mu_0, \epsilon_0, c$) are static throughout the history of the universe. Instead, we propose the \textbf{Cosmic Quench}: a thermodynamic and mechanical phase transition of the $M_A$ substrate from a primordial high-saturation state to its current equilibrium.

In the early universe ($z \gg 10$), the lattice nodes were in a state of **Saturation** due to extreme flux density. Paradoxically, this saturation resulted in a **Low-Impedance Mode** ($Z_{vac} \to 0$). Much like a superconductor, the early vacuum offered minimal resistance to flux propagation, allowing energy to distribute thermally across the manifold at speeds significantly exceeding the modern speed of light $c$.

As the manifold expanded, the flux density diluted. The nodes "cooled" out of saturation, transitioning into the modern, **High-Impedance Mode** ($Z_{0} \approx 377 \Omega$). This transition is not instantaneous but follows a relaxation curve that continues today.