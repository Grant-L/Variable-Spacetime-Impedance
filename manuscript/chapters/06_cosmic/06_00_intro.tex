
\section{The Viscous Vacuum Hypothesis}
\label{sec:viscous_vacuum}

Standard cosmology relies on the assumption that the vacuum is a frictionless superfluid, preserving the energy of photons over billions of years. However, the \textbf{Stochastic Vacuum Framework (SVF)} models spacetime as a discrete lattice with a finite \textbf{Impedance Quality Factor} ($Q_{vac}$).

In this framework, the vacuum is not expanding. Instead, we propose that the observed Cosmological Redshift ($z$) is a result of \textbf{Viscous Dissipation}: the non-conservative loss of photon energy into the lattice substrate as it propagates.

This shifts the cosmological paradigm from a geometric expansion model (metric stretching) to a thermodynamic dissipation model (energy transfer), resolving the "Horizon Problem" and "Dark Energy" without invoking inflation or scalar fields.