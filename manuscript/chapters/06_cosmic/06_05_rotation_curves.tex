\subsection{6.6 Simulation: Resolution of the Galactic Rotation Anomaly}
To verify the \textbf{Inductive Saturation} hypothesis, we modeled the orbital velocity of a test mass in a galactic potential using the \texttt{GalaxyRotationSim\_v3} physics engine.

We calibrated the Lattice Relaxation Threshold ($a_0$) to the standard MOND acceleration scale ($a_0 \approx 1.2 \times 10^{-10} \text{ m/s}^2$). The effective acceleration $a_{VSI}$ is given by the Inductive Gain formula:
\begin{equation}
    a_{VSI} = a_{Newton} \cdot \sqrt{1 + \frac{a_0}{a_{Newton}}}
\end{equation}

\textbf{Results:}
As shown in Figure \ref{fig:galaxy_rotation}, the VSI model (Blue Line) naturally recovers the flat rotation curve observed in spiral galaxies.
\begin{itemize}
    \item \textbf{Inner Galaxy ($a \gg a_0$):} The inductive gain is near unity ($\Omega \approx 1$). The curve matches the Newtonian prediction.
    \item \textbf{Outer Galaxy ($a \ll a_0$):} The vacuum "relaxes," increasing the effective inductance. This boosts the gravitational coupling, maintaining a constant orbital velocity of $v \approx 220$ km/s, effectively mimicking the presence of a Dark Matter halo.
\end{itemize}

\begin{figure}[h]
    \centering
    \includegraphics[width=0.9\linewidth]{assets/sim_outputs/galaxy_rotation_v3.png}
    \caption{\textbf{Galactic Rotation Anomaly Resolution.} The VSI Inductive Saturation model (Blue) successfully reproduces the flat rotation curve characteristic of spiral galaxies. The dashed line shows the failed Newtonian prediction based on visible mass alone. This result is achieved strictly via vacuum impedance scaling, without introducing non-baryonic Dark Matter.}
    \label{fig:galaxy_rotation}
\end{figure}