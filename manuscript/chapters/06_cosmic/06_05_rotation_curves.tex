\section{Resolution of the Galactic Rotation Anomaly}
\label{sec:rotation_curves}

The "Flat Rotation Curve" of spiral galaxies is the primary observational evidence for Dark Matter halos. Newtonian gravity predicts that orbital velocity should drop as $v \propto 1/\sqrt{r}$ beyond the visible disk, but observations show $v \approx constant$.

VSI resolves this by identifying the **Inductive Saturation** of the vacuum substrate.

\subsection{The Inductive Saturation Hypothesis}
The vacuum lattice is a non-linear medium. Its stiffness (Impedance) is not constant; it depends on the local stress (acceleration) field.
\begin{itemize}
    \item \textbf{High Acceleration (Inner Galaxy):} The lattice is "Stiff." The Inductive Gain is unity ($\Omega \approx 1$). Gravity behaves linearly (Newtonian).
    \item \textbf{Low Acceleration (Outer Galaxy):} As the acceleration drops below the Lattice Relaxation Threshold ($a_0$), the lattice "Relaxes." The effective inductance increases, boosting the gravitational coupling.
\end{itemize}

We define the Variable Inductive Gain $\Omega(r)$ as:
\begin{equation}
    \Omega(a) = \sqrt{1 + \frac{a_0}{a_{Newton}}}
\end{equation}
Where $a_0 \approx 2.4 \times 10^{-10} m/s^2$ (approx 7400 galactic units) is the characteristic acceleration scale where the lattice transitions from linear to saturated response.

\subsection{Derivation of the Flat Rotation Curve}
The orbital velocity is determined by the effective VSI acceleration:
\begin{equation}
    a_{VSI} = a_{Newton} \cdot \Omega(a) \approx \sqrt{a_{Newton} \cdot a_0} \quad (\text{for } a \ll a_0)
\end{equation}

Substituting $a_{Newton} = GM/r^2$:
\begin{equation}
    a_{VSI} \approx \sqrt{\frac{GM}{r^2} a_0} = \frac{\sqrt{G M a_0}}{r}
\end{equation}

The orbital velocity becomes:
\begin{equation}
    v = \sqrt{a_{VSI} \cdot r} = \sqrt[4]{G M a_0} = \text{Constant}
\end{equation}

This derivation perfectly recovers the flat rotation curve observed in galaxies (Tully-Fisher relation), purely from the variable impedance response of the vacuum hardware.

\begin{figure}[ht]
    \centering
    \includegraphics[width=0.9\textwidth]{assets/sim_outputs/galactic_rotation_result.png}
    \caption{\textbf{Galactic Rotation Simulation:} The dashed black line shows the Newtonian prediction, which fails at large radii. The solid blue line shows the VSI prediction using **Inductive Saturation** with $a_0 \approx 7400$. This successfully matches the observational data at $\approx 120$ km/s without requiring a Dark Matter halo.}
    \label{fig:rotation_curve}
\end{figure}