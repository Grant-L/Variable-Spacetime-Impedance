\section{The Lensing Theorem: Deriving the Schwarzschild Metric}

A historic challenge for scalar gravity theories is their inability to predict the full deflection of light around a star. A pure 1D scalar metric ($n_{scalar} = 1 + GM/rc^2$) natively yields only half of the required deflection (The Newtonian Deflection). In the AVE framework, the full Einstein deflection emerges strictly and perfectly from the exact physical \textbf{Poisson's Ratio} of the Cosserat solid.

\subsection{The Transverse Optical Index ($n_\perp$)}
Unlike massive particles, a photon is a purely transverse, massless shear wave. It does not couple to the isotropic bulk scalar volume; it couples \textit{exclusively} to the transverse spatial strain of the solid. 

In classical continuum mechanics, the transverse metric strain ($h_\perp$) induced by a primary volumetric strain ($\chi_{vol}$) is governed exactly by the solid's \textbf{Poisson's Ratio} ($\nu_{vac}$). As derived from first principles in Chapter 1, the perfect trace-reversal stabilization of the Cosserat vacuum rigidly locks this ratio to exactly $\nu_{vac} \equiv 2/7$.

Therefore, the transverse metric strain physically perceived exclusively by light is identically:
\begin{equation}
    h_\perp = \nu_{vac} \chi_{vol}(r) = \frac{2}{7} \left( \frac{7GM}{c^2 r} \right) = \mathbf{\frac{2GM}{c^2 r}}
\end{equation}

This solid-mechanics transformation proves that the effective refractive index governing transverse optical photons is natively:
\begin{tcolorbox}[colback=white, colframe=black]
\begin{equation}
    n_\perp(r) = 1 + h_\perp = \mathbf{1 + \frac{2GM}{c^2 r}}
\end{equation}
\end{tcolorbox}
The exact Schwarzschild weak-field optical metric ($1 + 2\Phi$) is derived flawlessly from classical continuum mechanics without a single free parameter. The famous Einstein ``factor of 2'' is not a mystic property of warped 4D geometry; it is the strict mathematical manifestation of a transverse shear wave refracting through a Cosserat solid possessing a $\nu=2/7$ Poisson's ratio.

\subsection{Deflection of Light (Einstein Bending)}
With the transverse refractive profile $n_\perp(r)$ rigorously derived, we calculate the bending of light purely via Snell's Law and Huygens' Principle.

Consider a photon passing a mass $M$ with impact parameter $b$. In AVE, light curves not because "space is bent," but because the wavefront phase velocity is physically slower in the denser compressed lattice near the mass ($v = c/n_\perp$), causing the ray to physically refract inward. The deflection angle ($\delta$) is governed exactly by the spatial gradient of the refractive index perpendicular to the path ($\nabla_\perp n_\perp$):
\begin{equation}
    \delta = \int_{-\infty}^{\infty} \nabla_\perp n_\perp \, dz = \int_{-\infty}^{\infty} \frac{2GM}{c^2} \frac{b}{(b^2 + z^2)^{3/2}} \, dz
\end{equation}
Evaluating this standard optical geometric line-integral smoothly and analytically yields exactly:
\begin{equation}
    \delta = \mathbf{\frac{4GM}{bc^2}}
\end{equation}
This perfectly and mechanically recovers the exact Einstein deflection angle solely through fluidic refraction (see Figure \ref{fig:tensor_lensing}).

\begin{figure}[htbp]
    \centering
    \includegraphics[width=0.9\textwidth]{chapters/07_gravitation_metric_refraction/simulations/outputs/tensor_lensing.png}
    \caption{\textbf{Gravitational Lensing: Emergence of the Schwarzschild Metric.} A purely scalar coupling yields only half the required optical deflection. The full Einstein deflection natively emerges when the transverse photon wave is properly coupled to the transverse physical strain of the lattice, governed entirely by the $\nu_{vac} \equiv 2/7$ Poisson Ratio of the Cosserat solid.}
    \label{fig:tensor_lensing}
\end{figure}

\subsection{Shapiro Delay (The Refractive Transit Delay)}
The apparent "slowing" of light as it passes near a massive body is measured observationally as the Shapiro time delay ($\Delta t$). In the AVE framework, this is not the dilation of a metaphysical "time" dimension; it is simply the literal physical transit time integral of a mechanical wave traversing a denser dielectric fluid medium.
\begin{equation}
    \Delta t = \int_{path} \left(\frac{1}{v(r)} - \frac{1}{c}\right) dl = \frac{1}{c} \int_{path} (n_\perp(r) - 1) dl
\end{equation}

Substituting $n_\perp(r) = 1 + \frac{2GM}{rc^2}$ recovers the exact empirical continuous Shapiro Delay:
\begin{equation}
    \Delta t \approx \frac{4GM}{c^3} \ln\left(\frac{4 x_e x_p}{b^2}\right)
\end{equation}
This confirms unequivocally that the Shapiro Delay is identically a Dielectric Delay. The spatial vacuum near the sun is physically "thicker," mechanically increasing the node-to-node topological phase-transport time.