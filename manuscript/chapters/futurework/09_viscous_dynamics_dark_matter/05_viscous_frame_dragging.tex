\section{The Flyby Anomaly: Viscous Frame Dragging}

Spacecraft performing precise gravity-assist maneuvers past Earth often exhibit a small but highly distinct, unexplained macroscopic velocity shift ($\Delta v \approx$ mm/s). The Standard Model and standard General Relativity continually fail to explain this phenomenon via standard conservative gravitational fields. The AVE framework natively identifies this anomaly as a direct, localized macroscopic measurement of the \textbf{Kinematic Viscosity} of the vacuum entrained near a rapidly rotating mass.

As established by standard fluid dynamics (and paralleling the Lense-Thirring effect in GR), a massive rotating celestial body physically and fluidically drags the local viscous vacuum substrate along with its rotation (Fluid Entrainment). A spacecraft entering this localized shear zone couples directly to the viscous vorticity flow ($\mathbf{\Omega}_{vac} = \nabla \times \mathbf{v}_{vac}$) of the substrate. 

The thermodynamic kinetic energy transfer to the spacecraft is strictly non-zero because the un-yielded vacuum, while possessing extremely low friction in the solar regime, still formally possesses a non-zero Lattice Viscosity ($\eta_{vac}$):
\begin{equation}
    \Delta E_{craft} = \int \eta_{vac} \left( \mathbf{v}_{craft} \cdot \mathbf{\Omega}_{vac} \right) dt
\end{equation}

If the spacecraft executes a \textit{prograde} flyby, it moves kinematically \textit{with} the local rotational vacuum fluid flow, physically reducing the expected mechanical drag and appearing to Earth-bound observers as an anomalous orbital energy and velocity gain. Conversely, a \textit{retrograde} flyby forces the craft to move directly \textit{against} the entrained fluid flow, structurally increasing viscous drag and resulting in an anomalous energy loss. 

The Flyby Anomaly is not a glitch in NASA telemetry; it is a direct, localized laboratory-scale measurement of the exact continuous fluid dynamics and visco-elastic entrainment that generate the phenomenon of Dark Matter at the macroscopic galactic scale.