\section{VCFD Benchmark: Discrete Graph Calculus}

To computationally and rigorously validate the foundational continuous VCFD model against classical fluidics, we analytically evaluate the classical ``Lid-Driven Cavity'' fluid dynamics benchmark utilizing the exact topological discrete operators of the physical $\mathcal{M}_A$ graph.

Rather than relying blindly on abstract continuous partial differential equations (which inherently and mathematically break down at the $l_{node}$ limit), the true micro-physics of the vacuum must be computationally evaluated via exact finite-difference operations across adjacent, discrete spatial nodes. The continuous graph divergence ($\mathbf{D}$) and gradient ($\mathbf{G}$) matrices meticulously map scalar potentials from the discrete nodes to the connecting flux edges, strictly mathematically conserving local topological flux. 

The discrete graph Laplacian operator ($\mathbf{L} = \mathbf{D} \mathbf{G}$) allows us to mathematically solve the continuous Pressure-Poisson equation exactly and stably on the underlying $\mathcal{M}_A$ discrete hardware framework:
\begin{equation}
    \mathbf{L} P^{n+1} = \frac{\rho_{bulk}}{\Delta t} \mathbf{D} \mathbf{u}^*
\end{equation}

Where $\mathbf{u}^*$ is the transient intermediate fluid velocity field. Evaluating this purely algebraic, deterministic matrix equation under a constant kinematic applied shear from a moving spatial boundary, using the rigorously derived vacuum viscosity ($\nu_{vac} \approx 8.45 \times 10^{-7} \text{ m}^2/\text{s}$), flawlessly and deterministically generates a stable, macroscopic central continuous fluidic vortex. 

In the AVE physical theoretical framework, this macroscopic rotational fluidic stability is identically the required exact mechanical and hydrodynamic precursor strictly necessary to establish the non-linear Topological Matter geometries (Spin) derived in Chapters 3 and 4.