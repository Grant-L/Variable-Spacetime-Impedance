\section{Topological Fractionalization: The Origin of Quarks}

A stringent requirement for any unified model of the Proton is the derivation of fractional electric charges for its constituent quarks ($+2/3, +2/3, -1/3$). In the AVE framework, where charge is defined strictly as an integer topological Winding Number ($N \in \mathbb{Z}$), true continuous fractional twists are mechanically forbidden, as they would permanently tear the continuous $\mathcal{M}_A$ manifold.

\subsection{Falsification of Geometric ``Stenciling''}
Earlier hypotheses suggested these fractions arose because the loops physically ``stenciled'' or blocked 1/3 or 2/3 of the macroscopic solid angle. This classical analogy fails at the hardware level, where charge is strictly governed by the discrete Aharonov-Bohm phase topology, not optical shadow-casting.

\subsection{Rigorous Derivation: The Witten Effect and $\mathbb{Z}_3$ Symmetry}
We resolve the fractional charge paradox cleanly via the rigorous mathematics of \textbf{Topological Fractionalization} on a highly frustrated discrete graph.

The proton possesses a total, strictly integer effective electric charge topological winding number of $Q_{total} = +1e$. However, this integer flux is trapped within the tri-partite symmetry of the $6^3_2$ Borromean linkage. Because the three loops are topologically entangled such that the removal of any one loop unlinks the others, the total global phase twist is forcibly distributed across a degenerate structural ground state.

In a non-linear dielectric substrate, a composite topological defect with internal permutation symmetry natively generates a discrete CP-violating $\theta$-vacuum phase. By the exact application of the \textbf{Witten Effect}, a topological magnetic defect embedded in a $\theta$-vacuum mathematically acquires a fractionalized effective electric charge shift proportional to its phase angle:
\begin{equation}
    q_{eff} = n + \frac{\theta}{2\pi}e
\end{equation}

As proven in Figure \ref{fig:proton_borromean}, the $6^3_2$ Borromean linkage possesses a strict three-fold permutation symmetry ($\mathbb{Z}_3$). This rigid topological constraint restricts the allowed degenerate phase angles of the local trapped vacuum strictly to perfect mathematical thirds: 
\begin{equation}
    \theta \in \left\{0, \pm\frac{2\pi}{3}, \pm\frac{4\pi}{3}\right\}
\end{equation}

Substituting these precise discrete topological $\mathbb{Z}_3$ angles into the Witten charge equation rigorously and inescapably yields the exact effective fractional charges observed in nature:
\begin{tcolorbox}[colback=white, colframe=black]
\begin{equation}
    q_{eff} \in \left\{\pm \frac{1}{3}e, \pm \frac{2}{3}e\right\}
\end{equation}
\end{tcolorbox}

\textbf{Conclusion:} Quarks are not independent fundamental point-particles possessing intrinsically fractional hardware charges. They are strictly \textit{deconfined topological quasiparticles} emerging from a heavily frustrated topology. The global integer hardware charge of the proton ($+1e$) is mathematically partitioned by the fundamental group $\pi_1$ of the Borromean knot complement.