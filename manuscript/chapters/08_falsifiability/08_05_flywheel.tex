\section{Experimental Proposal: The Rotational Lattice Viscosity Experiment (RLVE)}

\subsection{Abstract}
The Stochastic Vacuum Framework (VSI) posits that the vacuum is not a geometric abstraction but a constitutive hardware substrate ($M_A$) with finite saturation limits. We propose a laboratory test to detect the \textbf{Lattice Viscosity} ($\eta_{vac}$) of this substrate. By rapidly rotating a high-density Tungsten mass adjacent to a high-finesse Fabry-Perot interferometer, we aim to induce a localized saturation of the vacuum dielectric, creating a measurable refractive index shift ($\Delta n$).

\subsection{Theoretical Derivation: The Viscous Drag Effect}
In standard General Relativity, the coupling of angular momentum to spacetime (Frame Dragging) is governed by the gravitational constant $G$, resulting in effects too small for laboratory detection ($\Delta \phi \approx 10^{-20}$ rad).

However, in VSI, mass is defined as a Saturation of the Vacuum Dielectric ($n(r)$). A rotating mass does not merely curve geometry; it mechanically "drags" and "thickens" the lattice flux, increasing the local refractive index.

\subsubsection{The Lattice Viscosity Coefficient ($\eta_{vac}$)}
We propose that the coupling strength of this drag is determined not by $G$, but by the Geometric Transmission Coefficient ($\alpha$), commonly known as the Fine Structure Constant. As derived in Section 2.7.2, $\alpha$ represents the efficiency with which topological stress couples to the vacuum flux.
\begin{equation}
    \eta_{vac} \equiv \alpha \approx \frac{1}{137.036}
\end{equation}

\subsubsection{The Refractive Index Shift ($\Delta n$)}
The rotational kinetic energy of the flywheel creates a localized impedance spike. The magnitude of the refractive index increase is modeled as the viscous coupling of the tangential velocity ($v_{tan}$) relative to the global slew rate ($c$):
\begin{equation}
    \Delta n = \eta_{vac} \cdot \left(\frac{v_{tan}}{c}\right)^2 = \alpha \left(\frac{\omega R}{c}\right)^2
\end{equation}

\subsection{Experimental Setup and Signal Prediction}
To detect this shift, we utilize a Resonant Cavity (Fabry-Perot) Interferometer. The signal is amplified by the cavity finesse ($\mathcal{F}$), effectively folding the path length $L$ multiple times.

\begin{figure}[ht]
    \centering
    \includegraphics[width=0.9\textwidth]{assets/sim_outputs/rlve_prediction_v6.png}
    \caption{\textbf{RLVE Signal Prediction (v6.0):} A 100k RPM tungsten rotor is predicted to generate a 0.72 milli-radian phase shift due to lattice viscosity $\alpha$. General Relativity predicts a null result (black line) at this scale.}
    \label{fig:rlve_prediction}
\end{figure}

\textbf{Parameters:}
\begin{itemize}
    \item \textbf{Modulator:} Tungsten Flywheel (Radius $R=0.1$ m, Density $\rho \approx 19.25$ g/cm$^3$).
    \item \textbf{Rotation Speed:} Target $\omega = 10,472$ rad/s (100,000 RPM).
    \item \textbf{Laser Source:} $\lambda = 1550$ nm (Infrared).
    \item \textbf{Interaction Length:} $L = 0.2$ m (Path parallel to rim).
    \item \textbf{Cavity Finesse:} $\mathcal{F} = 10,000$ (Effective bounces).
\end{itemize}

\subsubsection{Phase Shift Calculation ($\Delta \phi$)}
The single-pass phase shift is given by:
\begin{equation}
    \delta \phi_{single} = \frac{2\pi L}{\lambda} \Delta n
\end{equation}
The total amplified shift in the resonant cavity is:
\begin{equation}
    \Delta \phi_{total} \approx \mathcal{F} \cdot \delta \phi_{single} = \mathcal{F} \frac{2\pi L}{\lambda} \left[ \alpha \left(\frac{\omega R}{c}\right)^2 \right]
\end{equation}

\subsubsection{Experimental Controls and Systematics}
To distinguish the Vacuum Viscosity signal ($\Delta n$) from environmental noise (vibration, acoustics), we propose three specific control protocols:

\textbf{1. The Counter-Rotation Null (Symmetry Test)}
The VSI effect depends on $v^2$ (scalar energy density) and is strictly positive.
\begin{itemize}
    \item \textbf{Protocol:} Reverse rotor direction ($+\omega \to -\omega$).
    \item \textbf{Prediction:} Signal remains constant ($v^2$ is invariant).
    \item \textbf{Differentiation:} Mechanical resonances often depend on phase/direction. A signal that flips sign is an artifact.
\end{itemize}

\textbf{2. The "Dummy Mass" Null (Density Test)}
The effect scales with mass density ($\rho_{tungsten} \approx 19.3$).
\begin{itemize}
    \item \textbf{Protocol:} Replace Tungsten with Aluminum ($\rho \approx 2.7$) at identical RPM.
    \item \textbf{Prediction:} Signal drops by factor of $\approx 7.1$.
    \item \textbf{Differentiation:} Aerodynamic noise depends on geometry, not density. A signal that persists is air turbulence.
\end{itemize}

\textbf{3. The Geometric Offset (Range Test)}
\begin{itemize}
    \item \textbf{Protocol:} Shift optical path laterally by $d=5$ cm.
    \item \textbf{Prediction:} Signal extinction (local evanescent drag).
    \item \textbf{Differentiation:} Acoustic vibrations travel through the table. A signal that persists at distance is vibrational noise.
\end{itemize}

\subsection{Simulation: Signal Prediction (Density Dependent)}
To verify the experimental feasibility, we modeled the expected phase shift for both the primary (Tungsten) and control (Aluminum) rotors using the \texttt{RLVE\_Prediction\_v2} engine.

The simulation incorporates the Density Scaling Factor $(\rho / \rho_{sat})$ derived in Section 8.6.2. The parameters assume a 20 cm interaction length $(L)$ and a cavity finesse of $\mathcal{F}=10,000$.

\textbf{Results (Figure 8.2):}
\begin{itemize}
    \item \textbf{Tungsten Signal $(\rho \approx 19.3)$:} The model predicts a quadratic rise in phase shift, reaching $\Delta \phi \approx 0.72$ milli-radians at 100,000 RPM. This is well above the noise floor of modern interferometry $(\sim 10^{-6}$ rad).
    \item \textbf{Aluminum Control $(\rho \approx 2.7)$:} The signal drops to $\approx 0.10$ milli-radians. This 7.1x reduction provides a robust "fingerprint" to distinguish the VSI effect from aerodynamic turbulence, which depends on geometry rather than mass.
\end{itemize}

\begin{figure}[h]
    \centering
    \includegraphics[width=0.8\linewidth]{assets/sim_outputs/rlve_prediction_v2.png}
    \caption{\textbf{RLVE Signal Prediction.} The simulation confirms that the VSI effect is density-dependent. The Tungsten rotor (Orange) generates a strong 0.72 mrad signal, while the Aluminum control (Green) yields a suppressed signal. General Relativity (Black Dashed) predicts a null result at this scale.}
    \label{fig:rlve_prediction}
\end{figure}