\section{Experimental Proposal: The Rotational Lattice Viscosity Experiment (RLVE)}
\label{sec:experiment_rlve}

\subsection{Abstract}
\citestart The Stochastic Vacuum Framework (VSI) posits that the vacuum is not a geometric abstraction but a constitutive hardware substrate ($M_A$) with finite saturation limits\cite{1161}\citeend. We propose a laboratory test to detect the \textbf{Lattice Viscosity} ($\eta_{vac}$) of this substrate. By rapidly rotating a high-density Tungsten mass adjacent to a high-finesse Fabry-Perot interferometer, we aim to induce a localized saturation of the vacuum dielectric, creating a measurable refractive index shift ($\Delta n$). This section documents the derivation, experimental parameters, and simulation code required to replicate the prediction.

\subsection{Theoretical Derivation: The Viscous Drag Effect}
In standard General Relativity, the coupling of angular momentum to spacetime (Frame Dragging) is governed by the gravitational constant $G$, resulting in effects too small for laboratory detection ($\Delta\phi \approx 10^{-20}$ rad).

\citestart However, in VSI, mass is defined as a Saturation of the Vacuum Dielectric ($n(r)$)\cite{1168}\citeend. A rotating mass does not merely curve geometry; it mechanically "drags" and "thickens" the lattice flux, increasing the local refractive index.

\subsubsection{The Lattice Viscosity Coefficient ($\eta_{vac}$)}
We propose that the coupling strength of this drag is determined not by $G$, but by the \textbf{Geometric Transmission Coefficient} ($\alpha$), commonly known as the Fine Structure Constant. \citestart As derived in Section 2.7.2, $\alpha$ represents the efficiency with which topological stress couples to the vacuum flux\cite{1173}\citeend.

\begin{equation}
    \eta_{vac} \equiv \alpha \approx \frac{1}{137.036}
\end{equation}

\subsubsection{The Refractive Index Shift ($\Delta n$)}
The rotational kinetic energy of the flywheel creates a localized impedance spike. The magnitude of the refractive index increase is modeled as the viscous coupling of the tangential velocity ($v$) relative to the global slew rate ($c$):

\begin{equation}
    \Delta n = \eta_{vac} \cdot \left( \frac{v_{tan}}{c} \right)^2 = \alpha \left( \frac{\omega R}{c} \right)^2
\end{equation}

\subsection{Experimental Setup and Signal Prediction}
To detect this shift, we utilize a Resonant Cavity (Fabry-Perot) Interferometer. The signal is amplified by the cavity finesse ($\mathcal{F}$), effectively folding the path length $L$ multiple times.

\textbf{Parameters:}
\begin{itemize}
    \item \textbf{Modulator:} Tungsten Flywheel (Radius $R=0.1$ m, Density $\rho \approx 19.25$ g/cm$^3$).
    \item \textbf{Rotation Speed:} Target $\omega = 10,472$ rad/s (100,000 RPM).
    \item \textbf{Laser Source:} $\lambda = 1550$ nm (Infrared).
    \item \textbf{Interaction Length:} $L = 0.2$ m (Path parallel to rim).
    \item \textbf{Cavity Finesse:} $\mathcal{F} = 10,000$ (Effective bounces).
\end{itemize}

\subsubsection{Phase Shift Calculation ($\Delta\phi$)}
The single-pass phase shift is given by:
\begin{equation}
    \delta\phi_{single} = \frac{2\pi L}{\lambda} \Delta n
\end{equation}

The total amplified shift in the resonant cavity is:
\begin{equation}
    \Delta\phi_{total} \approx \mathcal{F} \cdot \delta\phi_{single} = \mathcal{F} \frac{2\pi L}{\lambda} \left[ \alpha \left( \frac{\omega R}{c} \right)^2 \right]
\end{equation}

\subsection{Simulation and Verification}
To ensure reproducibility, the following Python script calculates the predicted phase shift across a range of RPMs. This script serves as the "Digital Twin" of the RLVE protocol.

\begin{lstlisting}[language=Python, caption=RLVE Signal Prediction Model]
import numpy as np
import matplotlib.pyplot as plt

def calculate_viscous_shift(rpm, radius, length, wavelength, finesse):
    """
    Calculates the VSI Viscous Drag Phase Shift for a given RPM.
    
    Parameters:
      rpm: Angular velocity in Revolutions Per Minute
      radius: Radius of flywheel (meters)
      length: Interaction path length (meters)
      wavelength: Laser wavelength (meters)
      finesse: Optical cavity quality factor (amplification)
      
    Returns:
      phase_shift: Total phase shift in milli-radians
    """
    # Constants
    c = 299792458.0          # Speed of light (m/s)
    alpha = 1.0 / 137.035999 # Lattice Viscosity
    
    # Kinematics
    omega = rpm * (2 * np.pi / 60.0) # rad/s
    v_tan = omega * radius           # Tangential Velocity
    
    # 1. Calculate Refractive Index Shift (Viscous Drag Effect)
    # Delta_n = alpha * (v/c)^2
    delta_n = alpha * (v_tan / c)**2
    
    # 2. Calculate Single-Pass Phase Shift (radians)
    # phi = (2*pi*L / lambda) * delta_n
    phi_single = (2 * np.pi * length / wavelength) * delta_n
    
    # 3. Apply Resonant Cavity Amplification
    phi_total = phi_single * finesse
    
    return phi_total * 1000.0 # Convert to milli-radians

# --- Experimental Parameters ---
R_wheel = 0.1          # 10 cm radius
L_path = 0.2           # 20 cm interaction zone
lambda_laser = 1.55e-6 # 1550 nm
Finesse = 10000        # Cavity bounces

# --- Simulation Sweep ---
rpm_range = np.linspace(0, 100000, 100) # 0 to 100k RPM
shifts = [calculate_viscous_shift(r, R_wheel, L_path, lambda_laser, Finesse) 
          for r in rpm_range]

# --- Output Specific Prediction ---
target_rpm = 100000
predicted_mrad = calculate_viscous_shift(target_rpm, R_wheel, L_path, 
                                       lambda_laser, Finesse)

print(f"--- RLVE Prediction Summary ---")
print(f"Input: {target_rpm} RPM Tungsten Flywheel")
print(f"Theory: Lattice Viscosity (alpha = {1/137.036:.5f})")
print(f"Predicted Signal: {predicted_mrad:.4f} milli-radians")
\end{lstlisting}

\subsubsection{Predicted Results}
Running the simulation with the parameters defined above yields the following testable prediction:
\begin{itemize}
    \item \textbf{Input:} 100,000 RPM Tungsten Rotor.
    \item \textbf{Lattice Viscosity ($\alpha$):} $7.297 \times 10^{-3}$.
    \item \textbf{Predicted Signal ($\Delta\phi$):} \textbf{0.72 milli-radians}.
\end{itemize}

\subsection{Conclusion: The Falsification Threshold}
Standard General Relativity predicts a phase shift of $\approx 10^{-20}$ radians for this setup, which is undetectable. A positive detection of a signal in the \textbf{milli-radian range} that scales with $\omega^2$ would constitute a high-sigma falsification of the geometric vacuum hypothesis and a confirmation of the VSI Viscous Lattice model.