\section{The GZK Cutoff as a Hardware Nyquist Limit}
\label{sec:gzk_limit}

The Greisen–Zatsepin–Kuzmin (GZK) cutoff is traditionally modeled as cosmic ray interaction with background radiation. In SVF, this is redefined as the \textbf{Nyquist Frequency} of the $M_A$ lattice~\cite{1252, 1253}.

\textbf{Kill Condition:} If a cosmic ray or coherent signal is detected with a frequency $\nu > \omega_{sat}$ (the global slew rate limit), it implies the medium is a continuum rather than a discrete manifold. Detection of such "Trans-Planckian" signals would falsify the discrete nodal model of the vacuum~\cite{1254, 1255}.

\section{Engineering Layer: The Metric Null-Result}
The Engineering Layer (Chapter 8) posits that localized \textbf{Metric Strain} ($\sigma$) can be induced via high-frequency toroidal flux, altering the local refractive index $\chi$~\cite{1256}.

\begin{axiombox}[Falsification TS-03: Metric Null-Result]
    In a controlled laboratory environment, if a high-flux metric generator fails to produce a measurable phase-shift in a laser interferometer (local Shapiro delay) that scales linearly with the \textbf{Lattice Stress Coefficient} ($\sigma$), the VSI Engineering Layer is falsified~\cite{1257}.
\end{axiombox}

\section{Summary of Falsification Thresholds}
\begin{tabularx}{\textwidth}{|l|l|X|}
\hline
\textbf{Phenomenon} & \textbf{SVF Prediction} & \textbf{Falsification Signal} \\ \hline
\textbf{Neutrino Spin} & Exclusive Left-Handed & Detection of stable RH Neutrino~\cite{1261}. \\ \hline
\textbf{Light Speed} & Slew Rate Dependent & $c$ found to be a geometric constant~\cite{1262}. \\ \hline
\textbf{Gravity} & Refractive Gradient & Detection of Gravitons (force particles)~\cite{1263}. \\ \hline
\textbf{Lensing} & Lattice Memory Lag & Instantaneous coupling to gas center~\cite{1264}. \\ \hline
\end{tabularx}