\section{The Breakdown Limit ($V_0$): An Energetic Definition}
\label{sec:breakdown_limit}

To avoid circular definitions and maintain strict dimensional homogeneity, we do not calibrate the Breakdown Voltage ($V_0$) using arbitrary Planck units. Instead, we define it strictly via the Topological Yield Energy ($E_{sat}$) of the discrete amorphous manifold ($\mathcal{M}_A$).

\subsection{Resolving Inductance vs. Mass}
The vacuum acts as a reactive transmission network. We must strictly differentiate between Inductance (measured in Henries, $[H]$) and Mass (measured in Kilograms, $[kg]$). In the AVE framework, mass is not a fundamental substance; it is defined as the equivalent inertial resistance of stored inductive energy.

For a nodal flux current $I$, the stored magnetic energy is $E_L = \frac{1}{2}L_{node} I^2$, where $L_{node} = \mu_0 l_0$. Equating this to rest mass-energy ($E = m c^2$), the equivalent mass of a node is:
\begin{equation}
    m_{node} = \frac{E_L}{c^2} = \frac{L_{node} I^2}{2 c^2} \quad \text{[kg]}
\end{equation}
This rigorous separation ensures that all downstream topological mass derivations strictly adhere to SI dimensional analysis without conflating electrical and kinematic properties.

\subsection{The Dielectric Saturation Limit}
We postulate a fundamental Yield Energy ($E_{sat}$) representing the maximum energy a single capacitive flux edge can store before the local dielectric physically ruptures, creating a vacancy defect (singularity).

Using the lumped capacitive modulus of the node ($C_{node} = \epsilon_0 l_0$), the electrostatic potential required to stress a node to its yield energy is given by the standard capacitive energy equation, $E_{sat} = \frac{1}{2} C_{node} V_0^2$. Solving for $V_0$:
\begin{equation}
    V_0 = \sqrt{\frac{2 E_{sat}}{\epsilon_0 l_0}} \quad \text{[Volts]}
\end{equation}

\textbf{Dimensional Proof:} This formulation ensures strict dimensional homogeneity:
\begin{equation}
    [V_0] = \sqrt{\frac{[J]}{[F]}} = \sqrt{\frac{[J \cdot V]}{[C]}} = \sqrt{\frac{[V \cdot C \cdot V]}{[C]}} = \sqrt{[V^2]} = [V]
\end{equation}
This establishes the breakdown limit as a dimensionally exact, self-contained hardware specification.