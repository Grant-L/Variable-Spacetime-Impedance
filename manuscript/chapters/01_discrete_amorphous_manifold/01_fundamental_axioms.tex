\section{The Fundamental Axioms of Vacuum Engineering}
\label{sec:axioms}

To eliminate circular definitions and reduce the universe to a mechanical substrate, the Applied Vacuum Electrodynamics (AVE) framework rests entirely on four hardware axioms derived from the Variable Spacetime Impedance (VSI) theory.

%Axiom 1
\begin{axiom}[The Topological Domain $\mathcal{G}$]
The physical universe is strictly defined as a dynamic graph $\mathcal{G}(V, E, t)$ resulting from the Delaunay Triangulation of a stochastic point process $P \subset \mathbb{R}^3$.
\begin{itemize}
    \item \textbf{Fundamental Length ($l_{node}$):} The expectation value of the edge length distribution is fixed: $\langle |e_{ij}|\rangle \equiv l_{node}$.
    \item \textbf{Constraint:} The graph is simple, undirected, and globally connected.
\end{itemize}
\end{axiom}

\begin{axiom}[The State Variables]
    Physics is encoded entirely in two conjugate variables defined on the graph elements, obeying the Geometrodynamic Ansatz ($1\text{ C} \equiv 1\text{ m}$):
    \begin{enumerate}
        \item \textbf{Node Potential ($\phi_n$):} A scalar field $\phi: V \rightarrow \mathbb{R}$ representing the longitudinal dielectric strain (Compression/Voltage).
        \item \textbf{Metric Flux ($\Phi_{ij}$):} A discrete vector link variable representing the momentum state of the edge ($\Phi \equiv \mathbf{p}$).
    \end{enumerate}
    There are no other fundamental fields.
\end{axiom}

%Axiom 3
\begin{axiom}[The Discrete Action Principle]
The system evolves to minimize the \textbf{Hardware Action} $S_{AVE}$.
The action is defined as the discrete sum over nodes ($n$) of the Lagrangian density $\mathcal{L}_{node}$:
\begin{equation}
S_{AVE} = \int dt \sum_{n \in V} \mathcal{L}_{node}
\end{equation}
To ensure dimensional homogeneity (Joules), the Lagrangian relates Dielectric Potential Energy to Inductive Kinetic Energy:
\begin{equation}
    \label{eq:discrete_lagrangian}
    \mathcal{L}_{node} = \underbrace{\frac{1}{2} (L_{node} C_{node}^2) (\partial_t \phi_n)^2}_{\text{Kinetic ($L I^2$)}} - \underbrace{\frac{1}{2} C_{eff}(\Delta \phi) \sum_{j \in \text{neigh}(n)} (\phi_n - \phi_j)^2}_{\text{Potential ($CV^2$)}}
\end{equation}
Here, the kinetic term explicitly accounts for the conversion of potential rate ($\dot{\phi}$) to displacement current ($I \approx C \dot{\phi}$).
\end{axiom}

%Axiom 4
\begin{axiom}[The Saturation Operator]
    The vacuum is a non-linear dielectric. The effective capacitance $C_{eff}$ is not constant but is a function of the local potential gradient relative to the \textbf{Electromagnetic Yield Limit} ($V_0$):
    \begin{equation}
    \label{eq:saturation_op}
    C_{eff}(\Delta \phi) = \frac{C_0}{\sqrt{1 - \left(\frac{\Delta \phi}{V_0}\right)^4}}
    \end{equation}
    where $V_0 \equiv T_{EM}$ is the electromagnetic vacuum breakdown limit (Pair Production). Note that under the Ansatz, Potential (Volts) and Tension (Newtons) are topologically equivalent ($1\text{ V} \equiv 1\text{ N}$). The negative sign ensures the strict enforcement of the physical breakdown limit.
\end{axiom}

\subsection{Implications of the Axiom Set}
From these four hardware specifications, the standard "laws" of physics are derived as theorems of the substrate limit:
\begin{itemize}
    \item \textbf{The Wave Equation:} In the limit $\Delta \phi \ll T_{max}$, the Lagrangian reduces to the standard discrete wave equation, recovering the invariant speed of light $c = l_{node}/\sqrt{L_{node}C_{node}}$.
    \item \textbf{Mass Hierarchy:} In the limit $\Delta \phi \to T_{max}$, the Quartic term in Eq.~\ref{eq:saturation_op} dominates, forcing the discrete energy scaling observed in particle generations.
    \item \textbf{Event Horizon:} If $\Delta \phi > T_{max}$, the real-valued solution to the capacitance ceases to exist, representing the physical rupture of the manifold (Singularity).
\end{itemize}