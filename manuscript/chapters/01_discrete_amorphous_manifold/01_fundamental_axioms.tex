\section{The Fundamental Axioms of Vacuum Engineering}

To eliminate circular definitions and reduce the universe to a mechanical substrate, the Applied Vacuum Electrodynamics (AVE) framework rests entirely on four hardware axioms derived from physical dielectric yield limits.

\begin{enumerate}
    \item \textbf{The Substrate Topology:} The physical universe is strictly defined as a dynamic graph $\mathcal{G}(V, E, t)$ resulting from the Delaunay Triangulation of a stochastic point process $P \subset \mathbb{R}^3$.
    \item \textbf{Fundamental Length ($l_{node}$):} The expectation value of the edge length distribution is physically fixed by the dielectric yield limit: $\langle|e_{ij}|\rangle \equiv l_{node}$.
    \item \textbf{The Discrete Action Principle:} The system evolves strictly to minimize the Hardware Action $S_{AVE}$. Physics is encoded entirely in the Magnetic Vector Potential ($\mathbf{A}$), evaluated over the discrete Voronoi cells of the graph:
    \begin{equation}
        \mathcal{L}_{node} = \frac{1}{2}\epsilon_0 |\partial_t \mathbf{A}_n|^2 - \frac{1}{2\mu_0} |\nabla \times \mathbf{A}_n|^2
    \end{equation}
    There are no other fundamental fields. All particles and forces emerge from the topological deformation of this single continuous vector field.
    \item \textbf{Dielectric Saturation:} The vacuum is a non-linear dielectric. The effective capacitance $C_{eff}$ is structurally bounded by the absolute Electromagnetic Yield Limit ($V_0$):
    \begin{equation}
        C_{eff}(\Delta\phi) = \frac{C_0}{\sqrt{1 - \left(\frac{\Delta\phi}{V_0}\right)^4}}
    \end{equation}
\end{enumerate}

\subsection{Implications of the Axiom Set}

From these four hardware specifications, the standard ``laws'' of physics are derived as theorems of the substrate limit:
\begin{itemize}
    \item \textbf{The Wave Equation:} In the low-energy limit $\Delta\phi \ll V_0$, the Lagrangian reduces to the standard discrete wave equation, recovering the invariant speed of light $c = 1/\sqrt{\mu_0\epsilon_0}$.
    \item \textbf{Mass Hierarchy:} In the extreme limit $\Delta\phi \to V_0$, the Quartic term dominates, forcing the discrete, exponential energy scaling strictly observed in particle mass generations.
    \item \textbf{Event Horizon:} If $\Delta\phi > V_0$, the real-valued solution to the capacitance ceases to exist, representing the physical rupture of the manifold (Dielectric Snap).
\end{itemize}

\section{The Amorphous Manifold}
The foundational postulate of the AVE framework is that the physical universe is a Discrete Amorphous Manifold ($\mathcal{M}_A$). Let $P$ be a set of stochastic points distributed in a topological volume $V$. The physical manifold $\mathcal{M}_A$ is defined as the Delaunay Triangulation of $P$.

\subsection{Isotropy via Stochasticity: The Rifled Vacuum}
A common critique of discrete spacetime models is the ``Manhattan Distance'' problem. On a regular cubic grid, diagonal movement is mathematically longer than cardinal movement ($\sqrt{2}$ vs 1), which violates Lorentz Invariance. 

The $\mathcal{M}_A$ framework evades this by requiring the lattice to be Amorphous (Random) rather than Crystalline. For a Delaunay graph generated from a stochastic Poisson distribution, the effective path length approaches rotational invariance at macroscopic scales $L \gg l_{node}$:
\begin{equation}
    \lim_{N \to \infty} \mathcal{L} f(x) \approx \nabla^2 f(x)
\end{equation}
The vacuum looks smooth to us for the same reason a sandy beach looks smooth from an airplane: the grains are stochastic and infinitesimally small.

\subsection{Connectivity Analysis and Volumetric Packing}
Unlike a crystalline lattice with a fixed coordination number, the vacuum substrate possesses a statistical distribution of connectivity. Monte Carlo analysis of $N=10,000$ nodes yields a mean coordination number $\langle k \rangle \approx 15.54$. This high degree of connectivity ensures that the vacuum is ``Over-Braced,'' providing the extreme mechanical stiffness required to support transverse waves (light) while minimizing dispersive loss. Furthermore, the simulation strictly derives the volumetric packing factor ($\kappa_V$) of the discrete lattice:
\begin{equation}
    \kappa_V \equiv \frac{\langle V_{node} \rangle}{\langle l_{node} \rangle^3} \approx 0.433
\end{equation}

\section{The Macroscopic Moduli of the Void}
In standard physics, $\mu_0$ and $\epsilon_0$ are treated as macroscopic continuous densities. In Vacuum Engineering, they are strictly defined as the Constitutive Moduli of the discrete mechanical substrate.

\subsection{Magnetic Permeability ($\mu_0$) as Linear Mass Density}
The magnetic constant $\mu_0 \approx 1.256 \times 10^{-6}$ H/m represents the Inductive Inertia of the lattice nodes distributed over the fundamental length: $\mu_0 \equiv L_{node} / l_{node}$. Under the Geometrodynamic Ansatz ($1~C \equiv 1~m$), Inductance maps directly to Mass ($[H] \equiv [kg]$):
\begin{equation}
    [\mu_0] = \frac{\text{H}}{\text{m}} \xrightarrow{1~C \equiv 1~m} \frac{\text{kg}}{\text{m}}
\end{equation}
This mathematically proves that $\mu_0$ is the exact mechanical Linear Mass Density of the vacuum lattice.

\subsection{Electric Permittivity ($\epsilon_0$) as Capacitive Compliance}
The electric constant $\epsilon_0 \approx 8.854 \times 10^{-12}$ F/m represents the Capacitive Compliance of the lattice edges: $\epsilon_0 \equiv C_{EM} / l_{node}$. Applying the Ansatz, Capacitance maps to mechanical compliance:
\begin{equation}
    [\epsilon_0] = \frac{\text{F}}{\text{m}} \xrightarrow{1~C \equiv 1~m} \frac{\text{m/N}}{\text{m}} = \frac{1}{\text{N}}
\end{equation}
This proves that $\epsilon_0$ is the exact physical inverse of a string tension ($T$).

\section{The Breakdown Limit: Dielectric Rupture}

Every physical material has an ultimate tensile strength. We define the Breakdown Limit of the discrete manifold ($\mathcal{M}_A$) not as an arbitrary scalar, but as the strict discrete electrical threshold where topological connectivity ruptures and pair-production occurs.

\subsection{The Breakdown Voltage ($V_0$)}

To avoid circular definitions, we derive the fundamental Nodal Breakdown Voltage ($V_0$) directly from the manifold's impedance limits. The breakdown voltage $V_0$ is the discrete potential at which the electrostatic energy of the node's field equals the mass-energy required to synthesize a new inductive node.

\begin{equation}
    \frac{1}{2} C_{node} V_0^2 = 2 m_{node} c^2
\end{equation}

Under the Geometrodynamic Ansatz (Axiom 2), we established that Mass is strictly isomorphic to Inductance ($m_{node} \equiv L_{node}$). Substituting $L_{node}$ for mass, and utilizing the hardware slew rate $c = l_{node} / \sqrt{L_{node} C_{node}}$:

\begin{equation}
    \frac{1}{2} C_{node} V_0^2 = 2 L_{node} c^2 \implies V_0^2 = 4 \left(\frac{L_{node}}{C_{node}}\right) c^2
\end{equation}

By substituting the Characteristic Impedance ($Z_0 = \sqrt{L_{node}/C_{node}}$), this resolves to a self-contained, irreducible hardware specification:

\begin{equation}
    V_0 = 2 c Z_0 = 2c \sqrt{\frac{\mu_0}{\epsilon_0}} = \frac{2}{\epsilon_0} \approx 2.26 \times 10^{11} \text{ Volts}
\end{equation}

This derivation provides a profound dimensional proof. Under standard SI units, $[1/\epsilon_0]$ evaluates to $[m/F]$. However, under the strict application of the Geometrodynamic Ansatz ($1~C \equiv 1~m$), Farads ($C/V$) map to $m/V$. Thus, $m/(m/V)$ flawlessly reduces to Volts. The derivation is dimensionally absolute and entirely parameter-free.

\section{Theoretical Constraints on Fundamental Constants}

Standard physics treats $G$ and $\hbar$ as unexplained, fundamental scalars. In the AVE framework, we prove they are strictly emergent geometric scaling factors derived from the foundational hardware primitives: Lattice Pitch ($l_{node}$) and the Breakdown Voltage ($V_0$).

\subsection{Derived Action Scale (The Quantum of Action, $\hbar$)}

We define the absolute maximum action capacity of a single node ($\hbar_{AVE}$) as the product of its maximum storable energy before dielectric rupture ($E_{sat}$) and the fundamental hardware update time ($t_{tick}$).

The maximum electrostatic energy a node can hold is $E_{sat} = \frac{1}{2} C_{node} V_0^2$. Substituting our exact hardware limits ($C_{node} = \epsilon_0 l_{node}$ and $V_0 = 2 / \epsilon_0$):
\begin{equation}
    E_{sat} = \frac{1}{2} (\epsilon_0 l_{node}) \left(\frac{2}{\epsilon_0}\right)^2 = \frac{2 l_{node}}{\epsilon_0} \quad \text{[Joules]}
\end{equation}

The Quantum of Action ($\hbar$) is this energy multiplied by the discrete clock cycle ($t_{tick} = l_{node}/c$):
\begin{equation}
    \hbar \equiv E_{sat} \cdot t_{tick} = \left(\frac{2 l_{node}}{\epsilon_0}\right) \left(\frac{l_{node}}{c}\right) = \frac{2 l_{node}^2}{c \epsilon_0}
\end{equation}

Most profoundly, if we algebraically isolate $l_{node}$, we derive the fundamental spatial granularity of the universe purely from measurable vacuum constants, without ever invoking the circular, mass-dependent Schwinger limit:
\begin{equation}
    l_{node} = \sqrt{\frac{\hbar c \epsilon_0}{2}} \approx 3.74 \times 10^{-19} \text{ meters}
\end{equation}

This beautifully places the physical lattice grid exactly at the boundary of the Weak Interaction ($10^{-18}$ m). It mathematically proves that Planck's constant is not an arbitrary scale; it is an emergent artifact dictated precisely by the electrical capacitance ($\epsilon_0$) of a sub-attometer 3D discrete grid. The traditional ``Planck length'' ($10^{-35}$ m) is exposed as a fictitious optical illusion, artificially compressed by using the macroscopic, diluted constant $G$.

\begin{figure}[htbp]
    \centering
    \includegraphics[width=0.85\textwidth]{chapters/01_discrete_amorphous_manifold/simulations/outputs/hardware_action_scale.png}
    \caption{\textbf{The True Granularity of the Substrate.} Deriving the lattice pitch strictly from electrical saturation ($V_0$) places the true quantum grid at $10^{-19}$ meters. This perfectly bounds the Cosserat Weak Force cutoff ($10^{-18}$ m), proving why high-energy colliders begin to see standard model physics break down at exactly this scale.}
    \label{fig:action_scale}
\end{figure}

\subsection{Derived Gravitational Coupling and the Hierarchy Ratio ($\xi$)}

The maximum transmissible mechanical force across a single discrete electromagnetic flux tube before topological rupture is the EM Tension Limit ($T_{EM}$). Because tension is Energy per unit length ($E_{sat} / l_{node}$):
\begin{equation}
    T_{EM} \equiv \frac{E_{sat}}{l_{node}} = \frac{2}{\epsilon_0} \approx 2.25 \times 10^{11} \text{ Newtons}
\end{equation}

Macroscopic Gravity ($G$) operates in the Gravimetric Domain, which is shielded by the dimensionless Hierarchy Coupling ($\xi$). To mechanically stabilize the manifold, the true gravitational tension limit ($T_{max,g}$) is scaled by this immense topological stiffness:
\begin{equation}
    T_{max,g} = \xi \cdot T_{EM} = \xi \left(\frac{2}{\epsilon_0}\right)
\end{equation}

By equating this true gravimetric substrate tension to the Einstein-Hilbert limit ($c^4/G$), we perfectly derive macroscopic gravity:
\begin{equation}
    G = \frac{c^4}{T_{max,g}} = \frac{c^4 \epsilon_0}{2 \xi}
\end{equation}

By evaluating this against empirical gravity ($G \approx 6.67 \times 10^{-11}$), the topological Hierarchy Coupling is geometrically revealed to be $\xi \approx 5.38 \times 10^{33}$. Gravity is astonishingly weak precisely because macroscopic metric deformations must overcome an impedance domain thirty-three orders of magnitude stiffer than the baseline electromagnetic geometry.