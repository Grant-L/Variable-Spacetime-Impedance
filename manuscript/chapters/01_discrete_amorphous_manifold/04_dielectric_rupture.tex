\section{The Breakdown Limit: Dielectric Rupture}
\label{sec:breakdown_limit}

Every physical material has an ultimate tensile strength. We define the Breakdown Limit of the discrete manifold ($M_A$) not as an arbitrary scalar, but as the strict discrete threshold where topological connectivity ruptures and pair-production occurs.

\subsection{The Schwinger Yield Energy ($u_{sat}$)}
We anchor the physical breakdown limit of the discrete amorphous manifold to the experimentally established onset of dielectric pair-production: the \textbf{Schwinger Limit} ($E_{crit} \approx 1.32 \times 10^{18} \text{ V/m}$).

In standard linear dielectrics, the volumetric energy density $u$ is defined as $u = \frac{1}{2} \epsilon_0 |\mathbf{E}|^2$. Therefore, the ultimate Yield Energy Density ($u_{sat}$) of the vacuum substrate is dimensionally exact:
\begin{equation}
    u_{sat} = \frac{1}{2} \epsilon_0 E_{crit}^2 \approx 7.71 \times 10^{25} \left[\frac{\text{J}}{\text{m}^3}\right]
\end{equation}

For a single discrete lattice node occupying a fundamental Voronoi cell of volume $V_{node} = \kappa_V l_{node}^3$, the maximum discrete energy capacity before topological rupture (particle genesis) is strictly bounded. The maximum energetic yield per individual node is:
\begin{equation}
    E_{sat} = u_{sat} V_{node} = \frac{1}{2} \epsilon_0 E_{crit}^2 (\kappa_V l_{node}^3) \quad [\text{Joules}]
\end{equation}

\subsection{The Breakdown Voltage ($V_0$): A Geometric Proof}
To avoid circular definitions, we derive the Nodal Breakdown Voltage ($V_0$) directly from the manifold's impedance limits. The breakdown voltage $V_0$ is the discrete potential at which the electrostatic energy of the node's field equals the mass-energy required to synthesize a new inductive node.
\begin{equation}
\frac{1}{2} C_{node} V_0^2 = 2 m_{node} c^2
\end{equation}

Under the Geometrodynamic Ansatz, we established that Mass is Inductance ($m_{node} \equiv L_{node}$). Substituting $L_{node}$ for mass, and utilizing the slew rate $c = l_{node} / \sqrt{L_{node}C_{node}}$:
\begin{equation}
\frac{1}{2} C_{node} V_0^2 = 2 L_{node} c^2 \implies V_0^2 = 4 \left( \frac{L_{node}}{C_{node}} \right) c^2
\end{equation}
By substituting the Characteristic Impedance ($Z_0 = \sqrt{L_{node}/C_{node}}$), this resolves to a self-contained hardware specification:
\begin{equation}
V_0 = 2c Z_0 \approx 2.26 \times 10^{11} \text{ Volts}
\end{equation}
A 226 Billion Volt potential difference occurring across a single microscopic spatial step generates a localized electric field on the order of $10^{46}$ V/m. This mathematically proves why the node experiences catastrophic topological failure long before reaching mathematical singularities.

Furthermore, this derivation provides a profound dimensional proof. Under standard SI units, $[c Z_0]$ evaluates to $[kg \cdot m^3 / (C^2 \cdot s^2)]$, which breaks standard Volts. However, under the strict application of the Geometrodynamic Ansatz ($1\text{ C} \equiv 1\text{ m}$), the units reduce flawlessly to Newtons ($[kg \cdot m / s^2]$). Because Volts also topologically reduce to Newtons under the Ansatz ($[J/C] \to [J/m] = [N]$), the derivation becomes dimensionally absolute.

\section{Theoretical Constraints on Fundamental Constants}
\label{sec:theoretical_constraints}

Standard physics treats $G$ and $\hbar$ as unexplained, fundamental scalars. In the AVE framework, we prove they are strictly emergent geometric scaling factors derived from the foundational hardware primitives: Lattice Pitch ($l_{node}$) and the Schwinger Yield Energy Density ($u_{sat}$). 

\subsection{Derived Action Scale (The Quantum of Action, $\hbar$)}
We define the absolute maximum action capacity of a single node ($\hbar_{AVE}$) as the product of its maximum storable energy ($E_{sat}$) and the fundamental hardware update time ($t_{tick}$). 

Given the volumetric saturation limit $E_{sat} = u_{sat} (\kappa_V l_{node}^3)$ and the lattice clock speed $t_{tick} = l_{node} / c$:
\begin{equation}
    \hbar_{AVE} \equiv E_{sat} \cdot t_{tick} = \left(u_{sat} \kappa_V l_{node}^3\right) \left(\frac{l_{node}}{c}\right) = \kappa_V \frac{u_{sat} l_{node}^4}{c}
\end{equation}
This derivation is dimensionally exact: $[\text{J}/\text{m}^3] \cdot [\text{m}^4] / [\text{m}/\text{s}] = [\text{J} \cdot \text{s}]$. 

Most profoundly, if we algebraically isolate $l_{node}$ and evaluate it using the known empirical constants:
\begin{equation}
    l_{node} = \left( \frac{\hbar c}{\kappa_V u_{sat}} \right)^{1/4} \approx 1.75 \times 10^{-13} \text{ meters}
\end{equation}
This beautifully resolves directly to the characteristic topological scale of the electron. It mathematically proves that Planck's constant is not an arbitrary scale; it is an emergent artifact dynamically dictated by the volumetric topology ($\kappa_V$) of an electron-scale amorphous 3D lattice.

\subsection{Derived Gravitational Coupling as Lattice Tension}
To connect the microscopic substrate to macroscopic gravity, we identify the universal stiffness of the vacuum. The maximum transmissible mechanical force across a single 1D lattice cell before topological rupture is the Vacuum Tension Limit ($T_{max}$):
\begin{equation}
    T_{max} \equiv \frac{E_{sat}}{l_{node}} = u_{sat} \kappa_V l_{node}^2 \quad [\text{Newtons}]
\end{equation}
Equating this physical substrate tension to the Einstein-Hilbert proportionality constant ($c^4 / G$), we perfectly derive the emergent gravitational coupling:
\begin{equation}
    G_{AVE} = \frac{c^4}{T_{max}} = \frac{c^4}{u_{sat} \kappa_V l_{node}^2}
\end{equation}
Gravity is thus mathematically recast as the macroscopic mechanical compliance of the discrete hardware layer. A stiffer vacuum dielectric (higher $u_{sat}$) or a denser amorphous packing (higher $\kappa_V$) physically necessitates a proportionally weaker gravitational coupling ($G$).