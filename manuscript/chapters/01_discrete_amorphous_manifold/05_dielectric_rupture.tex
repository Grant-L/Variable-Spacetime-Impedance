\section{The Breakdown Limit: Dielectric Rupture}
\label{sec:breakdown_limit}

Every physical material has an ultimate tensile strength. We define the Breakdown Limit of the discrete manifold ($M_A$) not as an arbitrary scalar, but as the strict discrete threshold where topological connectivity ruptures and pair-production occurs.

\subsection{The Schwinger Yield Energy ($u_{sat}$)}
In standard linear dielectrics, the volumetric energy density $u$ is defined as $u = \frac{1}{2} \epsilon_0 |\mathbf{E}|^2$. Therefore, the ultimate Yield Energy Density ($u_{sat}$) of the vacuum substrate is dimensionally exact:
\begin{equation}
    u_{sat} = \frac{1}{2} \epsilon_0 E_{crit}^2 \approx 7.71 \times 10^{24} \left[\frac{\text{J}}{\text{m}^3}\right]
\end{equation}

For a single discrete lattice node occupying a fundamental Voronoi cell of volume $V_{node} = \kappa_V l_{node}^3$, the maximum discrete energy capacity before topological rupture (particle genesis) is strictly bounded. The maximum energetic yield per individual node is:
\begin{equation}
    E_{sat} = u_{sat} V_{node} = \frac{1}{2} \epsilon_0 E_{crit}^2 (\kappa_V l_{node}^3) \quad [\text{Joules}]
\end{equation}

\subsection{The Breakdown Voltage ($V_0$): A Geometric Proof}
To avoid circular definitions, we derive the Nodal Breakdown Voltage ($V_0$) directly from the manifold's impedance limits. The breakdown voltage $V_0$ is the discrete potential at which the electrostatic energy of the node's field equals the mass-energy required to synthesize a new inductive node.
\begin{equation}
\frac{1}{2} C_{node} V_0^2 = 2 m_{node} c^2
\end{equation}

Under the Geometrodynamic Ansatz, we established that Mass is Inductance ($m_{node} \equiv L_{node}$). Substituting $L_{node}$ for mass, and utilizing the slew rate $c = l_{node} / \sqrt{L_{node}C_{node}}$:
\begin{equation}
\frac{1}{2} C_{node} V_0^2 = 2 L_{node} c^2 \implies V_0^2 = 4 \left( \frac{L_{node}}{C_{node}} \right) c^2
\end{equation}
By substituting the Characteristic Impedance ($Z_0 = \sqrt{L_{node}/C_{node}}$), this resolves to a self-contained hardware specification:
\begin{equation}
V_0 = 2c Z_0 \approx 2.26 \times 10^{11} \text{ Volts}
\end{equation}
A 226 Billion Volt potential difference occurring across a single microscopic spatial step generates a localized electric field on the order of $10^{46}$ V/m. This mathematically proves why the node experiences catastrophic topological failure long before reaching mathematical singularities.

Furthermore, this derivation provides a profound dimensional proof. Under standard SI units, $[c Z_0]$ evaluates to $[kg \cdot m^3 / (C^2 \cdot s^2)]$, which breaks standard Volts. However, under the strict application of the Geometrodynamic Ansatz ($1\text{ C} \equiv 1\text{ m}$), the units reduce flawlessly to Newtons ($[kg \cdot m / s^2]$). Because Volts also topologically reduce to Newtons under the Ansatz ($[J/C] \to [J/m] = [N]$), the derivation becomes dimensionally absolute.