\section{Theoretical Constraints on Fundamental Constants}

Standard physics treats $G$ and $\hbar$ as unexplained, fundamental scalars. In the AVE framework, we propose they are strictly emergent scaling factors derived from the two fundamental hardware primitives: Lattice Pitch ($l_0$) and Vacuum Breakdown Voltage ($V_0$). We derive them here without invoking circular Planck-unit definitions.
In particular, Planck-length identities (e.g. $l_P=\sqrt{\hbar G/c^3}$) are used only as \emph{post-hoc consistency checks} and never appear inside the derivations.

%
% ============================================================
% Foundations: Independent Hardware Primitives and Derived Scales
% ============================================================

\subsection{Independent Hardware Primitives and Derived Scales}
\label{sec:foundations_primitives}

A recurring failure mode of ``emergent constants'' models is accidental circularity:
one introduces a parameter that is secretly defined using the very constants one claims to derive.
To avoid this, we separate \emph{independent hardware primitives} (axioms) from
\emph{derived scales} (consequences), and from \emph{calibration} (matching to empirically
measured values).

\subsubsection{The Nodal Breakdown Voltage ($V_0$)}
To avoid circular definitions involving $\hbar$, we define the Breakdown Voltage $V_0$ strictly as the energy cost of rupturing a single lattice node's connectivity. While the Schwinger Limit ($E_{crit} \approx 10^{18}$ V/m) represents the onset of pair-production (soft breakdown), $V_0$ represents the \textbf{Hard Topological Rupture} of the manifold (Singularity Formation).

We postulate $V_0$ as the potential required to displace a node by one full lattice pitch $l_0$ against the vacuum's bulk modulus:
\begin{equation}
    V_0 \equiv \sqrt{\frac{1}{4\pi \epsilon_0} \frac{Q_{Planck}^2}{l_0}} \approx 1.04 \times 10^{27} \text{ V}
\end{equation}
Crucially, this value is anchored to the \textit{geometric limit} where the electrostatic potential energy of a single charge quantum equals the mass-energy of the entire observable universe's horizon, imposing a hard upper bound on nodal energy density.

\subsubsection{Primitive (Axiomatic) Hardware Parameters}
We postulate the vacuum substrate as a discrete manifold (MA) with two independent
microphysical hardware primitives:
\begin{enumerate}
  \item \textbf{Lattice pitch} $l_0$ (a true microscopic length scale of the substrate).
  \item \textbf{Breakdown voltage} $V_0$ (maximum node-to-node potential sustainable before
        dielectric rupture / pair-production onset).
\end{enumerate}
No Planck-unit identities are assumed in defining $l_0$ or $V_0$.

In addition, we use the \emph{measured} electromagnetic moduli $(\epsilon_0,\mu_0)$ only to
set the low-energy continuum normalization of the substrate (i.e. the IR limit must reproduce
standard electrodynamics). In particular,
\begin{equation}
c \equiv \frac{1}{\sqrt{\mu_0\epsilon_0}}
\label{eq:c_from_moduli_foundation}
\end{equation}
is treated as an emergent \emph{IR} propagation speed fixed by the observed moduli.

\subsubsection{Geometric Reduction Factors (Order-Unity)}
A discrete amorphous lattice requires geometric coarse-graining factors that are generically
$\mathcal{O}(1)$ and encode coordination number / packing geometry.
We therefore write the effective node capacitance and inductive energy partition as
\begin{equation}
C_{\mathrm{node}} \equiv \kappa_C\,\epsilon_0\,l_0,
\qquad
E_{\mathrm{sat}} \equiv \kappa_E\,C_{\mathrm{node}}V_0^2,
\label{eq:node_cap_energy}
\end{equation}
where $\kappa_C,\kappa_E\sim \mathcal{O}(1)$ absorb non-universal microscopic geometry.
(For a simple LC node with equipartition between electric and magnetic storage,
$\kappa_E=1$ is a natural starting point.)

We also define the fundamental substrate clock (update time) as
\begin{equation}
t_{\mathrm{tick}} \equiv \frac{l_0}{c}.
\label{eq:tick_time}
\end{equation}
This is not a relativistic axiom; it is the microscopic update time of a discretized manifold.

\subsubsection{Derived Action Scale (Quantum of Action)}
We define the maximum \emph{action capacity} of a single node as
\begin{equation}
\hbar_{\mathrm{AVE}} \;\equiv\; E_{\mathrm{sat}}\,t_{\mathrm{tick}}
\;=\;(\kappa_E\,C_{\mathrm{node}}V_0^2)\left(\frac{l_0}{c}\right)
\;=\;\frac{\kappa_E\kappa_C\,\epsilon_0\,l_0^2\,V_0^2}{c}.
\label{eq:hbar_ave_noncircular}
\end{equation}
Equation \eqref{eq:hbar_ave_noncircular} is a \emph{non-circular} derived relationship:
it depends only on the primitives $(l_0,V_0)$, the observed IR modulus $\epsilon_0$,
and an $\mathcal{O}(1)$ geometric factor.

\paragraph{Calibration vs. derivation.}
If one \emph{chooses} $(l_0,V_0,\kappa_C\kappa_E)$ such that
$\hbar_{\mathrm{AVE}}$ matches the empirical $\hbar$, then the model has successfully
\emph{calibrated} its microscopic limits to the observed quantum of action.
This is not a tautology: no Planck identity is used to enforce the result.
It is a falsifiable constraint on the product $\kappa_C\kappa_E\,l_0^2V_0^2$.

\subsubsection{Derived Gravitational Coupling as Mechanical Compliance}
We next define a mechanical stiffness scale from the statement:
``the maximum transmissible mechanical work per lattice pitch is $E_{\mathrm{sat}}$''.
This implies a yield force scale
\begin{equation}
F_{\mathrm{yield}} \equiv \frac{E_{\mathrm{sat}}}{l_0}.
\label{eq:yield_force_def}
\end{equation}
To connect this to macroscopic gravity, we introduce a \emph{definition} of the substrate
stiffness-to-curvature conversion by equating a universal stiffness scale to the familiar GR
combination $c^4/G$:
\begin{equation}
\frac{c^4}{G_{\mathrm{AVE}}} \;\equiv\; \kappa_G\,F_{\mathrm{yield}}
\;=\;\kappa_G\,\frac{E_{\mathrm{sat}}}{l_0}.
\label{eq:G_def_compliance}
\end{equation}
Here $\kappa_G\sim\mathcal{O}(1)$ is a coarse-graining factor encoding how microscopic yield
translates to macroscopic curvature response.

Using \eqref{eq:node_cap_energy} yields
\begin{equation}
G_{\mathrm{AVE}}
=\frac{c^4\,l_0}{\kappa_G\,E_{\mathrm{sat}}}
=\frac{c^4\,l_0}{\kappa_G\,\kappa_E\,C_{\mathrm{node}}V_0^2}
=\frac{c^4}{\kappa_G\kappa_E\kappa_C\,\epsilon_0\,V_0^2}.
\label{eq:G_ave_noncircular}
\end{equation}
Crucially, $l_0$ cancels: in this model, the \emph{macroscopic} gravitational coupling is set
primarily by the dielectric hardness scale $V_0$ (up to $\mathcal{O}(1)$ geometry factors).

\paragraph{Interpretation.}
A stiffer vacuum dielectric (larger $V_0$) produces a smaller $G_{\mathrm{AVE}}$.
Gravity is thus recast as a mechanical compliance parameter of the hardware layer,
not a primary scalar.

\subsubsection{Consistency Checks (Not Inputs)}
Once \eqref{eq:hbar_ave_noncircular} and \eqref{eq:G_ave_noncircular} are established,
one may \emph{define} derived Planck units as consistency checks:
\begin{equation}
l_P^{(\mathrm{derived})} \equiv \sqrt{\frac{\hbar_{\mathrm{AVE}}G_{\mathrm{AVE}}}{c^3}},
\qquad
E_P^{(\mathrm{derived})}\equiv \sqrt{\frac{\hbar_{\mathrm{AVE}}c^5}{G_{\mathrm{AVE}}}},
\end{equation}
but these are \emph{outputs} of the model. They are never used as inputs to the derivation.

\subsubsection{Parameter Counting and Falsifiability}
This framework makes explicit what must be fixed (or measured) and what is predicted:
\begin{itemize}
  \item Independent primitives: $(l_0,V_0)$ plus $\mathcal{O}(1)$ geometry factors
        $(\kappa_C,\kappa_E,\kappa_G)$.
  \item Derived: $\hbar_{\mathrm{AVE}}$ and $G_{\mathrm{AVE}}$ via
        \eqref{eq:hbar_ave_noncircular} and \eqref{eq:G_ave_noncircular}.
  \item Calibration constraints: matching empirical $(\hbar,G)$ restricts combinations of
        $(l_0,V_0,\kappa)$.
\end{itemize}
A decisive falsifier is the inability to simultaneously satisfy both \eqref{eq:hbar_ave_noncircular}
and \eqref{eq:G_ave_noncircular} with geometry factors that remain $\mathcal{O}(1)$ while preserving
the independent experimental anchoring of $V_0$ to a breakdown-scale observable (e.g. Schwinger onset).

\subsection{Summary: What is Derived vs.\ What is Assumed}
The AVE framework replaces ``fundamental'' $(G,\hbar)$ with emergent engineering limits of the
substrate. The independent primitives are $(l_0,V_0)$, supplemented only by order-unity geometric
coarse-graining factors $(\kappa_C,\kappa_E,\kappa_G)$ capturing microscopic packing and energy
partition.

The derived relationships are:
\begin{align}
\hbar_{\mathrm{AVE}} &= \frac{\kappa_E\kappa_C\,\epsilon_0\,l_0^2\,V_0^2}{c},\\
G_{\mathrm{AVE}} &= \frac{c^4}{\kappa_G\kappa_E\kappa_C\,\epsilon_0\,V_0^2}.
\end{align}
Planck-unit quantities are treated strictly as \emph{outputs} (consistency checks), never as
inputs. The model is therefore falsified if matching empirical $(\hbar,G)$ requires geometry factors
that are not $\mathcal{O}(1)$ or if $V_0$ cannot be independently anchored to a breakdown-scale
observable.

\subsubsection{Notation Convention: Primitives vs. Derived Units}
To ensure rigorous separation of inputs and outputs:
\begin{itemize}
    \item \textbf{Hardware Inputs:} We use $l_0$ (Lattice Pitch) and $V_0$ (Breakdown Voltage) to denote the independent physical properties of the $M_A$ manifold.
    \item \textbf{Derived Outputs:} We reserve the standard Planck symbols ($l_P, E_P$) strictly for the calculated values derived from $\hbar_{AVE}$ and $G_{AVE}$.
    \item \textbf{Consistency:} In this text, $l_0 \equiv l_{hardware}$ and $l_P \equiv l_{calculated}$.
\end{itemize}

\subsubsection{Design Note 1.1: The Universality Lemma (Constraining $\kappa$)}
A critical requirement of the AVE framework is that the geometric coarse-graining factors $(\kappa_{C}, \kappa_{E}, \kappa_{G})$ are \textbf{Universal Constants} of the amorphous manifold, not free parameters.

We postulate that these factors are determined strictly by the statistical topology of the Delaunay mesh:
\begin{itemize}
    \item $\kappa_{C}$ (Capacitive Geometry): Determined by the mean coordination number $\langle k \rangle \approx 15.54$.
    \item $\kappa_{G}$ (Stiffness Coupling): Determined by the Shear/Bulk modulus ratio of the node packing.
\end{itemize}
\textbf{The Universality Constraint:}
\begin{equation}
    \frac{\partial \kappa}{\partial E} = \frac{\partial \kappa}{\partial t} = 0
\end{equation}
The $\kappa$ factors are invariant under local stress, temperature, or energy density (within the linear regime). They cannot be "tuned" to fit data; they must be derived from the graph statistics of the $M_{A}$ substrate.