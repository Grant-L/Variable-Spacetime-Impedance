\section{Theoretical Constraints on Fundamental Constants}
\label{sec:theoretical_constraints}

Standard physics treats $G$ and $\hbar$ as unexplained, fundamental scalars. In the DCVE framework, we propose they are strictly emergent scaling factors derived from the foundational hardware primitives: Lattice Pitch ($l_0$) and the Schwinger Yield Energy Density ($u_{sat}$). 

\subsection{Derived Action Scale (Quantum of Action)}
We define the maximum action capacity of a single node as the product of its maximum storable energy ($E_{sat}$) and the fundamental hardware update time ($t_{tick}$). 

Given the volumetric saturation limit $E_{sat} = u_{sat} l_0^3$ and the lattice clock speed $t_{tick} = l_0 / c$:
\begin{equation}
    \hbar_{AVE} \equiv E_{sat} \cdot t_{tick} = (u_{sat} l_0^3) \left(\frac{l_0}{c}\right) = \frac{u_{sat} l_0^4}{c}
\end{equation}
This derivation is dimensionally exact: $[\text{J}/\text{m}^3] \cdot [\text{m}^4] / [\text{m}/\text{s}] = [\text{J} \cdot \text{s}]$. It dynamically links the quantum of action to the volumetric dielectric yield limit of the substrate without relying on fabricated scaling voltages or circular Planck-unit definitions.

\subsection{Derived Gravitational Coupling as Lattice Tension}
To connect the microscopic substrate to macroscopic gravity, we identify the universal stiffness of the vacuum. The maximum transmissible mechanical force across a single 1D lattice edge before topological rupture is the Vacuum String Tension ($T_{vac}$):
\begin{equation}
    T_{vac} \equiv \frac{E_{sat}}{l_0} = u_{sat} l_0^2 \quad [\text{Newtons}]
\end{equation}
Equating this physical substrate tension to the Einstein-Hilbert proportionality constant ($c^4 / G$), we derive the emergent gravitational coupling:
\begin{equation}
    \frac{c^4}{G_{AVE}} \propto u_{sat} l_0^2 \implies G_{AVE} \propto \frac{c^4}{u_{sat} l_0^2}
\end{equation}
Gravity is thus mathematically recast as the mechanical compliance parameter of the Cosserat hardware layer. A stiffer vacuum dielectric (higher $u_{sat}$) produces a smaller $G_{AVE}$.