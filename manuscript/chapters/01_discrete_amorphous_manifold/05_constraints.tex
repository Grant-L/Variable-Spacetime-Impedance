\section{Theoretical Constraints on Fundamental Constants}
\label{sec:constraints}

We propose that $G$ and $\hbar$ are not arbitrary scalars, but emergent geometric properties of the lattice packing and saturation limits. To avoid circular definitions, we derive them strictly from \textbf{Axiom VI} ($V_{break}$).

\subsection{Derivation of the Planck Action ($\hbar$)}
We derive the Planck Action as the Maximum Action Capacity of a single node. The energy stored in a node at breakdown voltage is $E_{sat} = \frac{1}{2} C_{node} V_{break}^2$. Substituting the local capacitance $C_{node} \approx \epsilon_0 l_P$, and multiplying by the hardware clock cycle ($t_{tick} = l_P/c$):
\begin{equation}
\hbar \equiv E_{sat} \cdot t_{tick} = \left(\frac{1}{2} \epsilon_0 l_P V_{break}^2\right)\left(\frac{l_P}{c}\right)
\end{equation}
This rigorously identifies $\hbar$ not as a primary constant, but as the derived action limit of the capacitive substrate.

\subsection{The Gravitational Constant ($G$) as Lattice Compliance}
The Gravitational Constant is a derived measure of the lattice's Mechanical Compliance. The Yield Force $F_{yield}$ is the force required to displace the lattice by one pitch ($l_P$) against the saturation energy $E_{sat}$. Equating this to the Einstein Stiffness ($c^4/G$):
\begin{equation}
\frac{c^4}{G} = \frac{E_{sat}}{l_P} \implies G = \frac{c^4 l_P}{E_{sat}}
\end{equation}
Gravity is mechanically defined: a ``stiffer'' lattice (higher $V_{break}$) results in a weaker gravitational coupling $G$.