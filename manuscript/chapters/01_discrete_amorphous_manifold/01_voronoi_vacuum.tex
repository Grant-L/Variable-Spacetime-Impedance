
\section{The Amorphous Manifold}
The foundational postulate of the AVE framework is that the physical universe is a Discrete Amorphous Manifold ($M_A$). Let $P$ be a set of stochastic points distributed in a topological volume $V$. The physical manifold $M_A$ is defined as the Delaunay Triangulation of $P$.

\begin{definition}[\textbf{The Amorphous Manifold}]
    Let $\mathcal{P}$ be a set of stochastic points distributed in a topological volume $\mathcal{V}$ with mean density $\rho_{node}$. The physical manifold $M_A$ is defined as the \textbf{Delaunay Triangulation} of $\mathcal{P}$.
    \begin{itemize}
        \item \textbf{Nodes ($V$):} The active processing elements of the vacuum.
        \item \textbf{Edges ($E$):} The flux transmission lines connecting nearest neighbors.
        \item \textbf{Cells ($\Phi$):} The Voronoi cells representing the effective volume of each node.
    \end{itemize}
\end{definition}

\subsection{The Fundamental Lattice Pitch ($l_{0}$)}
Just as a digital image has a pixel size, the vacuum has a fundamental granularity. We define the \textbf{Lattice Pitch} ($l_{0}$) as the mean edge length of the graph:
\begin{equation}
    l_{0} = \langle |e_{ij}| \rangle \approx 1.6 \times 10^{-35} \text{ m}
\end{equation}
This length scale is the physical separation between the inductive nodes of the substrate. It imposes a "Hardware Cutoff" frequency ($\omega_{max} \approx c/l_{0}$) on all physical signals, naturally preventing ultraviolet divergences.

\textit{Calibration Note:} While $l_{0}$ is numerically close to the derived Planck length ($l_P = \sqrt{\hbar G/c^3}$), in AVE $l_{0}$ is a primary input parameter of the mesh. We assume a calibration such that the hardware limit matches the observable high-energy cutoff.

\subsection{Isotropy via Stochasticity: The Rifled Vacuum}
\label{sec:isotropy}

A common critique of discrete spacetime models is the "Manhattan Distance" problem. On a regular cubic grid, diagonal movement is mathematically longer than cardinal movement ($\sqrt{2}$ vs 1), which violates Lorentz Invariance and would cause the speed of light to vary with direction.

The $M_A$ framework evades this by requiring the lattice to be \textbf{Amorphous} (Random) rather than Crystalline.

\subsubsection*{Theorem 1.2 (Isotropic Averaging)}
For a Delaunay graph generated from a stochastic Poisson distribution, the effective path length approaches rotational invariance at macroscopic scales ($L \gg l_P$).

\begin{equation}
    \lim_{N \to \infty} \mathcal{L} f(x) \approx \nabla^2 f(x)
\end{equation}

While the photon performs a random walk at the micro-scale (The Jagged Path), the Graph Laplacian ($\mathcal{L}$) converges to the continuous Laplace-Beltrami operator ($\nabla^2$) at the macro-scale. The vacuum looks smooth to us for the same reason a sandy beach looks smooth from an airplane: the grains ($l_P$) are stochastic, and the signal is gyroscopically stabilized.

\textbf{Physical Result:} Light travels at the same speed in every direction. The vacuum looks smooth to us for the same reason a sandy beach looks smooth from an airplane: the grains ($l_P$) are stochastic and infinitesimally small.

\subsection{Connectivity Analysis}
Unlike a crystalline lattice with a fixed coordination number (e.g., 6 for cubic, 12 for FCC), the vacuum substrate possesses a statistical distribution of connectivity. Monte Carlo analysis of $N=10,000$ nodes yields a mean coordination number:
\begin{equation}
    \langle k \rangle \approx 15.54
\end{equation}
This high degree of connectivity ensures that the vacuum is "Over-Braced," providing the extreme mechanical stiffness required to support the propagation of transverse waves (light) at $c$ while minimizing dispersive loss.