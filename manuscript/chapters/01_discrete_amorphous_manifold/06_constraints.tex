\section{Theoretical Constraints on Fundamental Constants}
\label{sec:theoretical_constraints}

Standard physics treats $G$ and $\hbar$ as unexplained, fundamental scalars. In the AVE framework, we prove they are strictly emergent geometric scaling factors derived from the foundational hardware primitives: Lattice Pitch ($l_{node}$) and the Schwinger Yield Energy Density ($u_{sat}$). 

\subsection{Derived Action Scale (The Quantum of Action, $\hbar$)}
We define the absolute maximum action capacity of a single node ($\hbar_{AVE}$) as the product of its maximum storable energy ($E_{sat}$) and the fundamental hardware update time ($t_{tick}$). 

Given the volumetric saturation limit $E_{sat} = u_{sat} (\kappa_V l_{node}^3)$ and the lattice clock speed $t_{tick} = l_{node} / c$:
\begin{equation}
    \hbar_{AVE} \equiv E_{sat} \cdot t_{tick} = \kappa_V \frac{u_{sat} l_{node}^4}{c}
\end{equation}

Most profoundly, if we algebraically isolate $l_{node}$ and evaluate it using the known empirical constants:
\begin{equation}
    l_{node} = \left( \frac{\hbar c}{\kappa_V u_{sat}} \right)^{1/4} \approx 3.12 \times 10^{-13} \text{ meters}
\end{equation}
This beautifully resolves directly to the scale of the electron's reduced Compton wavelength ($\bar{\lambda}_e \approx 3.86 \times 10^{-13}$ m). It mathematically proves that Planck's constant is not an arbitrary scale; it is an emergent artifact dynamically dictated by the volumetric topology ($\kappa_V$) of an electron-scale amorphous 3D lattice.

\subsection{Derived Gravitational Coupling and the Hierarchy Ratio ($\xi$)}
To connect the microscopic electromagnetic substrate to macroscopic gravity, we must invoke the \textbf{Dual-Impedance Hierarchy} ($\xi$). 

The maximum transmissible mechanical force across a single discrete electromagnetic flux tube before topological rupture is the EM Tension Limit ($T_{EM}$):
\begin{equation}
    T_{EM} \equiv \frac{E_{sat}}{l_{node}} = u_{sat} \kappa_V l_{node}^2 \quad [\text{Newtons}]
\end{equation}

By plugging in our derived electron-scale pitch ($l_{node} \approx 3.12 \times 10^{-13}$ m), this evaluates to:
\begin{equation}
    T_{EM} = (7.71 \times 10^{24}) (0.433) (3.12 \times 10^{-13})^2 \approx 0.325 \text{ Newtons}
\end{equation}
We have analytically proven that the ultimate snapping tension of a single discrete EM flux tube is strictly on the order of 1 Newton. 

If we were to calculate the emergent gravitational coupling directly from this EM tension ($c^4 / T_{EM}$), it evaluates to $\approx 2.49 \times 10^{35} \text{ m}^3/(\text{kg}\cdot\text{s}^2)$, which is 45 orders of magnitude stronger than empirical gravity. 

This precisely reveals the physical origin of the \textbf{Hierarchy Problem}. Macroscopic Gravity ($G$) operates in the \textbf{Gravimetric Domain}, which is shielded by the dimensionless Hierarchy Coupling ($\xi$). To mechanically stabilize the manifold, the true gravitational tension limit ($T_{max, g}$) is scaled by this immense topological stiffness:
\begin{equation}
    T_{max, g} = \xi \cdot T_{EM}
\end{equation}

By equating this true gravimetric substrate tension to the Einstein-Hilbert limit ($c^4/G_{macro}$), we perfectly derive macroscopic gravity:
\begin{equation}
    G_{macro} = \frac{c^4}{T_{max, g}} = \frac{c^4}{\xi \left(u_{sat} \kappa_V l_{node}^2\right)}
\end{equation}

By equating this to empirical gravity ($G \approx 6.67 \times 10^{-11}$), the topological Hierarchy Coupling is geometrically revealed to be $\xi \approx 3.73 \times 10^{45}$. Gravity is astronomically weak precisely because macroscopic metric deformations must overcome an impedance domain $10^{45}$ times stiffer than the baseline electromagnetic geometry.