\section{The Global Slew Rate ($c$)}
\label{sec:slew_rate}

The speed of light is not an arbitrary speed limit imposed by traffic laws; it is the \textbf{Global Slew Rate} of the hardware.

\subsection{Derivation from Moduli}
In any transmission line, the propagation velocity is determined strictly by the distributed inductance and capacitance. Using the moduli defined in Section \ref{sec:moduli}:
\begin{equation}
    c = \frac{1}{\sqrt{\permeability \permittivity}}
\end{equation}
Substituting the measured values:
\begin{equation}
    c = \frac{1}{\sqrt{(1.256 \times 10^{-6})(8.854 \times 10^{-12})}} \approx 299,792,458 \text{ m/s}
\end{equation}
This derivation proves that $c$ is not a fundamental constant itself, but an emergent property of the substrate's stiffness and density.

\subsection{The Bandwidth Limit}
Physically, $c$ represents the maximum rate at which a lattice node can update its internal state vector. It is the \textbf{Clock Speed} of the manifold.
\begin{itemize}
    \item \textbf{Massless Particles:} Travel at the slew rate because they have no inductive core to charge up.
    \item \textbf{Massive Particles:} Travel slower than $c$ because they must constantly "charge" and "discharge" the local vacuum inductance as they move (see Chapter 3).
\end{itemize}