\section{Dielectric Saturation Limit}
\label{sec:saturation}

Every physical material has a breakdown voltage. The vacuum is no exception. We define the \textbf{Planck Voltage} ($V_P$) as the saturation limit of the lattice.

\subsection{The Schwinger Limit}
Standard QED predicts that at an electric field strength of $E_{crit} \approx 1.32 \times 10^{18}$ V/m, the vacuum "boils," spontaneously generating electron-positron pairs. In Vacuum Engineering, this is the point where the capacitive edges of the graph ($E$) rupture.

\subsection{Non-Linear Response}
Below this limit, the vacuum acts as a linear medium (Hooke's Law). Near this limit, the stress-strain curve becomes non-linear.
\begin{equation}
    D = \permittivity E + \chi^{(3)} E^3 + \dots
\end{equation}
This non-linearity is crucial for:
\begin{enumerate}
    \item \textbf{Particle Genesis:} Creating stable topological knots (Matter).
    \item \textbf{Black Holes:} Regions where the lattice is stressed to maximal density.
\end{enumerate}
We postulate that the \textbf{Planck Energy} is simply the total energy storage capacity of a single lattice cell before dielectric breakdown occurs.