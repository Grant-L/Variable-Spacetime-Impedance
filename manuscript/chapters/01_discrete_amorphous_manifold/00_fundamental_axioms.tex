\section{The Fundamental Axioms of Vacuum Engineering}
\label{sec:axioms}

To eliminate circular definitions and reduce the universe to a mechanical substrate, the \textbf{Applied Vacuum Electrodynamics (AVE)} framework rests entirely on six fundamental hardware axioms. All other physics are derived as emergent behaviors of these limits.

\begin{itemize}
    \item \textbf{Axiom I: The Discrete Substrate Limit ($l_0$).} The universe is not a continuous geometry, but a discrete, amorphous transmission network. The mean edge length between nodes is the fundamental Lattice Pitch ($l_0$). This is an empirical hardware primitive, bounded by high-energy cosmic ray cutoffs to approximately $1.616 \times 10^{-35} \text{ m}$.
    \item \textbf{Axiom II: The Constitutive Moduli ($\mu_0, \epsilon_0$).} Each node acts as a reactive circuit element possessing Inductance Density ($\mu_0$, resistance to flux displacement) and Capacitance Density ($\epsilon_0$, elastic charge storage).
    \item \textbf{Axiom III: The Global Slew Rate ($c$).} The speed of light is the maximum signal propagation slew rate of the discrete network: $c = 1/\sqrt{\mu_0\epsilon_0}$.
    \item \textbf{Axiom IV: The Saturable Dielectric Condition.} Near breakdown ($U \approx U_{sat}$), the capacitance clamps to a maximum saturation value, localizing energy as stable topological knots (Matter).
    \item \textbf{Axiom V: The Generative Manifold ($H_0$).} The continuous quantum potential underlying the graph constantly crystallizes into new discrete nodes at the Genesis Rate ($H_0 \approx 2.3 \times 10^{-18} \text{ Hz}$).
    \item \textbf{Axiom VI: The Fundamental Breakdown Voltage ($V_0$).} The absolute maximum potential difference a single node can sustain before the lattice bonds rupture is an empirically anchored hardware limit: $V_0 \approx 1.04 \times 10^{27} \text{ V}$.
\end{itemize}