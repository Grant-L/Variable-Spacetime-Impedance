\section{The Fundamental Axioms of Vacuum Engineering}
\label{sec:axioms}

To eliminate circular definitions and reduce the universe to a mechanical substrate, the Applied Vacuum Electrodynamics (AVE) framework operationalizes the Variable Spacetime Impedance (VSI) theory into four hardware axioms.

\begin{axiom}[The Topological Domain $\mathcal{G}$]
The physical universe is strictly defined as a dynamic graph $\mathcal{G}(V, E, t)$ resulting from the Delaunay Triangulation of a stochastic point process $P \subset \mathbb{R}^3$.
\begin{itemize}
    \item \textbf{Fundamental Length ($l_{node}$):} The expectation value of the edge length distribution is fixed: $\langle |e_{ij}|\rangle \equiv l_{node}$.
    \item \textbf{Constraint:} The graph is simple, undirected, and globally connected.
\end{itemize}
\end{axiom}

\begin{axiom}[The State Variables]
Physics is encoded entirely in two conjugate variables defined on the graph elements, governed by the Geometrodynamic Ansatz ($1V \equiv 1m$):
\begin{enumerate}
    \item \textbf{Node Potential ($\phi_n$):} A scalar field $\phi: V \rightarrow \mathbb{R}$ representing the longitudinal dielectric strain (Displacement/Voltage).
    \item \textbf{Metric Flux ($\Phi_{ij}$):} A vector link variable representing the momentum density state of the edge.
\end{enumerate}
There are no other fundamental fields.
\end{axiom}

\begin{axiom}[The Discrete Action Principle]
The system evolves to minimize the \textbf{Hardware Action} $S_{AVE}$. The action is the discrete sum of the Nodal Lagrangian:
\begin{equation}
S_{AVE} = \int dt \sum_{n \in V} \mathcal{L}_{node}
\end{equation}
To ensure dimensional homogeneity (Joules), the Lagrangian relates the Dielectric Potential Energy to the Inductive Kinetic Energy:
\begin{equation}
\label{eq:discrete_lagrangian}
\mathcal{L}_{node} = \underbrace{\frac{1}{2} C_{eff}(\Delta \phi) \sum_{j \in \text{neigh}(n)} (\phi_n - \phi_j)^2}_{\text{Dielectric Potential ($CV^2$)}} - \underbrace{\frac{1}{2} (L_{node} C_{node}^2) (\partial_t \phi_n)^2}_{\text{Inductive Kinetic ($LI^2$)}}
\end{equation}
Here, the kinetic term explicitly accounts for the conversion of potential rate ($\dot{\phi}$) to displacement current ($I \approx C \dot{\phi}$).
\end{axiom}

\begin{axiom}[The Saturation Operator]
The vacuum is a non-linear dielectric. The effective capacitance $C_{eff}$ is a function of the local potential gradient relative to the \textbf{Lattice Tension Limit} ($V_{max}$):
\begin{equation}
\label{eq:saturation_op}
C_{eff}(\Delta \phi) = \frac{C_0}{\sqrt{1 + \left(\frac{\Delta \phi}{V_{max}}\right)^4}}
\end{equation}
where $V_{max} \equiv T_{max} \cdot C_0 / C_{node}$ represents the potential equivalent of the mechanical breaking force.
\end{axiom}

\subsection{Implications of the Axiom Set}
From these four hardware specifications, the standard "laws" of physics are derived as theorems of the substrate limit:
\begin{itemize}
    \item \textbf{The Wave Equation:} In the limit $\Delta \phi \ll V_{max}$, the Lagrangian recovers the invariant speed of light $c = l_{node}/\sqrt{L_{node}C_{node}}$.
    \item \textbf{Mass Hierarchy:} In the limit $\Delta \phi \to V_{max}$, the Quartic term in Eq.~\ref{eq:saturation_op} dominates, forcing the discrete scaling laws observed in particle generations.
    \item \textbf{Event Horizon:} If $\Delta \phi > V_{max}$, the real-valued solution to the capacitance ceases to exist, representing the physical rupture of the manifold (Singularity).
\end{itemize}