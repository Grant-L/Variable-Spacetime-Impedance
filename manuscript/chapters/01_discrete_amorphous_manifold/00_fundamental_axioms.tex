\section{The Fundamental Axioms of Vacuum Engineering}
\label{sec:axioms}

To eliminate circular definitions and reduce the universe to a mechanical substrate, the Applied Vacuum Electrodynamics (AVE) framework rests entirely on four fundamental hardware axioms. Unlike standard physics, which postulates laws (e.g., Maxwell's Equations) as primary, AVE postulates a \textit{Discrete Action Principle} from which these laws emerge.

\begin{axiom}[The Topological Domain $\mathcal{G}$]
The physical universe is strictly defined as a dynamic graph $\mathcal{G}(V, E, t)$ resulting from the Delaunay Triangulation of a stochastic point process $P \subset \mathbb{R}^3$.
\begin{itemize}
    \item \textbf{Fundamental Length ($l_0$):} The expectation value of the edge length distribution is fixed: $\langle |e_{ij}| \rangle \equiv l_0$.
    \item \textbf{Constraint:} The graph is simple, undirected, and globally connected.
\end{itemize}
\end{axiom}

\begin{axiom}[The State Variables]
Physics is encoded entirely in two conjugate variables defined on the graph elements:
\begin{enumerate}
    \item \textbf{Node Potential ($\phi_n$):} A scalar field $\phi: V \rightarrow \mathbb{R}$ representing the longitudinal dielectric strain (Compression).
    \item \textbf{Edge Flux ($U_{ij}$):} A unitary link variable $U: E \rightarrow U(1)$ representing the transverse phase transport (Twist).
\end{enumerate}
There are no other fundamental fields.
\end{axiom}

\begin{axiom}[The Discrete Action Principle]
The system evolves to minimize the \textbf{Hardware Action} $S_{AVE}$. The action is defined not as a continuous integral, but as a discrete sum over nodes ($n$) and edges ($ij$):
\begin{equation}
S_{AVE} = \int dt \sum_{n \in V} \mathcal{L}_{node}
\end{equation}
Where the discrete Lagrangian density $\mathcal{L}_{node}$ is:
\begin{equation}
\label{eq:discrete_lagrangian}
\mathcal{L}_{node} = \underbrace{\frac{1}{2} C_{eff}(\Delta \phi) \sum_{j \in \text{neigh}(n)} (\phi_n - \phi_j)^2}_{\text{Dielectric Strain (Potential)}} - \underbrace{\frac{1}{2} L_{node} (\partial_t \phi_n)^2}_{\text{Inductive Inertia (Kinetic)}}
\end{equation}
Here, $L_{node} \equiv \mu_0 l_0$ represents the inertial mass of the node.
\end{axiom}

\begin{axiom}[The Saturation Operator]
The vacuum is a non-linear dielectric. The effective capacitance $C_{eff}$ is not constant but is a function of the local potential gradient, governed by the Breakdown Voltage ($V_0$):
\begin{equation}
\label{eq:saturation_op}
C_{eff}(\Delta \phi) = \frac{C_0}{\sqrt{1 + \left(\frac{\Delta \phi}{V_0}\right)^4}}
\end{equation}
where $C_0 \equiv \epsilon_0 l_0$ is the vacuum capacitance in the linear (low-energy) limit.
\end{axiom}

\subsection{Implications of the Axiom Set}
From these four hardware specifications, the standard "laws" of physics are derived as theorems of the substrate limit:
\begin{itemize}
    \item \textbf{The Wave Equation:} In the limit $\Delta \phi \ll V_0$, the Lagrangian in Eq.~\ref{eq:discrete_lagrangian} reduces to the standard discrete wave equation, recovering the speed of light $c = 1/\sqrt{L_{node}C_{node}}$.
    \item \textbf{Mass Hierarchy:} In the limit $\Delta \phi \to V_0$, the Quartic term in Eq.~\ref{eq:saturation_op} dominates, forcing the $N^9$ scaling law observed in lepton generations.
    \item \textbf{Breakdown:} If $\Delta \phi > V_0$, the real-valued solution to the capacitance ceases to exist, representing the physical rupture of the manifold (Event Horizon).
\end{itemize}