\section{Real vs. Reactive Power: The Orbital Friction Paradox}

A historical and persistent critique of analog fluidic spacetime models is the "Friction Paradox": \textit{If a planet is physically moving through a dense spatial condensate, why doesn't fluidic drag drain its kinetic energy, causing its orbit to decay over cosmological timescales?}

Within the VCA framework, this paradox is resolved flawlessly by rigorously distinguishing between non-conservative fluidic drag and conservative AC Power Analysis. As established in Chapter 11, exceeding the Bingham yield limit ($\tau > \tau_{yield}$) does not merely result in a classical viscous fluid; it triggers an avalanche dielectric phase-transition. The local metric structurally melts into an irrotational, continuous quantum fluid. Because this continuous melted phase mathematically cannot support transverse shear vectors, the localized fluidic viscosity strictly collapses to zero ($\eta \to 0$). 

Therefore, the anti-parallel fluidic drag force ($F_{drag}$) that would typically drain real energy at a $180^\circ$ phase angle mathematically evaluates to exactly zero Newtons.

With non-conservative drag structurally eliminated via this topological phase transition, we evaluate the remaining thermodynamic interaction using electrical engineering power principles. Total apparent power ($S$) is divided into two distinct components depending on the phase angle ($\theta$) between Voltage ($V$) and Current ($I$):
\begin{enumerate}
\item \textbf{Real Power ($P$):} Measured in Watts. $P = VI \cos(\theta)$. This represents energy physically dissipated from the system (e.g., heat, mechanical friction).
\item \textbf{Reactive Power ($Q$):} Measured in Volt-Amperes Reactive (VARs). $Q = VI \sin(\theta)$. This represents energy conservatively exchanged back and forth without permanent dissipation.
\end{enumerate}

By applying the Topo-Kinematic Identity to the remaining conservative interactions, the radial Gravitational Force vector acts identically as the AC Voltage ($V_{condensate} \propto F_g$), and the tangential Orbital Velocity vector acts as the AC Current ($I_{condensate} \propto v_{orb}$). In a stable, circular planetary orbit, the radial gravitational force vector is perfectly and mathematically orthogonal ($90^\circ$) to the tangential velocity vector. Therefore, the phase angle between the vacuum Voltage and Current is exactly $\theta = 90^\circ$. 

Evaluating the Real Power physically dissipated by the planetary body into the vacuum fluid via the conservative gravity well yields:
\begin{equation}
P_{real} = F_g \cdot v_{orb} \cdot \cos(90^\circ) \equiv 0\ \text{Watts}
\end{equation}

Because fluidic drag is neutralized by the dielectric phase transition, and the remaining gravitational coupling is purely orthogonal, the orbiting body experiences absolutely zero macroscopic energy dissipation. A stable planetary orbit is the macroscopic mechanical equivalent of a \textbf{Lossless LC Tank Circuit} operating purely in the reactive power domain (VARs), continuously conserving its stored energy without thermodynamically heating the ambient vacuum fluid.