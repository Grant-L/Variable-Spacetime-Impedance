\section{Topological Defects as Resonant LC Solitons}

As established in prior chapters, a fundamental particle is a stable topological defect---a highly tensioned phase vortex permanently locked into the discrete graph structure. In classical electrical engineering, a localized, trapped electromagnetic standing wave that permanently cycles reactive energy without radiative loss is defined as a \textbf{Resonant LC Tank Circuit}. By applying the Topo-Kinematic mapping to the electron's rest mass, its equivalent localized Inductance evaluates to $L_e \equiv \xi_{topo}^{-2} m_e$. The local lattice compliance acts as the restoring capacitor ($C_e \equiv \xi_{topo}^2 k^{-1}$).

\subsection{Recovering the Virial Theorem and $E=mc^2$}
We can rigorously verify this structural mapping by evaluating the stored energy of the resonant soliton. In an ideal LC tank, the peak internal dynamic (inductive) energy is defined as $E_{mag} = \frac{1}{2} L_e I_{max}^2$. Substituting the hardware velocity limit ($I_{max} = \xi_{topo} c$) evaluates to:
\begin{equation}
E_{mag} = \frac{1}{2} (\xi_{topo}^{-2} m_e)(\xi_{topo} c)^2 = \frac{1}{2} m_e c^2
\end{equation}

In a stable LC resonant soliton, the classical Virial Theorem rigidly dictates that the capacitive (electric/strain) energy stored in the static topological twist of the core must exactly equal the inductive kinetic energy ($E_{elec} = E_{mag} = \frac{1}{2}m_e c^2$). Summing the two isolated energy ledgers perfectly recovers $E_{total} = m_e c^2$ \cite{einstein1916}. Einstein's mass-energy equivalence principle is mechanically and mathematically identical to the Total Stored Electrical Energy of a classical macroscopic Resonant LC Tank Circuit ringing natively within the analog vacuum metric.

\section{Real vs. Reactive Power: The Orbital Friction Paradox}

A historical and persistent critique of analog fluidic spacetime models is the "Friction Paradox": \textit{If a planet is physically moving through a dense spatial condensate, why doesn't fluidic drag drain its kinetic energy, causing its orbit to decay over cosmological timescales?}

Within the VCA framework, this paradox is resolved flawlessly by rigorously distinguishing between non-conservative fluidic drag and conservative AC Power Analysis. As established in Chapter 11, exceeding the Bingham yield limit ($\tau > \tau_{yield}$) does not merely result in a classical viscous fluid; it triggers an avalanche dielectric phase-transition. The local metric structurally melts into an irrotational, continuous quantum fluid. Because this continuous melted phase mathematically cannot support transverse shear vectors, the localized fluidic viscosity strictly collapses to zero ($\eta \to 0$). Therefore, the anti-parallel fluidic drag force ($F_{drag}$) mathematically evaluates to exactly zero Newtons \cite{flyby2008}.

With non-conservative drag structurally eliminated, we evaluate the remaining thermodynamic interaction using electrical engineering power principles. Total apparent power ($S$) is divided into two distinct components depending on the phase angle ($\theta$) between Voltage ($V$) and Current ($I$):
\begin{enumerate}
\item \textbf{Real Power ($P$):} Measured in Watts. $P = VI \cos(\theta)$. This represents energy physically dissipated from the system.
\item \textbf{Reactive Power ($Q$):} Measured in Volt-Amperes Reactive (VARs). $Q = VI \sin(\theta)$. This represents energy conservatively exchanged back and forth without permanent dissipation.
\end{enumerate}

By applying the Topo-Kinematic Identity to the remaining conservative interactions, the radial Gravitational Force vector acts identically as the AC Voltage ($V_{condensate} \propto F_g$), and the tangential Orbital Velocity vector acts as the AC Current ($I_{condensate} \propto v_{orb}$). In a stable, circular planetary orbit, the radial gravitational force vector is perfectly and mathematically orthogonal ($90^\circ$) to the tangential velocity vector. Therefore, the phase angle between the vacuum Voltage and Current is exactly $\theta = 90^\circ$. 

Evaluating the Real Power physically dissipated by the planetary body into the vacuum fluid via the conservative gravity well yields:
\begin{equation}
P_{real} = F_g \cdot v_{orb} \cdot \cos(90^\circ) \equiv 0\ \text{Watts}
\end{equation}

Because fluidic drag is neutralized by the dielectric phase transition, and the remaining gravitational coupling is purely orthogonal, the orbiting body experiences absolutely zero macroscopic energy dissipation. A stable planetary orbit is the macroscopic mechanical equivalent of a \textbf{Lossless LC Tank Circuit} operating purely in the reactive power domain.

\begin{figure}[htbp]
    \centering
    \includegraphics[width=\textwidth]{orbital_reactive_power.png}
\caption{\textbf{Orbital Mechanics as Reactive AC Power.} Because the topological voltage (gravitational force) is perfectly 90-degrees out of phase with the spatial current (orbital velocity), the Real Power (Watts) dissipated by the planetary body evaluates identically to zero. The orbit operates as a pure LC reactive circuit, elegantly resolving the classical fluid friction paradox in condensed-matter models of the vacuum.}
    \label{fig:orbital_power}
\end{figure}