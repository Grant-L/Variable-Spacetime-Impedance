\chapter{Vacuum Circuit Analysis: Equivalent Network Models}
\label{ch:vca}

A primary goal of the Applied Vacuum Engineering (AVE) framework is to construct a rigorous, analytical bridge between theoretical topological physics and applied macroscopic engineering. Because the vacuum substrate is formally modeled as an Effective Field Theory (EFT) of a structurally constrained, non-linear discrete condensate ($\mathcal{M}_A$), the macroscopic kinematics of spacetime can be mathematically approximated using the established tools of Transient Circuit Analysis and Equivalent Circuit Modeling.

By translating physical continuum mechanics into their lumped-element electrical equivalents, we can utilize standard Electronic Design Automation (EDA) methodologies to explore complex relativistic phenomena---including inertial damping, dielectric breakdown limits, and non-linear wave propagation. We formalize this translation as \textbf{Spacetime Circuit Analysis (SCA)}.


\section{The Topo-Kinematic Circuit Identity}
To map continuum mechanics to electrical networks, we rely on the Topological Conversion Constant ($\xi_{topo} \equiv e/l_{node}$), which defines the fundamental dimensional isomorphism between spatial dislocation and electrical charge. 

In standard SI units, electrical charge ($Q$) is the time integral of current ($Q = \int I \, dt$). By substituting our kinematic mapping for current ($I \equiv \xi_{topo} v$), we derive the absolute mechanical identity of charge within the condensate:

\begin{equation}
Q = \int (\xi_{topo} v) \, dt = \xi_{topo} \int v \, dt = \xi_{topo} x
\end{equation}

Electrical charge is physically isomorphic to \textbf{Macroscopic Spatial Displacement} ($x$). We can rigorously verify this through the Work-Energy Theorem. The physical work done to charge a capacitor is $W = \int V \, dQ$. By substituting our topological identities for Voltage ($V \equiv \xi_{topo}^{-1} F$) and Charge ($dQ \equiv \xi_{topo} \, dx$):

\begin{equation}
W = \int (\xi_{topo}^{-1} F)(\xi_{topo} \, dx) = \int F \, dx
\end{equation}

The scaling constants flawlessly cancel. A capacitor storing electrical charge is mathematically identical to a mechanical lattice storing localized elastic spatial strain. Dielectric breakdown occurs precisely when the continuous spatial lattice is dynamically displaced beyond its absolute physical yield limit.
\section{Constitutive Circuit Models for Vacuum Non-Linearities}

Standard circuit simulators rely on ideal, linear RLC components[cite: 2085]. However, physical topological condensates exhibit highly non-linear behaviors under extreme mechanical stress[cite: 2086]. By applying the Topo-Kinematic identity, we can construct the exact non-linear equivalent circuit components of the spatial metric[cite: 2087].

\subsection{1. The Metric Varactor (Modeling Dielectric Yield)}
As defined by Axiom 4, the effective compliance (capacitance) of the spatial substrate is structurally bounded by the absolute classical dielectric saturation limit ($V_{crit} \equiv \alpha$)[cite: 2088]. As the local topological potential approaches this limit, the effective capacitance increases non-linearly[cite: 2089]. This structurally mirrors a Voltage-Dependent Varactor Diode, rigorously yielding the 4th-order bounding required to satisfy standard optical Kerr effect limits[cite: 2090]:
\begin{equation}
C_{vac}(V) = \frac{C_0}{\sqrt{1 - (V/V_{crit})^4}}
\end{equation}

\subsection{2. The Relativistic Inductor (Lorentz Saturation)}
Because inertia maps to spatial inductance, and velocity maps to spatial current, the phenomenon of Special Relativity is identically modeled in Spacetime Circuit Analysis (SCA) as a non-linear inductor[cite: 2091]. The effective inductance saturates as the macroscopic current approaches the fundamental hardware propagation limit ($I_{max} = \xi_{topo} c$)[cite: 2092]:
\begin{equation}
L_{vac}(I) = \frac{L_0}{\sqrt{1 - (I/I_{max})^2}}
\end{equation}
This provides the mechanical rationale for why standard SPICE simulators natively cannot push current (matter) past $c$; the localized inductive drag asymptotes to infinity, perfectly mirroring the aerodynamic Prandtl-Glauert singularity[cite: 2092, 2093].

\subsection{3. The Viscoelastic TVS Zener Diode (Bingham Transition)}
In a Bingham Plastic continuum, viscosity yields strictly when subjected to extreme shear stress ($\tau > \tau_{yield}$)[cite: 2094]. Because macroscopic shear stress is proportional to mechanical force, vacuum liquefaction must act as a Voltage-Driven Breakdown[cite: 2095]. The vacuum substrate acts electrically as a Transient Voltage Suppression (TVS) Zener Diode[cite: 2096]. Below $V_{yield}$, it acts as a highly resistive solid (kinematically gripping matter)[cite: 2097]. Above $V_{yield}$, it enters avalanche breakdown, allowing frictionless superfluid slip[cite: 2098].

\subsection{4. The Vacuum Memristor (Thixotropic Hysteresis)}
Because the Bingham-plastic transition of the $\mathcal{M}_A$ condensate requires a finite geometric relaxation time ($\tau_{macro} \approx L/c$) to physically liquefy, the vacuum cannot alter its fluidic resistance instantaneously[cite: 2099]. Its state is rigidly dependent on the historical integral of the stress applied to it[cite: 2100].  Consequently, the physical vacuum completes the fundamental electronic quartet by acting as a \textbf{Macroscopic Memristor}, exhibiting a strict pinched hysteresis loop when subjected to high-frequency AC topological stress[cite: 2101].

\subsection{The Superfluid Skin Effect (Metric Faraday Cages)}
In standard electrical engineering, high-frequency alternating currents (AC) do not penetrate deeply into conductors; they are pushed to the surface by opposing eddy currents[cite: 2102, 2103]. The penetration depth ($\delta$) of the signal is strictly proportional to the square root of the medium's electrical resistance ($\delta \propto \sqrt{R_{elec}}$)[cite: 2104]. Because the AVE framework rigorously maps Vacuum Resistance identically to Vacuum Viscosity ($R_{vac} \equiv \eta_{vac}$), the Electromagnetic Skin Effect and the Hydrodynamic Boundary Layer are mathematically identical phenomena[cite: 2105].

As the local metric yields past the Bingham limit ($V > V_{yield}$) and the vacuum transitions into a superfluid, the local resistance of the metric collapses to near-zero ($R_{vac} \to 0$)[cite: 2106]. Because the resistance drops, the Metric Skin Depth mathematically collapses to zero[cite: 2107]. This provides a profound engineering constraint: the destructive, high-shear superfluid slipstream generated by macroscopic metric manipulation is strictly confined to the exterior boundary (the hull) of the vessel[cite: 2108]. The interior metric acts as a \textbf{Topological Faraday Cage}, physically shielding the interior from extreme structural shear[cite: 2109].

\begin{figure}[htbp]
    \centering
    \includegraphics[width=\textwidth]{chapters/13_spacetime_circuit_analysis/simulations/outputs/memristor_and_skineffect.png}
\caption{\textbf{The Vacuum Memristor and Superfluid Skin Effect.} Left: Because the Bingham-plastic vacuum requires a finite thixotropic relaxation time to yield, it acts as a Macroscopic Memristor, producing a classic Pinched Hysteresis loop under AC drive[cite: 2110]. Right: As the applied topological voltage exceeds the Bingham yield limit (Red Line) and the vacuum liquefies, the AC skin depth ($\delta$) drops to zero, proving the destructive shear layer cannot penetrate the interior metric[cite: 2111].}
    \label{fig:skin_effect}
\end{figure}
\section{The Impedance of Free Space ($Z_0$)}

A foundational parameter in classical electromagnetism is the Characteristic Impedance of Free Space ($Z_0 = \sqrt{\mu_0/\epsilon_0} \approx 376.73\ \Omega$)[cite: 2112]. In Spacetime Circuit Analysis, this possesses a literal mechanical identity[cite: 2113]. By applying our mapping, electrical impedance ($Z = V/I$) translates directly to Mechanical Acoustic Impedance ($Z_m = F/v$)[cite: 2114]:
\begin{equation}
Z_{elec} = \frac{V}{I} = \frac{\xi_{topo}^{-1} F}{\xi_{topo} v} = \xi_{topo}^{-2} \left(\frac{F}{v}\right) = \xi_{topo}^{-2} Z_m
\end{equation}

Rearranging for the mechanical impedance reveals an exact physical identity[cite: 2114]:
\begin{equation}
Z_m = \xi_{topo}^2 \cdot Z_0 = \xi_{topo}^2 \sqrt{\frac{\mu_0}{\epsilon_0}} \approx 6.48 \times 10^{-11} \left[ \frac{\text{kg}}{\text{s}} \right]
\end{equation}
The $376.7\ \Omega$ impedance of free space is structurally isomorphic to the Absolute Mechanical Acoustic Impedance of the physical $\mathcal{M}_A$ substrate[cite: 2114].

\section{Gravitational Stealth (S-Parameter Analysis)}

In classical RF engineering, when a wave transitions into a denser physical medium (e.g., from air to glass), the refractive index ($n$) rises asymmetrically, forcing the characteristic impedance to drop[cite: 2115, 2116]. This impedance mismatch causes the signal to partially reflect, measured logarithmically as Return Loss ($S_{11}$)[cite: 2116]. This introduces a profound paradox for analog gravity models: \textit{If a gravity well represents a physical increase in the localized optical density of the vacuum, why does light seamlessly enter a black hole without scattering or reflecting off the boundary?} [cite: 2117]

In the SCA transmission line model, macroscopic gravity operates strictly as a 3D Volumetric Compression of the Cosserat solid[cite: 2118]. This localized geometric crowding proportionately and \textit{symmetrically} increases both the effective inductive mass density ($\mu_{local} = n(r) \cdot \mu_0$) and the capacitive compliance ($\epsilon_{local} = n(r) \cdot \epsilon_0$)[cite: 2118]. Evaluating the Characteristic Impedance of the vacuum down to the extreme metric divergence of an Event Horizon ($r \to R_s$) reveals a perfect mathematical invariant[cite: 2119]:
\begin{equation}
Z_{local}(r) = \sqrt{\frac{\mu_{local}}{\epsilon_{local}}} = \sqrt{\frac{n(r) \cdot \mu_0}{n(r) \cdot \epsilon_0}} = \sqrt{\frac{\mu_0}{\epsilon_0}} \equiv Z_0 \approx 376.73\ \Omega
\end{equation}

The $\mathcal{M}_A$ condensate is mathematically and perfectly Impedance-Matched to itself everywhere, absolutely regardless of extreme gravitational strain[cite: 2119]. Because the spatial derivative of the impedance remains strictly zero ($\partial_r Z_0 = 0$), the Reflection Coefficient ($\Gamma$) is mathematically forced to zero[cite: 2120]. The universe structurally possesses an \textbf{$S_{11}$ Return Loss of $-\infty$ dB}[cite: 2121]. This provides the exact continuum-mechanics mechanism for why localized gravitational gradients act as perfect RF-absorbing stealth structures rather than optical mirrors[cite: 2122].

\begin{figure}[htbp]
    \centering
    \includegraphics[width=\textwidth]{chapters/13_spacetime_circuit_analysis/simulations/outputs/log_impedance_s_parameters.png}
\caption{\textbf{S-Parameter Analysis of a Gravity Well.} Top: As a wave approaches a gravitational core, the density $n(r)$ diverges[cite: 2123]. Because analog macroscopic gravity compresses volumetric space, it scales $L$ and $C$ symmetrically, ensuring the Characteristic Impedance ($Z_0$) remains perfectly invariant[cite: 2124]. Bottom: If gravity behaved like an unmatched optical dielectric, the resulting impedance drop would generate massive reflection ($S_{11} > -10$ dB)[cite: 2125]. The symmetric volumetric scaling of the AVE EFT forces $S_{11} \to -\infty$ dB, providing the precise mechanism for why intense gravity wells do not act as RF mirrors[cite: 2126].}
    \label{fig:s_parameters}
\end{figure}

\subsection{The Condensate Transmission Line (Emergence of $c$)}
To computationally prove that macroscopic Special Relativity emerges deterministically from these discrete components, we modeled the 1D spatial vacuum grid as a cascaded LC transmission line using the AVE-SPICE ordinary differential equation solver[cite: 2127]. By normalizing the discrete Inductors ($\mu_0 l_{node}$) and Capacitors ($\epsilon_0 l_{node}$) to the hardware pitch, the injection of a transient topological voltage pulse confirms that the signal propagates through the discrete components at exactly the continuous group velocity $v_g = 1/\sqrt{LC} \equiv c$[cite: 2127]. The continuous, invariant speed of light is mathematically identically the macroscopic slew-rate of a discrete transmission line[cite: 2128].

\begin{figure}[htbp]
    \centering
    \includegraphics[width=\textwidth]{chapters/13_spacetime_circuit_analysis/simulations/outputs/condensate_transmission_line.png}
\caption{\textbf{The EFT Transmission Line.} A time-domain simulation of a discrete 100-node vacuum grid[cite: 2129]. By cascading the discrete inductive mass and capacitive compliance of the analog lattice, the signal propagates flawlessly at $v_g = c$, proving that continuous spacetime kinematics emerge natively from lumped-element circuit analysis[cite: 2130].}
    \label{fig:transmission_line}
\end{figure}
\section{Topological Defects as Resonant LC Solitons}
If the unperturbed spacetime condensate is modeled as a passive, linear cascaded transmission line, physical matter (Fermions) can be evaluated dynamically within the SCA framework. As established in prior chapters, a fundamental particle is a stable topological defect---a highly tensioned phase vortex permanently locked into the discrete graph structure.

In classical electrical engineering, a localized, trapped electromagnetic standing wave that permanently cycles reactive energy without radiative loss is defined as a \textbf{Resonant LC Tank Circuit}. 

By applying the Topo-Kinematic mapping to the electron's rest mass, its equivalent localized Inductance evaluates to $L_e \equiv \xi_{topo}^{-2} m_e$. The local lattice compliance acts as the restoring capacitor ($C_e \equiv \xi_{topo}^2 k^{-1}$). 

\subsection{Recovering the Virial Theorem and $E=mc^2$}
We can rigorously verify this structural mapping by evaluating the stored energy of the resonant soliton. In an ideal LC tank, the peak internal dynamic (inductive) energy is $E_{mag} = \frac{1}{2} L_e I_{max}^2$. Substituting the hardware velocity limit ($I_{max} = \xi_{topo} c$) evaluates to:

\begin{equation}
E_{mag} = \frac{1}{2} (\xi_{topo}^{-2} m_e)(\xi_{topo} c)^2 = \frac{1}{2} m_e c^2
\end{equation}

This elegantly resolves the historical "4/3 Electromagnetic Mass Paradox" regarding internal Poincar\'e Stresses. In a stable LC resonant soliton, the classical Virial Theorem rigidly dictates that the capacitive (electric/strain) energy stored in the static topological twist of the core must exactly equal the inductive kinetic energy ($E_{elec} = E_{mag} = \frac{1}{2}m_e c^2$). 

Summing the two isolated energy ledgers perfectly recovers $E_{total} = m_e c^2$. Einstein's mass-energy equivalence principle is mechanically and mathematically identical to the Total Stored Electrical Energy of a classical macroscopic Resonant LC Tank Circuit ringing natively within the analog vacuum metric. 
\section{Topological Defects as Resonant LC Solitons}

As established in prior chapters, a fundamental particle is a stable topological defect---a highly
tensioned phase vortex permanently locked into the discrete graph structure. In classical
electrical engineering, a localized, trapped electromagnetic standing wave that permanently
cycles reactive energy without radiative loss is defined as a \textbf{Resonant LC Tank Circuit}.

By applying the Topo-Kinematic mapping to the electron's rest mass, its equivalent localized
Inductance evaluates to $L_e \equiv \xi_{topo}^{-2} m_e$. The local lattice compliance acts as the restoring
capacitor ($C_e \equiv \xi_{topo}^2 k^{-1}$).

\subsection{Recovering the Virial Theorem and $E=mc^2$}

We can rigorously verify this structural mapping by evaluating the stored energy of the
resonant soliton. In an ideal LC tank, the peak internal dynamic (inductive) energy is defined
as $E_{mag} = \frac{1}{2} L_e I_{max}^2$. Substituting the hardware velocity limit ($I_{max} = \xi_{topo} c$) evaluates to:

\begin{equation}
E_{mag} = \frac{1}{2} (\xi_{topo}^{-2} m_e)(\xi_{topo} c)^2 = \frac{1}{2} m_e c^2
\end{equation}

In a stable LC resonant soliton, the classical Virial Theorem rigidly dictates that the
capacitive (electric/strain) energy stored in the static topological twist of the core must exactly
equal the inductive kinetic energy ($E_{elec} = E_{mag} = \frac{1}{2}m_e c^2$). Summing the two isolated energy
ledgers perfectly recovers $E_{total} = m_e c^2$ \cite{einstein1916}. Einstein's mass-energy equivalence principle is
mechanically and mathematically identical to the Total Stored Electrical Energy of a classical
macroscopic Resonant LC Tank Circuit ringing natively within the analog vacuum metric.

\subsection{Total Internal Reflection: The Confinement Bubble}

A fundamental requirement for any discrete particle (soliton) model is explaining why the
localized wave-packet does not instantly disperse its stored energy into the ambient vacuum.
In the AVE framework, this geometric stability is mathematically guaranteed by the extreme
flux crowding at the particle's boundary, which generates a perfect macroscopic impedance
mismatch.

Unlike the symmetric volumetric compression of macroscopic gravity (which keeps $Z_{0}$
perfectly invariant, preventing scattering), the localized topological twist of a particle core
induces extreme dielectric saturation. As the local topological strain ($\Delta\phi$) approaches the
Axiom 4 hardware limit ($\alpha$), the effective geometric capacitance (compliance) of the boundary
nodes diverges to infinity:

\begin{equation}
\lim_{\Delta\phi \to \alpha} C_{eff}(\Delta\phi) = \lim_{\Delta\phi \to \alpha} \frac{C_{0}}{\sqrt{1-\left(\frac{\Delta\phi}{\alpha}\right)^{2}}} = \infty
\end{equation}

Because the characteristic impedance of a spatial cell is dictated by $Z=\sqrt{L/C}$, this
massive spike in boundary capacitance drives the localized impedance of the particle boundary
strictly to zero:

\begin{equation}
\lim_{C_{eff} \to \infty} Z_{core} = \lim_{C_{eff} \to \infty} \sqrt{\frac{\mu_{0}}{C_{eff}}} = 0\,\Omega
\end{equation}

In standard wave mechanics, the Reflection Coefficient ($\Gamma$) governing the transmission of
energy across a boundary is defined by the impedance differential between the two media.
Evaluating the boundary between the saturated particle core ($0\,\Omega$) and the unperturbed
ambient vacuum ($Z_{0}\approx376.7\,\Omega$) yields:

\begin{equation}
\Gamma=\frac{Z_{core}-Z_{0}}{Z_{core}+Z_{0}}=\frac{0-376.7}{0+376.7}=-1
\end{equation}

A reflection coefficient of $\Gamma=-1$ constitutes a \textbf{Perfect Short-Circuit Boundary}.

This mathematical limit proves that 100\% of the kinetic energy attempting to radiate
outward from the saturated flux tube hits this impedance wall, undergoes a perfect $180^{\circ}$ phase
inversion, and reflects internally. Mechanically, the nodes at the saturation boundary are
geometrically jammed at the absolute hard-sphere exclusion limit. The local phase velocity
($c_{local}=1/\sqrt{LC}$) strictly collapses to zero, creating a hyper-rigid, localized envelope. The
particle dynamically weaves its own perfect topological mirror, forming an impenetrable,
hyper-highly-reluctant ``Local Bubble'' that perfectly confines the internal LC resonance without
radiative loss.

\textbf{Deriving the QCD Linear Potential:} Furthermore, this provides the strict deter-
ministic mechanism for Strong Force flux collimation. Rather than radiating isotropically
($1/r^{2}$), the energy traveling between nucleons undergoes Total Internal Reflection (TIR) off
the impedance walls of the highly strained vacuum, acting as a Topological Fiber-Optic Cable.

By applying Gauss's Law to a confined 1D cylinder of constant cross-sectional area,
the electric flux density ($D$) mathematically cannot spread radially outward. The electric
flux remains perfectly constant along the entire length of the tube, absolutely regardless
of separation distance. Consequently, the restorative force ($F(r) = \text{constant}$) inherently
generates the exact \textbf{Linear Confinement Potential} ($V(r)\propto r$) empirically observed in
Quantum Chromodynamics. The phenomenological ``MIT Bag Model'' is directly exposed
as a macroscopic impedance wall woven natively by the non-linear varactor limits of the
continuous vacuum.

\subsection{The Mechanical Origin of the Pauli Exclusion Principle}

The establishment of the saturated particle boundary as a perfect topological mirror ($\Gamma = -1$)
provides a rigorous, continuous-mechanical derivation for the Pauli Exclusion Principle. 

In standard quantum mechanics, the inability of fermions to occupy the same quantum
state is treated as an abstract statistical postulate. In the AVE framework, it is an unavoidable
consequence of classical macroscopic impedance boundaries. 

When massless Bosons (photons) propagate, they act as linear transverse shear waves.
Because they do not possess a static inductive core, they do not geometrically saturate
the dielectric lattice ($\Delta\phi \ll \alpha$). The local metric impedance remains perfectly matched at
$Z_{0} \approx 376.7\,\Omega$. With a reflection coefficient of $\Gamma \approx 0$, boson waves pass cleanly through one
another, permitting infinite superposition.

Conversely, Fermions are massive topological defects bounded by strictly saturated $Z_{core} =
0\,\Omega$ envelopes. If two fermions are forced into the same spatial volume, their boundaries
collide. Because both boundaries possess a reflection coefficient of strictly $\Gamma = -1$, their
internal localized wave-functions cannot mathematically penetrate one another. The kinetic
energy of Fermion A perfectly reflects off the infinite-compliance wall of Fermion B. The Pauli
Exclusion Principle is therefore physically identical to the hard-sphere collision of perfectly
impedance-mismatched dielectric bubbles.

\section{Real vs. Reactive Power: The Orbital Friction Paradox}

A historical and persistent critique of analog inductive spacetime models is the ``Friction Paradox'':
\textit{If a planet is physically moving through a dense spatial condensate, why doesn't inductive drag
drain its kinetic energy, causing its orbit to decay over cosmological timescales?}

Within the VCA framework, this paradox is resolved flawlessly by rigorously distinguishing
between non-conservative inductive drag and conservative AC Power Analysis. As established
in Chapter 11, exceeding the Dielectric Saturation limit ($\tau > \tau_{yield}$) does not merely result in a
classical highly-reluctant network; it triggers an avalanche dielectric phase-transition. The local metric
structurally melts into an irrotational, continuous quantum network. Because this continuous
melted phase mathematically cannot support transverse shear vectors, the localized inductive
mutual inductance strictly collapses to zero ($\eta \to 0$). Therefore, the anti-parallel inductive drag force
($F_{drag}$) mathematically evaluates to exactly zero Newtons \cite{flyby2008}.

With non-conservative drag structurally eliminated, we evaluate the remaining thermody-
namic interaction using electrical engineering power principles. Total apparent power ($S$) is
divided into two distinct components depending on the phase angle ($\theta$) between Voltage ($V$)
and Current ($I$):
\begin{enumerate}
\item \textbf{Real Power ($P$):} Measured in Watts. $P = VI \cos(\theta)$. This represents energy physically
dissipated from the system.
\item \textbf{Reactive Power ($Q$):} Measured in Volt-Amperes Reactive (VARs). $Q = VI \sin(\theta)$.
This represents energy conservatively exchanged back and forth without permanent
dissipation.
\end{enumerate}

By applying the Topo-Kinematic Identity to the remaining conservative interactions, the
radial Gravitational Force vector acts identically as the AC Voltage ($V_{condensate} \propto F_g$), and
the tangential Orbital Velocity vector acts as the AC Current ($I_{condensate} \propto v_{orb}$). In a stable,
circular planetary orbit, the radial gravitational force vector is perfectly and mathematically
orthogonal ($90^\circ$) to the tangential velocity vector. Therefore, the phase angle between the
vacuum Voltage and Current is exactly $\theta = 90^\circ$. 

Evaluating the Real Power physically dissipated by the planetary body into the vacuum
network via the conservative gravity well yields:
\begin{equation}
P_{real} = F_g \cdot v_{orb} \cdot \cos(90^\circ) \equiv 0\ \text{Watts}
\end{equation}

Because inductive drag is neutralized by the dielectric phase transition, and the remaining
gravitational coupling is purely orthogonal, the orbiting body experiences absolutely zero
macroscopic energy dissipation. A stable planetary orbit is the macroscopic mechanical
equivalent of a \textbf{Lossless LC Tank Circuit} operating purely in the reactive power domain.
\section{Condensate IMD Spectroscopy: The Harmonic Fingerprint}
By modeling the universe as a non-linear network, we can extract the exact theoretical signature of the AVE framework using standard RF analysis techniques \cite{nyquist1928, shannon1949}.

\paragraph{The 3rd-Order Falsification Test:}
Standard Quantum Electrodynamics (QED) models the vacuum as a linear medium at low energies, predicting that photon-photon scattering (light-by-light scattering) only occurs via extraordinarily weak perturbative quantum fluctuations. However, Axiom 4 mandates a strict, macroscopic classical squared geometric saturation limit ($1 - V^2$) for the physical vacuum condensate.
\begin{equation}
    C_{vac}(V) = \frac{C_0}{\sqrt{1 - (V/V_{crit})^2}}
\end{equation}

\begin{figure}[h]
    \centering
    \includegraphics[width=0.9\textwidth]{vacuum_dielectric_saturation.png}
\caption{\textbf{The Squared Dielectric Saturation Limit.} Unlike standard perturbative QED (Dashed Black), the AVE condensate (Red) imposes a macroscopic geometric varactor asymptote at $V_{crit}$. This classical structural non-linearity is the specific source of the macroscopic intermodulation products predicted in the simulation.}
    \label{fig:dielectric_saturation}
\end{figure}

\paragraph{Predicted Signal:}
This specific non-linear varactor curvature strictly forces the physical vacuum to act as a macroscopic RF mixer. Simulations using the AVE-SPICE solver demonstrate that when driven by a dual-tone macroscopic signal ($f_1, f_2$) approaching the breakdown voltage, the vacuum generates highly distinct \textbf{3rd-Order Intermodulation Products} (specifically $2f_1 - f_2$ and $2f_2 - f_1$). Measuring the exact amplitude trajectory of these 3rd-order sidebands against the $1/\sqrt{1-V^2}$ varactor limit provides a direct, accessible tabletop falsification test of the macroscopic physical hardware graph versus the standard continuous, linear vacuum.