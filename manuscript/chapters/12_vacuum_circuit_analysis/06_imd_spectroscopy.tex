\section{Condensate IMD Spectroscopy: The Harmonic Fingerprint}
By modeling the universe as a non-linear network, we can extract the exact theoretical signature of the AVE framework using standard RF analysis techniques \cite{nyquist1928, shannon1949}.

\paragraph{The 3rd-Order Falsification Test:}
Standard Quantum Electrodynamics (QED) models the vacuum as a linear medium at low energies, predicting that photon-photon scattering (light-by-light scattering) only occurs via extraordinarily weak perturbative quantum fluctuations. However, Axiom 4 mandates a strict, macroscopic classical squared geometric saturation limit ($1 - V^2$) for the physical vacuum condensate.
\begin{equation}
    C_{vac}(V) = \frac{C_0}{\sqrt{1 - (V/V_{crit})^2}}
\end{equation}

\begin{figure}[h]
    \centering
    \includegraphics[width=0.9\textwidth]{vacuum_dielectric_saturation.png}
\caption{\textbf{The Squared Dielectric Saturation Limit.} Unlike standard perturbative QED (Dashed Black), the AVE condensate (Red) imposes a macroscopic geometric varactor asymptote at $V_{crit}$. This classical structural non-linearity is the specific source of the macroscopic intermodulation products predicted in the simulation.}
    \label{fig:dielectric_saturation}
\end{figure}

\paragraph{Predicted Signal:}
This specific non-linear varactor curvature strictly forces the physical vacuum to act as a macroscopic RF mixer. Simulations using the AVE-SPICE solver demonstrate that when driven by a dual-tone macroscopic signal ($f_1, f_2$) approaching the breakdown voltage, the vacuum generates highly distinct \textbf{3rd-Order Intermodulation Products} (specifically $2f_1 - f_2$ and $2f_2 - f_1$). Measuring the exact amplitude trajectory of these 3rd-order sidebands against the $1/\sqrt{1-V^2}$ varactor limit provides a direct, accessible tabletop falsification test of the macroscopic physical hardware graph versus the standard continuous, linear vacuum.