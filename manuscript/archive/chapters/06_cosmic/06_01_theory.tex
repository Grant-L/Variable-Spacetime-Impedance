\section{Generative Cosmology: The Crystallizing Vacuum}
\label{sec:generative_theory}

\subsection{The Lattice Genesis Hypothesis}
We propose that the vacuum manifold $M_A$ is \textbf{Generative}. The Lattice Tension ($P_{vac}$) identified in Chapter 2 drives a continuous phase transition: the crystallization of new lattice nodes from the quantum substrate.

\subsection{Derivation of the Genesis Rate ($H_0$)}
Let $N(t)$ be the total number of nodes along a line of sight. The Lattice Tension induces a proliferation of nodes proportional to the existing volume (geometric growth):
\begin{equation}
    \frac{dN}{dt} = R_g N(t)
\end{equation}
Where $R_g$ is the \textbf{Node Genesis Rate} (Hz). Solving for $N(t)$:
\begin{equation}
    N(t) = N_0 e^{R_g t}
\end{equation}

\subsection{Recovering the Hubble Parameter}
The physical distance $D$ is the node count $N$ times the Lattice Pitch $l_P$. The recession velocity $v$ is the rate of growth:
\begin{equation}
    v = \frac{dD}{dt} = l_P \frac{dN}{dt} = l_P (R_g N) = R_g D
\end{equation}
Comparing this to Hubble's Law ($v = H_0 D$), we identify the Hubble Constant mechanically:
\begin{equation}
    H_0 \equiv R_{genesis}
\end{equation}
\textbf{Conclusion:} The "Expansion of the Universe" is simply the real-time refresh rate of the vacuum hardware.