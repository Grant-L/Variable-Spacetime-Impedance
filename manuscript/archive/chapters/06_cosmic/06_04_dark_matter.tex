\section{Resolution of the Dark Matter Anomaly}
\label{sec:dark_matter}

The Bullet Cluster (1E 0657-558) represents the strongest evidence for particulate Dark Matter. Observations show a separation between the visible baryonic gas (which interacts and slows down) and the gravitational potential (which passes through unhindered).
Standard Cosmology $(\Lambda CDM)$ explains this by postulating that 85\% of the mass is collisionless, invisible Dark Matter.

VSI provides an alternative resolution based on \textbf{Inductive Crowding} (Section 4.4).

\subsection{The Density-Inductance Non-Linearity}
In VSI, mass is not merely a count of nucleons; it is a measure of Total Inductive Energy.
\begin{equation}
    M_{total} \propto \sum N_{knots} \cdot \Omega_{complexity}
\end{equation}
As derived in the Lepton Hierarchy ($\gamma \approx 9$), condensed topological structures (high crossing density) exhibit significantly higher inductance per nucleon than diffuse structures due to mutual flux coupling.

\begin{itemize}
    \item \textbf{Galaxies (Condensed):} Stars and Black Holes are regions of extreme topological density. They benefit from the $N^9$ inductive multiplier. They act as "Heavy Inductors."
    \item \textbf{Intergalactic Gas (Diffuse):} The plasma is diffuse, with negligible mutual inductance between particles. It acts as "Linear Resistance."
\end{itemize}

\subsection{The Collision Mechanism}
When two clusters collide:
\begin{enumerate}
    \item \textbf{The Gas:} Being diffuse and collisional, the plasma interacts via direct lattice friction (viscosity), converting kinetic energy into heat (X-Rays) and slowing down.
    \item \textbf{The Galaxies:} Being compact, high-inductance solitons, they possess enormous "Inductive Inertia" (Back-EMF). They plow through the vacuum with minimal braking.
    \item \textbf{The Gravity:} Since the Galaxies (Condensed) carry the vast majority of the Inductive Mass (despite having fewer baryons than the gas), the gravitational potential follows the galaxies, not the gas.
\end{enumerate}

\begin{figure}[ht]
    \centering
    \includegraphics[width=1.0\textwidth]{assets/sim_outputs/bullet_cluster_result.png}
    \caption{\textbf{VSI Bullet Cluster Simulation:} The red heatmap shows the X-Ray gas slowing down due to viscosity. The blue contours show the gravitational potential following the galaxies. This separation arises naturally from the "Inductive Crowding" gain of condensed matter, without requiring non-baryonic Dark Matter.}
    \label{fig:bullet_cluster}
\end{figure}

\textbf{Conclusion:} "Dark Matter" is an illusion caused by assuming mass scales linearly with baryon count. In a VSI universe, \textbf{Structure Weighs More Than Dust.}