\section{The Origin of Inertia as Back-EMF}
\label{sec:inertia}

In classical mechanics, inertia is an axiom ($F=ma$). In the VSI framework, inertia is emergent \textbf{Back-Electromotive Force (Back-EMF)}.

\subsection{The Inductive Resistance}
Because the manifold is inductive ($L_{node} \equiv \mu_0$), any attempt to change the flux current $I_\phi$ of a node (acceleration) is met with an opposing potential $\mathcal{E}$ generated by the lattice:
\begin{equation}
    \mathcal{E}_{back} = -L \frac{dI_\phi}{dt}
\end{equation}
Identifying the flux current change with acceleration ($dI/dt \propto a$) and the Back-EMF with the inertial force ($F_{inertial}$):
\begin{equation}
    F_{inertial} = - (m_{eff}) a
\end{equation}
\textbf{Conclusion:} Inertia is simply the manifold's inductive resistance to the change in flux density. "Mass" is the effective inductance of the topological knot.