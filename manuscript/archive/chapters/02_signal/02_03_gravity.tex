\section{Gravity as Metric Refraction}
\label{sec:gravity_refraction}

We have established (Chapter 1) that a mass $M$ creates a refractive index gradient $n(r)$ in the surrounding lattice:
\begin{equation}
    n(r) = 1 + \frac{2GM}{rc^2}
\end{equation}
In this section, we treat the propagation of light through this gradient as an optical problem.

\subsection{Fermat's Principle on the Lattice}
Light follows the path of least time (geodesic). In a variable index medium $n(r)$, the trajectory is governed by Snell's Law of Refraction.
The total deflection angle $\delta$ for a photon passing a mass $M$ at impact parameter $b$ is the integral of the refractive gradient perpendicular to the path:
\begin{equation}
    \delta = \int_{-\infty}^{\infty} \nabla_\perp n \, dz \approx \frac{4GM}{bc^2}
\end{equation}

\subsection{The No-Birefringence Proof}
A critical constraint is that gravity must not be birefringent (polarization-dependent).
This is satisfied by the \textbf{Impedance Matching Condition}:
\begin{equation}
    Z(r) = \sqrt{\frac{L(r)}{C(r)}} = \sqrt{\frac{L_0 \chi}{C_0 \chi}} = Z_0 \approx 377 \Omega
\end{equation}
Because the metric strain $\chi$ affects Inductance and Capacitance equally (Equipartition), the characteristic impedance $Z$ remains invariant.
\textbf{Result:} Light slows down (Gravity), but does not reflect or split (No Birefringence), perfectly matching General Relativity observational constraints.