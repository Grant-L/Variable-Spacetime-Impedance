\section{Introduction: The Activated Substrate}
\label{sec:signal_intro}

In Part I, we defined the vacuum as a static hardware substrate ($M_{A}$) characterized by finite inductance ($L_0$) and capacitance ($C_0$). However, a static lattice explains nothing. To describe the universe we observe—populated by light, matter, and energy—we must transition from \textbf{Hardware Architecture} to \textbf{Signal Dynamics}.

The "Signal Layer" treats the $M_{A}$ substrate as a 3D Transmission Line Grid. In this framework, "Physics" is simply the study of signal propagation through a reactive medium.

\subsection{The Transmission Line Analogy}
Classical mechanics treats space as a passive stage upon which particles move. The VSI framework inverts this relationship:
\begin{itemize}
    \item \textbf{The Medium is the Machine:} The vacuum nodes *are* the physics. A particle is not a distinct object moving *through* the lattice; it is a persistent state of excitation *of* the lattice.
    \item \textbf{Propagation is Handoff:} Motion is the sequential transfer of flux energy from one node to its neighbor. The speed of this transfer is strictly governed by the local impedance ($Z_0 = \sqrt{L/C}$).
\end{itemize}

\subsection{Time as Nodal Update Rate}
Time is not a fundamental dimension; it is the \textbf{Global Clock Rate} of the manifold.
\begin{equation}
    t_{tick} = \sqrt{L_{0}C_{0}} \cdot l_P \approx 5.39 \times 10^{-44} \text{ s}
\end{equation}
"Time Dilation" is mechanically defined as \textbf{Lattice Latency}: when a node is saturated by high energy density (mass), it requires more "cycles" to process a signal update, slowing the local effective clock.