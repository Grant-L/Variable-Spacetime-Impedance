\section{Practical VSI: The Circuit Board Trace}
\label{sec:circuit_inductance}

To visualize VSI in a non-astrophysical context, consider a standard copper trace on a high-speed Printed Circuit Board (PCB).

\subsection{The Copper Lattice as Standing Waves}
In standard electronics, we view the copper trace as a passive "pipe" for electrons. In VSI, the copper atoms are dense knots of standing-wave energy. The vacuum within the trace is "pre-stressed" by the presence of this matter.

\subsection{Signal Propagation and Saturation}
When a high-speed signal ($dI/dt$) propagates down the trace, it is not merely moving electrons; it is a **Flux Wave** attempting to displace the vacuum nodes.
\begin{enumerate}
    \item \textbf{Mutual Crowding:} The propagating flux must "push" against the standing waves of the copper lattice. This resistance is \textbf{Mutual Inductance}.
    \item \textbf{Dielectric Saturation:} High current density forces the local vacuum nodes near their breakdown limit ($U \to U_{sat}$).
    \item \textbf{Back-EMF:} As the lattice stiffness increases near saturation, it resists further changes in flux. This resistance is measured as **Parasitic Inductance** ($L_{p}$).
\end{enumerate}

\textbf{Conclusion:} The "Parasitic Inductance" that plagues high-speed CPU design is simply the Inertial Back-Reaction of the vacuum hardware. We are hitting the slew-rate limit of the space between the atoms.