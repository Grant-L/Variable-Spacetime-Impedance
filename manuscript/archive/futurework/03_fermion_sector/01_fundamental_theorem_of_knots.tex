\section{The Fundamental Theorem of Knots}

In the DCVE framework, ``Matter'' is not a substance distinct from the vacuum; it is a localized, self-sustaining topological knot in the vacuum's flux field. We posit that every stable elementary particle corresponds to a discrete graph topology. The physical properties of the particle must be derived strictly from the non-linear topology of this knot.

\subsection{Mass as Inductive Energy}

We have defined the vacuum edges as possessing distributed inductance $\mu_0$. Therefore, any closed loop of topological flux stores energy in the localized magnetic field:

\begin{equation}
    E_{mass} = \frac{1}{2} L_{eff} |A|^2
\end{equation}

Where $L_{eff}$ is the Effective Inductance of the knotted manifold. Mass is simply the Stored Inductive Energy required to maintain the topological integrity of the knot against the elastic pressure of the vacuum.

\textbf{Circuit Analogy: The Inductive Flywheel.} Why does mass resist acceleration? In DCVE, we replace the concept of ``Mass'' with the electrical concept of \textbf{Inductive Inertia}. A heavy flywheel resists changes in rotation; when you try to spin it up, it fights you (Back-EMF). An elementary particle is a knot of flux spinning so fast it acts as a Gyroscopic Flywheel. It resists acceleration not because it has ``stuff'' inside it, but because the magnetic field possesses Lenz's Law Inertia.
