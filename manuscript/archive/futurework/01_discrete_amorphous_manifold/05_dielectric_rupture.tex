\section{The Breakdown Limit: Dielectric Rupture}
\label{sec:breakdown_limit}

Every physical material has an ultimate tensile strength. We define the Breakdown Limit of the discrete manifold ($\mathcal{M}_A$) as the strict threshold where topological electrostatic connectivity ruptures, triggering pair-production (the Dielectric Snap).

\subsection{The Schwinger Yield Density ($u_{sat}$)}
In standard linear dielectrics, the volumetric energy density $u$ is defined as $u = \frac{1}{2} \epsilon_0 |\mathbf{E}|^2$. In Quantum Electrodynamics, the absolute critical electric field ($E_{crit}$) required to rip an electron-positron pair from the vacuum is strictly defined by the rest-mass limit: $E_{crit} = m_e^2 c^3 / (e \hbar) \approx 1.32 \times 10^{18}$ V/m. 

Therefore, the ultimate Yield Energy Density ($u_{sat}$) of the continuous vacuum substrate is dimensionally exact:
\begin{equation}
    u_{sat} = \frac{1}{2} \epsilon_0 E_{crit}^2 \approx 7.75 \times 10^{24} \left[\frac{\text{J}}{\text{m}^3}\right]
\end{equation}

\subsection{The Breakdown Voltage ($V_0$)}
Because the physical node size is identical to the pitch ($l_{node}$), the absolute maximum discrete electrical potential difference that can exist between two adjacent nodes before the string permanently snaps is the Nodal Breakdown Voltage ($V_0$).
\begin{equation}
    V_0 = E_{crit} \cdot l_{node} = \left( \frac{m_e^2 c^3}{e \hbar} \right) \left( \frac{\hbar}{m_e c} \right) = \mathbf{\frac{m_e c^2}{e}} \approx \mathbf{511.0 \text{ kV}}
\end{equation}
A 511 Kilovolt potential localized across a singular microscopic spatial step ($l_{node} \approx 3.86 \times 10^{-13}$ m) acts as the exact fundamental structural failure bound of the physical spatial metric.