\section{The Proton Mass: Resolving the 3D Tensor Deficit}

A fundamental mystery of the Standard Model is why the proton (938.27 MeV) is roughly 100 times heavier than the arithmetic mass sum of its three constituent quarks. In the AVE framework, this mass is not an arithmetic sum of independent parts; it is identically the integrated geometric impedance of the highly tensioned $6^3_2$ orthogonal linkage.

\subsection{The 1D Scalar Bound and the Tensor Gap}
In Chapter 1, we computed the 1D Scalar Baseline Limit for the $Q_H=9$ mass generation. Bounded purely by the scalar limit of the Axiom 4 dielectric saturation ($\alpha$), the analytical minimum bounded to $\approx 1162\times$ the mass of the electron. We analytically proved that the remaining $\sim 36\%$ structural deficit between $1162$ and the empirical $1836$ ratio was identically the magnitude of the missing \textbf{3D Transverse Torsional Tensor Strain ($\mathcal{I}_{tensor}$)}—energy generated by anisotropic flux tubes crossing orthogonally over each other, which a 1D spherical model truncates.

The precise mapping of the Proton to the Borromean linkage ($6^3_2$) is the triumphant physical realization of this exact geometric prediction. 

\subsection{Computational Bounding of the Borromean Manifold}
The mass of the proton emerges from the exact same topological field theory constraints applied to the lepton sector. We evaluate the Proton as a three-component linked defect in the Cosserat vacuum, mapped to the Faddeev-Skyrme non-linear Hamiltonian bounded by $\alpha$:
\begin{equation}
    E_{proton} = \min_{\mathbf{n}} \int_{\mathcal{M}_A} d^3x \left[ \frac{1}{2} (\partial_\mu \mathbf{n})^2 + \frac{1}{4} \kappa_{FS}^2 \frac{(\partial_\mu \mathbf{n} \times \partial_\nu \mathbf{n})^2}{\sqrt{1 - (\Delta\phi / \alpha)^4}} \right]
\end{equation}

Because the Borromean linkage cannot be untied without cutting a loop, it physically forces three distinct, mutually orthogonal flux tubes into the exact same minimal saturated core volume ($1~l_{node}^3$). As visualized in Figure \ref{fig:proton_borromean}, the loops must cross each other orthogonally in pairs. This structural frustration generates extreme \textbf{Orthogonal Tensor Strain}.

\begin{figure}[htbp]
    \centering
    \includegraphics[width=0.9\textwidth]{chapters/04_baryon_sector/simulations/outputs/baryon_mass_saturation.png}
    \caption{\textbf{Unification of Lepton and Baryon Masses.} The massive ratio of the Proton emerges natively from the exact same Axiom 4 saturation denominator that governs the Lepton generations. The structural frustration of forcing three orthogonal loops into the minimal core uniquely bridges the exact deficit between the 1D scalar bound ($\sim 1162$) and the full 3D tensor reality ($\sim 1836$).}
    \label{fig:baryon_mass_saturation}
\end{figure}

The empirical mass ratio $m_p / m_e \approx 1836.15$ is not an arbitrary arithmetic constant or a phenomenological tuning parameter. It is the exact, unyielding eigenvalue of non-linear inductive resonance. This extreme volumetric flux crowding drives the local electrical potential ($\Delta\phi$) asymptotically to the absolute spatial breakdown voltage ($\alpha$). The geometric capacitance crashes, causing the stored inductive mass-energy to spike exponentially. The exact mass emerges organically as the asymptotic lower-energy bound of this 3D non-linear gradient relaxation.