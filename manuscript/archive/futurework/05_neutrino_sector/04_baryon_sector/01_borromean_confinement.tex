\section{The Composite Baryon Sector}

In Chapter 3, we successfully derived the Lepton hierarchy (Electron, Muon, Tau) as single, isolated flux loops mapped to the $(p,2)$ torus knot sequence. However, the Baryon sector (Protons and Neutrons) introduces a fundamentally different class of topology. 

Baryons are not isolated single loops; they are \textbf{Composite Topological Linkages}. While leptons are defined by their internal crossing number, baryons are defined by the mutual entanglement of multiple distinct loops of momentum flux ($\mathbf{A}$) traversing the $\mathcal{M}_A$ Cosserat vacuum. The physical properties of the Baryon—including Confinement, the Strong Force, and Fractional Quarks—must be derived strictly from the non-linear topology of these composite linkages.

\section{Borromean Confinement: Deriving the Strong Force}

In the Standard Model, the Strong Nuclear Force is mediated by the continuous exchange of virtual gluons between point-like quarks carrying an abstract mathematical property called ``Color Charge.'' In the Applied Vacuum Engineering (AVE) framework, we permanently discard these abstract symmetries, replacing them with rigorous \textbf{Topological Geometry}.

We identify the Proton not as a bag of independent probabilistic point particles, but as a rigid \textbf{Borromean Linkage} of three continuous phase-flux loops ($6^3_2$) tensioned within the discrete $\mathcal{M}_A$ substrate.

\subsection{The Borromean Topology}
The Borromean Rings consist of three loops interlinked such that no two individual loops are linked to each other directly, but the three together form an inseparable topological triad.

\begin{itemize}
    \item \textbf{The Quark ($q$):} A single topological flux loop. Mathematically and physically unstable on its own (it cannot exist in isolation without instantly shedding its inductive energy and relaxing into the vacuum).
    \item \textbf{Topological Confinement:} If any single loop is cut or removed, the other two immediately fall apart into unknots.
\end{itemize}

This geometry intrinsically and rigidly enforces \textbf{Quark Confinement}. It is topologically impossible to isolate a single quark because the Borromean linkage requires the complete triad to establish the structural integrity of the localized topological defect.

\subsection{The Gluon Field as 1D Lattice Tension}
In standard Quantum Chromodynamics (QCD), the strong force does not drop off with distance like electromagnetism ($1/r^2$); it remains constant, forming a ``flux tube'' that binds quarks together. The Standard Model inserts this linear potential phenomenologically ($V(r) \propto \sigma r$). AVE derives it strictly from the absolute hardware limits of the continuous field.

Because the vacuum is an over-braced Cosserat solid governed by non-linear dielectric saturation (Axiom 4), extreme spatial separation causes the phase-flux lines connecting the Borromean loops to collimate tightly into a 1D cylindrical tube rather than spreading out isotropically into 3D space. 

The force required to stretch this collimated flux tube is exactly the absolute tensile breaking strength of the discrete edges. As mathematically derived in Chapter 1, the maximum force a discrete electromagnetic flux tube can sustain before the lattice ruptures is identically the \textbf{EM Tension Limit ($T_{EM}$)}:
\begin{equation}
    F_{confinement} = T_{EM} = \frac{m_e c^2}{l_{node}} \approx \mathbf{0.212 \text{ Newtons}}
\end{equation}

``Gluons'' are not discrete particles flying magically between quarks. They are the mathematical representation of the extreme \textbf{Static Elastic Stress} of the vacuum lattice physically trapped between the separating topological loops. As the loops are pulled apart, the restoring force remains absolutely constant at exactly $0.212$ N. The flux tube does not break until the stored elastic strain energy exceeds the classical pair-production threshold ($E > 2m_q c^2$), at which point the over-tensioned continuous field mathematically snaps and re-triangulates, creating a new stable linkage (a meson) rather than releasing a free quark.

\begin{figure}[htbp]
    \centering
    \includegraphics[width=0.9\textwidth]{chapters/04_baryon_sector/simulations/outputs/proton_borromean_tensor.png}
    \caption{\textbf{The Borromean Proton ($6^3_2$).} The discrete physical representation of Quark Confinement. The three distinct topological loops are mutually entangled. The ``Gluon Field'' is mathematically identical to the extreme mechanical strain ($0.212$ N) exerted on the $\mathcal{M}_A$ lattice nodes occupying the interstitial volume. The pure $\mathbb{Z}_3$ permutation symmetry naturally dictates the origin of SU(3) color rules (Cyan, Magenta, Yellow).}
    \label{fig:proton_borromean}
\end{figure}