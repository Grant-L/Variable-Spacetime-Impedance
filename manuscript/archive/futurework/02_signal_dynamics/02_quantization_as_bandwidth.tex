\section{Deriving the Quantum Formalism from Signal Bandwidth}
\label{sec:quantization_as_bandwidth}

Standard Quantum Mechanics posits its formalism—complex Hilbert spaces and non-commuting operators—as axiomatic magic. In the AVE framework, these are rigorously derived algebraic consequences of transmitting finite-bandwidth signals across a discrete mechanical graph ($\mathcal{M}_A$).

\subsection{The Paley-Wiener Hilbert Space ($\mathcal{H}$)}
Because the $\mathcal{M}_A$ lattice has a fundamental pitch $l_{node}$, it acts as an absolute spatial Nyquist sampling grid. The maximum spatial frequency the lattice can support without aliasing is the strict geometric boundary: $k_{max} = \pi / l_{node}$.

By the \textbf{Whittaker-Shannon Interpolation Theorem}, any physical continuous signal $\mathbf{A}(\mathbf{x})$ propagating through this discrete lattice that is perfectly band-limited can be reconstructed uniquely and continuously everywhere in space using a superposition of orthogonal sinc functions. Mathematically, the set of all such band-limited functions formally constitutes a Reproducing Kernel Hilbert Space known as the \textbf{Paley-Wiener Space} ($PW_{\pi/l_{node}}$). 

To cleanly map the real-valued physical lattice potential $\mathbf{A}(\mathbf{x},t)$ to the complex continuous quantum state vector $\Psi(\mathbf{x},t)$, we apply the standard signal-processing \textbf{Analytic Signal} representation using the Hilbert Transform ($\mathcal{H}_{transform}$):
\begin{equation}
    \Psi(\mathbf{x},t) = \mathbf{A}(\mathbf{x},t) + i \mathcal{H}_{transform}[\mathbf{A}(\mathbf{x},t)]
\end{equation}
\textit{Conclusion:} The complex continuous Hilbert space of quantum field theory is identically the Paley-Wiener signal-processing space of the discrete vacuum hardware.

\subsection{The Authentic Generalized Uncertainty Principle}
In standard QM, the non-commutativity of position and momentum ($[\hat{x}, \hat{p}] = i\hbar$) is an assumed axiom. On a discrete graph with pitch $l_{node}$, continuous coordinate translation is physically impossible. Furthermore, continuous momentum ($\hat{p}_c$) is strictly bounded by the Brillouin zone. 

For a macroscopic wave propagating through a stochastic 3D amorphous solid, the effective continuous momentum operator $\langle \hat{P} \rangle$ must be defined as an isotropic ensemble average of the symmetric central finite-difference operator across adjacent nodes:
\begin{equation}
    \langle \hat{P} \rangle \approx \frac{\hbar}{l_{node}} \sin\left(\frac{l_{node} \hat{p}_c}{\hbar}\right)
\end{equation}

By evaluating the exact commutator of the continuous position operator with this discrete lattice momentum ($[\hat{x}, f(\hat{p}_c)] = i\hbar f'(\hat{p}_c)$), we find:
\begin{equation}
    [\hat{x}, \langle \hat{P} \rangle] = i\hbar \cos\left(\frac{l_{node} \hat{p}_c}{\hbar}\right)
\end{equation}

Applying the generalized Robertson-Schr\"odinger relation yields the rigorously exact \textbf{Generalized Uncertainty Principle (GUP)} for the discrete vacuum:
\begin{tcolorbox}[colback=white, colframe=black]
\begin{equation}
    \Delta x \Delta P \ge \frac{\hbar}{2} \left| \left\langle \cos\left(\frac{l_{node} \hat{p}_c}{\hbar}\right) \right\rangle \right|
\end{equation}
\end{tcolorbox}
In the low-energy limit ($p_c \ll \hbar/l_{node}$), the cosine perfectly evaluates to $1$, flawlessly recovering Heisenberg's continuous principle ($\Delta x \Delta p \ge \hbar/2$). However, at extreme kinetic energies approaching the Brillouin boundary, the expectation value shrinks to zero, mathematically defining a hard, physical minimum length cutoff dictated exclusively by graph mechanics, completely eliminating ultraviolet singularities.

\subsection{Deriving the Schr\"odinger Equation from Circuit Resonance}
When a topological defect (mass) is synthesized within the graph, it acts as a localized inductive load, imposing a fundamental circuit resonance frequency ($\omega_m = mc^2/\hbar$). This mathematically transforms the massless wave equation into the massive \textbf{Klein-Gordon Equation}:
\begin{equation}
    \nabla^2 \mathbf{A} - \frac{1}{c^2}\frac{\partial^2 \mathbf{A}}{\partial t^2} = \left(\frac{mc}{\hbar}\right)^2 \mathbf{A}
\end{equation}

To map this relativistic classical evolution to non-relativistic quantum states, we apply the \textbf{Paraxial Approximation}, factoring out the rest-mass Compton frequency via a slow-varying envelope function $\mathbf{A}(\mathbf{x},t) = \Psi(\mathbf{x},t) e^{-i \omega_m t}$. 

For non-relativistic speeds ($v \ll c$), the second time derivative of the envelope ($\partial_t^2 \Psi$) is mathematically negligible. The strict mass resonance terms precisely cancel out, leaving exactly:
\begin{equation}
    \nabla^2 \Psi + \frac{2im}{\hbar} \frac{\partial \Psi}{\partial t} = 0 \quad \implies \quad i\hbar \frac{\partial \Psi}{\partial t} = -\frac{\hbar^2}{2m} \nabla^2 \Psi
\end{equation}
The Schr\"odinger Equation is mathematically proven to be the paraxial envelope equation of a classical macroscopic pressure wave propagating through the discrete massive $LC$ circuits of the vacuum.