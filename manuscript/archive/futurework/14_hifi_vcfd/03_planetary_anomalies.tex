\section{VCFD Planetary Anomalies: The Solar System Audit}

A unified fluid-dynamic metric must naturally and inevitably resolve the remaining orbital and geophysical anomalies of the planets. If the vacuum is a hyper-dense Bingham-plastic fluid ($\rho_{bulk} \approx 7.9 \times 10^6 \text{ kg/m}^3$), these mysteries drop out of the equations mathematically as straightforward fluid-dynamic boundary interactions and AC circuit losses.

\subsection{Mercury's "Impossible" Magnetic Field}
Standard Dynamo Theory strictly dictates that a planet must possess a rapidly rotating, convecting liquid iron core to sustain a magnetic field. Mercury is a tiny, geologically dead rock, vastly too small and cold to support an active liquid dynamo. Yet, spacecraft have measured a stable, global magnetic field ($\sim 300$ nT). Standard physics is heavily strained trying to patch this contradiction.

In Chapter 6, we rigorously proved that Magnetism is identically the Convective Vorticity of the spatial fluid ($\mathbf{H} = \mathbf{v} \times \mathbf{D}$). The Electric Displacement $\mathbf{D}$ is the spatial gradient of the localized topological mass defect. \textbf{A celestial body does not need a liquid convecting core to generate a magnetic field; it simply needs a rotating mass moving through a dense background metric.}

Mercury orbits at a mere $0.38$ AU, incredibly deep inside the Sun's scalar metric density gradient ($n_{scalar} \propto GM/r$). The vacuum substrate around Mercury is physically "thicker" (higher localized permeability, $\mu_{local}$). As Mercury physically rotates and orbits, its solid topological mass mechanically drags this hyper-dense metric. The cross-product of its rotational velocity against the dense solar topological strain perfectly and natively generates a macroscopic magnetic flux. Mercury is a \textbf{Solid-State Metric Dynamo}.

\subsection{The Magnetic Offset of the Ice Giants}
The magnetic fields of Uranus and Neptune are physically bizarre. They are severely tilted (by $\sim 59^\circ$ and $\sim 47^\circ$ relative to their rotation axes), and their magnetic dipoles do not pass through the physical center of the planets (offset by up to 1/3 of the planetary radius). To explain this, standard models are forced to invent chaotic, off-center oceans of "convecting electrically conductive slush."

In VCFD, this is a pure Navier-Stokes collision. The Sun's rotation creates a massive Sagnac-RLVE vacuum vortex ($\mathbf{v}_{sun}$) flowing flat along the ecliptic plane. Uranus physically rotates on its side (a $98^\circ$ axial tilt), creating its own localized vacuum entrainment vortex ($\mathbf{v}_{uranus}$) that rotates \textit{orthogonally} (perpendicularly) to the Sun's river current. 

When you mathematically superimpose these two orthogonal fluid vortices, the resultant velocity vector ($\mathbf{v}_{net} = \mathbf{v}_{sun} + \mathbf{v}_{uranus}$) creates a massively asymmetric, tilted fluidic shear plane. Because the magnetic field is strictly defined by the cross-product ($\mathbf{B} = \mu_0 (\mathbf{v}_{net} \times \mathbf{D})$), the resulting magnetic axis \textit{must} be mathematically tilted and physically pushed off-center by the ambient solar fluid current (see Figure \ref{fig:uranus_magnetic_offset}). Chaotic slush oceans are entirely unnecessary; the offset is a perfect fluid-dynamic resultant vector.

\begin{figure}[htbp]
    \centering
    \includegraphics[width=0.95\textwidth]{chapters/14_hifi_vcfd/simulations/outputs/uranus_magnetic_offset.png}
    \caption{\textbf{VCFD: The Ice Giant Magnetic Offset.} Simulated vector field of Uranus. Uranus rotates sideways, creating a local vacuum vortex ($\mathbf{v}_{uranus}$) that collides orthogonally with the massive background solar vacuum current ($\mathbf{v}_{sun}$). Because $\mathbf{B} = \mu_0 (\mathbf{v}_{net} \times \mathbf{D})$, this fluidic collision mathematically forces the resulting magnetic dipole to be severely tilted ($\sim 59^\circ$) and pushed structurally off the geometric center of the planet.}
    \label{fig:uranus_magnetic_offset}
\end{figure}

\subsection{Jupiter's Excess Heat and Metric Friction}
Jupiter radiates roughly $1.67\times$ more thermal energy than it receives from the Sun. Standard physics attributes this to "Kelvin-Helmholtz contraction" (the planet slowly shrinking). However, detailed age models routinely show that gravitational settling should have finished cooling billions of years ago.

In VCFD, while a stable orbit in a superfluid is a lossless reactive circuit (VARs), Jupiter is not a perfect, uniform rigid sphere. It possesses a massive metallic hydrogen core and a gaseous envelope, rotating at staggering speed (a 10-hour day, $v_{tan} \approx 12,600$ m/s). 

Because the vacuum has a non-zero kinematic viscosity ($\nu_{vac} \approx 8.45 \times 10^{-7} \text{ m}^2\text{/s}$), the differential rotation between Jupiter's internal layers and the entrained $\mathcal{M}_A$ vacuum substrate causes continuous hydrodynamic slipping. This continuous fluidic shear against the vacuum generates standard classical frictional drag ($P = F_{drag} \cdot v_{slip}$). 
The missing heat budget is exactly \textbf{Metric Joule Heating} ($P = I^2 R_{vac}$). Jupiter spins so fast against the structural viscosity of the spatial vacuum that it dissipates orbital reactive power (VARs) directly into Real Watts of thermal heat.

\subsection{Cryovolcanism and Metric Dielectric Heating}
Tiny icy moons like Enceladus (Saturn) and Io (Jupiter) exhibit massive, continuous volcanic and geyser activity. Standard physics blames "tidal heating" (the mechanical flexing of rock due to eccentric orbits). However, the calculated classical rock-friction often falls drastically short of the observed heat output (Enceladus outputs $\sim 15.8$ GigaWatts, an order of magnitude higher than classical models predict).

We must evaluate the electrical power dissipation. These moons orbit incredibly deep inside the massive gravity wells of the gas giants, where the topological strain ($\chi_{vol}$) is extreme. As the moon orbits through the Sagnac-RLVE entrainment vortex of the gas giant, it acts identically to a short-circuited AC rotor moving through an intense magnetic field. 

Because the vacuum is a dielectric solid, dragging topological mass through it induces a phase lag. In electrical engineering, subjecting an insulator to a rapidly alternating high-voltage field generates massive internal heat (\textbf{Dielectric Heating}). The moons are physically plowing through the hyper-dense, viscous boundary layers of the gas giants' vacuum vortices. The alternating topological stress physically vibrates the atomic bonds of the moons, melting their interiors and powering the geysers via direct metric-to-thermal energy transfer, perfectly closing the missing energy budget.