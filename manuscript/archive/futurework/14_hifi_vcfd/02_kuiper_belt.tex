\section{The Kuiper Belt and Oort Cloud: The Freezing Point of Space}

In standard astrophysics, there is a glaring topological contradiction at the edge of our solar system. The planets, the asteroid belt, and the \textbf{Kuiper Belt} (from 30 to 50 AU) form a nearly perfect, flat, 2D planar disk. However, the \textbf{Oort Cloud} (from roughly 2,000 out to 100,000 AU) completely abandons this flat disk geometry, forming a massive, chaotic, highly diffuse 3D sphere of cometary bodies extending outward into interstellar space.

Standard models attribute this transition to random galactic tidal perturbations. In the AVE framework, this transition is not random; it drops out algebraically as an exact, mathematically absolute fluid-dynamic boundary layer.

\subsection{Zone 1: The Superfluid Vortex (The Kuiper Belt)}
The Kuiper Belt resides incredibly deep inside the Sun's $7,442$ AU Bingham Yield boundary. Here, the immense gravitational shear of the Sun physically liquefies the spatial metric, dropping its structural viscosity to near-zero ($\eta_{eff} \to 0$). 

As the Sun rotates, it fluidically entrains this liquid vacuum (The Sagnac-RLVE effect), creating a massive, swirling whirlpool in the ecliptic plane. Any matter in this region (such as the icy bodies of the Kuiper Belt) is physically dragged by the spatial current. The continuous lateral hydrodynamic pressure slowly but inevitably grinds down their vertical orbital inclinations, forcing the celestial bodies to settle perfectly into a flat, 2D spinning disk.

\subsection{Zone 2: The Viscous Boundary (The Oort Cloud)}
The Oort cloud begins around $2,000$ AU and extends out to $100,000$ AU. It directly intersects and completely engulfs the \textbf{7,442 AU Vacuum Freezing Line}. 

As we cross this absolute hardware threshold ($g < a_{genesis}$), the gravitational shear drops below the Bingham yield limit. The vacuum geometrically snaps back into a highly viscous, rigid Cosserat solid (the Dark Matter regime). 

In classical mechanics, you cannot stir a solid. The Sun's rotational vacuum vortex violently crashes and dies against this rigid metric boundary. Without the swirling fluidic river to pull them into a 2D plane, the distant icy comets of the Oort Cloud are subjected to isotropic drag and random multi-body scattering against the rigidified $\mathcal{M}_A$ lattice. They become permanently locked into a chaotic, 3D spherical halo (see Figure \ref{fig:kuiper_oort_boundary}).

\subsection{Resolving the "Planet Nine" Anomaly (Aerodynamic Herding)}
In recent years, astronomers have observed that Extreme Trans-Neptunian Objects (ETNOs, such as Sedna) possess highly elliptical orbits that are bizarrely "clustered" or herded together in one specific directional quadrant. Because standard astrophysics assumes the vacuum is a featureless void, theorists have mathematically invented an invisible, 10-Earth-mass planet (\textbf{"Planet Nine"}) to herd them via gravity.

The AVE framework completely eliminates the need to invent invisible planets. We only require fundamental Fluid Dynamics. 

These extreme ETNOs possess highly elliptical orbits with aphelions extending from $1,000$ to $2,000$ AU. They are rapidly approaching the freezing boundary of the Bingham Bubble. In continuum fluid mechanics, when a rapidly flowing fluid approaches a highly viscous, slow-moving boundary, it creates a \textbf{Hydraulic Jump} (a fluid-dynamic pressure shockwave). 

The ETNOs are not being pulled by an invisible planet; they are experiencing \textbf{Aerodynamic Herding}. As their orbits stretch outward toward the freezing vacuum limit, they strike the accumulating fluidic pressure wave of the stalling vacuum vortex. They are physically shepherded into aligned, clustered orbits by the macroscopic fluid boundary-layer of the solar system.

\begin{figure}[htbp]
    \centering
    \includegraphics[width=0.95\textwidth]{chapters/14_hifi_vcfd/simulations/outputs/kuiper_oort_boundary.png}
    \caption{\textbf{Topological Architecture of the Outer Solar System.} Rendered via 3D VCFD. \textbf{The Kuiper Belt (Cyan):} Deep inside the Bingham yield bubble, the vacuum is a superfluid. The Sun's rotation easily stirs this metric into a whirlpool, dragging the planets and the Kuiper Belt into a perfectly flat 2D ecliptic disk. \textbf{The Planet 9 Anomaly (Yellow):} As highly elliptical ETNOs approach the stalling edge of the fluid vortex, they experience aerodynamic herding via a fluidic hydraulic jump, clustering their orbits without an invisible planet. \textbf{The Oort Cloud (Red):} Crossing the 7,442 AU threshold, the vacuum freezes into a viscous solid. The vortex stops, robbing objects of planar momentum and scattering them into a chaotic 3D sphere.}
    \label{fig:kuiper_oort_boundary}
\end{figure}