% 10_generative_cosmology.tex
\chapter{Generative Cosmology and Thermodynamic Attractors}
\label{ch:cosmology}

\section{Lattice Genesis: The Origin of Metric Expansion}
Standard cosmology often models metric expansion as the continuous stretching of an unstructured coordinate geometry. The AVE framework restricts the macroscopic stretching of this fundamental limit. Because a discrete LC network cannot stretch macroscopically without altering its fundamental capacitance ($\epsilon_0$), metric expansion is modeled strictly as the discrete, real-time \textbf{crystallization} of new electromagnetic nodes.

To preserve the invariant optical density of the condensate globally ($\partial_t \rho_n = 0$), the Eulerian continuity equation dictates the discrete generative source term must identically match the macroscopic volumetric expansion divergence. We hypothesize that the Hubble Constant ($H_0$) is not a velocity, but the \textbf{LC Crystallization Rate} required to maintain the vacuum's structural impedance against the compressive polarization of gravity.

As derived in Chapter 4, evaluating the Machian boundary impedance against the quantum mass-gap establishes an absolute geometric relationship for the asymptotic expansion limit:
\begin{equation}
    H_{\infty} = \frac{28\pi m_e^3 c G}{\hbar^2 (p_c/8\pi)^2} = \frac{1792\pi^3 m_e^3 c G}{\hbar^2 p_c^2}
\end{equation}

\subsection{Verification: Resolving the Hubble Tension}
Substituting the fundamental constants ($m_e, c, \hbar, G$) and the derived geometric packing fraction ($p_c \approx 0.1834$) into this geometric bound evaluates to:
\begin{equation}
    H_{\infty} \approx 69.32 \text{ km/s/Mpc}
\end{equation}
This baseline relationship lies precisely between the Early Universe measurements (Planck 2018: $67.4 \pm 0.5$) and Late Universe measurements (SHOES: $73.0 \pm 1.4$). This suggests that the "Hubble Tension" is an artifact of measuring effective expansion across different thermodynamic regimes, while the underlying hardware generation rate asymptotes to this exact geometric bound.

\section{Dark Energy: The Stable Phantom Derivation}

During lattice genesis, the phase transition continuously expels a latent heat of fusion
($\rho_{latent}dV$) into the ambient photon gas (CMB). By the first law of thermodynamics, to physically fund the internal energy of the newly created spatial volume ($\rho_{vac}$) while simultaneously
expelling this latent heat, the total macroscopic mechanical pressure ($P_{tot}$) of the vacuum
must be strictly negative.

Calculating the Equation of State ($w=P/\rho$) for this generative process yields:

\begin{equation}
w_{vac}=-1-\frac{\rho_{latent}}{\rho_{vac}}
\end{equation}

Because the thermodynamic latent heat of structural fusion is strictly positive ($\rho_{latent} > 0$), this fundamental generative ratio algebraically guarantees a stable \textbf{Phantom Energy} state ($w < -1$). 

The AVE framework identifies ``Dark Energy'' not as a mysterious scalar field, but strictly as the
thermodynamic latent heat of the vacuum's continuous macroscopic crystallization. It natively drives cosmic acceleration without requiring heuristic parameter tuning, and structurally forbids a Big Rip singularity.

\section{The CMB as an Asymptotic Thermal Attractor}

The continuous injection of latent heat into the photon gas (Cosmic Microwave Background) dynamically forms a permanent asymptotic thermal floor. By modeling the universe as a standard radiation network ($P = \frac{1}{3}\rho$) with a continuous volumetric generative source term driven by the latent heat of lattice crystallization ($\Gamma = \frac{1}{V}\frac{dQ}{dt} = 3H\rho_{latent}$), the cosmological continuity equation rigorously evaluates to:

\begin{equation}
\dot{\rho}_{rad} + 4H\rho_{rad} = 3H\rho_{latent}
\end{equation}

Converting this differential equation to evaluate against the cosmological scale factor ($a$), the system natively integrates against standard adiabatic expansion cooling ($a^{-4}$) to strictly yield:

\begin{equation}
u_{rad}(a) = U_{hot}a^{-4} + \frac{3}{4}\rho_{latent}
\end{equation}

As $a \to \infty$, the standard adiabatic expansion cooling ($a^{-4}$) is perfectly offset by the continuous latent heat injection. The temperature smoothly asymptotes to the fundamental Unruh-Hawking temperature limit ($T_U \sim 10^{-30}\text{ K}$), structurally resolving the thermodynamic Heat Death paradox.

\section{Early Galaxy Accretion (The JWST Paradox)}
The James Webb Space Telescope (JWST) recently discovered massive, fully mature galaxies existing a mere 300 to 500 million years after the Big Bang ($z > 10$). Under the standard $\Lambda$CDM model, this is mathematically impossible. Gravity alone is far too weak; cosmological models strictly dictate that primordial gas requires billions of years to slowly clump into invisible Dark Matter halos via slow, collisionless hierarchical merging ($M \propto t^{2.5}$).

\textbf{AVE Resolution:} How does matter accrete in the AVE framework? As proven in Book 1, the deep cosmos operates in the "Low-Voltage" regime of the dielectric vacuum, where the network acts as a highly reluctant Chiral LC grid (The Dark Matter mutual inductance effect). 

In the ultra-dense early universe, the spatial metric possessed extreme inductive inertia. Instead of relying solely on the weak $1/r^2$ gravitational attraction, the macroscopic structural mutual inductance of the $\mathcal{M}_A$ network acted as a \textbf{Cosmic Sweep} (Mutual Inductive accretion). Because the accretion rate is proportional to the mass already collected ($\frac{dM}{dt} = \lambda M$), the mutual inductive drag yields a strict \textbf{Exponential Growth Law}:
\begin{equation}
    M(t) = M_{seed} \cdot e^{t / \tau_{ind}}
\end{equation}

If we evaluate the JWST empirical data (requiring a cluster to grow to $10^{10} M_\odot$ by $t=350$ Myr, and $10^{11} M_\odot$ by $t=500$ Myr), we can exactingly calculate the required exponential mutual inductance time constant ($\tau_{ind}$) of the primordial vacuum:
\begin{equation}
    \frac{10^{11}}{10^{10}} = \frac{e^{500 / \tau_{ind}}}{e^{350 / \tau_{ind}}} \implies 10 = e^{150 / \tau_{ind}}
\end{equation}
\begin{tcolorbox}[colback=white, colframe=black]
\begin{equation}
    \tau_{ind} = \frac{150}{\ln(10)} \approx \mathbf{65.1 \text{ Million Years}}
\end{equation}
\end{tcolorbox}

JWST does not break cosmology; it breaks the "zero-impedance void" assumption. The massive mutual inductance of the $\mathcal{M}_A$ network collapses primordial gas into galaxies exponentially faster than collisionless $\Lambda$CDM models permit. By establishing a rigid $\tau \approx 65.1$ Myr inductive herding limit, the AVE framework seamlessly predicts the formation of super-massive galaxies in millions, not billions, of years.

\section{Black Holes and The Absolute Impedance Mismatch ($\Gamma = -1$)}

No physical substrate compresses infinitely to a geometric singularity. As confined electromagnetic wave packets (matter) aggregate into a hyper-dense core, the macroscopic refractive index ($n_{\perp}=1+2GM/rc^{2}$) increases.

At the exact mathematical radius of the event horizon, the continuous tensor strain on
the discrete edges reaches the strictly squared (2nd-order) Axiom 4 dielectric saturation limit. At this
threshold, the spatial structure physically ruptures. The discrete nodes undergo a sudden
thermodynamic phase transition, melting back into an unstructured, pre-geometric continuous
plasma. The concept of the geometric singularity is replaced by a flat thermodynamic floor.

Because topological particles (knots) fundamentally require the discrete lattice edges to
maintain their invariants, crossing the event horizon destroys the structural canvas supporting
them. The knots mechanically unravel. The mass-energy is conserved strictly as latent heat,
but the geometric quantum information is physically, mathematically, and permanently erased.

The AVE framework explicitly sides with Hawking's original assessment: the thermody-
namic phase transition of the substrate dictates that quantum unitarity is macroscopically
violated at the event horizon, strictly enforcing information loss.