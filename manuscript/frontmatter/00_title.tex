\begin{titlepage}
    \centering
    \vspace*{1.5cm}
    
 
    
    \vspace{2.5cm}
    
    % --- MAIN TITLE ---
    {\huge\bfseries APPLIED VACUUM \par}
    \vspace{0.2cm}
    {\huge\bfseries ELECTRODYNAMICS \par}
    
    \vspace{1cm}
    
    % --- SUBTITLE ---
    {\Large\itshape The Mechanical Substrate of Physics \par}
    
    \vspace{3cm}
    
    % --- AUTHOR ---
    {\Large\bfseries Grant Lindblom \par}
    
    \vspace{3cm}
    
    % --- ABSTRACT / BLURB ---
    \begin{minipage}{0.85\textwidth}
        \centering
        \small
        \textbf{Abstract} \\
        Modern theoretical physics relies on continuous geometric and probabilistic abstractions to model the universe. This text proposes a constitutive mechanical alternative: \textbf{Applied Vacuum Electrodynamics (AVE)}. By defining the vacuum not as a passive void, but as a Discrete Amorphous Manifold ($M_A$) with finite inductive ($\mu_0$) and capacitive ($\epsilon_0$) limits, we derive the fundamental laws of nature as emergent hardware specifications. From this discrete substrate, we recover General Relativity as tensor metric refraction, Quantum Mechanics as bandwidth-limited signal processing, and the Standard Model particle zoo as topological knots (Matter) and spatial flux partitions (Quarks). Furthermore, the framework mechanically unifies the dark sector: resolving Dark Energy as the latent heat of ongoing lattice crystallization, and Dark Matter as the viscous fluid dynamics of the Hubble wake. By replacing singularities with closed thermodynamic phase transitions, this text serves as a formal engineering manual for the hardware of reality.
    \end{minipage}
    
    \vfill
    
    % --- FOOTER ---
   
    {\large \textit{2026} \par}
    {\scshape Vacuum Engineering Press \par}

\end{titlepage}