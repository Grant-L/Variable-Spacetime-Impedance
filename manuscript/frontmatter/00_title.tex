\title{Applied Vacuum Engineering \\ \large Understanding the Mechanics of Vacuum Rheology}
\author{Grant Lindblom}
\date{}

\maketitle

\vspace{2em}
\noindent\textbf{Applied Vacuum Engineering: Understanding the Mechanics of Vacuum Rheology} \\
This document is a technical specification. All constants derived herein are subject to the hardware limitations of the local vacuum manifold.

\begin{abstract}
Modern physics models the universe as a passive stage governed by abstract laws. Discrete Cosserat Vacuum Electrodynamics (DCVE) redefines spacetime as an active physical machine: a Discrete Amorphous Manifold ($\mathcal{M}_A$) governed by rigorous continuum mechanics and topological field theory. 

By postulating two fundamental hardware limits---the Lattice Pitch ($l_0$) and the Schwinger Yield Energy Density ($u_{sat}$)---we derive the "constants" of nature not as fixed scalars, but as emergent operating limits of a Cosserat substrate. From these axioms, we derive:
\begin{itemize}
    \item \textbf{Quantum Mechanics:} The Generalized Uncertainty Principle (GUP) as the exact finite-difference momentum bound of the discrete Brillouin zone.
    \item \textbf{Gravity:} The continuum limit of the Cosserat solid natively reproduces the transverse-traceless nature of the Einstein Field Equations, mathematically resolving the negative-bulk-modulus paradoxes of classical Cauchy aethers.
    \item \textbf{Topological Matter:} Particle masses scale strictly according to the mathematically rigorous Vakulenko-Kapitanski energy bounds for Faddeev-Skyrme $O(3)$ topological solitons. Fractional charge arises natively via the Witten Effect acting on the $\mathbb{Z}_3$ symmetry of the Borromean linkage.
    \item \textbf{The Dark Sector:} The flat galactic rotation curve ($v \propto M^{1/4}$) is derived rigorously via the Bekenstein-Milgrom AQUAL formulation as the asymptotic boundary layer solution to a shear-thinning vacuum fluid.
\end{itemize}
This framework is strictly falsifiable via the Rotational Lattice Viscosity Experiment (RLVE), predicting a density-dependent phase shift ($\Psi > 5$) that contradicts General Relativity.
\end{abstract}
\newpage