\section{Protocol 9: The Achromatic Impedance Lens}
\label{sec:achromatic_lens}

General Relativity models gravity strictly as a geometric curvature of spacetime. In contrast, the Applied Vacuum Electrodynamics (AVE) framework identifies gravity as a macroscopic phase strain ($h_{\perp}$) acting directly upon the discrete $6^3_2$ Borromean lattice.

A critical mathematical discriminator between these two models lies in the fundamental nature of the refractive boundary. In standard optics, changing the refractive index ($n$) of a medium by altering its permittivity ($\epsilon_r$) naturally creates an impedance mismatch ($Z \neq Z_0$), resulting in Fresnel reflection scattering at the boundary interface.

However, the AVE framework mathematically requires that the vacuum phase strain scales both the inductive permeability ($\mu$) and capacitive permittivity ($\epsilon$) perfectly proportionally:
\begin{equation}
    Z_{\text{gravity}} = \sqrt{\frac{\mu(r)}{\epsilon(r)}} \equiv \sqrt{\frac{\mu_0 \cdot n(r)}{\epsilon_0 \cdot n(r)}} = Z_0
\end{equation}

Thus, gravity acts as an \textit{Achromatic Impedance Lens}. It diffracts light perfectly without any boundary reflection loss. 

\subsection{Experimental Hardware Falsification}
This opens a definitive vector for solid-state hardware falsification. If researchers fabricate a metamaterial dielectric lens where both $\mu_r$ and $\epsilon_r$ are spatially doped to scale at an identically proportional radial gradient, the resulting macroscopic structure will perfectly mimic a gravitational well.

Under testing with high-frequency RF or optical lasers, this Achromatic Impedance Lens should exhibit a physical reflection coefficient of identically zero ($0.0\%$) across all angles of incidence, fundamentally bypassing the Fresnel limits of standard optics and proving the unified topographic mechanism in the laboratory.
