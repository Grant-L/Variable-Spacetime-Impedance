\appendix
\chapter{Mathematical Proofs and Formalism}
\label{app:proofs}

\section{The Discrete-to-Continuum Limit (Kirchhoff)}
We rigorously show that as the Lattice Pitch $\lp \to 0$, the discrete difference equations of the mesh converge to the continuous differential equations of Maxwell.

\begin{theorem}
    The Kirchhoff Current Law (KCL) for a node $n$ in the limit of $N \to \infty$ recovers the Continuity Equation:
    \begin{equation}
        \sum_{i} I_{n,i} = 0 \implies \nabla \cdot \mathbf{J} + \frac{\partial \rho}{\partial t} = 0
    \end{equation}
\end{theorem}

\section{The Madelung Internal Pressure (Q)}
The "Quantum Potential" $Q$ found in the Bohmian formulation is identified here as the **Internal Stress** of the lattice fluid.
\begin{equation}
    Q = -\frac{\hbar^2}{2m} \frac{\nabla^2 \sqrt{\rho}}{\sqrt{\rho}} \equiv \text{Lattice Tension}
\end{equation}

\chapter{Simulation Manifest and Codebase}
\label{app:code}

The following Python modules constitute the core of the Vacuum Engineering simulation suite (VSS).

\section{Core Code: Metric Lensing}
\begin{lstlisting}[language=Python, caption=Calculating Refractive Index from Mass]
def calculate_refractive_index(r, M):
    """
    Returns the vacuum refractive index n(r) based on
    Lattice Stress saturation near a mass M.
    """
    G = 6.674e-11
    c = 2.998e8
    
    # Gravitational Potential
    phi = -G * M / r
    
    # Refractive Index (Stress Equation 5.1)
    n = 1 - (2 * phi / c**2)
    return n
\end{lstlisting}

\chapter{The Rosetta Stone}
\label{app:rosetta}

\section{Mapping Table}
This table translates the abstract terminology of the Standard Model into the hardware specifications of Applied Vacuum Engineering.

\begin{table}[h]
    \centering
    \begin{tabularx}{\textwidth}{@{}l|X@{}}
        \toprule
        \textbf{Standard Physics Term} & \textbf{Vacuum Engineering Hardware Spec} \\
        \midrule
        Curvature of Spacetime & Refractive Gradient of Lattice Density ($\nabla n$) \\
        Speed of Light ($c$) & Global Slew Rate ($1/\sqrt{\mu_0 \epsilon_0}$) \\
        Mass ($m$) & Stored Inductive Energy of a Knot ($E_L$) \\
        Electric Charge ($q$) & Topological Winding Number ($N$) \\
        Gravitational Lensing & Dielectric Refraction (Snell's Law) \\
        Heisenberg Uncertainty & Nyquist Sampling Limit ($\Delta x < \lp$) \\
        The Big Bang & Lattice Crystallization Phase Transition \\
        Dark Matter & Viscosity of the Vacuum ($\eta_{vac}$) \\
        \bottomrule
    \end{tabularx}
    \caption{The Dictionary of Reality}
\end{table}