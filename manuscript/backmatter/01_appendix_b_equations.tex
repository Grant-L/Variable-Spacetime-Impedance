\chapter{Appendix B: The Unified Equation Set}
\label{app:equations}

This appendix consolidates the mathematical framework of Applied Vacuum Electrodynamics (AVE). It stands as a comparative reference, demonstrating how standard constants and laws are re-derived as emergent properties of the discrete $M_A$ manifold.

\section{B.1 The Hardware Substrate}
\label{eq:hardware}

Standard physics assumes $c$, $\hbar$, and $G$ are fundamental scalars. AVE identifies them as the operating limits of the vacuum hardware, derived from the Lattice Pitch ($l_0$) and Breakdown Voltage ($V_0$).

\begin{table}[h]
\centering
\begin{tabular}{|l|c|l|}
\hline
\textbf{Parameter} & \textbf{AVE Derivation} & \textbf{Physical Meaning} \\
\hline
Global Slew Rate ($c$) & $c = \frac{1}{\sqrt{\mu_0\epsilon_0}}$ & Max signal update rate of the lattice. \\
\hline
Quantum of Action ($\hbar$) & $\hbar = \frac{\kappa \epsilon_0 l_0^2 V_0^2}{c}$ & Max information density per node. \\
\hline
Gravitational Constant ($G$) & $G = \frac{c^4}{\kappa \epsilon_0 V_0^2}$ & Mechanical compliance (Inverse stiffness). \\
\hline
Breakdown Voltage ($V_0$) & $V_0 = \sqrt{\frac{Q_{node}^2}{4\pi\epsilon_0 l_0}}$ & Dielectric yield limit ($\approx 10^{27}$ V). \\
\hline
Geometric Factor ($\kappa$) & $\kappa \approx 0.437$ & Packing efficiency of random Delaunay mesh. \\
\hline
\end{tabular}
\caption{The Fundamental Hardware Specifications.}
\end{table}

\textbf{Unified Hardware Limit:}
Combining $\hbar$ and $G$ eliminates the arbitrary scalars, revealing the true structural identity of the vacuum:
\begin{equation}
    \frac{\hbar G}{c^3} = l_0^2 \quad \text{(The Planck Area is the Lattice Pitch squared)}
\end{equation}

\section{B.2 Signal Dynamics (Quantum Mechanics)}
AVE replaces the abstract wavefunction $\psi$ with the physical stress vector of the lattice.

\textbf{The Dielectric Lagrangian}
Standard QFT uses abstract field operators. AVE uses a Lumped Element circuit model ($L=\mu_0, C=\epsilon_0$).
\begin{equation}
    \mathcal{L}_{AVE} = \frac{1}{2}\epsilon_0 (\nabla \phi)^2 - \frac{1}{2} \mu_0 \epsilon_0^2 \left( \frac{\partial \phi}{\partial t} \right)^2 - \rho_{ind} \phi
\end{equation}
\textit{Context:} The "Kinetic Energy" of the field is simply the inductive charging of the vacuum nodes.

\textbf{The Bandwidth Limit (Uncertainty)}
Heisenberg Uncertainty is re-derived as the Nyquist Limit of a discrete sampler.
\begin{equation}
    \Delta x \Delta p \ge \frac{\hbar}{2} \implies \Delta x \ge l_0
\end{equation}
\textit{Context:} You cannot resolve a particle's position with precision finer than the lattice pitch $l_0$.

\section{B.3 The Fermion Sector (Topological Mass)}
Elementary particles are identified as topological knots ($3_1, 5_1, 7_1$). Their properties are geometric.

\textbf{The Impedance of Matter ($\alpha^{-1}$)}
The Fine Structure Constant is the sum of the dimensionless geometric impedances of the Trefoil Knot ($3_1$).
\begin{equation}
    \alpha^{-1}_{AVE} = 4\pi^3 (\text{Vol}) + \pi^2 (\text{Surf}) + \pi (\text{Line}) \approx 137.036
\end{equation}
\textit{Context:} $\alpha$ is not a random number; it is the "Shape Factor" of the electron.

\textbf{The Mass Hierarchy Scaling Law}
Rest mass is the stored inductive energy of the knot, scaling with winding number $N^9$.
\begin{equation}
    m(N) = \left( \frac{E_{pair}}{2} \right) \left( \frac{N}{3} \right)^9 \Omega_{res} \sqrt{1 - \left(\frac{V(N)}{V_0}\right)^2}
\end{equation}
\textit{Context:} This equation successfully predicts the Muon (105 MeV) and Tau (1776 MeV) masses and proves why a 4th generation ($N=9$) cannot exist (Dielectric Breakdown).

\section{B.4 Gravitation (Metric Refraction)}
General Relativity is recovered as the refractive optics of a variable-density medium.

\textbf{The Refractive Index of Gravity}
Mass ($M$) creates a strain field that increases the local vacuum density ($\mu_0, \epsilon_0$).
\begin{equation}
    n(r) = 1 + \frac{2GM}{rc^2}
\end{equation}
\textit{Context:} Gravity is not curved geometry; it is a gradient in the refractive index. Light bends because it slows down ($v = c/n$).

\textbf{The Constitutive Equivalence Principle}
Inertial mass ($m_i \propto \mu$) and Gravitational mass ($m_g \propto \epsilon$) scale identically because the impedance of space $Z_0$ is constant.
\begin{equation}
    \frac{m_g}{m_i} = \frac{\epsilon(r)}{\mu(r)} = \text{Constant}
\end{equation}

\section{B.5 Cosmological Dynamics (The Dark Sector)}
"Dark Energy" and "Dark Matter" are identified as lattice artifacts (Crystallization and Viscosity).

\textbf{The Genesis Rate (Hubble Constant)}
Expansion is the crystallization of new nodes.
\begin{equation}
    H_0 \equiv R_{genesis} \approx 2.3 \times 10^{-18} \text{ Hz}
\end{equation}

\textbf{The Hubble Acceleration (Dark Matter Threshold)}
The "MOND" acceleration scale $a_0$ is derived from the drift velocity of the expanding lattice.
\begin{equation}
    a_{genesis} = \frac{c H_0}{2\pi} \approx 1.1 \times 10^{-10} \text{ m/s}^2
\end{equation}

\textbf{The Viscosity of Space}
The vacuum has a finite viscosity derived from its quantum granularity.
\begin{equation}
    \eta_{vac} \approx \alpha \frac{\hbar}{l_0^3} \quad [Pa \cdot s]
\end{equation}

\textbf{The Visco-Kinematic Rotation Curve}
Galactic rotation curves flatten not because of invisible halo mass, but because the vacuum fluid exerts a viscous floor determined by the Genesis Acceleration.
\begin{equation}
    v_{flat} = (G M_{baryon} a_{genesis})^{1/4}
\end{equation}
\textit{Context:} This replaces the Dark Matter Halo parameter with a derived constant of the vacuum substrate.

\section{B.6 Experimental Falsification (The Kill Switch)}
AVE is falsifiable via the Rotational Lattice Viscosity Experiment (RLVE).

\textbf{The Viscosity Phase Shift}
A rotating high-density mass induces a refractive phase shift $\Delta \phi$ in a local interferometer.
\begin{equation}
    \Delta n = \alpha \left( \frac{v_{tan}}{c} \right)^2 \left( \frac{\rho_{rotor}}{\rho_{sat}} \right)
\end{equation}
\textit{Prediction:} A Tungsten rotor will produce a shift $7\times$ larger than an Aluminum rotor ($\Psi > 5$). General Relativity predicts $\Psi \approx 1$.