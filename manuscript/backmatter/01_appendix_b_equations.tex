\chapter{Appendix B: The Unified Equation Set}
\label{app:unified_equations}

This appendix consolidates the rigorous mathematical framework of Discrete Cosserat Vacuum Electrodynamics (DCVE). It demonstrates how standard constants and laws are re-derived as emergent properties of the discrete $\mathcal{M}_A$ manifold, strictly preserving SI dimensional homogeneity and classical continuum mechanics.

\section{B.1 The Hardware Substrate}
Standard physics assumes $c, \hbar$, and $G$ are fundamental scalars. DCVE identifies them as the emergent operating limits of the vacuum hardware, derived entirely from the Lattice Pitch ($l_{node}$) and the Schwinger Yield Energy Density ($u_{sat}$).

\begin{table}[h]
\centering
\begin{tabular}{|l|c|l|}
\hline
\textbf{Parameter} & \textbf{DCVE Derivation} & \textbf{Physical Meaning} \\ \hline
Global Slew Rate ($c$) & $c = 1 / \sqrt{\mu_0\epsilon_0}$ & Max transverse signal update rate. \\ \hline
Yield Energy ($E_{sat}$) & $E_{sat} = u_{sat} l_{node}^3$ & Dielectric topological rupture limit. \\ \hline
Cosserat Bulk Modulus ($K$) & $K = \lambda + \frac{2}{3}\mu > 0$ & Ensures absolute thermodynamic stability. \\ \hline
Quantum of Action ($\hbar$) & $\hbar = u_{sat} l_{node}^4 / c$ & Maximum action capacity per node. \\ \hline
Lattice Tension ($T_{vac}$) & $T_{vac} = u_{sat} l_{node}^2 = c^4/4\pi G$ & Linear yield force of the substrate. \\ \hline
\end{tabular}
\caption{The Fundamental Hardware Specifications.}
\end{table}

\section{B.2 Signal Dynamics (Quantum Mechanics)}
\textbf{The Dimensionally Exact Lagrangian:}\\
DCVE uses the Magnetic Vector Potential ($\mathbf{A}$) to ensure exact $[\text{J}/\text{m}^3]$ energy density.
\begin{equation}
    \mathcal{L}_{DCVE} = \frac{1}{2} \epsilon_0 \left| \frac{\partial \mathbf{A}}{\partial t} \right|^2 - \frac{1}{2\mu_0} |\nabla \times \mathbf{A}|^2
\end{equation}

\textbf{The Authentic Generalized Uncertainty Principle:}\\
Derived without Taylor truncation errors from the exact finite-difference lattice shift operator acting within the Brillouin zone limits:
\begin{equation}
    \Delta x \Delta P \ge \frac{\hbar}{2} \left| \left\langle \cos\left(\frac{l_{node} \hat{p}_c}{\hbar}\right) \right\rangle \right|
\end{equation}

\textbf{The Thermodynamic Born Rule:}\\
Probability emerges classically via intensity-coupled thermodynamic thresholding:
\begin{equation}
    P(click | x_n) = \frac{|\mathbf{A}(x_n)|^2}{\int |\mathbf{A}(x)|^2 dx}
\end{equation}

\section{B.3 Topological Matter}
\textbf{The Vakulenko-Kapitanski Mass Bound:}\\
The rest mass of a knotted soliton is bounded by its Hopf winding number ($Q_H$), replacing heuristic integer scaling laws with strict $O(3)$ topological bounds.
\begin{equation}
    M_{rest}(Q_H) \ge C_{vac} \cdot |Q_H|^{3/4}
\end{equation}

\textbf{The Witten Effect (Fractional Charge):}\\
The constrained $\mathbb{Z}_3$ permutation symmetry of the Borromean linkage ($6^3_2$) naturally fractionalizes charge via the discrete $\theta$-vacuum.
\begin{equation}
    q_{eff} = n + \frac{\theta}{2\pi} e \implies \pm \frac{1}{3}e, \pm \frac{2}{3}e
\end{equation}

\section{B.4 Gravitation and The Weak Force}
\textbf{Trace-Reversed Cosserat Gravity:}\\
General Relativity emerges dynamically from the trace-reversed elastic strain tensor ($\bar{h}_{\mu\nu}$) mapped to the coupled twist-shear modes of the stable Cosserat solid.
\begin{equation}
    -\frac{1}{2} \Box \bar{h}_{\mu\nu} = \frac{8\pi G}{c^4} T_{\mu\nu}
\end{equation}

\textbf{The Weak Force (Micropolar Cutoff):}\\
The massive W and Z bosons are natively identified as the rigid acoustic gap frequencies of the lattice's intrinsic microrotational stiffness.
\begin{equation}
    \omega_{cutoff} = \sqrt{\frac{4\alpha_c}{J}} \implies m_{W,Z} = \frac{\hbar}{c^2} \sqrt{\frac{4\alpha_c}{J}}
\end{equation}

\section{B.5 Cosmological Dynamics (The Dark Sector)}
\textbf{The Visco-Kinematic Rotation Curve (MOND):}\\
Galactic rotation flattens natively as the exact boundary layer solution to the Bekenstein-Milgrom AQUAL non-linear fluid equation for a shear-thinning vacuum.
\begin{equation}
    v_{flat} = (G M a_{genesis})^{1/4} \quad \text{where} \quad a_{genesis} = \frac{c \cdot H_0}{2\pi}
\end{equation}