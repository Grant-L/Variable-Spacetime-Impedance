\chapter{Topological Thrust Mechanics (Acoustic Rectification)}

Standard electrodynamics strictly forbids the continuous generation of unidirectional thrust within a sealed vacuum environment (e.g., a contained asymmetric capacitor) as it violates the conservation of momentum. 

However, the Applied Vacuum Engineering (AVE) framework proves the vacuum is a rigid LC dielectric lattice ($\mathcal{M}_A$ metric). By inducing an extreme non-linear asymmetric gradient of electric field energy density ($\nabla |\mathbf{E}|^2$) and driving it dynamically, an engineered emitter can theoretically grip the static continuous background modes of the matrix. This process, formally known as \textbf{Macroscopic Acoustic Rectification}, actively rectifies symmetric vacuum fluctuations into continuous, unidirectional thrust \textit{against} the physical weight of the spatial lattice itself.

In this chapter, we derive the exact coupling transfer coefficient ($k_{topo}$) required to translate purely electromagnetic gradients into physical longitudinal force ($F_{thrust}$).

\begin{figure}[ht]
    \centering
    \includegraphics[width=1.0\textwidth]{ponder_01_thrust_vectoring.png}
    \caption{\textbf{Topological Thrust Vectoring:} The simulated macroscopic force output of a $25 \text{ cm}^2$ asymmetric electrode array driven at $30 \text{ kV}$ RMS. The extreme non-linear $\nabla |\mathbf{E}|^2$ gradient acts as a geometric drag anchor against the continuous string-lattice. To breach the $1 \mu\text{N}$ detection floor of a vacuum torsion balance, the array must be pumped dynamically in the VHF band. Operating at $100 \text{ MHz}$ yields a highly detectable $45 \mu\text{N}$ continuous anomaly.}
    \label{fig:ponder_01_thrust_vectoring}
\end{figure}
