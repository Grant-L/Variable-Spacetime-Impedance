\chapter{30kV VHF Driver Mechanics and LC Filtering}

The topological coupling coefficient to the string lattice increases violently with the $\partial_t \mathbf{D}$ displacement frequency. An asymmetric electrode generating a static DC gradient merely polarizes the vacuum. To generate continuous Ponderomotive thrust via acoustic rectification, the geometry must be pumped dynamically. 

Calculations prove that standard 1-2 MHz flyback topologies are incapable of crossing the $1 \mu\text{N}$ torsion-balance detection threshold. The PONDER-01 test article requires a continuous-wave AC excitation voltage of $30 \text{ kV}$ RMS operating explicitly in the VHF Band ($100 \text{ MHz}$) to achieve macroscopically detectable propulsion ($\sim 45 \mu\text{N}$). This chapter covers the exact SPICE-level avalanche architecture required to drive a capacitive load at these extreme VHF regimes.

\begin{figure}[ht]
    \centering
    \includegraphics[width=1.0\textwidth]{ponder_01_vhf_drive_transient.png}
    \caption{\textbf{100 MHz VHF Avalanche Drive Transient:} When driving the asymmetric geometry at $30 \text{ kV}$ RMS, the sharp $1 \mu\text{m}$ tip undergoes field emission (avalanche breakdown) at the waveform peaks. The electrical driving circuitry must source extreme transient current bursts through the $50 \Omega$ match to prevent the LC voltage from sagging mid-oscillation and crashing the topological drag thrust.}
    \label{fig:ponder_01_vhf_drive_transient}
\end{figure}

\begin{figure}[ht]
    \centering
    \includegraphics[width=1.0\textwidth]{ponder_01_s11_match.png}
    \caption{\textbf{Resonant Load Matching:} The purely reactive $100 \text{ pF}$ PCBA test article requires an ultra-high-$Q$ series inductor of roughly $25 \text{ nH}$ to achieve a pure $50 \Omega$ transmission line match, preventing the $30 \text{ kV}$ VHF source from catastrophically reflecting power back into the amplifier.}
    \label{fig:ponder_01_s11_match}
\end{figure}

\begin{figure}[ht]
    \centering
    \includegraphics[width=1.0\textwidth]{ponder_01_thermal_dissipation.png}
    \caption{\textbf{Thermal Runaway Catastrophe Limits:} Operating in a convective-dead hard vacuum at extreme VHF frequencies forces skin-effect and massive $\tan \delta$ dielectric heating. Standard FR-4 substrate delaminates in milliseconds. Even military-spec Rogers PTFE guarantees only a sub-second firing window before the geometry physically evaporates.}
    \label{fig:ponder_01_thermal_dissipation}
\end{figure}
