\chapter{HOPF-01: Chiral Antenna Verification}
\label{ch:hopf_01}

Before constructing a mechanical thrust measurement, the AVE framework offers a purely electromagnetic falsification test that requires nothing more than a printed circuit board and a \$70 vector network analyzer. If the vacuum possesses intrinsic chirality (Axiom 1), then the resonant frequency of a torus knot antenna must deviate from the standard Maxwell prediction by an amount that scales exactly with the knot's topological winding number.

\section{The Chiral Coupling Prediction}

A microstrip antenna trace of length $L_{trace}$ on a substrate with relative permittivity $\varepsilon_r$ resonates at:
\begin{equation}
    f_{std} = \frac{c}{2\pi L_{trace} \sqrt{\varepsilon_r}}
\end{equation}
This is the prediction of standard Maxwell electrodynamics. Any commercial HFSS or CST simulation will reproduce this result to within manufacturing tolerances.

The AVE framework predicts an additional correction. Because the $(p,q)$ torus knot topology couples to the intrinsic chirality of the $\mathcal{M}_A$ lattice, the effective refractive index acquires a topological term:
\begin{equation}
    n_{AVE} = \sqrt{\varepsilon_r}\left(1 + \alpha \frac{pq}{p+q}\right)
    \label{eq:n_ave}
\end{equation}
where $\alpha \approx 1/137$ is the fine-structure constant and $pq/(p+q)$ is the harmonic mean of the torus knot winding numbers. This is \textbf{not a free parameter}: $\alpha$ is fixed by Axiom 2, and $p,q$ are fixed by the physical trace geometry.

The resulting frequency shift is:
\begin{equation}
    \frac{\Delta f}{f_{std}} = \alpha \frac{pq}{p+q}
\end{equation}

\section{The Four-Knot Test Panel}

To distinguish the chiral coupling from manufacturing tolerances ($\varepsilon_r$ variation, trace width error, etching undercut), all four knot topologies must be fabricated on a \textbf{single FR-4 panel}. This ensures identical substrate properties across all antennas. The predicted shifts are:

\begin{center}
\begin{tabular}{|c|c|c|c|c|c|}
\hline
Torus Knot & $pq/(p{+}q)$ & $L_{trace}$ & $f_{std}$ (MHz) & $\Delta f$ (MHz) & Shift (ppm) \\ \hline
$(2,3)$ Trefoil    & 1.200 & 60 mm  & 383.5 & 3.33 & 8{,}681  \\
$(2,5)$ Cinquefoil & 1.429 & 90 mm  & 255.7 & 2.64 & 10{,}317 \\
$(3,7)$            & 2.100 & 120 mm & 191.7 & 2.89 & 15{,}093 \\
$(3,11)$           & 2.357 & 150 mm & 153.4 & 2.59 & 16{,}910 \\ \hline
\end{tabular}
\end{center}

All shifts exceed 2.5 MHz---easily resolvable with a NanoVNA-H4 (50 kHz -- 1.5 GHz, 1601 points per sweep). Each measurement requires fewer than 201 frequency points.

\section{The Falsification Protocol}

\begin{enumerate}
    \item \textbf{Fabrication:} Order a single 2-layer FR-4 panel from JLCPCB or PCBWay ($\varepsilon_r = 4.3 \pm 0.05$, 1.6 mm thickness, 1 oz Cu, ENIG finish). All four knot traces share the panel. Each trace terminates in a 50$\Omega$ SMA edge-launch connector.

    \item \textbf{Calibration:} Perform SOLT (Short-Open-Load-Thru) calibration of the NanoVNA at the SMA reference plane using a calibration kit matched to the connector type.

    \item \textbf{Measurement:} Sweep each antenna individually. Record the fundamental resonant frequency $f_{res}$ (the deepest S$_{11}$ dip below $-10$ dB). Measure each antenna 10 times, rotating the cable to average connector repeatability.

    \item \textbf{HFSS Comparison:} Import the exact Gerber geometry into Ansys HFSS or openEMS. Simulate S$_{11}$ using the measured $\varepsilon_r$ (from a dedicated test coupon on the same panel). Record the HFSS-predicted $f_{res}$.

    \item \textbf{Extract the Anomaly:} For each knot, compute:
    \begin{equation}
        \Delta f_i = f_{measured,i} - f_{HFSS,i}
    \end{equation}

    \item \textbf{Test the Scaling Law:} Plot $\Delta f$ vs.\ $pq/(p+q)$ for all four knots.
    \begin{itemize}
        \item \textbf{AVE confirmed:} $\Delta f$ is linear through the origin with slope $\alpha \cdot f_0 / \sqrt{\varepsilon_r}$.
        \item \textbf{AVE falsified:} $\Delta f$ is zero, random, or does not scale as $pq/(p+q)$.
    \end{itemize}

    \item \textbf{Substrate Independence:} Repeat on Rogers RO4003C ($\varepsilon_r = 3.38$). The fractional shift $\Delta f / f$ must be identical---it depends only on $\alpha$ and $pq/(p+q)$, not on $\varepsilon_r$.
\end{enumerate}

\section{Bill of Materials}

\begin{center}
\begin{tabular}{|l|c|c|}
\hline
Item & Quantity & Est.\ Cost \\ \hline
FR-4 PCB panel (4 knots + test coupon) & 5 pcs & \$30 \\
SMA edge-launch connectors & 4 & \$12 \\
NanoVNA-H4 & 1 & \$70 \\
SOLT calibration kit & 1 & \$15 \\ \hline
\textbf{Total} & & \textbf{\$127} \\ \hline
\end{tabular}
\end{center}

\section{Decision Gate}

If the HOPF-01 scaling law is confirmed, the AVE framework has produced a genuine, zero-parameter electromagnetic prediction that no existing Maxwell solver can reproduce. This justifies the significantly greater investment required for the mechanical thrust measurements described in the following chapters.

If the scaling law is \textit{not} confirmed, the chiral coupling term in Eq.~\ref{eq:n_ave} is falsified, and the PONDER thrust predictions must be re-examined from first principles.
