\chapter{Hyperboloid Geometric Optimization}

To maximize the Ponderomotive ($\nabla |\mathbf{E}|^2$) thrust drag vector against the LC lattice, the electrode symmetry must be broken as violently as possible without exceeding the $E_{yield}$ dielectric spark threshold ($1.13 \times 10^{17} \text{ V/m}$).

This chapter details the 3D finite-element geometric modeling of the PONDER-01 asymmetric electrode array, specifically transitioning from classical needle-plane geometries to ideal Hyperboloid structures parameterized for extreme $\nabla |\mathbf{E}|^2$ divergence.

By transitioning the rigid parallel plates into a dense matrix of hyperboloids pointing at a flat grounded plane, the effective macroscopic $\nabla |\mathbf{E}|^2$ gradient achieves petawatt equivalent intensity locally at each tip head, massively amplifying the aggregate topological coupling force vector.

\begin{figure}[ht]
    \centering
    \includegraphics[width=1.0\textwidth]{ponder_01_electrostatic_mesh.png}
    \caption{\textbf{Asymmetric PCBA Gradient Overload:} Finite element model overlay of the dense $400 \mu\text{m}$ pitch geometry. Etching the thruster topology down to $1 \mu\text{m}$ sharp cones forces the $30 \text{ kV}$ potential to pinch geometrically, generating extreme localized $\nabla |\mathbf{E}|^2$ vectors across the vacuum gap.}
    \label{fig:ponder_01_electrostatic_mesh}
\end{figure}
