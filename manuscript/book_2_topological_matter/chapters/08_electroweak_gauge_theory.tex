% 08_electroweak_gauge_theory.tex
\chapter{Electroweak Mechanics and Gauge Symmetries}
\label{ch:electroweak}

\section{Electrodynamics: The Gradient of Topological Phase}
A localized charged node permanently exerts a continuous rotational phase twist ($\theta$) on the surrounding LC condensate. Because the unsaturated vacuum acts as a linear dielectric in the far-field, the static structural phase strain must strictly obey the 3D \textbf{Laplace Equation} ($\nabla^2 \theta = 0$).

The spherically symmetric geometric solution dictates that the twist amplitude decays exactly inversely with distance ($\theta(r) \propto 1/r$). The continuous electric displacement field ($\mathbf{D}$) is physically identical to the spatial gradient of this structural phase twist ($\mathbf{D} = \nabla\theta \propto -1/r^2 \mathbf{\hat{r}}$), analytically deriving Coulomb's Law.

\subsection{Magnetism as Convective Vorticity}
When a twisted node translates at a velocity $\mathbf{v}$, it induces a convective shear flow in the momentum field. In classical network dynamics, the time evolution of a translating steady-state strain field $\mathbf{D}(\mathbf{r} - \mathbf{v}t)$ is governed by the convective material derivative:
\begin{equation}
    \partial_t \mathbf{D} = -(\mathbf{v} \cdot \nabla)\mathbf{D} \implies \nabla \times (\mathbf{v} \times \mathbf{D})
\end{equation}
Equating this to the Maxwell-Ampere law derives the macroscopic magnetic field strictly from network dynamics: $\mathbf{H} = \mathbf{v} \times \mathbf{D}$.

This relationship is rigorously supported by dimensional analysis. Applying the topological conversion constant ($\xi_{topo} \equiv e/\ell_{node}$), the displacement field reduces to $[\mathbf{D}] = \xi_{topo}[1/\text{m}]$. Evaluating the cross product $[\mathbf{v} \times \mathbf{D}]$ yields strictly $\xi_{topo}[1/\text{s}]$. Standard SI units for magnetic field intensity $\mathbf{H}$ ($[\text{A/m}]$) identically reduce to this exact same dimensional basis ($\xi_{topo}[1/\text{s}]$). Magnetism is thereby dimensionally proven to represent the continuous kinematic vorticity of the vacuum condensate.

\subsection{The Inductive Origin of Gauge Invariance}
Standard Quantum Field Theory mandates that the vector potential is a gauge field, where transformations of the form $\mathbf{A} \to \mathbf{A} + \nabla \Lambda$ leave physical observables ($\mathbf{B}$ and $\mathbf{E}$) unchanged. A common critique of identifying $\mathbf{A}$ as a physical momentum field is that this gauge freedom would imply the unphysical, spontaneous shifting of macroscopic mass, violating Noether's theorem.

This paradox is resolved rigorously via the \textbf{Helmholtz Decomposition Theorem} in classical network dynamics. Any continuous vector field can be decomposed into a solenoidal (divergence-free) component and an irrotational (curl-free) component. Adding the gradient of a scalar field ($\nabla \Lambda$) to the mass flow strictly introduces a uniform, irrotational velocity potential to the background network.

Because the $\mathcal{M}_A$ vacuum is highly incompressible ($K = 2G$), an irrotational flow field generates no localized compression ($-\partial_t \mathbf{A}$), no transverse vorticity ($\nabla \times \mathbf{A}$), and no topological defects. It is physically isomorphic to performing a \textbf{Galilean or Lorentz coordinate boost} of the observer's reference frame. Gauge invariance is not violated; it is strictly revealed to be the classical network-dynamic freedom to shift the irrotational background coordinate velocity without altering the physical transverse observables.

\section{The Weak Interaction: Inductive Cutoff Dynamics}
In classical electrodynamics, the ratio of the LC network's microrotational bending inductance ($\gamma_c$) to the macroscopic optical shear modulus ($G_{vac}$) rigidly defines a fundamental \textbf{Characteristic Length Scale} ($l_c = \sqrt{\gamma_c/G_{vac}}$). This length scale is identified as the physical origin of the weak force range ($r_W \approx 10^{-18}$ m).

Weak interactions lack the kinetic energy required to overcome the ambient LC rotational inductance. Any physical excitation operating \textit{below} a medium's natural cutoff frequency is mathematically forced to become an \textbf{Evanescent Wave}. The static field equation transforms from the Laplace equation to the massive Helmholtz equation ($\nabla^2 \theta - \frac{1}{l_c^2}\theta = 0$). The solution natively yields the exact \textbf{Yukawa Potential}:
\begin{equation}
    V_{weak}(r) \propto \frac{e^{-r/l_c}}{r}
\end{equation}

\subsection{Deriving the Gauge Bosons (\texorpdfstring{$W^{\pm}/Z^{0}$}{W/Z}) as Evanescent Modes}

The gauge bosons of the weak interaction represent the fundamental macroscopic evanescent cutoff excitations required to mechanically induce a localized phase twist.

\begin{itemize}
\item The charged $W^{\pm}$ bosons correspond to the pure longitudinal-torsional evanescent mode ($k\propto G_{vac}J$).
\item The neutral $Z^{0}$ boson corresponds to the transverse-bending evanescent mode ($k\propto E_{vac}I$).
\end{itemize}

Because Axiom 1 strictly bounds the physical diameter of a fundamental flux tube to exactly $d \equiv 1 l_{node}$ (the hard-sphere exclusion limit), these topological connections mechanically act as volume-bearing physical 3D continuous cylinders at the macroscopic limit. Furthermore, because the tube is formed by a radially symmetric dielectric displacement field, the Perpendicular Axis Theorem strictly dictates that its polar moment of inertia evaluates exactly to $J=2I$. This is a geometric absolute for any circular cross-section, not an assumed relationship.

Because the rest mass of an evanescent cutoff mode scales exactly with the square root of its structural stiffness ($m \propto \sqrt{k}$), the mass ratio evaluates to $m_W/m_Z = \sqrt{GJ / EI}$. Substituting the fundamental cylinder geometry ($J=2I$) strictly yields $\sqrt{2G/E}$. Applying the standard isotropic elastic continuous identity ($E = 2G(1+\nu)$) mathematically reduces this stiffness ratio to:

\begin{equation}
\frac{m_W}{m_Z} = \sqrt{\frac{2G}{2G(1+\nu_{vac})}} = \frac{1}{\sqrt{1+\nu_{vac}}}
\end{equation}

By substituting the geometric Chiral LC trace-reversed limit mathematically proven in Chapter 4 ($\nu_{vac} \equiv 2/7$), the weak mixing angle emerges as an exact analytical prediction:

\begin{equation}
\frac{m_W}{m_Z} = \frac{1}{\sqrt{1+2/7}} = \frac{1}{\sqrt{9/7}} = \frac{\sqrt{7}}{3} \approx 0.881917
\end{equation}

This derivation matches the experimental ratio to within 0.05\% error, offering a direct mechanical origin for the mass splitting without invoking symmetry-breaking scalar fields. The corresponding on-shell weak mixing angle is:
\begin{equation}
    \sin^2\theta_W^{\text{on-shell}} \equiv 1 - \left(\frac{M_W}{M_Z}\right)^2 = 1 - \frac{7}{9} = \frac{2}{9} \approx 0.2222
\end{equation}
This matches the PDG on-shell value ($0.2230$) to within $0.35\%$. The commonly quoted MSbar value ($0.2312$) incorporates radiative corrections and is a distinct quantity.

\subsection{The Absolute $W$ Boson Mass: Chirality Mismatch Self-Energy}

A twist defect in the vacuum creates a torsional field that obeys the same 3D Laplace equation as the Coulomb field: $\nabla^2\theta = 0$, giving $\theta(r) \propto 1/r$ and $|\nabla\theta|^2 \propto 1/r^4$. The self-energy integral is:
\begin{equation}
    E_{\text{twist}} = \frac{T_{EM}^2}{4\pi\,\varepsilon_T\, r_0}
\end{equation}
where $T_{EM}$ is the lattice tension (torsional ``charge''), $r_0 = \ell_{node}/(2\pi)$ is the flux tube UV cutoff (Axiom 1), and $\varepsilon_T$ is the torsional permittivity of the chiral lattice.

The Chiral SRS net (Axiom 2) has an intrinsic handedness. A twist that \emph{matches} the lattice chirality propagates freely (this is why left-handed neutrinos are nearly massless). A twist that \emph{opposes} the chirality fights the LC ground state, incurring a stiffness penalty.

The factor $\alpha^2$ is derived from the interaction Lagrangian. The twist field $\phi$ couples to the EM background through the Axiom~4 dielectric susceptibility $\varepsilon(\phi) = \varepsilon_0(1 + \alpha f(\phi))$, giving:
\begin{equation}
    \mathcal{L}_{\text{int}} = \frac{\varepsilon_0 \alpha}{2}\,\phi\,|\mathbf{E}|^2
\end{equation}
The self-energy is a \textbf{two-vertex process} (second-order perturbation theory):
\begin{equation}
    E_{\text{self}} = \iint \mathcal{L}_{\text{int}}(\mathbf{x})\, G(\mathbf{x}-\mathbf{x}')\, \mathcal{L}_{\text{int}}(\mathbf{x}')\, d^3x\, d^3x' \;\propto\; \alpha \times \alpha = \alpha^2
\end{equation}
\section{Electroweak Mechanics: Summary of Key Results}

The electroweak sector is derived in full in Chapter \ref{ch:electroweak_higgs}. Here we summarize only the critical results that interlock with the gauge structure derived in this chapter.

\subsection{The Weak Mixing Angle and Boson Masses}
The weak mixing angle is derived from the Perpendicular Axis Theorem (PAT) applied to cylindrical flux tubes with Poisson ratio $\nu_{vac} = 2/7$:
\begin{equation}
    \sin^2\theta_W = 1 - \frac{7}{9} = \frac{2}{9} \approx 0.2222 \qquad (\text{PDG on-shell: } 0.2230, \; -0.35\%)
\end{equation}

The W and Z masses follow from the torsional self-energy of the unknot:
\begin{equation}
    M_W = \frac{m_e}{8\pi\alpha^3\sqrt{3/7}} \approx 79{,}923 \text{ MeV}, \qquad M_Z = \frac{3}{\sqrt{7}}M_W \approx 90{,}624 \text{ MeV}
\end{equation}

The Schwinger anomalous magnetic moment is derived from the on-site impedance correction of the hopping unknot (full derivation in Chapter \ref{ch:electroweak_higgs}):
\begin{equation}
    a_e = \frac{\alpha}{2\pi} \approx 0.001161
\end{equation}

\subsection{From Photon to Electron: The Self-Trapping Transition}

A circularly polarized photon on the Chiral LC lattice is a torsional helix---the $\vec{E}$ and $\vec{H}$ vectors rotate around the propagation axis with pitch equal to the wavelength. The transverse radius of this helix is set by the angular momentum coherence length:
\begin{equation}
    R_{\text{helix}} = \frac{\lambda}{2\pi} = \frac{c}{2\pi f}
\end{equation}
This is a \emph{scaling law}: higher frequency $\to$ tighter helix $\to$ smaller polarization footprint on the lattice. The photon's spin angular momentum is concentrated in a cylinder of radius $R_{\text{helix}}$ around the propagation axis.

\paragraph{The critical frequency.} As $f$ increases, $R$ shrinks. When the frequency reaches the Compton frequency,
\begin{equation}
    f_C = \frac{m_e c^2}{\hbar} \approx 1.24 \times 10^{20} \text{ Hz},
\end{equation}
the helix radius collapses to
\begin{equation}
    R_{\text{helix}} \to \frac{\ell_{\text{node}}}{2\pi}
\end{equation}
---a fraction of a single lattice cell. The photon's field now wraps entirely within one node. Its leading wavefront interferes with its own trailing edge. \textbf{The photon catches its own tail.}

\paragraph{Topological trapping.} At this point the helix self-closes into an unknot---a topologically stable loop confined to a single lattice site. The circulating EM energy can no longer propagate; it is trapped. The photon has become an electron:
\begin{itemize}
    \item \textbf{Mass} $= \hbar\omega/c^2$: the trapped photon's energy, now localized.
    \item \textbf{Charge} $= e$: the topological winding number of the unknot.
    \item \textbf{Spin-1/2}: half the photon's spin-1, because the loop closes after a $2\pi$ rotation, not $4\pi$.
    \item \textbf{$g{-}2$ anomaly}: the residual interaction of the trapped helix with the node it occupies.
\end{itemize}
The fine structure constant $\alpha = (V_{\text{peak}}/V_{\text{snap}})^2 / (4\pi)$ measures exactly \emph{how much} the trapped photon strains the lattice node it sits on. The entire spectrum of matter---from free photons to confined electrons---is a single excitation at different frequencies, separated by the self-trapping threshold at $f = f_C$.

\begin{figure}[h]
    \centering
    \includegraphics[width=1.0\textwidth]{electroweak_acoustic_modes.png}
    \caption{\textbf{Electroweak Unification: Discrete LC Acoustic Resonance.} (Simulation Output). A continuous frequency-domain solver tracking the spatial LC phase logic. While macroscopic scales natively exhibit completely decoupled Electric ($Z_C$) and Magnetic ($Z_L$) field reactances, when the incident wavelength identically matches the discrete network grid spacing ($f_{res}$) the continuous continuous metric breaks. The reactive vectors abruptly merge into a single symmetric mechanical acoustic phase. The Weak Interaction dynamically arises as the macroscopic breakdown limit of the discrete LC structure, rather than an independent abstract field.}
    \label{fig:electroweak_acoustic_modes}
\end{figure}

\section{The Gauge Layer: From Topology to Symmetry}

\subsection{U(1) Electromagnetism from the Lattice Plaquette}
The physical continuous connection between adjacent nodes $i$ and $j$ is mathematically described by a unitary link variable $U_{ij} = e^{i\theta_{ij}}$, where $\theta_{ij}$ is the phase accumulated along the edge. The simplest gauge-invariant geometric quantity is the triangular plaquette---the product of link variables around a closed 3-node loop:
\begin{equation}
    U_P = U_{ij}U_{jk}U_{ki} = e^{i(\theta_{ij} + \theta_{jk} + \theta_{ki})}
\end{equation}

The total phase around the plaquette is the discrete lattice curl of the gauge connection. For small phase gradients ($\theta_{ij} \approx A_\mu \ell_{node}$), the Taylor expansion of $U_P$ yields:
\begin{equation}
    \theta_{ij} + \theta_{jk} + \theta_{ki} = \oint \mathbf{A} \cdot d\mathbf{l} = \iint (\nabla \times \mathbf{A}) \cdot d\mathbf{S} = \iint \mathbf{B} \cdot d\mathbf{S} \equiv \Phi_P
\end{equation}

The lattice action is constructed by summing over all plaquettes the deviation from unit phase:
\begin{equation}
    S_{lattice} = \sum_P \left(1 - \text{Re}\, U_P\right) = \sum_P \left(1 - \cos\Phi_P\right) \approx \sum_P \frac{1}{2}\Phi_P^2 \longrightarrow \int \frac{1}{4}F_{\mu\nu}F^{\mu\nu}\, d^4x
\end{equation}

The continuum limit ($\ell_{node} \to 0$) identically recovers the Maxwell Lagrangian ($-\frac{1}{4}F_{\mu\nu}F^{\mu\nu}$). \textbf{U(1) Electromagnetism} is therefore the strict enforcement of unitary topological continuity across the discrete graph---the standard Wilson formulation of lattice gauge theory, here derived from the physical structure of the $\mathcal{M}_A$ hardware.

\subsection{SU(3) Color Charge from the Borromean Linkage}
The Borromean proton ($6^3_2$) consists of three topologically indistinguishable interlocked flux loops. Because no two loops are individually linked, the mathematical permutation symmetry of the three-loop system is the symmetric group $S_3$. This discrete symmetry classifies the allowed ``color'' states of the composite defect.

To parallel-transport the continuous phase field $\mathbf{A}$ smoothly across a tri-partite symmetric graph, the connection must locally respect $S_3$ permutation invariance while preserving the unitarity of phase transport. The smallest continuous Lie group whose discrete quotient contains $S_3$ as a subgroup of its Weyl group is $SU(3)$. Explicitly:
\begin{itemize}
    \item The $S_3$ permutation group is the Weyl group of $SU(3)$.
    \item The three fundamental flux loops of the Borromean linkage transform under the fundamental representation (\textbf{3}) of $SU(3)$.
    \item The $\mathbb{Z}_3$ center of $SU(3)$ enforces the strict topological constraint that only color-singlet ($\mathbf{1}$) composite states---where all three loops are linked---can propagate as free particles. This is confinement.
\end{itemize}

\textbf{SU(3) Color Charge} is derived as the exact effective field theory limit of a three-loop topological defect traversing a discrete condensate grid. The ``color'' quantum number is the permutation label of which flux loop carries the dominant phase winding at any given lattice site.