% 08_electroweak_gauge_theory.tex
\chapter{Electroweak Mechanics and Gauge Symmetries}
\label{ch:electroweak}

\section{Electrodynamics: The Gradient of Topological Phase}
A localized charged node permanently exerts a continuous rotational phase twist ($\theta$) on the surrounding LC condensate. Because the unsaturated vacuum acts as a linear dielectric in the far-field, the static structural phase strain must strictly obey the 3D \textbf{Laplace Equation} ($\nabla^2 \theta = 0$).

The spherically symmetric geometric solution dictates that the twist amplitude decays exactly inversely with distance ($\theta(r) \propto 1/r$). The continuous electric displacement field ($\mathbf{D}$) is physically identical to the spatial gradient of this structural phase twist ($\mathbf{D} = \nabla\theta \propto -1/r^2 \mathbf{\hat{r}}$), analytically deriving Coulomb's Law.

\subsection{Magnetism as Convective Vorticity}
When a twisted node translates at a velocity $\mathbf{v}$, it induces a convective shear flow in the momentum field. In classical network dynamics, the time evolution of a translating steady-state strain field $\mathbf{D}(\mathbf{r} - \mathbf{v}t)$ is governed by the convective material derivative:
\begin{equation}
    \partial_t \mathbf{D} = -(\mathbf{v} \cdot \nabla)\mathbf{D} \implies \nabla \times (\mathbf{v} \times \mathbf{D})
\end{equation}
Equating this to the Maxwell-Ampere law derives the macroscopic magnetic field strictly from network dynamics: $\mathbf{H} = \mathbf{v} \times \mathbf{D}$.

This relationship is rigorously supported by dimensional analysis. Applying the topological conversion constant ($\xi_{topo} \equiv e/\ell_{node}$), the displacement field reduces to $[\mathbf{D}] = \xi_{topo}[1/\text{m}]$. Evaluating the cross product $[\mathbf{v} \times \mathbf{D}]$ yields strictly $\xi_{topo}[1/\text{s}]$. Standard SI units for magnetic field intensity $\mathbf{H}$ ($[\text{A/m}]$) identically reduce to this exact same dimensional basis ($\xi_{topo}[1/\text{s}]$). Magnetism is thereby dimensionally proven to represent the continuous kinematic vorticity of the vacuum condensate.

\subsection{The Inductive Origin of Gauge Invariance}
Standard Quantum Field Theory mandates that the vector potential is a gauge field, where transformations of the form $\mathbf{A} \to \mathbf{A} + \nabla \Lambda$ leave physical observables ($\mathbf{B}$ and $\mathbf{E}$) unchanged. A common critique of identifying $\mathbf{A}$ as a physical momentum field is that this gauge freedom would imply the unphysical, spontaneous shifting of macroscopic mass, violating Noether's theorem.

This paradox is resolved rigorously via the \textbf{Helmholtz Decomposition Theorem} in classical network dynamics. Any continuous vector field can be decomposed into a solenoidal (divergence-free) component and an irrotational (curl-free) component. Adding the gradient of a scalar field ($\nabla \Lambda$) to the mass flow strictly introduces a uniform, irrotational velocity potential to the background network.

Because the $\mathcal{M}_A$ vacuum is highly incompressible ($K = 2G$), an irrotational flow field generates no localized compression ($-\partial_t \mathbf{A}$), no transverse vorticity ($\nabla \times \mathbf{A}$), and no topological defects. It is physically isomorphic to performing a \textbf{Galilean or Lorentz coordinate boost} of the observer's reference frame. Gauge invariance is not violated; it is strictly revealed to be the classical network-dynamic freedom to shift the irrotational background coordinate velocity without altering the physical transverse observables.

\section{The Weak Interaction: Inductive Cutoff Dynamics}
In classical electrodynamics, the ratio of the LC network's microrotational bending inductance ($\gamma_c$) to the macroscopic optical shear modulus ($G_{vac}$) rigidly defines a fundamental \textbf{Characteristic Length Scale} ($l_c = \sqrt{\gamma_c/G_{vac}}$). This length scale is identified as the physical origin of the weak force range ($r_W \approx 10^{-18}$ m).

Weak interactions lack the kinetic energy required to overcome the ambient LC rotational inductance. Any physical excitation operating \textit{below} a medium's natural cutoff frequency is mathematically forced to become an \textbf{Evanescent Wave}. The static field equation transforms from the Laplace equation to the massive Helmholtz equation ($\nabla^2 \theta - \frac{1}{l_c^2}\theta = 0$). The solution natively yields the exact \textbf{Yukawa Potential}:
\begin{equation}
    V_{weak}(r) \propto \frac{e^{-r/l_c}}{r}
\end{equation}

\subsection{Deriving the Gauge Bosons (\texorpdfstring{$W^{\pm}/Z^{0}$}{W/Z}) as Evanescent Modes}

The gauge bosons of the weak interaction represent the fundamental macroscopic evanescent cutoff excitations required to mechanically induce a localized phase twist.

\begin{itemize}
\item The charged $W^{\pm}$ bosons correspond to the pure longitudinal-torsional evanescent mode ($k\propto G_{vac}J$).
\item The neutral $Z^{0}$ boson corresponds to the transverse-bending evanescent mode ($k\propto E_{vac}I$).
\end{itemize}

Because Axiom 1 strictly bounds the physical diameter of a fundamental flux tube to exactly $d \equiv 1 l_{node}$ (the hard-sphere exclusion limit), these topological connections mechanically act as volume-bearing physical 3D continuous cylinders at the macroscopic limit. Furthermore, because the tube is formed by a radially symmetric dielectric displacement field, the Perpendicular Axis Theorem strictly dictates that its polar moment of inertia evaluates exactly to $J=2I$. This is a geometric absolute for any circular cross-section, not an assumed relationship.

Because the rest mass of an evanescent cutoff mode scales exactly with the square root of its structural stiffness ($m \propto \sqrt{k}$), the mass ratio evaluates to $m_W/m_Z = \sqrt{GJ / EI}$. Substituting the fundamental cylinder geometry ($J=2I$) strictly yields $\sqrt{2G/E}$. Applying the standard isotropic elastic continuous identity ($E = 2G(1+\nu)$) mathematically reduces this stiffness ratio to:

\begin{equation}
\frac{m_W}{m_Z} = \sqrt{\frac{2G}{2G(1+\nu_{vac})}} = \frac{1}{\sqrt{1+\nu_{vac}}}
\end{equation}

By substituting the geometric Chiral LC trace-reversed limit mathematically proven in Chapter 4 ($\nu_{vac} \equiv 2/7$), the weak mixing angle emerges as an exact analytical prediction:

\begin{equation}
\frac{m_W}{m_Z} = \frac{1}{\sqrt{1+2/7}} = \frac{1}{\sqrt{9/7}} = \frac{\sqrt{7}}{3} \approx 0.881917
\end{equation}

This derivation matches the experimental ratio to within 0.05\% error, offering a direct mechanical origin for the mass splitting without invoking symmetry-breaking scalar fields. The corresponding on-shell weak mixing angle is:
\begin{equation}
    \sin^2\theta_W^{\text{on-shell}} \equiv 1 - \left(\frac{M_W}{M_Z}\right)^2 = 1 - \frac{7}{9} = \frac{2}{9} \approx 0.2222
\end{equation}
This matches the PDG on-shell value ($0.2230$) to within $0.35\%$. The commonly quoted MSbar value ($0.2312$) incorporates radiative corrections and is a distinct quantity.

\subsection{The Absolute $W$ Boson Mass: Chirality Mismatch Self-Energy}

A twist defect in the vacuum creates a torsional field that obeys the same 3D Laplace equation as the Coulomb field: $\nabla^2\theta = 0$, giving $\theta(r) \propto 1/r$ and $|\nabla\theta|^2 \propto 1/r^4$. The self-energy integral is:
\begin{equation}
    E_{\text{twist}} = \frac{T_{EM}^2}{4\pi\,\varepsilon_T\, r_0}
\end{equation}
where $T_{EM}$ is the lattice tension (torsional ``charge''), $r_0 = \ell_{node}/(2\pi)$ is the flux tube UV cutoff (Axiom 1), and $\varepsilon_T$ is the torsional permittivity of the chiral lattice.

The Chiral SRS net (Axiom 2) has an intrinsic handedness. A twist that \emph{matches} the lattice chirality propagates freely (this is why left-handed neutrinos are nearly massless). A twist that \emph{opposes} the chirality fights the LC ground state, incurring a stiffness penalty.

The factor $\alpha^2$ is derived from the interaction Lagrangian. The twist field $\phi$ couples to the EM background through the Axiom~4 dielectric susceptibility $\varepsilon(\phi) = \varepsilon_0(1 + \alpha f(\phi))$, giving:
\begin{equation}
    \mathcal{L}_{\text{int}} = \frac{\varepsilon_0 \alpha}{2}\,\phi\,|\mathbf{E}|^2
\end{equation}
The self-energy is a \textbf{two-vertex process} (second-order perturbation theory):
\begin{equation}
    E_{\text{self}} = \iint \mathcal{L}_{\text{int}}(\mathbf{x})\, G(\mathbf{x}-\mathbf{x}')\, \mathcal{L}_{\text{int}}(\mathbf{x}')\, d^3x\, d^3x' \;\propto\; \alpha \times \alpha = \alpha^2
\end{equation}
This is the \emph{same} mechanism that gives $e^2$ in the Coulomb self-energy: two factors of the coupling constant, one for each vertex. Higher-order (loop) corrections contribute $\alpha^3, \alpha^4, \ldots$, accounting for the $0.57\%$ tree-level deviation.

The torsional permittivity relative to the shear modulus decomposes as:
\begin{equation}
    \frac{\varepsilon_T}{\mu} = \pi \cdot \alpha^2 \cdot p_c \cdot \sqrt{3/7}
\end{equation}
\begin{center}
\begin{tabular}{|c|c|c|}
\hline
Factor & Origin & Source \\ \hline
$\pi$ & Spherical geometry of $1/r^2$ integral & Mathematics \\
$\alpha^2$ & Two-vertex coupling ($\mathcal{L}_{\text{int}} \times \mathcal{L}_{\text{int}}$) & Axiom 4 (Dielectric) \\
$p_c = 8\pi\alpha$ & Volumetric packing fraction & Axiom 4 (Saturation) \\
$\sqrt{3/7}$ & Torsion--shear projection & PAT $+$ $\nu_{vac} = 2/7$ \\ \hline
\end{tabular}
\end{center}

Evaluating yields:
\begin{equation}
    M_W = \frac{m_e}{\alpha^2 \, p_c \, \sqrt{3/7}} = \frac{m_e}{8\pi\alpha^3 \sqrt{3/7}} \approx 79{,}923 \text{ MeV} \qquad (0.57\%)
\end{equation}
Combined with the mixing angle:
\begin{equation}
    M_Z = \frac{3}{\sqrt{7}} M_W \approx 90{,}624 \text{ MeV} \qquad (0.62\%)
\end{equation}

The tree-level Fermi constant follows from the on-shell relation:
\begin{equation}
    G_F = \frac{\sqrt{2}\,\pi\alpha}{2\sin^2\theta_W \, M_W^2} \approx 1.142 \times 10^{-5} \text{ GeV}^{-2} \qquad (-2.1\%)
\end{equation}

No adjustable parameters appear: $\alpha$, $p_c$, $\nu_{vac}$ are all independently derived geometric constants of the Chiral LC lattice. The chirality mismatch mechanism provides the physical origin: the $W$ mass is the self-energy cost of forcing a wrong-handed twist against the chiral vacuum.

\subsection{The Anomalous Magnetic Moment: On-Site Impedance Correction}

The electron $g{-}2$ has long been the gold standard of quantum field theory. Schwinger's 1948 result, $a_e = \alpha/(2\pi)$, emerges here from the nonlinear self-interaction of the unknot with each lattice node it visits.

\paragraph{The electron as impedance discontinuity.} The vacuum impedance of the Chiral LC lattice is $Z_0 = \sqrt{\mu_0/\varepsilon_0} = 376.73\;\Omega$. Near the electron's core, the magnetic field of the Bohr magneton exceeds the saturation threshold ($B > B_{\text{snap}}$), driving $\mu_{\text{eff}} \to 0$. Like a saturated inductor, this shorts the node: $Z = \sqrt{\mu_{\text{eff}}/\varepsilon_0} \to 0$. The electron is a zero-impedance core embedded in the $377\;\Omega$ vacuum---an impedance discontinuity that reflects its own field.

\paragraph{The hopping unknot.} The unknot does not sit at a fixed lattice site; it hops from node to node as a quantum excitation. At each node it visits, all $m_e c^2$ is momentarily stored as EM field energy in one cell of volume $\ell^3$, split equally between the electric and magnetic sectors. The peak electric strain is:
\begin{equation}
    \left(\frac{V_{\text{peak}}}{V_{\text{snap}}}\right)^2 = 4\pi\alpha \qquad [\text{exact identity}]
\end{equation}
This follows directly from $\tfrac{1}{2}\varepsilon_0 E_{\text{peak}}^2 \ell^3 = \tfrac{1}{2}m_e c^2$ and $V_{\text{snap}} = m_e c^2/e$. \textbf{The fine structure constant is the on-site electric strain of the unknot.}

\paragraph{The nonlinear back-reaction.} The Axiom~4 nonlinear dielectric $\varepsilon_{\text{eff}} = \varepsilon_0\sqrt{1 - (V/V_s)^2}$ modifies the node capacitance. Time-averaged over the LC oscillation ($\langle\sin^2\rangle = 1/2$):
\begin{equation}
    \left\langle\frac{\delta C}{C}\right\rangle = -\pi\alpha, \qquad
    \frac{\delta\omega}{\omega} = \frac{\pi\alpha}{2}
\end{equation}
This is the total on-site frequency shift, which goes into mass renormalization. The anomalous magnetic moment receives only the \emph{form factor fraction}: the part of the correction that falls within the ring's topological domain. The unknot has diameter $2R = \ell/\pi$ (Axiom~1: circumference $= \ell$), giving an effective cross-section $(2R)^2 = \ell^2/\pi^2$ within the cell face $\ell^2$. The \textbf{form factor} is:
\begin{equation}
    F = \frac{(2R)^2}{\ell^2} = \frac{1}{\pi^2}
\end{equation}
The total on-site correction decomposes cleanly:
\begin{equation}
    \underbrace{\frac{\pi\alpha}{2}}_{\text{total}} \;=\;
    \underbrace{\left(1 - \frac{1}{\pi^2}\right)\frac{\pi\alpha}{2}}_{\text{mass renormalization}} \;+\;
    \underbrace{\frac{1}{\pi^2}\cdot\frac{\pi\alpha}{2}}_{\text{g-2 anomaly}}
\end{equation}
The anomalous magnetic moment is:
\begin{equation}
    \boxed{a_e = \frac{1}{\pi^2} \cdot \frac{\pi\alpha}{2} = \frac{\alpha}{2\pi} \approx 0.001161}
\end{equation}
This reproduces Schwinger's 1948 result from the lattice without renormalization. The lattice spacing $\ell_{\text{node}}$ \emph{is} the physical UV regulator---there are no sub-lattice degrees of freedom to generate a divergence. The ``point-like'' behaviour observed in deep inelastic scattering corresponds to the zero-crossing of the AC standing wave at the node centre, where the field intensity momentarily vanishes.

\subsection{From Photon to Electron: The Self-Trapping Transition}

A circularly polarized photon on the Chiral LC lattice is a torsional helix---the $\vec{E}$ and $\vec{H}$ vectors rotate around the propagation axis with pitch equal to the wavelength. The transverse radius of this helix is set by the angular momentum coherence length:
\begin{equation}
    R_{\text{helix}} = \frac{\lambda}{2\pi} = \frac{c}{2\pi f}
\end{equation}
This is a \emph{scaling law}: higher frequency $\to$ tighter helix $\to$ smaller polarization footprint on the lattice. The photon's spin angular momentum is concentrated in a cylinder of radius $R_{\text{helix}}$ around the propagation axis.

\paragraph{The critical frequency.} As $f$ increases, $R$ shrinks. When the frequency reaches the Compton frequency,
\begin{equation}
    f_C = \frac{m_e c^2}{\hbar} \approx 1.24 \times 10^{20} \text{ Hz},
\end{equation}
the helix radius collapses to
\begin{equation}
    R_{\text{helix}} \to \frac{\ell_{\text{node}}}{2\pi}
\end{equation}
---a fraction of a single lattice cell. The photon's field now wraps entirely within one node. Its leading wavefront interferes with its own trailing edge. \textbf{The photon catches its own tail.}

\paragraph{Topological trapping.} At this point the helix self-closes into an unknot---a topologically stable loop confined to a single lattice site. The circulating EM energy can no longer propagate; it is trapped. The photon has become an electron:
\begin{itemize}
    \item \textbf{Mass} $= \hbar\omega/c^2$: the trapped photon's energy, now localized.
    \item \textbf{Charge} $= e$: the topological winding number of the unknot.
    \item \textbf{Spin-1/2}: half the photon's spin-1, because the loop closes after a $2\pi$ rotation, not $4\pi$.
    \item \textbf{$g{-}2$ anomaly}: the residual interaction of the trapped helix with the node it occupies (Section above).
\end{itemize}
The fine structure constant $\alpha = (V_{\text{peak}}/V_{\text{snap}})^2 / (4\pi)$ measures exactly \emph{how much} the trapped photon strains the lattice node it sits on. The entire spectrum of matter---from free photons to confined electrons---is a single excitation at different frequencies, separated by the self-trapping threshold at $f = f_C$.

\subsection{The Neutrino: Torsional Screw Defect}


While the electron is an unknot (a closed flux tube loop---an edge dislocation), the neutrino is a pure \textbf{screw dislocation}: a helical twist of the lattice that carries spin-$1/2$ but no topological charge. Its mass is suppressed relative to the electron by two factors:
\begin{itemize}
    \item $\alpha$: the dielectric coupling between the torsional and translational sectors;
    \item $m_e/M_W$: the ratio of the translational to torsional energy scale.
\end{itemize}
This yields:
\begin{equation}
    m_\nu = m_e \cdot \alpha \cdot \frac{m_e}{M_W} \approx 24 \text{ meV}
    \label{eq:neutrino_mass}
\end{equation}

The three neutrino flavors arise from the torus knot ladder: each flavor pairs with a baryon resonance via the crossing number $c$, with mass splitting proportional to $1/c$:
\begin{center}
\begin{tabular}{|c|c|c|c|}
\hline
Flavor & Paired Baryon & $c$ & $m_\nu$ (meV) \\ \hline
$\nu_1$ & Proton (938 MeV) & 5 & 23.8 \\
$\nu_2$ & $\Delta(1232)$ & 7 & 17.0 \\
$\nu_3$ & $\Delta(1620)$ & 9 & 13.2 \\ \hline
\multicolumn{3}{|c|}{$\sum m_\nu$} & \textbf{54.1 meV} \\ \hline
\end{tabular}
\end{center}

The predicted sum $\sum m_\nu \approx 0.054$ eV lies comfortably within the Planck 2018 cosmological bound ($\sum m_\nu < 0.12$ eV) and near the normal-ordering hint ($\sum m_\nu \sim 0.06$ eV). This prediction is falsifiable: upcoming measurements from DESI and CMB-S4 will constrain $\sum m_\nu$ to $\pm 0.02$ eV.

\subsection{The Charged Lepton Spectrum: Three Cosserat Sectors}

All three charged leptons are topologically identical---each is an unknot (a closed flux tube loop). Their mass hierarchy arises from the three sectors of the Cosserat Lagrangian, each contributing one coupling channel:

\begin{enumerate}
    \item \textbf{Translation ($\mu$-sector, Electron):} The unknot ground state: $m_e = T_{EM} \cdot \ell_{node} / c^2$. No torsional excitation.
    \item \textbf{Rotation ($\kappa$-sector, Muon):} The unknot absorbs one quantum of torsional coupling $\alpha\sqrt{3/7}$ (one chirality interaction projected through the PAT):
    \begin{equation}
        m_\mu = \frac{m_e}{\alpha \sqrt{3/7}} \approx 107.0 \text{ MeV} \qquad (+1.24\%)
    \end{equation}
    Only \emph{one} factor of $\alpha$ enters because the muon is a static defect absorbing one quantum; the $W$ boson needs $\alpha^2$ because it creates \emph{and} destroys.
    \item \textbf{Curvature-twist ($\gamma_C$-sector, Tau):} The unknot is promoted to the full bending stiffness scale:
    \begin{equation}
        m_\tau = \frac{m_e \cdot p_c}{\alpha^2} = \frac{8\pi \, m_e}{\alpha} \approx 1{,}760 \text{ MeV} \qquad (-0.95\%)
    \end{equation}
    This is the maximum excitation before packing saturation.
\end{enumerate}

\begin{center}
\begin{tabular}{|c|c|c|c|c|}
\hline
Gen & Lepton & Cosserat Sector & Predicted & Dev \\ \hline
1 & $e$ & Translation ($\mu$) & 0.511 MeV & Exact \\
2 & $\mu$ & Rotation ($\kappa$) & 107.0 MeV & $+1.24\%$ \\
3 & $\tau$ & Curvature ($\gamma_C$) & 1760 MeV & $-0.95\%$ \\ \hline
\end{tabular}
\end{center}

The existence of \textbf{exactly three} generations follows from the structure of the Cosserat Lagrangian. An isotropic micropolar medium admits exactly three independent deformation sectors:
\begin{enumerate}
    \item \textbf{Translation} (shear modulus $\mu$): symmetric strain $\varepsilon_{ij}$. Governs the electromagnetic sector. Particle: $e$.
    \item \textbf{Microrotation} (Cosserat coupling $\kappa$): antisymmetric strain $e_{[ij]}$. Governs the weak torsional sector. Particle: $\mu$.
    \item \textbf{Curvature-twist} (bending stiffness $\gamma_C$): gradient of rotation $\kappa_{ij} = \phi_{i,j}$. Governs the dielectric saturation sector. Particle: $\tau$.
\end{enumerate}
A hypothetical fourth generation would require a \emph{strain-gradient} coupling ($\kappa_{ij,k}$), the gradient of the gradient of rotation. This is the domain of Mindlin--Toupin theory, which demands internal degrees of freedom beyond translation and rotation at each node. The discrete Chiral LC lattice, with one unknot (rigid body) per site, possesses exactly six DOF per node (3 translations $+$ 3 rotations) and cannot support strain-gradient modes. Three generations is a \textbf{topological bound}, not a parameter.

\begin{figure}[h]
    \centering
    \includegraphics[width=1.0\textwidth]{electroweak_acoustic_modes.png}
    \caption{\textbf{Electroweak Unification: Discrete LC Acoustic Resonance.} (Simulation Output). A continuous frequency-domain solver tracking the spatial LC phase logic. While macroscopic scales natively exhibit completely decoupled Electric ($Z_C$) and Magnetic ($Z_L$) field reactances, when the incident wavelength identically matches the discrete network grid spacing ($f_{res}$) the continuous continuous metric breaks. The reactive vectors abruptly merge into a single symmetric mechanical acoustic phase. The Weak Interaction dynamically arises as the macroscopic breakdown limit of the discrete LC structure, rather than an independent abstract field.}
    \label{fig:electroweak_acoustic_modes}
\end{figure}

\section{The Gauge Layer: From Topology to Symmetry}
The physical continuous connection between nodes is mathematically described by a unitary link variable $U_{ij}$. The simplest gauge-invariant geometric quantity is the 3-node triangular plaquette ($U_P = U_{ij}U_{jk}U_{ki}$). Expanding this topologically continuous loop via Taylor series natively recovers the Maxwell Lagrangian ($-\frac{1}{4}F_{\mu\nu}F^{\mu\nu}$). \textbf{U(1) Electromagnetism} represents the strict enforcement of unitary topological continuity across the discrete graph.

Furthermore, because the Borromean proton ($6^3_2$) consists of three topologically indistinguishable interlocked loops, its discrete mathematical permutation symmetry is exactly $S_3$. The continuous mathematical envelope required to locally parallel-transport the phase smoothly across a tri-partite symmetric graph is exactly the $SU(3)$ Lie group. \textbf{SU(3) Color Charge} is derived as the exact effective field theory limit of a three-loop topological defect traversing a discrete condensate grid.