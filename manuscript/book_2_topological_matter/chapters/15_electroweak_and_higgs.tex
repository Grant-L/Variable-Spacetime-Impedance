% 15_electroweak_and_higgs.tex
\chapter{The Subatomic Scale: Electroweak and Higgs Sectors}
\label{ch:electroweak_higgs}

\section{Reinterpretation of the Higgs Mechanism}
The Standard Model posits the existence of a scalar "Higgs Field" carrying a non-zero Vacuum Expectation Value (VEV) of $v = 246 \text{ GeV}$. According to the Glashow-Weinberg-Salam model, fundamental particles gain mass exclusively by interacting ("coupling") with this pervasive scalar field.

In the AVE framework, mass generation does not require a separate scalar field. Instead, it emerges naturally from the impedance structure of the LC condensate. The framework proposes that the empirical $125 \text{ GeV}$ resonance observed at the LHC corresponds to a transient acoustic mode---a topological node undergoing rapid structural relaxation upon high-energy impact.

Mass generation requires no exclusive scalar field because the "vacuum" is already a structured, continuous LC mesh. The $246 \text{ GeV}$ VEV parameterizes physically as the explicit \textit{Characteristic Impedance of Free Space}:
\begin{equation}
    Z_0 = \sqrt{\frac{\mu_0}{\epsilon_0}} \approx 376.73 \ \Omega
\end{equation}

When a topological knot of magnetic flux (e.g., a $0_1$ electron unknot) accelerates through this baseline LC grid, it encounters Lenz's Law induction drag from the vacuum itself. The energy required to propagate the knot against this $376.73 \ \Omega$ characteristic limit is precisely what is measured as Inertial Mass. 
\begin{equation}
    M_{inertial} \equiv L_{drag}(Z_0)
\end{equation}

Thus, physical "mass" is strictly Macroscopic Electromagnetic Resistance.

\section{The Weak Mixing Angle from the Perpendicular Axis Theorem}

The electroweak mixing angle is not a free parameter in the AVE framework. It is derived analytically from the Poisson ratio of the vacuum ($\nu_{vac} = 2/7$) and the Perpendicular Axis Theorem (PAT) applied to cylindrical flux tubes.

The W and Z bosons are torsional (Cosserat) excitations of the chiral LC lattice. Their mass ratio is set by the relationship between shear and torsional stiffness in a cylindrical medium. For a cylinder with Poisson ratio $\nu$, the PAT relates the polar moment of area $J$ to the two planar moments $I_x$, $I_y$:
\begin{equation}
    J = I_x + I_y = 2I
\end{equation}

The torsional-to-shear stiffness ratio is therefore $GJ/(EI) = 2G/E = 2G/(2G(1+\nu))$. For the flux tube eigenmode, this ratio sets the W/Z pole mass ratio:
\begin{equation}
    \frac{m_W}{m_Z} = \frac{1}{\sqrt{1 + \nu_{vac}}} = \frac{1}{\sqrt{1 + 2/7}} = \sqrt{\frac{7}{9}} = \frac{\sqrt{7}}{3} \approx 0.8819
\end{equation}

The on-shell weak mixing angle follows directly:
\begin{equation}
    \sin^2\theta_W = 1 - \left(\frac{M_W}{M_Z}\right)^2 = 1 - \frac{7}{9} = \frac{2}{9} \approx 0.2222
\end{equation}

\noindent\textbf{Comparison:} The PDG on-shell value is $\sin^2\theta_W = 0.2230$, a deviation of $-0.35\%$. The commonly cited $\overline{\text{MS}}$ value ($0.2312$) differs from the on-shell value by radiative corrections; the AVE derivation predicts the pole mass ratio, which is the correct physical comparison.

An internal coupling constant used in the W mass derivation is the torsion-shear projection from the PAT:
\begin{equation}
    \sin\theta_W^{(\text{PAT})} = \sqrt{\frac{3}{7}} \approx 0.6547
\end{equation}

This is the ratio of torsional to total mechanical coupling in the cylindrical flux tube geometry.

\section{The W and Z Boson Masses}

\subsection{Derivation of $M_W$}

The W boson mass is derived from the torsional ring self-energy of the unknot, with a chirality mismatch coupling.

A twist defect in the chiral LC lattice creates a $1/r^2$ torsional field (Laplace solution, identical in form to Coulomb). For a point source, the self-energy is:
\begin{equation}
    E_{\text{point}} = \frac{T_{EM}^2}{4\pi \varepsilon_T r_0}
\end{equation}

But the unknot is a \textit{ring}, not a point. The circumference integral enhances the energy by $2\pi R/a = 2\pi$ (for the minimal-ropelength unknot where $R = a$):
\begin{equation}
    E_{\text{ring}} = E_{\text{point}} \times 2\pi = \frac{T_{EM}^2}{2\varepsilon_T r_0}
\end{equation}

The torsional permittivity $\varepsilon_T$ relative to the shear modulus is:
\begin{equation}
    \frac{\varepsilon_T}{\mu} = \pi \cdot \alpha^2 \cdot p_c \cdot \sqrt{3/7}
\end{equation}

Each factor has a first-principles geometric origin:
\begin{enumerate}
    \item $\pi$ --- spherical geometry of the $1/r^2$ integral
    \item $\alpha^2$ --- two-vertex coupling (Axiom 4 dielectric $\times 2$)
    \item $p_c = 8\pi\alpha$ --- packing fraction (Axiom 4: Saturation)
    \item $\sqrt{3/7}$ --- torsion-shear projection from the PAT and $\nu = 2/7$
    \item $2\pi$ --- ring topology of the unknot (Axiom 1)
\end{enumerate}

\textbf{The $\alpha^2$ factor} arises because the twist field $\phi$ couples to the EM background through the Axiom 4 dielectric susceptibility. The self-energy is a two-vertex process (second-order perturbation theory):
\begin{itemize}
    \item Vertex 1: twist $\to$ dielectric perturbation (factor $\alpha$)
    \item Vertex 2: dielectric perturbation $\to$ twist (factor $\alpha$)
\end{itemize}
This is the same mechanism that gives $e^2$ in the Coulomb self-energy: two factors of the coupling constant, one per vertex.

Evaluating $E_{\text{ring}}$ with all substitutions gives:
\begin{equation}
    \boxed{M_W = \frac{m_e}{\alpha^2 \cdot p_c \cdot \sqrt{3/7}} = \frac{m_e}{\alpha^2 \cdot 8\pi\alpha \cdot \sqrt{3/7}} = \frac{m_e}{8\pi\alpha^3\sqrt{3/7}}}
\end{equation}

Evaluating numerically: $M_W c^2 \approx 79{,}923 \text{ MeV}$ (CODATA: $80{,}379 \text{ MeV}$, deviation $+0.57\%$).

\subsection{Derivation of $M_Z$}

From the pole mass ratio derived via the Perpendicular Axis Theorem:
\begin{equation}
    M_Z = M_W \cdot \frac{3}{\sqrt{7}} \approx 90{,}624 \text{ MeV} \quad (\text{CODATA: } 91{,}188 \text{ MeV, } -0.62\%)
\end{equation}

\subsection{The Cosserat Characteristic Length}

The weak force range is the Compton wavelength of the W boson:
\begin{equation}
    \ell_C = \frac{\hbar}{M_W c} \approx 2.46 \times 10^{-18} \text{ m}
\end{equation}

This defines the evanescent decay length of the Cosserat (torsional) sector of the lattice.

\begin{figure}[H]
    \centering
    \includegraphics[width=0.85\textwidth]{electroweak_dielectric_spark.pdf}
    \caption{\textbf{Topological Mass Generation (The Higgs Field).} (Simulation Output). A 2D elastodynamic transient solver charting localized high-energy kinetic collisions ($E > E_{crit}$). Once applied strain formally eclipses the fundamental discrete grid's structural yielding point ($\epsilon_{sat}$), the localized spatial compliance physically breaks ($C \to 0$). The transient kinetic energy is instantly forced to condense into permanently trapped inertial tensor inductance ($L$). The Higgs Mechanism is rigorously derived as explicit spatial coordinate rupture, establishing the $W/Z$ bosons as macroscopic dielectric vacuum arcs bridging the severed continuous metric.}
    \label{fig:w_boson_spark}
\end{figure}

\section{W and Z Bosons as Dielectric Plasma Arcs}
The Weak Nuclear Force is allegedly mediated by massive W ($\sim 80 \text{ GeV}$) and Z ($\sim 91 \text{ GeV}$) bosons. Because they are so massive, Heisenberg's Uncertainty Principle restricts their existence to vanishingly tiny fractions of a second, necessitating their classification as "virtual" mediators during Beta Decay.

In the AVE framework, W and Z bosons are reinterpreted as transient dielectric breakdown events rather than fundamental gauge mediators.

During Beta Decay (such as a Neutron breaking into a Proton and an Electron), the primary topological knot undergoes extreme mechanical shear and must structurally split to shed phase-frequency. This splitting process breaks the continuous magnetic flux loop open for a fraction of an attosecond. 

The immense stored inductive energy of the knot attempts to cross this severed vacuum gap. Because the vacuum is a dielectric, this massive potential difference causes instantaneous \textit{Dielectric Breakdown} (Yield Limit fracture). The resulting $80 \text{ GeV}$ energy spike is physically a macroscopic phase-arc, or "Spark", traversing the grid. 

Once the arc bridges the gap, continuity is reestablished, and the resulting topologies phase-lock into their lower-energy states (Proton, Electron, and the transverse recoil acoustic wave/Neutrino). Electroweak theory is therefore completely absorbed into the fluid dynamics of High-Voltage Circuit Breakdown.

\section{The Three-Generation Lepton Spectrum}

Each charged lepton maps to one sector of the Cosserat micropolar Lagrangian applied to the unknot ground state:

\subsection{Generation 1: Translation (Shear Modulus $\mu$)}

The electron is the $0_1$ unknot ground state. No torsional excitation is present:
\begin{equation}
    m_e = \frac{T_{EM} \cdot \ell_{node}}{c^2} = \frac{\hbar}{\ell_{node} \cdot c} = 0.511 \text{ MeV}
\end{equation}

\subsection{Generation 2: Rotation (Cosserat Coupling $\kappa$)}

The muon is the unknot absorbing one quantum of torsional coupling. The coupling constant is $\alpha\sqrt{3/7}$, where $\alpha$ is the dielectric compliance (one chirality interaction) and $\sqrt{3/7}$ is the PAT torsion-shear projection:
\begin{equation}
    \boxed{m_\mu = \frac{m_e}{\alpha \sqrt{3/7}} \approx 107.0 \text{ MeV}} \quad (\text{Exp: } 105.66 \text{ MeV, } +1.24\%)
\end{equation}

Only \textit{one} factor of $\alpha$ appears because the muon is a static defect; the W boson requires $\alpha^2$ because it creates \textit{and} destroys a torsional perturbation (two vertices).

\begin{figure}[h]
    \centering
    \includegraphics[width=0.75\textwidth]{topology_muon.png}
    \caption{\textbf{Muon Topology: Cosserat Torsional Excitation.} (Simulation Output). The muon is the $0_1$ unknot absorbing one quantum of rotational (torsional) coupling, producing a visibly rippled flux tube. The torsional mode wavelength is set by $\alpha\sqrt{3/7}$, the PAT projection of the Cosserat coupling.}
    \label{fig:topology_muon}
\end{figure}

\subsection{Generation 3: Curvature-Twist (Bending Stiffness $\gamma_C$)}

The tau is the unknot promoted to the full bending energy scale:
\begin{equation}
    \boxed{m_\tau = m_e \cdot \frac{p_c}{\alpha^2} = \frac{8\pi m_e}{\alpha} \approx 1{,}760 \text{ MeV}} \quad (\text{Exp: } 1{,}776.9 \text{ MeV, } -0.95\%)
\end{equation}

This is the maximum excitation before packing saturates. The hierarchy of Cosserat sectors yields exactly three generations:
\begin{equation}
    m_e \xrightarrow{\alpha\sqrt{3/7}} m_\mu \xrightarrow{\alpha \cdot p_c} m_\tau \xrightarrow{\alpha \cdot p_c} M_W
\end{equation}

\begin{figure}[h]
    \centering
    \includegraphics[width=0.75\textwidth]{topology_tau.png}
    \caption{\textbf{Tau Topology: Full Curvature-Twist Excitation.} (Simulation Output). The tau is the $0_1$ unknot promoted to the maximum bending stiffness sector ($\gamma_C$). The intense curvature ripple reflects the $p_c/\alpha^2 = 8\pi/\alpha$ mass amplification, the highest excitation before packing saturates.}
    \label{fig:topology_tau}
\end{figure}

\begin{center}
\begin{tabular}{|l|c|c|c|c|}
\hline
Particle & AVE Formula & Predicted & Experiment & Deviation \\ \hline
$e$    & $m_e$                            & 0.511 MeV   & 0.511 MeV   & Input  \\
$\mu$  & $m_e/(\alpha\sqrt{3/7})$         & 107.0 MeV   & 105.66 MeV  & $+1.24\%$ \\
$\tau$ & $m_e \cdot p_c/\alpha^2$         & 1,760 MeV   & 1,776.9 MeV & $-0.95\%$ \\
$W$    & $m_e/(\alpha^2 p_c \sqrt{3/7})$  & 79,923 MeV  & 80,379 MeV  & $-0.57\%$ \\
$Z$    & $M_W \cdot 3/\sqrt{7}$           & 90,624 MeV  & 91,188 MeV  & $-0.62\%$ \\ \hline
\end{tabular}
\end{center}

\section{The Neutrino Mass Spectrum}

The neutrino is a pure torsional (screw) defect---a propagating twist wave in the Cosserat sector. Its mass is set by the ratio of torsional to translational coupling, multiplied by the dielectric compliance:
\begin{equation}
    m_\nu = m_e \cdot \alpha \cdot \frac{m_e}{M_W}
\end{equation}

\textbf{Physical meaning:} $m_e/M_W$ is the ratio of translational to torsional energy scale, and $\alpha$ is the dielectric coupling between sectors. Together, the neutrino mass is suppressed by $\alpha \times (m_e/M_W)$ relative to the electron. Evaluating:
\begin{equation}
    m_\nu \approx 0.024 \text{ eV per flavor}
\end{equation}

\subsection{Flavor Splitting via the Torus Knot Ladder}

Three neutrino flavors arise from the torus knot ladder: each flavor pairs with a baryon resonance via the crossing number. The mass splitting scales as $1/c$ where $c$ is the crossing number:
\begin{center}
\begin{tabular}{|c|c|c|c|}
\hline
Flavor & Baryon Partner & Crossing $c$ & Mass (meV) \\ \hline
$\nu_1$ & Proton $(2,5)$  & 5 & $\sim 24$ \\
$\nu_2$ & $\Delta(1232)$ $(2,7)$ & 7 & $\sim 17$ \\
$\nu_3$ & $\Delta(1620)$ $(2,9)$ & 9 & $\sim 13$ \\ \hline
\multicolumn{3}{|l|}{$\sum m_\nu$} & $\sim 0.054$ eV \\ \hline
\end{tabular}
\end{center}

\begin{figure}[h]
    \centering
    \includegraphics[width=0.75\textwidth]{topology_neutrino.png}
    \caption{\textbf{Neutrino Topology: Pure Torsional Screw Defect.} (Simulation Output). The neutrino is a dispersive twisted unknot---a propagating helical wave in the Cosserat (torsional) sector of the lattice. Unlike the closed-loop leptons, the neutrino's topology is an open helix, reflecting its near-zero mass and weak-only coupling. Its mass is suppressed by $\alpha \times (m_e/M_W)$ relative to the electron.}
    \label{fig:topology_neutrino}
\end{figure}

\noindent\textbf{Comparison:} The Planck 2018 cosmological bound is $\sum m_\nu < 0.12 \text{ eV}$, with hints at $\sim 0.06 \text{ eV}$. The AVE prediction of $0.054 \text{ eV}$ sits comfortably within this window.

\section{Schwinger's Anomalous Magnetic Moment ($g-2$)}

The anomalous magnetic moment of the electron is derived from the on-site impedance correction of the hopping unknot.

When the unknot visits a lattice node, all $m_e c^2$ is stored in that cell as EM field energy, split equally between E and B:
\begin{equation}
    U_E = \frac{1}{2}\epsilon_0 E_{\text{peak}}^2 \ell^3 = \frac{m_e c^2}{2}
\end{equation}

Solving for the peak electric strain:
\begin{equation}
    \left(\frac{V_{\text{peak}}}{V_{\text{snap}}}\right)^2 = 4\pi\alpha \quad [\text{EXACT}]
\end{equation}

This is an identity: $\alpha$ \textit{is} the on-site electric strain. The Axiom 4 nonlinear dielectric modifies the node capacitance:
\begin{equation}
    \epsilon_{\text{eff}} = \epsilon_0 \sqrt{1 - (V/V_s)^2}
\end{equation}

Time-averaged over the LC oscillation ($\langle\sin^2\rangle = 1/2$):
\begin{equation}
    \langle\delta C / C\rangle = \langle\delta\epsilon / \epsilon\rangle = -\pi\alpha
\end{equation}

This shifts the LC resonance frequency by $\delta\omega/\omega = \pi\alpha/2$. The anomalous magnetic moment is the fraction of this correction that falls within the ring's topological domain (the form factor). The ring has diameter $2R = \ell/\pi$ (from $R = \ell/(2\pi)$, Axiom 1). Its effective cross-section in the cell face is:
\begin{equation}
    F = \frac{A_{\text{ring}}}{A_{\text{cell}}} = \frac{(2R)^2}{\ell^2} = \frac{1}{\pi^2}
\end{equation}

The full on-site correction $\pi\alpha/2$ decomposes:
\begin{equation}
    a_e = \frac{1}{\pi^2} \times \frac{\pi\alpha}{2} = \boxed{\frac{\alpha}{2\pi}} \approx 0.001161
\end{equation}

\textbf{This is Schwinger's result (1948).} The AVE framework derives it from three structural constants: the Axiom 4 squared saturation operator, the unknot ropelength, and the lattice pitch. No Feynman diagrams or renormalization are required.

\section{Summary of Electroweak Predictions}

\begin{center}
\begin{tabular}{|l|c|c|c|}
\hline
Quantity & AVE Prediction & Experiment & Deviation \\ \hline
$\sin^2\theta_W$ (on-shell) & 0.2222 & 0.2230 & $-0.35\%$ \\
$M_W$ & 79,923 MeV & 80,379 MeV & $-0.57\%$ \\
$M_Z$ & 90,624 MeV & 91,188 MeV & $-0.62\%$ \\
$m_\mu$ & 107.0 MeV & 105.66 MeV & $+1.24\%$ \\
$m_\tau$ & 1,760 MeV & 1,776.9 MeV & $-0.95\%$ \\
$\sum m_\nu$ & $\sim 0.054$ eV & $<0.12$ eV & Within bound \\
$a_e$ (Schwinger) & $\alpha/(2\pi) = 0.001161$ & 0.001160 & $+0.09\%$ \\ \hline
\end{tabular}
\end{center}

Every entry in this table is computed from the same three calibration inputs ($m_e$, $\alpha$, $G$) plus the Poisson ratio $\nu_{vac} = 2/7$ and the packing fraction $p_c = 8\pi\alpha$. No Standard Model parameters (Yukawa couplings, CKM matrix elements, or Higgs VEV) are used.

\begin{figure}[h]
    \centering
    \includegraphics[width=1.0\textwidth]{topology_particle_zoo.png}
    \caption{\textbf{The AVE Particle Zoo.} (Simulation Output). Complete topological classification of all fundamental fermions and baryons derived in the AVE framework. Top row: the three charged leptons as Cosserat excitations of the $0_1$ unknot (electron, muon, tau) and the neutrino as a pure torsional screw defect. Bottom row: the proton as a $6^3_2$ Borromean linkage and the neutron as a Borromean link with a threaded unknot. Every mass is computed from the same three calibration inputs ($m_e$, $\alpha$, $G$) with zero Standard Model parameters.}
    \label{fig:particle_zoo}
\end{figure}
