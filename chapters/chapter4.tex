\chapter{The Topological Layer: Matter as Defects}

\section{Introduction: The Periodic Table of Knots}
Standard physics treats particles as point-like excitations of a quantum field. LCT proposes that fundamental particles are stable **Topological Defects** (Vortices) in the vacuum order parameter. Just as a knot in a rope cannot be untied without cutting the rope, a particle cannot decay unless it interacts with an anti-particle to unwind its topology.

\section{Vortices as Charge}
In Chapter 2, we identified Mass as Bandwidth Saturation. Here, we identify Charge as **Phase Winding** (Topological Twist).
The phase $\theta$ of the vacuum wavefunction $\psi = |\psi|e^{i\theta}$ winds around a singularity:
\begin{equation}
\oint \nabla \theta \cdot dl = 2\pi n
\end{equation}
Where $n$ is the integer charge quantum number.
\begin{itemize}
    \item **Positive Charge ($n=+1$):** A $360^\circ$ Clockwise Phase Winding (Vortex).
    \item **Negative Charge ($n=-1$):** A $360^\circ$ Counter-Clockwise Phase Winding (Anti-Vortex).
\end{itemize}



\section{The Proton as a Molecule}
We propose that Baryons (Protons/Neutrons) are not elementary, but **Topological Molecules**. A Proton is modeled as a stable triplet of vortices (Quarks) bound by the vacuum tension.
\textbf{The Strong Force:} This is simply the elastic tension of the lattice trying to unwind the shared phase field between the vortices.
\textbf{Computational Verification:} Our simulations demonstrate that three co-rotating vortices self-assemble into a stable triangular geometry. The "Gluon Field" is visible as the strained phase sheet connecting the cores.

\section{Computational Module: The Proton Simulation}
We initialized three vortices with $+1$ winding number in a triangular configuration and allowed the system to relax via the Ginzburg-Landau equation.
\begin{itemize}
    \item **Result:** The vortices did not merge or fly apart. They locked into a stable equilibrium distance determined by the balance of repulsive rotation and attractive lattice tension.
    \item **Interpretation:** The Proton is a "bound state" of vacuum defects.
\end{itemize}
*(See Appendix B.3 for the full Python source code.)*

\section{Bridge the Gap: From Standard Model to Topology}
To the Particle Physicist, a Proton is $uud$ quarks + gluons. To the Topologist, a Proton is a **Trefoil Knot**.
\begin{itemize}
    \item **Quarks:** The individual loops of the knot.
    \item **Gluons:** The crossing points where the loops interact.
    \item **Decay:** Only possible if the knot is cut by an Anti-Knot (Anti-Proton).
\end{itemize}