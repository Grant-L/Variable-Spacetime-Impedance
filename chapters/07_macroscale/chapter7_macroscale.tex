\chapter{Observational Signatures: Superfluid Vorticism and Phase Transitions}
\label{ch:macroscale}

\section{Introduction: The Crisis of the Continuum}
Modern cosmology faces two primary anomalies that threaten the validity of General Relativity: the nature of Dark Matter and the "Hubble Tension." LCT proposes that these are not mysteries of invisible particles, but macroscale signatures of the vacuum's superfluid hardware.

\section{Dark Matter: The Superfluid Vortex Lattice}
LCT identifies the "Dark Matter Halo" as a region of \textbf{Quantum Turbulence} in the superfluid vacuum. Unlike a classical gas, a superfluid cannot rotate as a rigid body; instead, it partitions rotation into a quantized \textbf{Vortex Lattice} (Abrikosov lattice).



\begin{itemize}
    \item \textbf{Kinetic Stiffness}: The additional gravitational "pull" attributed to Dark Matter is actually the kinetic energy density of this vacuum vortex lattice.
    \item \textbf{Viscous Coupling}: As a galaxy rotates, it drags the local vacuum substrate. This shear stress is relieved by the formation of microscopic vortex filaments.
\end{itemize}

\section{Explaining Flat Rotation Curves}
A fundamental signature of spiral galaxies is that stars at the edge rotate as fast as stars near the center, violating Newtonian expectations. In LCT, the constant rotational velocity $v_{\text{rot}}$ emerges from the uniform distribution of quantized vortices in the galactic vacuum:

\begin{equation}
v_{\text{rot}} \approx \frac{\hbar}{m} \sqrt{2\pi n_{v}(r)}
\end{equation}

Where $n_v(r)$ is the area density of the vortex filaments. This "Vacuum Stiffness" prevents the orbital drop-off, providing a hardware-level explanation for the observed dynamics.



\section{Numerical Verification: Galactic Rotation}
To validate this, we augment the standard Newtonian model with the LCT lattice-vorticity term.

\begin{simbox}[Vacuum Stiffness and Galactic Rotation]
As verified in \texttt{sim\_7\_galactic\_rotation.py}, the addition of the vacuum vortex lattice term ($k_{\text{lattice}}$) perfectly corrects the Newtonian decay. This simulation matches observed data for spiral galaxies, proving that the vacuum provides the necessary rotational "stiffness" without the need for additional mass-particles.
\begin{center}
    \includegraphics[width=0.8\textwidth]{assets/sim_outputs/galactic_rotation_result.png}
\end{center}
\end{simbox}

\section{The Hubble Tension: Late-Time Phase Transitions}
The discrepancy between early-universe and late-universe measurements of the Hubble constant ($H_0$) is resolved in LCT as a \textbf{Redshift-Dependent Phase Transition}.
\begin{itemize}
    \item \textbf{The Quench Step}: At approximately $z \approx 10$, the vacuum underwent a final crystallization step.
    \item \textbf{Latent Heat Pulse}: This transition released "latent heat" into the metric, manifesting as a sudden boost in expansion pressure (Dark Energy).
\end{itemize}
This explains why "Early" $H_0$ (from the CMB) and "Late" $H_0$ (from Supernovae) do not match: the hardware changed states between the two measurements.

\section{The Bullet Cluster: Decoupling Proof}
The Bullet Cluster represents a "smoking gun" for LCT. When two clusters collide, the visible gas slows down due to friction, but the gravitational lensing (the vacuum lattice) continues moving with the stars. 
\begin{itemize}
    \item \textbf{LCT Explanation}: The superfluid vortex lattice is frictionless and non-interacting with baryonic gas. It "decouples" from the matter during the collision, proving that gravity (the lattice state) is an independent physical entity.
\end{itemize}



\section{Exercises}
\begin{problembox}[Macroscale Observational Proofs]
\begin{enumerate}
    \item \textbf{Vortex Area Density}: Calculate the required $n_v$ for a galaxy with $v_{\text{rot}} = 200$ km/s.
    \item \textbf{Energy Density}: Prove that the kinetic energy density of an Abrikosov lattice scales with $1/r^2$, matching the required density profile of a Dark Matter halo.
    \item \textbf{Redshift Bias}: Calculate the expected shift in $H_0$ given a latent heat release of $10^{-9}$ J/m$^3$ at $z=10$.
\end{enumerate}
\end{problembox}

\section{Transition to Vacuum Engineering}
We have identified the macroscale signatures of the hardware substrate. In the final chapter, \textbf{Chapter 8: Engineering the Vacuum}, we move from passive observation to active manipulation of the lattice impedance for propulsion and energy extraction.