\chapter[Macroscale Dynamics]{Macroscale Dynamics: Galactic Rotation and the Radial Impedance Gradient}
\label{ch:macroscale}

\section{Introduction: The Dark Matter Fallacy}
Standard cosmology invokes ``Dark Matter''—an undetected, non-baryonic substance—to explain why the outer rims of galaxies rotate faster than Newtonian mechanics allow. The \textbf{Stochastic Vacuum Framework (SVF)} proposes a mechanical alternative: the \textbf{Radial Impedance Gradient}. Galaxies are not just collections of stars in a void; they are high-flux loads on a physical transmission medium.

\section{The Mass-Loading of the Manifold}
The \textit{Discrete Amorphous Manifold} ($M_A$) behaves as a non-linear transmission medium. Just as a heavy electrical load on a power grid causes localized voltage shifts and phase delays, a high concentration of baryonic mass (a ``Topological Load'') in a galactic core induces a localized decrease in the \textbf{Dynamic Metric Impedance} ($Z_{metric}$).



\subsection{The Radial Impedance Function}
We model the impedance of the vacuum as a function of the local mass density $\rho(r)$ and its resulting metric strain $\epsilon$:

\begin{equation}
    Z(r) = \Zvac \exp\left(-\frac{\Phi_{total}}{r \cdot \Zvac}\right)
\end{equation}

Where:
\begin{itemize}
    \item $\Zvac \approx 376.73 \, \Omega$ is the ground-state impedance of deep space.
    \item $\Phi_{total}$ is the total flux displacement (baryonic mass) of the galactic core.
    \item $r$ is the radial distance from the center of mass.
\end{itemize}

As one moves toward the galactic rim, the mass-loading effect diminishes, causing the vacuum to "stiffen" or return to its high-impedance ground state. This return to $\Zvac$ increases the lattice's elastic resistance to rotation, naturally flattening the velocity curves without requiring auxiliary particles.

\section{Falsification: The Bullet Cluster and Lattice Memory}
A primary ``Means Test'' for the SVF framework is the observation of the \textbf{Bullet Cluster}, where gravitational lensing appears offset from visible gas. 

\begin{itemize}
    \item \textbf{SVF Prediction}: The ``Impedance Wake'' left behind by colliding galaxies persists in the $M_A$ manifold as \textbf{Lattice Memory} (Phase Lag). Because the nodes possess a finite update frequency (the \textbf{Slew Rate}), the metric strain $\epsilon$ does not vanish instantly when the baryonic mass moves. 
    \item \textbf{Failure Condition}: If the gravitational lensing signal is found to be instantaneous or perfectly coupled to the center of mass of the gas (baryons) during the collision, the SVF ``Impedance Wake'' theory is false.
\end{itemize}



\section{The Tully-Fisher Relation as an Impedance Law}
The observed relationship between a galaxy's luminosity and its rotation velocity is derived here as a consequence of the manifold’s saturation limit. In SVF, the \textbf{Tully-Fisher Relation} is the macroscale equivalent of the **Saturation Threshold** ($\Wcut$) established in Chapter 2, where the total mass-load dictates the maximum rotational velocity the local lattice can support.

\section{Exercises}
\begin{problembox}[Chapter 7 Macroscale Challenges]
\begin{enumerate}
    \item \textbf{Rotation Flattening}: Given a galactic core mass $M$, find the radius $r$ where the radial impedance gradient exactly balances the Newtonian $1/r^2$ decay to produce a flat velocity profile.
    \item \textbf{Lattice Relaxation}: Calculate the time $\tau$ required for a region of metric strain to return to $\Zvac$ after a mass displacement, using the Quench Constant $\gamma$ from Chapter 6.
    \item \textbf{Refractive Lensing}: Show that the bending of light in a radial impedance gradient $Z(r)$ recovers the Einstein deflection angle $\theta = 4GM/rc^2$ without invoking four-dimensional geometric curvature.
\end{enumerate}
\end{problembox}