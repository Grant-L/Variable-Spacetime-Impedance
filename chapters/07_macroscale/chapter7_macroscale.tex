\chapter{The Galactic Layer: Dark Matter as Vacuum Stiffness}
\label{ch:galactic_layer}

\section{Introduction: The Rotation Problem}
Observations show that stars at the edge of galaxies orbit just as fast as those near the center, defying Keplerian mechanics. Standard models patch this by adding a "Dark Matter" halo. \LCT{} removes this ad hoc fix by modeling the galaxy as a vortex in a superfluid condensate.

\section{Numerical Verification: The Abrikosov Lattice}
To validate this, we model the galaxy as a vortex in a superfluid condensate using \texttt{sim\_7\_galactic\_rotation\_v2.py}.

\begin{simbox}[Verification of Vacuum Stiffness]
Figure 7.1 presents a dual verification of the Galactic Layer.
\begin{itemize}
    \item \textbf{Left Pane:} The LCT prediction (Green) corrects the Newtonian decay (Red) by adding the elastic tension of the vacuum, perfectly matching the flat rotation curves of observed galaxies.
    \item \textbf{Right Pane:} The \textbf{Vortex Density} ($n_v$) profile reveals the mechanical origin of the "Dark Halo." The density of vacuum defects scales as $1/r$, creating a stiffness gradient that mimics the gravitational pull of an isothermal mass distribution.
\end{itemize}
\begin{center}
    \includegraphics[width=1.0\textwidth]{assets/sim_outputs/galactic_rotation_v2.png}
\end{center}
\end{simbox}

\section{Numerical Verification: Galactic Rotation Curves}
We model the galactic disk as a fluid coupled to an elastic background lattice.

\begin{simbox}[Verification of Vacuum Stiffness]
As verified in \texttt{sim\_7\_galactic\_rotation.py}, adding the stiffness term $k \cdot r$ to the Newtonian prediction perfectly recovers the flat rotation curves observed in spiral galaxies.
\begin{center}
    \includegraphics[width=0.9\textwidth]{assets/sim_outputs/galactic_rotation.png}
\end{center}
The simulation proves that $v(r)$ flattens at large $r$ not due to hidden mass, but due to the linear increase in vacuum tension.
\end{simbox}

\section{The Tully-Fisher Relation}
The relationship between a galaxy's mass and its rotation speed ($L \propto v^4$) emerges naturally from the hardware physics. A more massive galaxy creates a deeper strain in the lattice, increasing the density of Abrikosov vortices and thus the stiffness of the vacuum background.