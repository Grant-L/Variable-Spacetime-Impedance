% --- Chapter 5: The Thermodynamic Vacuum and Decoherence ---

In the previous chapters, we established the lattice as a transmission line (Chapter 2) and a quantum pilot wave medium (Chapter 3). However, a critical boundary remains undefined: the transition between the Quantum (Laminar) and Classical (Turbulent) domains.

This chapter proposes that "Classicality" is not a fundamental state of matter, but a regime of \textbf{High Vacuum Turbulence}. We introduce the \textbf{Vacuum Reynolds Number ($Re_{vac}$)} and demonstrate that the "Collapse of the Wavefunction" is simply the scrambling of the Pilot Wave by local phase noise.

\section{The Signal-to-Noise Ratio of Reality}
We define the stability of the vacuum flow using the \textbf{Vacuum Reynolds Number}:

\begin{equation}
Re_{vac} = \frac{\rho \cdot v \cdot L}{\mu_{vac}}
\end{equation}

\begin{itemize}
    \item \textbf{Low $Re_{vac}$ (Laminar):} The pilot wave propagates without distortion. The system behaves "Quantumly."
    \item \textbf{High $Re_{vac}$ (Turbulent):} The background noise level exceeds the amplitude of the Pilot Wave. The system "Decoheres" into a Classical trajectory.
\end{itemize}

\section{Computational Module: Gravitational Decoherence}
We propose that an Event Horizon is not a geometric singularity, but a \textbf{Thermodynamic Phase Transition} (Lattice Liquefaction). As a quantum signal approaches the horizon, the increasing turbulence of the lattice scrambles the phase information.

\begin{lstlisting}[language=Python, caption=Simulating Decoherence at the Event Horizon]
import numpy as np
import matplotlib.pyplot as plt

def gen_decoherence():
    x = np.linspace(-10, 10, 500)
    y = np.linspace(-10, 10, 500)
    X, Y = np.meshgrid(x, y)
    R = np.sqrt(X**2 + Y**2)
    
    # Interference Pattern (Quantum Signal)
    k = 2.0
    psi = np.sin(k * (X + 2*Y)) + np.sin(k * (X - 2*Y))
    
    # Horizon Scrambling (Thermodynamic Noise)
    # Noise increases as R -> 0 (Event Horizon)
    noise_mask = 1.0 / (R + 0.5)
    # Scramble the signal near the horizon
    scrambled = psi * (1 - np.exp(-R/3)) + np.random.normal(0, 2, X.shape) * np.exp(-R/2)
    
    plt.figure(figsize=(6, 5))
    plt.imshow(scrambled, extent=[-10, 10, -10, 10], cmap='magma', origin='lower')
    plt.title("Gravitational Decoherence at the Horizon")
    
    # Draw Black Hole
    circle = plt.Circle((0, 0), 2, color='black')
    plt.gca().add_patch(circle)
    plt.axis('off')
    plt.savefig('gravitational_double_slit.png', dpi=300)

if __name__ == "__main__":
    gen_decoherence()
\end{lstlisting}

\begin{figure}[h]
    \centering
    \includegraphics[width=0.8\textwidth]{gravitational_double_slit.png}
    \caption{\textbf{Gravitational Decoherence.} Simulation results showing the evolution of a quantum state near an event horizon. As the signal approaches the "Turbulence Zone" of the horizon (center), the coherent interference fringes are scrambled into thermodynamic noise.}
\end{figure}

\section{Thermodynamic Scrambling (The Information Paradox)}
Standard black hole theory struggles with the loss of information at the singularity. In LCT, the singularity does not exist. Instead, matter falling into the horizon is dissolved into the superfluid core. This process preserves \textbf{Unitarity}. The information is not destroyed, but is \textbf{scrambled} into the thermal degrees of freedom of the superfluid.

\section{Exercises}

\begin{enumerate}
    \item \textbf{The Vacuum Reynolds Number.}
    We defined the Vacuum Reynolds Number as $Re_{vac} = \frac{\rho v L}{\mu_{vac}}$.
    Using dimensional analysis, if the "viscosity" of the vacuum $\mu_{vac}$ is proportional to Planck's constant $\hbar$, show that the transition from Laminar (Quantum) to Turbulent (Classical) occurs when the action $S$ of the system exceeds $\hbar$.
    
    \item \textbf{Entropic Gravity.}
    If a black hole is a region of maximum lattice entropy, derive the Bekenstein-Hawking entropy formula $S_{BH} = \frac{k_B A}{4 \ell_P^2}$ by counting the number of "lattice nodes" on the surface area $A$, assuming one bit of information per Planck area $\ell_P^2$.
    
    \item \textbf{Computational: The Fringe Visibility.}
    Open \texttt{sim\_f\_grav\_decoherence.py}. The code currently adds noise based on distance $R$.
    Add a calculation to measure the \textit{Visibility} $V$ of the interference fringes:
    \begin{equation*}
        V = \frac{I_{max} - I_{min}}{I_{max} + I_{min}}
    \end{equation*}
    Plot $V$ as a function of the noise magnitude. At what "Temperature" does the quantum signal become indistinguishable from classical noise?
\end{enumerate}

\section*{Bridge the Gap: Multidisciplinary Links}
\begin{itemize}
    \item \textbf{For the Physicist:} The horizon is a \textbf{Critical Point} in a phase diagram. Hawking Radiation is simply the thermal evaporation of the superfluid surface.
    \item \textbf{For the Engineer:} This is \textbf{Shannon Entropy}. A Black Hole is a maximum-entropy channel where the signal-to-noise ratio drops to zero.
\end{itemize}