% --- Chapter 5: The Thermodynamic Vacuum and Decoherence ---

In the previous chapters, we established the lattice as a transmission line (Chapter 2) and a quantum pilot wave medium (Chapter 3). However, a critical boundary remains undefined: the transition between the Quantum (Laminar) and Classical (Turbulent) domains.

This chapter proposes that "Classicality" is not a fundamental state of matter, but a regime of \textbf{High Vacuum Turbulence}. We introduce the \textbf{Vacuum Reynolds Number ($Re_{vac}$)} and demonstrate that the "Collapse of the Wavefunction" is simply the scrambling of the Pilot Wave by local phase noise.

\section{The Signal-to-Noise Ratio of Reality}
We define the stability of the vacuum flow using the \textbf{Vacuum Reynolds Number}:

\begin{equation}
Re_{vac} = \frac{\rho \cdot v \cdot L}{\mu_{vac}}
\end{equation}

\begin{itemize}
    \item \textbf{Low $Re_{vac}$ (Laminar):} The pilot wave propagates without distortion. The system behaves "Quantumly."
    \item \textbf{High $Re_{vac}$ (Turbulent):} The background noise level exceeds the amplitude of the Pilot Wave. The system "Decoheres" into a Classical trajectory.
\end{itemize}

\section{Computational Module: Gravitational Decoherence}
We propose that an Event Horizon is not a geometric singularity, but a \textbf{Thermodynamic Phase Transition} (Lattice Liquefaction). As a quantum signal approaches the horizon, the increasing turbulence of the lattice scrambles the phase information.

\begin{lstlisting}[language=Python, caption=Simulating Decoherence at the Event Horizon]
import numpy as np
import matplotlib.pyplot as plt

def gen_decoherence():
    x = np.linspace(-10, 10, 500)
    y = np.linspace(-10, 10, 500)
    X, Y = np.meshgrid(x, y)
    R = np.sqrt(X**2 + Y**2)
    
    # Interference Pattern (Quantum Signal)
    k = 2.0
    psi = np.sin(k * (X + 2*Y)) + np.sin(k * (X - 2*Y))
    
    # Horizon Scrambling (Thermodynamic Noise)
    # Noise increases as R -> 0 (Event Horizon)
    noise_mask = 1.0 / (R + 0.5)
    # Scramble the signal near the horizon
    scrambled = psi * (1 - np.exp(-R/3)) + np.random.normal(0, 2, X.shape) * np.exp(-R/2)
    
    plt.figure(figsize=(6, 5))
    plt.imshow(scrambled, extent=[-10, 10, -10, 10], cmap='magma', origin='lower')
    plt.title("Gravitational Decoherence at the Horizon")
    
    # Draw Black Hole
    circle = plt.Circle((0, 0), 2, color='black')
    plt.gca().add_patch(circle)
    plt.axis('off')
    plt.savefig('gravitational_double_slit.png', dpi=300)

if __name__ == "__main__":
    gen_decoherence()
\end{lstlisting}

\begin{figure}[h]
    \centering
    \includegraphics[width=0.8\textwidth]{gravitational_double_slit.png}
    \caption{\textbf{Gravitational Decoherence.} Simulation results showing the evolution of a quantum state near an event horizon