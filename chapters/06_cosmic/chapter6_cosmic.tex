\chapter{The Cosmic Layer: Genesis and Phase Transitions}
\label{ch:cosmic_layer}

\section{Introduction: The Big Bang as a Global Quench}
Standard cosmology models the Big Bang as an expansion from a singular point of infinite density. LCT proposes a mechanical alternative: the \textbf{Global Quench}. The early universe was a high-temperature, disordered phase fluid that underwent a rapid transition into the ordered ground state of the LC lattice. 



\section{The Kibble-Zurek Mechanism}
As the vacuum crystallized, independent "domains" of the lattice formed with mismatched phase orientations. Where these domains met, the resulting phase mismatch created permanent topological "knots."
\begin{itemize}
    \item \textbf{Primordial Scars}: Fundamental particles are the "cracks" trapped in the freezing vacuum.
    \item \textbf{Defect Density}: The total amount of matter in the universe is a direct function of the \textbf{Quench Rate}---the speed at which the vacuum cooled.
\end{itemize}

\section{Numerical Verification: Spontaneous Matter Creation}
We use the time-dependent Ginzburg-Landau (TDGL) equation to simulate the "freezing" of a disordered vacuum into an ordered state.

\begin{simbox}[The Cosmic Quench]
As verified in \texttt{sim\_6\_cosmic\_quench.py}, starting from a state of total phase randomness (high temperature), the vacuum substrate naturally relaxes into ordered domains. At the junctions where phase orientations conflict, topological vortices (matter) are spontaneously trapped.
\begin{center}
    \includegraphics[width=0.8\textwidth]{assets/sim_outputs/cosmic_quench_result.png}
\end{center}
The resulting distribution of vortices represents the primordial matter density of the LCT universe.
\end{simbox}

\section{Dark Energy as Latent Heat}
In LCT, the accelerated expansion of the universe (Dark Energy) is not a "lambda" constant, but the \textbf{Latent Heat} of the vacuum phase transition. As the lattice settles into its final ordered state, the energy released from the "freezing" process exerts an outward pressure on the metric.



\section{The Late-Time Phase Transition}
The "Hubble Tension"---the discrepancy in the measured expansion rate of the universe---is resolved in LCT as a signature of a \textbf{Late-Time Phase Transition}. At approximately $z \approx 10$, the vacuum underwent a final crystallization step, releasing a pulse of energy that boosted the local expansion rate.

\section{Exercises}
\begin{problembox}[Cosmic Layer Challenges]
\begin{enumerate}
    \item \textbf{Quench Rate Calculation}: Using the defect density from \texttt{sim\_6}, calculate the required cooling rate ($dT/dt$) to produce the observed baryon-to-photon ratio.
    \item \textbf{Domain Boundary Tension}: Model a "Domain Wall" as a region of infinite metric strain $\epsilon$. Prove that signals (photons) are reflected by these boundaries, creating the observed "CMB Cold Spots."
    \item \textbf{Latent Heat Pressure}: Derive the effective "Dark Energy" pressure $P$ as a function of the lattice condensation energy $U_{cond}$.
\end{enumerate}
\end{problembox}

\section{Transition to Observational Signatures}
With the origin of matter established as a byproduct of vacuum crystallization, we move to \textbf{Chapter 7: Observational Signatures}. Here, we verify the macroscale consequences of our superfluid vacuum, solving the mystery of \textbf{Dark Matter} as a quantized vortex lattice.