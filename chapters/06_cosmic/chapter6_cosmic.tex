\chapter{The Cosmic Layer: Genesis and Phase Transitions}
\label{ch:cosmic_layer}

\section{Introduction: The Big Bang as a Global Quench}
Standard cosmology models the Big Bang as an expansion from a singular point. \LCT{} replaces this with the \textbf{Global Quench}. The early universe was a high-temperature, disordered fluid that underwent a rapid cooling phase, "freezing" into the \textbf{Amorphous Topological Glass} (the vacuum lattice) we observe today.

\section{The Kibble-Zurek Mechanism: Matter Genesis}
As the vacuum crystallized, independent "domains" of the lattice formed with mismatched phase orientations. Where these domains met, the resulting phase mismatch created permanent topological "knots."
\begin{itemize}
    \item \textbf{Primordial Scars:} Fundamental particles (Protons/Electrons) are the "cracks" trapped in the freezing vacuum structure.
    \item \textbf{Defect Density:} The total amount of matter is a direct function of the \textbf{Quench Rate} ($dT/dt$). Faster cooling traps more defects (matter), while slower cooling allows for a clearer, empty vacuum.
\end{itemize}

\section{Dark Energy as Latent Heat}
\label{sec:latent_heat}
The most significant prediction of LCT is the mechanical origin of **Dark Energy**. The "accelerated expansion" is not caused by a cosmological constant ($\Lambda$), but by the **Latent Heat** released during the solidification of the vacuum substrate.

\begin{axiombox}[Dark Energy as Work]
"Dark Energy" is the potential energy stored in the nodal capacitance $C$ that is converted into mechanical work (expansion) as the amorphous lattice releases its internal stress and reaches equilibrium connectivity $\langle k \rangle \approx 15.54$.
\end{axiombox}

\section{Numerical Verification: The Expansion Pulse}
We verify this by modeling the universe's expansion history $H(t)$ as a function of matter density $\rho_m$ and the latent heat flux $\rho_{latent}$.

\begin{simbox}[Verification of the Hubble Pulse]
As verified in \texttt{sim\_6\_cosmological\_expansion.py}, the expansion curve in Figure 6.1 displays a distinct "pulse" corresponding to the vacuum's phase transition.
\begin{center}
    \includegraphics[width=0.9\textwidth]{assets/sim_outputs/cosmological_expansion.png}
\end{center}
This resolves the \textbf{Hubble Tension}: early-universe measurements (CMB) capture the pre-transition rate, while late-universe measurements (Supernovae) capture the post-quench acceleration driven by latent heat release.
\end{simbox}

\section{The Late-Time Phase Transition}
The simulation confirms that the vacuum is still "settling." At approximately $z \approx 10$, the lattice underwent a secondary crystallization step. This implies that the "Hubble Constant" is not constant, but a time-dependent variable $H(t)$ driven by the cooling rate of the hardware.

\section{Exercises}
\begin{problembox}[Cosmic Layer Challenges]
\begin{enumerate}
    \item \textbf{Quench Rate Calculation:} Using the standard defect density of the universe ($n \approx 0.25$ baryons/$m^3$), calculate the required cooling rate ($dT/dt$) of the LCT vacuum.
    \item \textbf{Latent Heat Pressure:} Derive the effective "Dark Energy" pressure $P$ as a function of the lattice condensation energy $U_{cond}$.
    \item \textbf{The Heat Death:} Prove that as $t \to \infty$, the latent heat $\rho_{latent} \to 0$, leading to a static, Euclidean lattice (Flat Space).
\end{enumerate}
\end{problembox}

\section{Transition to Observational Signatures}
With the origin of Dark Energy established as the cooling of the vacuum glass, we move to **Chapter 7: The Galactic Layer**. Here, we verify the macroscale consequences of our superfluid vacuum, solving the mystery of **Dark Matter** as the rotational stiffness of the Abrikosov Lattice.