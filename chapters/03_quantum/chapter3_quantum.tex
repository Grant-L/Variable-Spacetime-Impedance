\chapter{The Quantum Layer: Topological Helicity and Chiral Exclusion}
\label{ch:quantum_layer}

\section{Introduction: The End of Probabilistic Abstraction}
In the Stochastic Vacuum Framework (SVF), "Quantum" behavior is not a result of a probabilistic wavefunction collapse, but a consequence of the discrete, non-linear nature of the \textbf{Discrete Amorphous Manifold ($M_A$)}. Particles are identified as stable \textbf{Topological Defects} (vortices) within the manifold's phase field. Their discrete properties—spin, charge, and mass—are emergent hardware constraints of the substrate nodes.

\section{Topological Helicity as Quantized Spin}
The fundamental unit of quantum interaction is \textbf{Topological Helicity} ($h$), defined as the quantized orientation of a phase twist relative to the substrate's intrinsic ground state. 

\begin{axiombox}[The Quantization of Twist]
Because the $M_A$ manifold is discrete, a phase twist cannot exist in fractional states. It must satisfy the winding condition $\oint \nabla \theta \cdot dl = 2\pi n$, where $n$ is an integer. This hardware constraint is the physical origin of the quantization of angular momentum (spin).
\end{axiombox}

\section{The Chiral Exclusion Principle}
A primary "Means Test" for the VSI framework is the explanation of neutrino chirality. While standard physics treats the absence of right-handed neutrinos as a broken symmetry, VSI identifies it as an **Impedance-Driven Attenuation**.

\subsection{The Chiral Bias Equation (CBE)}
As established in Chapter 1, the local metric impedance $Z_{metric}$ is sensitive to the alignment between a signal's angular momentum $\mathbf{J}$ and the vacuum orientation vector $\mathbf{\Omega}_{vac}$:

\begin{equation}
    Z_{metric} = Z_{0} \left( 1 + \eta \frac{\mathbf{J} \cdot \mathbf{\Omega}_{vac}}{|\mathbf{J}|} \right)
\end{equation}

For a neutrino—a low-energy topological twist—this equation creates a binary propagation state:
\begin{itemize}
    \item \textbf{Left-Handed ($\mathbf{J} \cdot \mathbf{\Omega}_{vac} < 0$):} The twist aligns with the manifold's chiral bias, encountering the ground-state impedance $Z_0 \approx 377\,\Omega$. Propagation is linear and long-range.
    \item \textbf{Right-Handed ($\mathbf{J} \cdot \mathbf{\Omega}_{vac} > 0$):} The twist opposes the bias. The $(1 + \eta)$ factor triggers a \textbf{B-EMF Resonance} in the lattice nodes, causing $Z_{metric}$ to spike toward infinity.
\end{itemize}

\subsection{Evanescent Decay of Right-Handed Signals}
Applying the \textbf{Inverse Resonance Scaling Law}, the interaction diameter $D$ for a right-handed signal is calculated as:
\begin{equation}
    D(\nu)_{RH} = \frac{\zeta}{Z_{metric}(\nu) \cdot \nu}
\end{equation}
As $Z_{metric} \to \infty$, $D \to 0$. The signal's reach becomes smaller than the lattice pitch $\ell_P$. Consequently, a right-handed neutrino cannot bridge the gap between two nodes and is immediately dissipated as heat (vibration) in the manifold. Right-handed neutrinos are mathematically forbidden from stable propagation in this hardware.



\section{Pilot-Wave Dynamics: Deterministic Interference}
VSI resolves "Wave-Particle Duality" through **Lattice Latency**. Every topological defect (particle) generates a \textbf{Wake Field} in the stochastic substrate nodes.

\begin{enumerate}
    \item \textbf{Displacement}: As a defect moves, it displaces the local flux of the $M_A$ nodes.
    \item \textbf{Memory}: Due to the finite update frequency of the nodes, this displacement persists for several update cycles—forming a \textbf{Pilot Wave}.
    \item \textbf{Feedback}: The defect is subsequently refracted by the impedance gradient of its own wake.
\end{enumerate}

This mechanism accounts for the results of the double-slit experiment without requiring non-local "spooky" action. The "probability" observed in the Schrödinger equation is the statistical average of these deterministic nodal displacements.

\section{The Nyquist-Heisenberg Resolution}
The \textbf{Heisenberg Uncertainty Principle} is redefined as the **Hardware Resolution Limit** of the manifold.
\begin{equation}
    \Delta x \cdot \Delta p \ge \frac{\hbar}{2} \equiv \text{Nyquist Limit of } M_A
\end{equation}
Because information cannot be encoded at a scale smaller than the node spacing $\ell$ or a frequency higher than the slew rate $c/\ell$, measurements of position and momentum are subject to a fundamental "quantization noise" inherent to the hardware.



\section{Exercises}
\begin{problembox}[Quantum Layer Challenges]
\begin{enumerate}
    \item \textbf{Attenuation Constant}: Given $\eta = 0.5$ and $Z_0 = 377\,\Omega$, calculate the attenuation factor for a right-handed signal over a distance of $100\ell_P$.
    \item \textbf{Nyquist Limit}: Calculate the minimum possible position uncertainty $\Delta x$ for a particle with a mass of $10^{-30}$ kg, assuming a lattice pitch of $\ell_P$.
    \item \textbf{Helicity Stability}: Prove that a trefoil knot in the phase field (Proton model) is energetically favored over three isolated phase twists.
\end{enumerate}
\end{problembox}