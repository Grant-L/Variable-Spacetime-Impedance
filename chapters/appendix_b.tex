\chapter{B The Computational Verification Suite}

\section{B.1 Overview: The Numerical Foundation}
The Lindblom Coupling Theory (LCT) is verified through a suite of Python-based Finite-Difference Time-Domain (FDTD) and Ginzburg-Landau relaxation simulations. This appendix provides the technical documentation for the scripts found in the \texttt{simulations/} directory, ensuring reproducibility of the emergent phenomena described in Chapters 1–8[cite: 5, 488].

\section{B.2 Hardware and Signal Verification}

\subsection{B.2.1 Metric Strain and Geodesics (\texttt{sim\_a\_metric\_strain.py})}
This script validates the "Gravity as Metric Strain" postulate from Chapter 2[cite: 5, 132]. By locally modifying the distributed inductance $\Lvac$ and capacitance $\Cvac$ according to the Schwarzschild potential, the simulation demonstrates the refraction of wave packets.
\begin{itemize}
    \item \textbf{Postulate}: Light speed $v = 1/\sqrt{\Lvac'\Cvac'}$ drops near a mass[cite: 145].
    \item \textbf{Result}: Wavefronts exhibit gravitational lensing, matching General Relativity’s predictions through variable impedance rather than curved geometry[cite: 151].
\end{itemize}



\subsection{B.2.2 Dispersion and Group Velocity (\texttt{01\_Relativistic\_Limit.ipynb})}
This Jupyter notebook verifies the Lindblom Dispersion Relation (Eq. 2.11)[cite: 123].
\begin{itemize}
    \item \textbf{Validation}: Numerical results confirm that as signal frequency $\omega$ approaches the hardware cutoff $\Wcut$, the group velocity $v_g$ vanishes[cite: 127].
    \item \textbf{Significance}: This provides the numerical basis for mass as bandwidth saturation[cite: 129].
\end{itemize}

\section{B.3 Quantum and Topological Verification}

\subsection{B.3.1 The Hydrodynamic Pilot-Wave (\texttt{sim\_d\_born\_rule.py})}
This simulation supports Chapter 3 by modeling a particle as a bouncing soliton that generates a memory field in the vacuum lattice[cite: 191, 199].
\begin{itemize}
    \item \textbf{Mechanism}: The particle is guided by the gradient of the standing wave it creates[cite: 193].
    \item \textbf{Outcome}: The resulting probability distribution reproduces the Born Rule without requiring probabilistic collapse[cite: 5].
\end{itemize}



\subsection{B.3.2 Proton Triplet Assembly (\texttt{sim\_k\_proton\_triplet.py})}
This script uses Ginzburg-Landau relaxation to verify the "Proton as a Molecule" model from Chapter 4[cite: 254, 260].
\begin{itemize}
    \item \textbf{Procedure}: Three phase vortices (winding $n=1$) are initialized in proximity.
    \item \textbf{Result}: The lattice elastic tension forces the vortices into a stable triangular "Trefoil" configuration[cite: 258].
\end{itemize}

\section{B.4 Cosmic and Macroscale Verification}

\subsection{B.4.1 Galactic Rotation and Vortex Lattices (\texttt{sim\_l\_galactic\_rotation.py})}
This script validates the Dark Matter solution in Chapter 7[cite: 391, 404].
\begin{itemize}
    \item \textbf{Logic}: It adds a quantized vortex lattice term to a standard Newtonian rotation model[cite: 395].
    \item \textbf{Result}: The simulation produces a flat rotation curve that matches observed galactic data without the need for additional invisible particles[cite: 403].
\end{itemize}



\subsection{B.4.2 The Cosmic Quench (\texttt{sim\_b\_genesis.py})}
This simulation models the Big Bang as a vacuum phase transition (crystallization) as described in Chapter 5[cite: 347, 356].
\begin{itemize}
    \item \textbf{Observation}: As the "hot" disordered fluid cools, topological defects (matter) are spontaneously trapped at domain boundaries[cite: 353].
    \item \textbf{Cosmology}: This provides a mechanical origin for the observed matter density of the universe[cite: 355].
\end{itemize}

\section{B.5 Environment Setup and Requirements}
To run the LCT verification suite, ensure the following dependencies are installed via the \texttt{requirements.txt} file found in the root directory:
\begin{itemize}
    \item \texttt{numpy}: For high-performance numerical array operations[cite: 152].
    \item \texttt{matplotlib}: For generating the visual proofs and phase maps[cite: 262].
    \item \texttt{scipy}: For Ginzburg-Landau relaxation and integration.
\end{itemize}
Execute \texttt{setup.sh} to initialize the environment and link the \texttt{src/constants.py} file to the simulation modules.