\chapter{The Weak Layer: Chirality as a Filter}
\label{ch:weak_layer}

\section{Introduction: The Vacuum as a Polarized TVS}
Standard particle physics treats chirality as an abstract quantum number. LCT proposes that the vacuum acts as a non-linear, directional impedance filter, analogous to a specialized \textbf{Polarized Transient Voltage Suppressor (TVS)}. The "Weak Interaction" is identified as the mechanical response of the hardware lattice to topologically incompatible "screw" directions in the signal pulse.



\section{Helicity and Mechanical Impedance}
A propagating particle in LCT is a helical vortex pulse. As established in Chapter 2, this propagation induces a backlog of \textbf{Metric Strain}: the compression of nodes ahead and the stretching of nodes behind the wavefront. 



\subsection{Emergent Mixing: The Weinberg Angle}
The forward weak current proceeds at electromagnetic strength; the reverse is suppressed by the lattice's intrinsic bias. The effective mixing angle $\theta_W$ is derived from the lattice up-down mass difference ($V_b \approx 2.52~\text{MeV}$) and the nuclear excitation scale ($kT_{\text{eff}} \approx 1.8~\text{MeV}$):
\begin{equation}
    \sin^2 \theta_W \approx \exp\left( -\frac{V_b}{kT_{\text{eff}}} \right) \approx 0.231
\end{equation}
This derivation matches experimental observation without requiring the manual "tuning" of the Standard Model, revealing $\theta_W$ as a thermal-mechanical property of the vacuum substrate.

\section{The Impedance Clamping Equation}
We define the \textbf{Coupling Efficiency} of a propagating helix into the strained hardware lattice. The effective impedance ($Z_{\text{eff}}$) encountered by a vortex with winding $m$ and propagation vector $k$ is given by the \textbf{Impedance Clamping Equation}:

\begin{equation}
Z_{\text{eff}} = Z_0 \cdot e^{\sigma (m \cdot k)}
\end{equation}

Where:
\begin{itemize}
    \item \textbf{$Z_0$}: The baseline characteristic impedance of free space ($\approx 376.73\,\Omega$).
    \item \textbf{$\sigma$}: The local \textbf{Metric Strain Constant}.
    \item \textbf{$m \cdot k$}: The alignment of the vortex winding (chirality) with its direction of travel.
\end{itemize}

\section{Numerical Verification: Weak Clamping}
To prove this directional bias, we simulate the propagation of helical signals through a strained lattice segment.

\begin{simbox}[Weak Clamping and Chirality Filtering]
As verified in \texttt{sim\_5\_weak\_clamping.py}, the lattice update rate "clamps" incompatible helical signals. Left-handed pulses (low impedance) propagate with minimal loss, while right-handed pulses (high impedance) are reflected or attenuated, providing a hardware-level explanation for Parity Violation.
\begin{center}
    \includegraphics[width=0.8\textwidth]{assets/sim_outputs/weak_clamping_result.png}
\end{center}
\end{simbox}

\section{The Slew Rate Threshold}
The lattice update rate, defined by the hardware time constant, imposes a maximum rate of change for phase flux. If the "screw pitch" of a vortex exceeds this limit, the node fails to update, presenting an effectively infinite impedance:

\begin{equation}
\left| \frac{d\theta}{dt} \right| > \omega_{\text{cutoff}}
\end{equation}

This \textbf{Slew Rate Limit} "clamps" the signal, forcing right-handed neutrinos into evanescent, non-propagating modes that are reflected back into the vacuum substrate.

\section{Bridge to the Standard Model}
To the particle physicist, the Weak Force is mediated by bosons. To the LCT engineer, it is the \textbf{Automated Surge Protection} of the vacuum lattice.



\begin{itemize}
    \item \textbf{$W^{\pm}$ Bosons}: Localized lattice "breakdown" events (dielectric breakdown) that allow a change in winding number $n$.
    \item \textbf{$Z^0$ Boson}: A common-mode impedance spike that mediates neutral current interactions without altering the topology.
\end{itemize}

\section{Exercises}
\begin{problembox}[Weak Layer Challenges]
\begin{enumerate}
    \item \textbf{Impedance Gradient}: Graph $Z_{\text{eff}}$ for $\sigma$ values ranging from 0 to 1. Identify the point where the signal becomes purely reflective.
    \item \textbf{Neutrino Reflection}: Using the results from \texttt{sim\_5}, calculate the reflection coefficient $\Gamma$ for a right-handed configuration at a strain of 10\%.
    \item \textbf{Fermi Constant}: Relate the Fermi constant $G_F$ to the lattice spacing $\Delta x$ using the derived vev $v \sim 246~\text{GeV}$.
\end{enumerate}
\end{problembox}

\section{Transition to the Cosmic Layer}
With the forces now identified as hardware constraints, we move to the \textbf{Cosmic Layer} (Chapter 6) to analyze the **Genesis** of these topological defects during the global phase transition of the early universe.