\section{Appendix A: Electrodynamics (Hardware Derivation)}
We derive Maxwell's Equations not from abstract fields, but from the discrete energy balance of the LC network.

Consider the \textbf{Lagrangian Density} $\mathcal{L} = T - U$ for a 3D LC lattice, representing the difference between the Kinetic (Capacitive) and Potential (Inductive) energies:

\begin{equation}
\mathcal{L} = \sum_{n} \left[ \underbrace{\frac{1}{2}C_{vac}\left(\frac{dV_n}{dt}\right)^2}_{\text{Capacitive Energy (E-Field)}} - \underbrace{\frac{1}{2}\frac{1}{L_{vac}}(\nabla V_n)^2}_{\text{Inductive Energy (B-Field)}} \right]
\label{eq:lagrangian}
\end{equation}

Applying the Euler-Lagrange equation $\frac{\partial \mathcal{L}}{\partial \phi} - \partial_\mu \frac{\partial \mathcal{L}}{\partial(\partial_\mu \phi)} = 0$, we minimize the action to recover the scalar wave equation:

\begin{equation}
\frac{\partial^2 \phi}{\partial t^2} - \frac{1}{L_{vac}C_{vac}}\nabla^2 \phi = 0
\end{equation}

\textbf{Engineering Conclusion:} Maxwell's Equations are simply the continuum limit of Kirchhoff's Laws applied to the vacuum mesh. Light is the vibration of the lattice; the speed of light $c$ is the characteristic propagation velocity determined by the lattice constants $L_{vac}$ and $C_{vac}$.
\chapter{Appendix B: General Relativity (Acoustic Metric)}

\section{Deriving the Schwarzschild Metric}
We model gravity as a radial "sink flow" of the vacuum substrate toward a massive object[cite: 316]. Assuming a steady-state, irrotational flow, the velocity field is defined as:
\begin{equation}
v_{0}(r) = -\sqrt{\frac{2GM}{r}}\hat{r}
\end{equation}
[cite: 317]
We substitute this flow field into the acoustic metric line element $ds^{2}$, which represents the effective geometry experienced by sound-like fluctuations in the fluid[cite: 317]. By applying a coordinate transformation to remove the non-diagonal cross-terms ($dt dr$), we recover the standard Schwarzschild line element:
\begin{equation}
ds^{2} \approx -\left(1 - \frac{2GM}{c_{s}^{2}r}\right) c_{s}^{2} dt^{2} + \left(1 - \frac{2GM}{c_{s}^{2}r}\right)^{-1} dr^2 + r^2 d\Omega^2
\end{equation}
[cite: 319]
\section{Conclusion: Emergent Geometry}
General Relativity is an \textbf{Emergent Phenomenon}. The curvature of spacetime is not a property of the manifold itself, but the \textbf{Effective Geometry} experienced by fluctuations (matter and light) propagating through a moving superfluid substrate. The "Event Horizon" is physically identified as the surface where the background flow velocity $|v_{0}|$ exceeds the local speed of light $c_{s}$ in the lattice.
\section{Appendix A: Electrodynamics (Hardware Derivation)}
We derive Maxwell's Equations not from abstract fields, but from the discrete energy balance of the LC network.

Consider the \textbf{Lagrangian Density} $\mathcal{L} = T - U$ for a 3D LC lattice, representing the difference between the Kinetic (Capacitive) and Potential (Inductive) energies:

\begin{equation}
\mathcal{L} = \sum_{n} \left[ \underbrace{\frac{1}{2}C_{vac}\left(\frac{dV_n}{dt}\right)^2}_{\text{Capacitive Energy (E-Field)}} - \underbrace{\frac{1}{2}\frac{1}{L_{vac}}(\nabla V_n)^2}_{\text{Inductive Energy (B-Field)}} \right]
\label{eq:lagrangian}
\end{equation}

Applying the Euler-Lagrange equation $\frac{\partial \mathcal{L}}{\partial \phi} - \partial_\mu \frac{\partial \mathcal{L}}{\partial(\partial_\mu \phi)} = 0$, we minimize the action to recover the scalar wave equation:

\begin{equation}
\frac{\partial^2 \phi}{\partial t^2} - \frac{1}{L_{vac}C_{vac}}\nabla^2 \phi = 0
\end{equation}

\textbf{Engineering Conclusion:} Maxwell's Equations are simply the continuum limit of Kirchhoff's Laws applied to the vacuum mesh. Light is the vibration of the lattice; the speed of light $c$ is the characteristic propagation velocity determined by the lattice constants $L_{vac}$ and $C_{vac}$.

\subsection{C.2 The Ether Drift (Why Michelson-Morley Failed)}
\textbf{Critique:} If the vacuum is a fluid, the Earth's motion through it should create an "Ether Wind" detectable by interferometers (Michelson-Morley).
\\
\textbf{Defense: Fresnel Drag.}
In hydrodynamics, a moving fluid only "drags" light if it has a refractive index $n > 1$. The drag coefficient $k$ is given by:
\begin{equation}
k = 1 - \frac{1}{n^2}
\end{equation}
\begin{itemize}
    \item \textbf{Near Earth:} The vacuum is unstrained, so $n \approx 1.0$. Therefore, $k \approx 0$. The vacuum flows \textit{through} the interferometer without altering the light path.
    \item \textbf{Near a Black Hole:} The vacuum is highly strained ($n \gg 1$), so $k \rightarrow 1$. In this regime, the "Ether Wind" is fully coupled to light, manifesting as \textbf{Frame Dragging} (Lense-Thirring Effect).
\end{itemize}
\textbf{Conclusion:} Michelson-Morley didn't disprove the Ether; they simply confirmed that the vacuum near Earth has a refractive index of 1.