\chapter{Appendix A: Electrodynamics (Maxwell Derivation)}
We derive Maxwell's Equations from the discrete Lagrangian of the LC network.
Consider the Lagrangian density $\mathcal{L}$ for a 3D LC lattice:
\begin{equation}
\mathcal{L} = \sum_{n} \left[ \frac{1}{2} C_{vac} \left(\frac{dV_n}{dt}\right)^2 - \frac{1}{2} \frac{1}{L_{vac}} (\nabla V_n)^2 \right]
\end{equation}
Applying the Euler-Lagrange equation, we recover the scalar wave equation for the potential $\phi$:
\begin{equation}
\frac{1}{c^2} \frac{\partial^2 \phi}{\partial t^2} - \nabla^2 \phi = 0
\end{equation}
This confirms that the continuum limit of the LCT lattice is standard Electrodynamics.

\chapter{Appendix B: General Relativity (Acoustic Metric)}

\section{Deriving the Schwarzschild Metric}
We model gravity as a radial "sink flow" of the vacuum substrate toward a massive object[cite: 316]. Assuming a steady-state, irrotational flow, the velocity field is defined as:
\begin{equation}
v_{0}(r) = -\sqrt{\frac{2GM}{r}}\hat{r}
\end{equation}
[cite: 317]
We substitute this flow field into the acoustic metric line element $ds^{2}$, which represents the effective geometry experienced by sound-like fluctuations in the fluid[cite: 317]. By applying a coordinate transformation to remove the non-diagonal cross-terms ($dt dr$), we recover the standard Schwarzschild line element:
\begin{equation}
ds^{2} \approx -\left(1 - \frac{2GM}{c_{s}^{2}r}\right) c_{s}^{2} dt^{2} + \left(1 - \frac{2GM}{c_{s}^{2}r}\right)^{-1} dr^2 + r^2 d\Omega^2
\end{equation}
[cite: 319]
\section{Conclusion: Emergent Geometry}
General Relativity is an \textbf{Emergent Phenomenon}. The curvature of spacetime is not a property of the manifold itself, but the \textbf{Effective Geometry} experienced by fluctuations (matter and light) propagating through a moving superfluid substrate. The "Event Horizon" is physically identified as the surface where the background flow velocity $|v_{0}|$ exceeds the local speed of light $c_{s}$ in the lattice.
\chapter{Appendix C: Theoretical Stress Tests}

\section{The Isotropy Problem}
\textbf{Critique:} A discrete lattice violates Lorentz Invariance because wave speed should vary based on the axis of travel[cite: 163, 324].

\textbf{Defense: The Amorphous Limit.} 
Just as window glass is transparent and isotropic despite being disordered at the atomic scale, the vacuum is isotropic at the macroscopic scale[cite: 325]. By modeling the vacuum as an \textbf{Amorphous Solid} rather than a perfect crystal, the local anisotropies average to zero over distances much larger than the breakdown wavelength ($L \gg \lambda_{min}$)[cite: 164, 167, 325].

\section{The Ether Drift (Stellar Aberration)}
\textbf{Critique:} If the vacuum is a fluid dragged by mass, we should not see the annual shift in star positions (Stellar Aberration)[cite: 326].

\textbf{Defense: Fresnel Drag.} 
In hydrodynamics, a fluid drags light only if its refractive index $n > 1$[cite: 327]. The drag coefficient $k$ is defined by:
\begin{equation}
k = 1 - \frac{1}{n^2}
\end{equation}
[cite: 327]

\begin{itemize}
    \item \textbf{Near Earth:} The vacuum strain is negligible, and $n \approx 1$[cite: 327]. Therefore, $k \approx 0$, the vacuum is not "dragged" significantly, and Stellar Aberration is preserved[cite: 327].
    \item \textbf{Near Black Holes:} Here, $n \gg 1$, and $k \to 1$[cite: 328]. In this regime, the vacuum is fully dragged, which observationally manifests as the **Lense-Thirring effect** (Frame Dragging)[cite: 328].
\end{itemize}