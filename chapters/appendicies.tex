\appendix

\chapter{Mathematical Proofs and Formalism}
\label{app:math_proofs}

\section{A.1 The Discrete-to-Continuum Limit (Kirchhoff)}
To bridge the gap between electrical engineering and field theory, we expand the derivation from Section 1.2.2[cite: 559]. Consider a 3D discrete lattice where the nodal current balance at node $n$ is defined by:
\begin{equation}
\Cvac \frac{dV_{n}}{dt} = I_{n} - I_{n+1}
\end{equation}
Differentiating and substituting the voltage relation $\Lvac \frac{dI}{dt} = \Delta V$ yields the discrete wave equation[cite: 561]:
\begin{equation}
\Lvac\Cvac \frac{d^{2}V_{n}}{dt^{2}} = V_{n-1} - 2V_{n} + V_{n+1} \quad (4)
\end{equation}
In the limit $\Dx \rightarrow 0$, we define the spatial second derivative and recover the standard Wave Equation[cite: 564]:
\begin{equation}
\frac{\Lvac\Cvac}{\Dx^{2}} \frac{\partial^{2}V}{\partial t^{2}} = \frac{\partial^{2}V}{\partial x^{2}} \implies \frac{\partial^{2}V}{\partial t^{2}} - c^{2} \frac{\partial^{2}V}{\partial x^{2}} = 0 \quad (5)
\end{equation}

\section{A.2 The Madelung Internal Pressure ($Q$)}
In Chapter 3, the Quantum Potential $Q$ was identified as internal vacuum pressure[cite: 567]. Substituting the polar form $\psi = \sqrt{\rho}e^{iS/\hbar}$ into the Schrödinger Equation and separating the real part yields the \textbf{Quantum Hamilton-Jacobi Equation}[cite: 568]:
\begin{equation}
\frac{\partial S}{\partial t} + \frac{(\nabla S)^{2}}{2m} + V + Q = 0 \quad \text{where} \quad Q = -\frac{\hbar^{2}}{2m} \frac{\nabla^{2}\sqrt{\rho}}{\sqrt{\rho}} \quad (6)
\end{equation}
In LCT, $Q$ is the elastic potential energy density of the lattice nodes being displaced by the pilot wave[cite: 571].

\section{A.3 Impedance Clamping and Parity Violation}
The effective impedance $Z_{eff}$ for helical pulses is modified by the alignment of the vortex winding $m$ and momentum vector $k$[cite: 572, 580]:
\begin{equation}
Z_{eff}(\sigma, m, k) = Z_{0} e^{\sigma(m \cdot k)} \quad (7)
\end{equation}
As $\omega \rightarrow \Wcut$, the impedance for right-handed configurations ($m \cdot k > 0$) hits the hardware slew limit, reflecting the energy back into the substrate[cite: 582].

\chapter{The Computational Verification Suite}
\label{app:verification_suite}

\section{B.1 Overview: The Numerical Foundation}
LCT is verified through Python-based FDTD and Ginzburg-Landau relaxation simulations[cite: 588]. These scripts ensure reproducibility of the emergent phenomena described in Chapters 1–8.

\section{B.2 Hardware and Signal Verification}
\begin{itemize}
    \item \textbf{Metric Strain and Geodesics (\texttt{sim\_a\_metric\_strain.py})}: Validates the "Gravity as Metric Strain" postulate. Wavefronts exhibit gravitational lensing through variable impedance[cite: 591, 593].
    \item \textbf{Dispersion (\texttt{01\_Relativistic\_Limit.ipynb})}: Confirms that as frequency approaches $\Wcut$, the group velocity $v_g$ vanishes, providing the basis for mass as bandwidth saturation[cite: 594, 597].
\end{itemize}

\section{B.3 Quantum and Topological Verification}
\begin{itemize}
    \item \textbf{Pilot-Wave (\texttt{sim\_d\_born\_rule.py})}: Models a particle as a bouncing soliton. The resulting distribution reproduces the Born Rule without probabilistic collapse[cite: 603, 606].
    \item \textbf{Proton Triplet (\texttt{sim\_k\_proton\_triplet.py})}: Proves that three vortex cores naturally self-assemble into a stable triangular "Trefoil" configuration[cite: 607, 611].
\end{itemize}

\chapter{Simulation Code Repository}
\label{app:code_repo}

\section{C.1 Introduction: Numerical Hardware Verification}
All scripts utilize Finite-Difference Time-Domain (FDTD) methods and work with the global constants defined in \texttt{src/constants.py}[cite: 636, 637].

\begin{simbox}[Topological Defect Creation]
\begin{lstlisting}[language=Python]
import numpy as np
def simulate_quench(N=300, steps=1500):
    # Initial Hot Disordered Phase
    psi = np.exp(1j * np.random.uniform(-np.pi, np.pi, (N, N)))
    dt, dx = 0.001, 0.1
    for t in range(steps):
        lap = (np.roll(psi,1,0) + np.roll(psi,-1,0) + np.roll(psi,1,1) + np.roll(psi,-1,1) - 4*psi) / (dx**2)
        # Vacuum relaxation to ordered state
        psi += dt * (lap + psi * (1.0 - np.abs(psi)**2))
    return np.angle(psi)
\end{lstlisting}
\end{simbox}