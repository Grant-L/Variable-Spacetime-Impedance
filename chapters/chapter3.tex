\chapter{The Quantum Layer: Hydrodynamic Pilot-Wave Mechanics}

\section{3.1 Introduction: The End of "Spooky" Action}
The Copenhagen Interpretation posits that particles exist as probabilistic wavefunctions ($\psi$) that collapse upon measurement[cite: 148]. LCT proposes a \textbf{Hidden Variable} solution: the vacuum lattice stores the history of a particle's path[cite: 150]. This "Memory Field" acts as a physical Pilot Wave, guiding the particle through interference patterns[cite: 151].

\section{3.2 Deriving the Schrödinger Equation}
We derive the Schrödinger Equation as the hydrodynamic limit of the vacuum lattice[cite: 153]. By applying the \textbf{Madelung Transformation} ($\psi = \sqrt{\rho}e^{iS/\hbar}$), where $v = \nabla S/m$, we rewrite the classical Euler equations for a vacuum fluid density $\rho$ and velocity $v$ as[cite: 154]:

\begin{equation}
i\hbar\frac{\partial\psi}{\partial t} = -\frac{\hbar^2}{2m}\nabla^2\psi + V\psi + Q\psi
\label{eq:schrodinger_hydro}
\end{equation}

In this framework, $Q$ is the \textbf{Quantum Potential}[cite: 156]:
\begin{equation}
Q = -\frac{\hbar^2}{2m}\frac{\nabla^2\sqrt{\rho}}{\sqrt{\rho}}
\end{equation}

$Q$ represents the \textbf{Internal Pressure} of the vacuum substrate[cite: 159]. This proves that the Schrödinger equation is simply the equation of motion for a superfluid lattice[cite: 160].

\section{3.3 Pilot Wave Dynamics: The Walker Model}
A particle in LCT is a "Bouncing Soliton" oscillating at the \textbf{Compton Frequency} ($\omega_c$)[cite: 162]. Each oscillation injects energy into the lattice, creating a standing wave field[cite: 163]. The particle "surfs" the gradient of its own memory field[cite: 164]:

\begin{equation}
F_{particle} = -\nabla \Phi_{memory}
\end{equation}

This feedback loop causes the particle to exhibit diffraction and interference even when passing through a system one at a time[cite: 169]. \textbf{Heisenberg Uncertainty} is thus identified as dynamical "jitter" (Zitterbewegung) caused by the background noise of the pilot wave[cite: 170].

\subsection{3.3.1 Computational Module: The Quantum Walker}
The following simulation verifies that a classical walker guided by its own wave gradient reproduces quantum-like trajectories [cite: 608-610].

\begin{verbatim}
import numpy as np
def simulate_walker():
    Nx, Ny = 200, 200; dt = 0.5
    u = np.zeros((Nx, Ny)); u_prev = np.zeros((Nx, Ny))
    px, py = 50.0, 100.0; vx, vy = 0.8, 0.0 
    for t in range(1000):
        lap = (np.roll(u,1,0) + np.roll(u,-1,0) + np.roll(u,1,1) + np.roll(u,-1,1) - 4*u)
        u_next = (2*u - u_prev + 0.25*lap) * 0.98 # Memory decay
        u_next[int(px), int(py)] += 2.0 * np.sin(0.5 * t) # Impact
        # Pilot Wave Guidance
        grad_y = (u[int(px), int(py)+1] - u[int(px), int(py)-1]) / 2.0
        vy += 0.1 * grad_y 
        px += vx; py += vy
        u_prev, u = u.copy(), u_next.copy()
    return px, py
\end{verbatim}

\section{3.4 The Illusion of Choice: The Observer Effect}
LCT replaces the "Conscious Collapse" model with a hydrodynamic \textbf{Impedance Mismatch}[cite: 172]. 



\begin{itemize}
    \item \textbf{Wave Mode (Observer OFF)}: The pilot wave passes through both slits, creating interference fringes that guide the particle[cite: 173].
    \item \textbf{Particle Mode (Observer ON)}: A detector acts as a \textbf{Resistive Load} ($R_{load}$) on the vacuum[cite: 174]. It extracts energy from the pilot wave, damping the interference[cite: 175]. 
\end{itemize}

Without the wave to guide it, the particle follows a straight Newtonian path[cite: 176].

\section{3.5 The Emergent Atom: Deriving the Bohr Radius}
LCT observes atomic stability as a consequence of fluid resonance[cite: 178]. 
\begin{itemize}
    \item \textbf{The Lock-In}: As an electron spirals toward a nucleus, it perturbs the vacuum lattice, creating a "wake"[cite: 179].
    \item \textbf{Quantization}: At a specific radius, the electron's orbital frequency matches the resonant frequency of its own vacuum wake[cite: 180]. 
    \item \textbf{Stability}: The radiation pressure from the lattice balances the Coulomb attraction, creating a stable orbit at the \textbf{Bohr Radius} ($a_0$)[cite: 181].
\end{itemize}

\section{3.6 The Casimir Effect: Vacuum Filtration}
The Casimir force is modeled as a \textbf{Band-Stop Filter} within the noisy vacuum substrate[cite: 183]. Conducting plates act as short circuits ($V=0$) for vacuum noise[cite: 184]. Any mode with $\lambda/2 > d$ is excluded from the gap, creating a pressure deficit [cite: 185-186].