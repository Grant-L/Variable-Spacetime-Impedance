\chapter{The Quantum Layer: Emergent Mechanics}

\section{Introduction: The End of "Spooky" Action}
The Copenhagen Interpretation of Quantum Mechanics posits that particles exist as probabilistic wavefunctions ($\psi$) that collapse upon measurement. This introduces an irreconcilable break between the determinism of Gravity and the randomness of Matter.
LCT proposes a \textbf{Hidden Variable} solution: The vacuum lattice itself stores the history of a particle's path. This "Memory Field" acts as a \textbf{Pilot Wave}, guiding the particle through interference patterns.

\section{Deriving the Schrödinger Equation (Hydrodynamic Limit)}
We begin with the Euler equation for the vacuum fluid density $\rho$ and velocity $v$. By applying the **Madelung Transformation** ($\psi = \sqrt{\rho} e^{iS/\hbar}$), where $v = \nabla S/m$, we can rewrite the classical fluid equations as:
\begin{equation}
i\hbar \frac{\partial \psi}{\partial t} = -\frac{\hbar^2}{2m} \nabla^2 \psi + V \psi + Q \psi
\end{equation}
Here, $Q$ is the **Quantum Potential** ($Q = -\frac{\hbar^2}{2m} \frac{\nabla^2 \sqrt{\rho}}{\sqrt{\rho}}$), which represents the internal pressure of the vacuum fluid. This proves that the Schrödinger Equation is simply the equation of motion for a superfluid substrate.

\section{Pilot Wave Dynamics (The Walker Model)}
A particle in LCT is a "Bouncing Soliton" oscillating at the Compton Frequency ($\omega_c$). Each oscillation injects energy into the lattice, creating a standing wave field.
\begin{equation}
F_{particle} = - \nabla \Phi_{memory}
\end{equation}
The particle "surfs" the gradient of its own wave field. This feedback loop locks the particle into quantized orbits and causes it to exhibit diffraction through a double slit, even when passing through one slit at a time.
\textbf{Heisenberg Uncertainty as Jitter:} The "fuzziness" of position is not ontological; it is dynamical. The particle undergoes constant \textbf{Zitterbewegung} (jitter) due to the background noise of the pilot wave.

\section{Restoring Lorentz Invariance (The Glass Vacuum)}
A standard cubic lattice violates Special Relativity because the speed of light varies with direction (axial vs. diagonal).
To resolve this, we model the vacuum as an \textbf{Amorphous Solid} (Glass) rather than a Crystal. The nodes are distributed according to a Poisson process and connected via Delaunay Triangulation.
\begin{itemize}
    \item \textbf{Local Anisotropy:} At the micro-scale ($< \lambda_{min}$), the speed of light fluctuates.
    \item \textbf{Global Isotropy:} At the macro-scale, these fluctuations average to zero. The refractive index is statistically uniform in all directions.
\end{itemize}

\section{Computational Module: The Double Slit Simulation}
We simulated a deterministic "Walker" particle interacting with a wave equation solver.
\begin{itemize}
    \item **Setup:** A particle passes one-by-one through a double-slit barrier.
    \item **Result:** Although each particle has a definite trajectory, the ensemble builds up an interference pattern matching $|\psi|^2$.
    \item **Observation:** The "interference" exists in the vacuum memory, not in the particle itself.
\end{itemize}
*(See Appendix B.2 for the full Python source code.)*

\section{Bridge the Gap: From Copenhagen to Hydrodynamics}
To the Quantum Physicist, $\psi$ is a probability amplitude. To the Fluid Dynamicist, $\psi$ is a complex order parameter.
\begin{itemize}
    \item **Density:** $|\psi|^2$ is the fluid density $\rho$.
    \item **Phase:** The gradient of the phase $\nabla S$ is the fluid velocity $v$.
    \item **Collapse:** Is not a magical event, but a rapid equilibration of the pilot wave pressure when a measurement probe disturbs the fluid.
\end{itemize}