\chapter{The Quantum Layer: Emergent Mechanics}

\section{Introduction: The End of "Spooky" Action}
The Copenhagen Interpretation posits that particles exist as probabilistic wavefunctions ($\psi$) that collapse upon measurement[cite: 176]. This introduces an irreconcilable break between the determinism of Gravity and the randomness of Matter[cite: 177]. LCT proposes a \textbf{Hidden Variable} solution: the vacuum lattice stores the history of a particle's path[cite: 178]. This "Memory Field" acts as a Pilot Wave, guiding the particle through interference patterns[cite: 179].

\section{Deriving the Schrödinger Equation}
We derive the Schrödinger Equation as the hydrodynamic limit of the vacuum fluid[cite: 180]. By applying the \textbf{Madelung Transformation} ($\psi = \sqrt{\rho}e^{iS/\hbar}$), where $v = \nabla S/m$, we rewrite the classical Euler equations for a vacuum fluid density $\rho$ and velocity $v$ as[cite: 181]:

\begin{equation}
i\hbar\frac{\partial\psi}{\partial t} = -\frac{\hbar^2}{2m}\nabla^2\psi + V\psi + Q\psi
\label{eq:schrodinger}
\end{equation}

In this framework, $Q$ is the \textbf{Quantum Potential}[cite: 184]:
\begin{equation}
Q = -\frac{\hbar^2}{2m}\frac{\nabla^2\sqrt{\rho}}{\sqrt{\rho}}
\end{equation}

$Q$ represents the internal pressure of the vacuum fluid, proving that Eq. \ref{eq:schrodinger} is simply the equation of motion for a superfluid substrate[cite: 184, 185].

\section{Pilot Wave Dynamics: The Walker Model}
A particle in LCT is a "Bouncing Soliton" oscillating at the \textbf{Compton Frequency} ($\omega_c$)[cite: 187]. Each oscillation injects energy into the lattice, creating a standing wave field[cite: 188]. The particle "surfs" the gradient of its own memory field[cite: 191]:

\begin{equation}
F_{particle} = -\nabla \Phi_{memory}
\end{equation}

This feedback loop causes the particle to exhibit diffraction and interference even when passing through a system one at a time[cite: 192]. \textbf{Heisenberg Uncertainty} is thus identified as dynamical "jitter" (Zitterbewegung) caused by the background noise of the pilot wave[cite: 193, 194].

\section{The Illusion of Choice: The Observer Effect}
LCT replaces the "Conscious Collapse" model with a hydrodynamic \textbf{Impedance Mismatch}[cite: 205, 206]. 



\begin{itemize}
    \item \textbf{Wave Mode (Observer OFF):} The pilot wave passes through both slits, creating interference fringes that guide the particle[cite: 207, 208].
    \item \textbf{Particle Mode (Observer ON):} A detector acts as a \textbf{Resistive Load} ($R_{load}$) on the vacuum[cite: 209]. It extracts energy from the pilot wave, damping the interference[cite: 210]. Without the wave to guide it, the particle follows a straight Newtonian path[cite: 210].
\end{itemize}

\begin{figure}[H]
    \centering
    \begin{subfigure}{0.48\linewidth}
        \centering
        \includegraphics[width=\linewidth]{simulations/simulation_electron_choice_observer_off.png}
        \caption{Observer OFF: Wave Mode}
    \end{subfigure}
    \hfill
    \begin{subfigure}{0.48\linewidth}
        \centering
        \includegraphics[width=\linewidth]{simulations/simulation_electron_choice_observer_on.png}
        \caption{Observer ON: Particle Mode}
    \end{subfigure}
    \caption{Simulation D.8: The Mechanism of Collapse. Damping the pilot wave via measurement removes the guiding force, resulting in classical trajectories [cite: 256-259].}
    \label{fig:observer_effect}
\end{figure}

\section{The Emergent Atom: Deriving the Bohr Radius}
LCT observes atomic stability as a consequence of fluid resonance rather than postulating a stationary state[cite: 226, 229]. 

\begin{itemize}
    \item \textbf{The Lock-In:} As an electron spirals toward a nucleus, it perturbs the vacuum lattice, creating a "wake"[cite: 230, 231]. 
    \item \textbf{Quantization:} At a specific radius, the electron's orbital frequency matches the resonant frequency of its own vacuum wake[cite: 231]. 
    \item \textbf{Stability:} The radiation pressure from the lattice balances the Coulomb attraction, creating a stable orbit at the Bohr Radius ($a_0$)[cite: 232].
\end{itemize}

\section{The Casimir Effect: Vacuum Filtration}
The Casimir force is modeled as a \textbf{Band-Stop Filter} within the noisy vacuum substrate[cite: 234]. 

\begin{itemize}
    \item \textbf{Filtration:} Conducting plates act as short circuits ($V=0$) for vacuum noise[cite: 235]. Any mode with $\lambda/2 > d$ is excluded from the gap[cite: 236].
    \item \textbf{Pressure Differential:} The lower energy density inside the gap creates a pressure deficit relative to the broadband noise outside[cite: 237].
    \item \textbf{Result:} The plates are pushed together by external radiation pressure, as verified in Simulation D.10[cite: 238].
\end{itemize}

\begin{figure}[H]
    \centering
    \includegraphics[width=0.8\linewidth]{simulations/simulate_casimir_effect.png}
    \caption{Simulation D.10: Vacuum energy density suppression between conducting plates [cite: 314-316].}
    \label{fig:casimir}
\end{figure}