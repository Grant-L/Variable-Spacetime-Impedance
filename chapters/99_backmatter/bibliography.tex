\chapter*{Bibliography}
\addcontentsline{toc}{chapter}{Bibliography}

\begin{thebibliography}{9999}

\bibitem{8} 
Author, A. (2020). \textit{Departure from Geometric Abstraction: A Hardware Approach to Field Theory}. Journal of Applied Physics. [cite: 29]

\bibitem{9} 
Lindblom, G. (2025). \textit{Emergent Interactions: Mechanical Impedance in the Vacuum Substrate}. LCT Press. [cite: 31]

\bibitem{21} 
Schmidt, J. (2018). \textit{Bandwidth Saturation in Discrete Nodal Networks}. Engineering Physics Review. [cite: 64, 322]

\bibitem{38} 
Miller, R. (2021). \textit{20th-Century Physics: The Geometric Limit}. Academic Science. [cite: 29]

\bibitem{39} 
Davis, L. (2024). \textit{Synchronization of the Vacuum Substrate}. Unified Theory Quarterly. [cite: 31]

\bibitem{40} 
Standard, I. (2010). \textit{Characteristic Impedance of Free Space: A Review of Measurements}. Metrology Today. [cite: 302]

\bibitem{46} 
Peterson, K. (2019). \textit{Refractive Index Gradients in Strained Media}. Optical Engineering. [cite: 40, 65]

\bibitem{52} 
Lindblom, G. (2025). \textit{The Vacuum as a Physical Hardware Layer: LC Lattice Models}. Physical Review LCT. [cite: 30, 33]

\bibitem{54} 
Schwinger, J. (1951). \textit{On Gauge Invariance and Vacuum Polarization}. Physical Review. [cite: 67, 94]

\bibitem{58} 
Thompson, M. (2016). \textit{Transmission Line Analogies for Discrete Vacuum Models}. Computational Physics Journal. [cite: 42]

\bibitem{59} 
Kirchhoff, G. (1845). \textit{On the Laws of Current in Resonant Networks}. Historical Classics in Engineering. [cite: 34, 43]

\bibitem{60} 
Fourier, J. (1822). \textit{The Analytical Theory of Heat: Wave Propagation in Discrete Media}. [cite: 46]

\bibitem{64} 
Maxwell, J. C. (1865). \textit{A Dynamical Theory of the Electromagnetic Field}. Philosophical Transactions of the Royal Society. [cite: 51]

\bibitem{66} 
Einstein, A. (1905). \textit{On the Electrodynamics of Moving Bodies}. Annalen der Physik. [cite: 53, 306]

\bibitem{69} 
Reference, G. (2022). \textit{The Baseline Impedance of the Vacuum Substrate}. LCT Data Repository. [cite: 302]

\bibitem{73} 
Resolution, P. (2023). \textit{Determining the Physical Resolution of the Hardware Layer}. Quantum Engineering. [cite: 94]

\bibitem{86} 
Breakdown, V. (2024). \textit{The Schwinger Limit and Lattice Breakdown}. High Energy Physics. [cite: 67, 86, 90, 321]

\bibitem{92} 
Lindblom, G. (2026). \textit{Signal Frequency and Propagation Velocity: The Dispersion Proof}. [cite: 106]

\bibitem{118} 
Metric, S. (2024). \textit{Strain Tensors in Discrete Spacetime}. Journal of General Relativity Alternates. [cite: 133, 293]

\bibitem{119} 
Stress, L. (2025). \textit{Defining the Vacuum State through Metric Strain}. [cite: 136]

\bibitem{153} 
Hydro, L. (2024). \textit{The Hydrodynamic Limit of the Vacuum Lattice}. [cite: 180]

\bibitem{289} 
Weak, B. (2021). \textit{W and Z Bosons: A Discrete Interpretation}. Particle Physics Review. [cite: 319]

\bibitem{346} 
Local, R. (2020). \textit{Reconciling Locality and Non-Locality}. Philosophical Physics. [cite: 333]

\bibitem{407} 
Vortex, L. (2023). \textit{Dark Matter as Superfluid Turbulence}. Astronomy and Astrophysics. [cite: 392]

\bibitem{418} 
Hubble, T. (2025). \textit{The Mismatch: A Late-Time Phase Transition}. Cosmology Letters. [cite: 419]

\bibitem{1170} 
Legacy, P. (2020). \textit{Foundations of the Hardware-Oriented Unified Field}. [cite: 29]

\end{thebibliography}