\chapter{The Computational Verification Suite}
\label{app:verification_suite}

\section{B.1 Overview: The Numerical Foundation}
The LCT is verified through Python-based Finite-Difference Time-Domain (FDTD) and Ginzburg-Landau relaxation simulations[cite: 42, 43]. These ensure reproducibility of the emergent phenomena described in Chapters 1–8[cite: 43].

\section{B.2 Key Verification Modules}
\begin{itemize}
    \item \textbf{Metric Strain (\texttt{sim\_a\_metric\_strain.py})}: Validates the "Gravity as Metric Strain" postulate by locally modifying $\Lvac$ and $\Cvac$[cite: 44, 45].
    \item \textbf{Dispersion (\texttt{01\_Relativistic\_Limit.ipynb})}: Confirms that as signal frequency $\omega$ approaches $\Wcut$, the group velocity $v_g$ vanishes[cite: 47, 48].
    \item \textbf{Pilot-Wave (\texttt{sim\_d\_born\_rule.py})}: Reproduces the Born Rule using deterministic solitons without wavefunction collapse[cite: 50, 52].
    \item \textbf{Proton Triplet (\texttt{sim\_k\_proton\_triplet.py})}: Proves that three vortex cores self-assemble into a stable triangular "Trefoil" configuration[cite: 53, 54].
    \item \textbf{Galactic Rotation (\texttt{sim\_l\_galactic\_rotation.py})}: Validates the Dark Matter solution through quantized vortex lattices[cite: 55, 56].
    \item \textbf{The Cosmic Quench (\texttt{sim\_b\_genesis.py})}: Models the Big Bang as a vacuum phase transition where matter is trapped at domain boundaries[cite: 57, 58].
\end{itemize}

\section{B.3 Environment Setup}
To run the suite, use the following requirements[cite: 59, 60, 61, 62]:
\begin{itemize}
    \item \texttt{numpy}, \texttt{matplotlib}, \texttt{scipy}[cite: 59, 60, 61].
    \item Execute \texttt{setup.sh} to link \texttt{src/constants.py}[cite: 62].
\end{itemize}