\chapter{Mathematical Proofs and Formalism}
\label{app:math_proofs}

\section{A.1 The Discrete-to-Continuum Limit (Kirchhoff)}
To bridge the gap between electrical engineering and field theory, we expand the derivation of the vacuum wave equation from Section 1.2.2. Consider a 3D discrete lattice where each node is connected by inductors $\mathcal{L}_{vac}$ and capacitors $\mathcal{C}_{vac}$.

The nodal current balance at node $n$ is defined by Kirchhoff's Current Law:
\begin{equation}
    \mathcal{C}_{vac} \frac{dV_n}{dt} = I_n - I_{n+1}
\end{equation}
Differentiating with respect to time and substituting the inductive voltage relation $\mathcal{L}_{vac} \frac{dI}{dt} = V_{n-1} - V_n$, we obtain the discrete equation of motion:
\begin{equation}
    \mathcal{L}_{vac}\mathcal{C}_{vac} \frac{d^2 V_n}{dt^2} = V_{n-1} - 2V_n + V_{n+1}
\end{equation}
In the continuum limit where $\Delta x \rightarrow 0$, we apply the Taylor expansion $V_{n\pm1} \approx V(x) \pm \Delta x \frac{\partial V}{\partial x} + \frac{\Delta x^2}{2} \frac{\partial^2 V}{\partial x^2}$. This recovers the standard Maxwellian Wave Equation:
\begin{equation}
    \frac{\mathcal{L}_{vac}\mathcal{C}_{vac}}{\Delta x^2} \frac{\partial^2 V}{\partial t^2} = \frac{\partial^2 V}{\partial x^2} \implies \frac{\partial^2 V}{\partial t^2} - c^2 \nabla^2 V = 0
\end{equation}
Where $c = \Delta x / \sqrt{\mathcal{L}_{vac}\mathcal{C}_{vac}}$ represents the lattice slew rate limit.

\section{A.2 The Madelung Internal Pressure ($Q$)}
The Quantum Potential $Q$ is identified as the internal hydrostatic pressure of the vacuum superfluid. Substituting the polar form $\psi = \sqrt{\rho}e^{iS/\hbar}$ into the Schrödinger Equation separates the system into a Continuity Equation (Conservation of Probability) and a Quantum Hamilton-Jacobi Equation (Conservation of Momentum):
\begin{equation}
    \frac{\partial S}{\partial t} + \frac{(\nabla S)^2}{2m} + V + Q = 0
\end{equation}
The term $Q$ arises purely from the spatial curvature of the amplitude density $\sqrt{\rho}$:
\begin{equation}
    Q = -\frac{\hbar^2}{2m} \frac{\nabla^2\sqrt{\rho}}{\sqrt{\rho}}
\end{equation}
In LCT, this is the **elastic potential energy density** of the lattice nodes resisting compression, manifesting as the "force" that drives quantum interference.

\section{A.3 Impedance Clamping and Parity Violation}
The "Weak Interaction" is derived as a frequency-dependent impedance mismatch. For a helical pulse with winding number $m$ and momentum $k$, the effective lattice impedance $Z_{eff}$ is directionally biased:
\begin{equation}
    Z_{eff}(\sigma, m, k) = Z_0 \exp(m \cdot k)
\end{equation}
For left-handed modes ($m \cdot k < 0$), $Z_{eff} \approx Z_0$, allowing propagation. For right-handed modes ($m \cdot k > 0$), $Z_{eff}$ diverges as the signal frequency $\omega \rightarrow \omega_{cutoff}$. This triggers a **Total Internal Reflection** event, physically preventing the propagation of right-handed neutrinos.