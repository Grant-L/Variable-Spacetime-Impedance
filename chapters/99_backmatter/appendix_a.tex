\chapter{Mathematical Proofs and Formalism}

\section{A.1 The Discrete-to-Continuum Limit (Kirchhoff)}
To bridge the gap between electrical engineering and field theory, we expand the derivation in Section 1.2.2. [cite_start]Consider the 3D discrete lattice where each node is connected by inductors $\Lvac$ and capacitors $\Cvac$[cite: 745]. [cite_start]The nodal current balance at node $n$ is[cite: 747]:
\begin{equation}
\Cvac \frac{dV_n}{dt} = I_n - I_{n+1}
\end{equation}
[cite_start]Differentiating and substituting $\Lvac \frac{dI}{dt} = \Delta V$ yields the discrete wave equation[cite: 749]:
\begin{equation}
\Lvac\Cvac \frac{d^2 V_n}{dt^2} = V_{n-1} - 2V_n + V_{n+1}
\end{equation}
[cite_start]In the limit $\Dx \rightarrow 0$, we define the spatial second derivative and recover the standard Wave Equation[cite: 754, 758]:
\begin{equation}
\frac{\Lvac\Cvac}{\Dx^2} \frac{\partial^2 V}{\partial t^2} = \frac{\partial^2 V}{\partial x^2} \implies \frac{\partial^2 V}{\partial t^2} - c^2 \frac{\partial^2 V}{\partial x^2} = 0
\end{equation}

\section{A.2 The Madelung Internal Pressure (Q)}
[cite_start]In Chapter 3, the Quantum Potential $Q$ was identified as internal vacuum pressure[cite: 761]. [cite_start]Substituting the polar form $\psi = \sqrt{\rho}e^{iS/\hbar}$ into the Schrödinger Equation and separating the real part yields the \textbf{Quantum Hamilton-Jacobi Equation}[cite: 764]:
\begin{equation}
\frac{\partial S}{\partial t} + \frac{(\nabla S)^2}{2m} + V + Q = 0 \quad \text{where} \quad Q = -\frac{\hbar^2}{2m} \frac{\nabla^2\sqrt{\rho}}{\sqrt{\rho}}
\end{equation}
[cite_start]In LCT, $Q$ is the \textbf{elastic potential energy density} of the lattice nodes being displaced[cite: 772].

\section{A.3 Impedance Clamping and Parity Violation}
[cite_start]The effective impedance $Z_{eff}$ for helical pulses is modified by the alignment of the vortex winding $m$ and momentum vector $k$[cite: 776, 778]:
\begin{equation}
Z_{eff}(\sigma, m, k) = Z_0 e^{\sigma(m \cdot k)}
\end{equation}
[cite_start]As $\omega \rightarrow \Wcut$, the impedance for right-handed configurations $(m \cdot k > 0)$ hits the hardware slew limit, reflecting the energy back into the substrate[cite: 780, 781].