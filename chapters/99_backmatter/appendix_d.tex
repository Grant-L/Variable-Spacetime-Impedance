\chapter{Appendix D: Glossary of Electrical-to-Physical Analogies}
\label{app:rosetta}

\section{D.1 The Rosetta Stone of LCT}
To facilitate the transition from vacuum engineering to theoretical physics, this appendix maps the constitutive electrical properties of the hardware substrate to their emergent physical counterparts.

\begin{center}
\begin{tabular}{|l|l|l|}
\hline
\textbf{Hardware Term} & \textbf{Physical Equivalent} & \textbf{LCT Mechanical Role} \\ \hline
Inductance ($L$) & Permeability ($\mu_0$) & Inertial component resisting flux changes. \\ \hline
Capacitance ($C$) & Permittivity ($\epsilon_0$) & Elastic modulus storing potential energy. \\ \hline
Impedance ($Z_0$) & Vacuum "Thickness" & Baseline ratio defining signal propagation. \\ \hline
B-EMF & Inertia & Resistance to acceleration in saturated nodes. \\ \hline
TVS Analogy & Weak Interaction & Directional clamping of chiral vortex signals. \\ \hline
Slew Rate Limit & Speed of Light ($c$) & Maximum update frequency of a lattice node. \\ \hline
Saturation & Rest Mass & High-frequency flux trapped as a standing wave. \\ \hline
Winding ($n$) & Electric Charge & Quantized topological twist in the phase field. \\ \hline
$\nabla Z$ Gradient & Metric Curvature & Refractive index gradient causing signal delay. \\ \hline
Topological Short & Wormhole & Low-impedance bypass connecting distant nodes. \\ \hline
\end{tabular}
\end{center}