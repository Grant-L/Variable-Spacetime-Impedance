\chapter{Simulation Code Repository}
\label{app:code_repo}

\section{C.1 Introduction}
All scripts utilize FDTD and Ginzburg-Landau methods based on the global constants defined in \texttt{src/constants.py}. [cite: 859]

\section{C.2 Core Code: Metric Lensing}
\begin{lstlisting}[language=Python, caption=Gravitational Lensing Simulation]
import numpy as np

def run_metric_simulation(Nx=600, Ny=400, Nt=1200):
    u = np.zeros((Nx, Ny))
    u_prev = np.zeros((Nx, Ny))
    
    # Grid for metric strain mapping
    X, Y = np.meshgrid(np.arange(Nx), np.arange(Ny), indexing='ij')
    R = np.sqrt((X - Nx//2)**2 + (Y - (Ny//2+50))**2)
    
    # n = 1 + epsilon (refractive index gradient)
    n_map = 1.0 + 20.0 / (np.sqrt(R**2 + 10.0)) 
    v_map = 1.0 / n_map # Local phase velocity
    
    dt = 0.5
    for t in range(Nt):
        lap = (np.roll(u, 1, 0) + np.roll(u, -1, 0) + 
               np.roll(u, 1, 1) + np.roll(u, -1, 1) - 4*u)
        u_next = 2*u - u_prev + (v_map * dt)**2 * lap
        
        if t < 100: 
            u_next[5, Ny//2-50] += np.sin(0.6*t)
            
        u_prev, u = u.copy(), u_next.copy()
    return u
\end{lstlisting}

\section{C.3 Core Code: The Cosmic Quench}
\begin{lstlisting}[language=Python, caption=Vacuum Phase Transition (Genesis)]
def simulate_quench(N=300, steps=1500):
    # Initial Hot Disordered Phase
    psi = np.exp(1j * np.random.uniform(-np.pi, np.pi, (N, N)))
    dt, dx = 0.001, 0.1
    
    for t in range(steps):
        lap = (np.roll(psi, 1, 0) + np.roll(psi, -1, 0) + 
               np.roll(psi, 1, 1) + np.roll(psi, -1, 1) - 4*psi) / (dx**2)
        # GL Relaxation to ordered state
        psi += dt * (lap + psi * (1.0 - np.abs(psi)**2))
    return np.angle(psi)
\end{lstlisting}