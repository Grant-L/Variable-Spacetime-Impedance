\chapter{The Signal Layer: Variable Impedance and Mass Emergence}
\label{ch:signal_layer}

\section{The Lindblom Dispersion Relation}
In Chapter 1, we established the vacuum as a discrete LC lattice. We now derive the relationship between signal frequency and propagation velocity, identifying the mechanical origin of rest mass and relativistic scaling as a direct result of hardware bandwidth limitations.

\subsection{Derivation from Discrete Kirchhoff Laws}
Starting from the discrete equations of motion defined in Section 1.3:
\begin{equation}
\Lvac \frac{dI_{n}}{dt} = V_{n-1} - V_{n}, \quad \Cvac \frac{dV_{n}}{dt} = I_{n} - I_{n+1}
\end{equation}

By substituting a plane-wave solution $V_{n} = V_{0}e^{i(\omega t - nk\Dx)}$, we obtain the discrete dispersion relation for the vacuum substrate:
\begin{equation}
\omega(k) = \frac{2}{\sqrt{\Lvac\Cvac}} \sin\left(\frac{k\Dx}{2}\right)
\end{equation}

The Group Velocity ($v_{g}$), representing the speed of energy propagation through the hardware nodes, is:
\begin{equation}
v_{g} = \frac{d\omega}{dk} = \frac{\Dx}{\sqrt{\Lvac\Cvac}} \cos\left(\frac{k\Dx}{2}\right)
\end{equation}

Defining the global speed limit $c = \Dx/\sqrt{\Lvac\Cvac}$ and the Nyquist limit $\Wcut = 2/\sqrt{\Lvac\Cvac}$, we recover the \textbf{Lindblom Dispersion Relation}:
\begin{equation}
v_{g}(\omega) = c\sqrt{1 - \left(\frac{\omega}{\Wcut}\right)^{2}}
\end{equation}

\subsection{The Mechanical Origin of Lorentz Scaling}
Equation 2.4 is functionally identical to the relativistic velocity addition formula. In LCT, the Lorentz factor $\gamma$ is not a geometric artifact but the \textbf{Bandwidth Proximity Factor}. As a signal's frequency $\omega$ approaches the hardware's $\Wcut$, the lattice nodes require more "cycles" to update their state, slowing the group velocity.

\subsection{Identifying Rest Mass: The Back-EMF Effect}
When $\omega = \Wcut$, $v_{g}$ vanishes. The signal becomes a localized standing wave—a \textbf{Soliton}. 
\begin{itemize}
    \item \textbf{Inertia}: The resistance of this standing wave to acceleration is the mechanical \textbf{Back-EMF} generated by the lattice inductors attempting to shift the phase of a saturated node.
    \item \textbf{$E=mc^2$}: The rest energy is the total potential energy stored in the lattice capacitance $C$ at the saturation frequency.
\end{itemize}

\section{Gravity as Metric Refraction}
General Relativity's "curvature" is recast as the mechanical strain ($\epsmn$) of the hardware components. Gravity is not a force, but a \textbf{Refractive Index Gradient} in the vacuum substrate.

\subsection{The LCT Strain Tensor}
A massive object (a region of high-frequency flux) imposes a stress load on the surrounding lattice. We define the metric strain as a local modification of the lattice inductance:
\begin{equation}
\epsmn \approx \frac{\Delta\Lvac}{\Lvac}
\end{equation}

This increase in inductance raises the local \textbf{Characteristic Impedance} ($\Zvac$) and lowers the local speed of light $c(r)$. Light "bends" toward regions of high $\Lvac$ just as it bends toward glass in optics.

\section{Numerical Verification: Gravitational Lensing}
To validate this, we simulate a wavefront passing a high-impedance "mass" region using the FDTD solver established in the core modules.

\begin{simbox}[Metric Refraction and Lensing]
As verified in \texttt{sim\_2\_metric\_refraction.py}, modulating the local phase velocity $v(r)$ according to a $1/r$ strain profile produces the exact geodesic bending predicted by the Schwarzschild metric.
\begin{center}
    \includegraphics[width=0.8\textwidth]{assets/sim_outputs/metric_refraction_result.png}
\end{center}
The simulation proves that "curvature" is an emergent effect of signal delay in a strained hardware lattice.
\end{simbox}

\section{Worked Example: The Schwarzschild Impedance}
\begin{examplebox}[Calculating Vacuum Index of Refraction]
Calculate the effective refractive index $n$ at the surface of a neutron star.
\begin{enumerate}
    \item \textbf{Mass Load}: Define the strain $\epsilon = \frac{2GM}{rc^2}$.
    \item \textbf{Impedance Shift}: The local impedance becomes $Z(r) = Z_0(1 + \epsilon)$.
    \item \textbf{Refractive Index}: $n(r) = \frac{c_{vacuum}}{c_{local}} = \sqrt{(1+\epsilon)^2}$. For a neutron star, $n \approx 1.2$, meaning the vacuum hardware is "optically" denser near the mass.
\end{enumerate}
\end{examplebox}

\section{Exhaustive Problems and Exercises}
\begin{problembox}[Chapter 2 Signal Dynamics]
\begin{enumerate}
    \item \textbf{Frequency Shift}: Prove that a signal entering a region of high strain $\epsilon$ undergoes a frequency redshift to maintain energy conservation across the impedance mismatch.
    \item \textbf{The Black Hole Limit}: Calculate the strain $\epsilon$ required to make $v_g = 0$ for all frequencies. This defines the LCT "Event Horizon" as a \textbf{Total Internal Reflection} boundary.
    \item \textbf{Time Dilation}: Show that the delay in nodal updates in a strained lattice ($dt' = dt(1+\epsilon)$) recovers the gravitational time dilation formula.
\end{enumerate}
\end{problembox}

\section{Transition to the Quantum Layer}
With the origin of mass and gravity established as hardware signal delays, we move to the \textbf{Quantum Layer} (Chapter 3) to investigate the deterministic "jitter" of the lattice nodes and the emergence of pilot-wave hydrodynamics.