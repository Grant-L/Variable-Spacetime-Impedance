\chapter{The Signal Layer: Variable Impedance and Mass Emergence}
\label{ch:signal_layer}

\section{The Lindblom Dispersion Relation}
In Chapter 1, we established the vacuum as a discrete LC lattice. We now derive the relationship between signal frequency and propagation velocity, identifying the mechanical origin of rest mass and relativistic scaling as a direct result of hardware bandwidth limitations.

\subsection{Derivation from Discrete Kirchhoff Laws}
Starting from the discrete equations of motion defined in Section 1.3:
\begin{equation}
\Lvac \frac{dI_{n}}{dt} = V_{n-1} - V_{n}, \quad \Cvac \frac{dV_{n}}{dt} = I_{n} - I_{n+1}
\end{equation}

By substituting a plane-wave solution $V_{n} = V_{0}e^{i(\omega t - nk\Dx)}$, we obtain the discrete dispersion relation for the vacuum substrate:
\begin{equation}
\omega(k) = \frac{2}{\sqrt{\Lvac\Cvac}} \sin\left(\frac{k\Dx}{2}\right)
\end{equation}

The Group Velocity ($v_{g}$), representing the speed of energy propagation through the hardware nodes, is:
\begin{equation}
v_{g} = \frac{d\omega}{dk} = \frac{\Dx}{\sqrt{\Lvac\Cvac}} \cos\left(\frac{k\Dx}{2}\right)
\end{equation}

Defining the global speed limit $c = \Dx/\sqrt{\Lvac\Cvac}$ and the Nyquist limit $\Wcut = 2/\sqrt{\Lvac\Cvac}$, we recover the \textbf{Lindblom Dispersion Relation}:
\begin{equation}
v_{g}(\omega) = c\sqrt{1 - \left(\frac{\omega}{\Wcut}\right)^{2}}
\end{equation}

\subsection{The Mechanical Origin of Lorentz Scaling}
Equation 2.4 is functionally identical to the relativistic velocity addition formula. In LCT, the Lorentz factor $\gamma$ is not a geometric artifact but the \textbf{Bandwidth Proximity Factor}. As a signal's frequency $\omega$ approaches the hardware's $\Wcut$, the lattice nodes require more "cycles" to update their state, slowing the group velocity.

\section{Identifying Rest Mass: The Back-EMF Effect}
\label{sec:mass_emergence}

In \LCT{}, rest mass is not an intrinsic "property" of matter, but a result of \textbf{Bandwidth Saturation} in the lattice hardware[cite: 57, 138]. As a signal's frequency $\omega$ approaches the \Wcut{}, its group velocity $v_g$ vanishes, transforming the propagating wave into a localized \textbf{Bouncing Soliton}[cite: 156, 165, 257].

\begin{axiombox}[The Origin of Inertia]
Inertia is the mechanical \textbf{Back-EMF} generated by the lattice inductors as they resist changes in phase flux within a saturated node[cite: 40, 48, 166]. To accelerate a soliton, the hardware must shift the phase of a node already operating at its maximum update frequency.
\end{axiombox}

The rest energy $E=mc^2$ is the total potential energy stored in the nodal capacitance $C$ at the saturation threshold[cite: 167, 614]:
\begin{equation}
    m_e = \frac{\hbar \omega_{cutoff}}{c^2} \cdot \Phi_{geo}
\end{equation}
Where $\Phi_{geo}$ is the geometric scaling factor determined by the mean connectivity $\langle k \rangle$ of the amorphous mesh[cite: 133]. This identifies the electron as the \textit{fundamental resonant mode} of the vacuum glass[cite: 271, 276].

\section{Gravity as Metric Refraction}
\label{sec:gravity_refraction}

General Relativity's "curvature" is recast as the mechanical strain ($\epsilon_{\mu\nu}$) of the hardware components[cite: 28, 154]. Gravity is not a force, but a \textbf{Refractive Index Gradient} in the vacuum substrate[cite: 155].

\subsection{The LCT Strain Tensor}
A massive object—a region of high-frequency flux—imposes a stress load on the surrounding lattice[cite: 157]. We define the metric strain as a local modification of the lattice inductance:
\begin{equation}
    \epsilon_{\mu\nu} \approx \frac{\Delta\mathcal{L}}{\mathcal{L}}
\end{equation}
This increase in inductance raises the local Characteristic Impedance ($Z_0$) and lowers the local phase velocity $v(r)$[cite: 161]. 

\begin{figure}[h]
    \centering
    \includegraphics[width=0.7\textwidth]{assets/figures/refractive_gradient.png}
    \caption{LCT Metric Refraction: Light "bends" toward regions of high inductance (L) just as it bends toward glass in optics[cite: 162].}
\end{figure}

\section{Numerical Verification: Gravitational Lensing}
To validate this, we simulate a wavefront passing a high-impedance "mass" region using the FDTD solver established in the core modules[cite: 164, 165].

\begin{simbox}[Verification of Schwarzschild Impedance]
    As verified in \texttt{sim\_2\_metric\_refraction\_v2.py}, the dual-pane visualization in Figure 2.2 proves that signal trajectories bend not due to a force, but because the vacuum hardware is "optically" denser near a mass load[cite: 194]. The left pane captures the wavefront delay, while the right pane exposes the underlying \textbf{Impedance Heatmap} ($Z(r)$)[cite: 161, 191].
    \end{simbox}

\begin{figure}[h]
    \centering
    \includegraphics[width=0.9\textwidth]{assets/sim_outputs/impedance_heatmap.png}
    \caption{Split-pane simulation output: Left pane shows wavefront lensing[cite: 175]; Right pane shows the $Z(r)$ Impedance Heatmap illustrating the "optically" denser vacuum near the mass center[cite: 194].}
\end{figure}

\section{Worked Example: The Schwarzschild Impedance}
\label{sec:schwarzschild_impedance}
To calculate the effective refractive index $n$ at the surface of a neutron star[cite: 186]:
\begin{enumerate}
    \item \textbf{Mass Load}: Define the strain $\epsilon = \frac{2GM}{rc^2}$[cite: 189, 190].
    \item \textbf{Impedance Shift}: The local impedance becomes $Z(r) = Z_0(1 + \epsilon)$[cite: 191].
    \item \textbf{Refractive Index}: $n(r) = \frac{c_{vacuum}}{c_{local}} = \sqrt{(1 + \epsilon)^2}$.
\end{enumerate}
For a neutron star, $n \approx 1.2$, meaning signal propagation is slowed by 20\% relative to the vacuum ground state[cite: 192, 194].

\section{Exhaustive Problems and Exercises}
\begin{problembox}[Chapter 2 Signal Dynamics]
\begin{enumerate}
    \item \textbf{Frequency Shift}: Prove that a signal entering a region of high strain $\epsilon$ undergoes a frequency redshift to maintain energy conservation across the impedance mismatch.
    \item \textbf{The Black Hole Limit}: Calculate the strain $\epsilon$ required to make $v_g = 0$ for all frequencies. This defines the LCT "Event Horizon" as a \textbf{Total Internal Reflection} boundary.
    \item \textbf{Time Dilation}: Show that the delay in nodal updates in a strained lattice ($dt' = dt(1+\epsilon)$) recovers the gravitational time dilation formula.
\end{enumerate}
\end{problembox}

\section{Transition to the Quantum Layer}
With the origin of mass and gravity established as hardware signal delays, we move to the \textbf{Quantum Layer} (Chapter 3) to investigate the deterministic "jitter" of the lattice nodes and the emergence of pilot-wave hydrodynamics.