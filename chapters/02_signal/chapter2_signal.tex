\chapter{The Signal Layer: Variable Impedance and Mass Emergence}
\label{ch:signal_layer}

\section{The Vacuum Dispersion Relation}
In Chapter 1, we established the vacuum as a physical transmission medium composed of discrete LC nodes. We now derive the relationship between signal frequency and propagation velocity, identifying the mechanical origin of rest mass and relativistic scaling as a direct result of hardware bandwidth limitations.

\subsection{Derivation from Discrete Kirchhoff Laws}
Starting from the discrete equations of motion for the \textbf{Discrete Amorphous Manifold ($M_A$)}:
\begin{align}
    L_{node} \frac{dI_n}{dt} &= V_{n-1} - V_n \\
    C_{node} \frac{dV_n}{dt} &= I_n - I_{n+1}
\end{align}

By substituting a plane-wave solution $V_n = V_0 e^{i(\omega t - nk\Delta x)}$, we obtain the discrete dispersion relation for the vacuum substrate:
\begin{equation}
    \omega(k) = \frac{2}{\sqrt{L_{node}C_{node}}} \sin\left(\frac{k\Delta x}{2}\right)
\end{equation}

The \textbf{Group Velocity} ($v_g$), representing the speed of energy propagation through the hardware nodes, is:
\begin{equation}
    v_g = \frac{d\omega}{dk} = \frac{\Delta x}{\sqrt{L_{node}C_{node}}} \cos\left(\frac{k\Delta x}{2}\right)
\end{equation}

Defining the global speed limit $c = \Delta x/\sqrt{L_{node}C_{node}}$ and the saturation frequency $\nu_{sat} = 2/\sqrt{L_{node}C_{node}}$, we recover the VSI velocity equation:
\begin{equation}
    v_g = c \sqrt{1 - \left(\frac{\omega}{\omega_{sat}}\right)^2}
\end{equation}

\section{The Mechanical Origin of Lorentz Scaling}
The factor $\sqrt{1 - (v/c)^2}$ (the Lorentz factor) is traditionally a geometric postulate. In SVF, it is a consequence of the \textbf{Bandwidth Limit} of the vacuum nodes. As a topological defect (particle) is accelerated, its internal phase frequency $\omega$ increases. As $\omega \to \omega_{sat}$, the group velocity $v_g$ must drop to zero to satisfy the hardware's energy conservation.



\section{Metric Strain and Gravitational Refraction}
Gravity is not the "curvature" of a void, but a \textbf{Refractive Index Gradient} caused by \textbf{Metric Strain} ($\epsilon$). When a large baryonic mass saturates a region of nodes, the local $L$ and $C$ values are altered due to the physical displacement of the manifold.

\subsection{The Refractive Index of the Vacuum}
We define the local propagation speed $c'$ in a region of metric strain $\epsilon$ as:
\begin{equation}
    c' = \frac{c_{vacuum}}{\chi} = \frac{c_{vacuum}}{1 + \epsilon}
\end{equation}

Where $\chi$ is the \textbf{Dynamic Refractive Index}. Light passing near a massive body slows down because the nodes in that region are "loaded" and require more update cycles to process the same flux. This explains the \textbf{Shapiro Delay} and the bending of light as simple refraction through a variable-impedance medium.

\section{Time Dilation as Lattice Latency}
Time is the rate of nodal updates. In a high-impedance zone (high gravity or high velocity), the nodes spend more "hardware cycles" maintaining the saturation state of mass. Consequently, there are fewer cycles available for signal propagation. 
\begin{equation}
    dt' = dt \sqrt{1 + \frac{2\Phi}{c^2}}
\end{equation}
An observer in a high-strain zone perceives time moving slower because the hardware is running at a higher \textbf{Lattice Latency}.

\section{Exercises}
\begin{problembox}[Chapter 2 Signal Dynamics]
\begin{enumerate}
    \item \textbf{The Black Hole Limit}: Prove that at the "Event Horizon," the metric strain $\epsilon$ is sufficient to force the characteristic impedance $Z \to \infty$. Calculate the resulting $v_g$.
    \item \textbf{Redshift Derivation}: Show that a signal entering a region of high impedance must undergo a frequency shift to maintain phase continuity across the node boundary.
    \item \textbf{Slew Rate Calculation}: Using $\mu_0$ and $\epsilon_0$, calculate the maximum slew rate of a single Planck-scale node in Volts per second.
\end{enumerate}
\end{problembox}