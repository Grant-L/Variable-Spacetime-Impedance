\chapter{2 The Signal Layer: Variable Impedance and Mass Emergence}

\section{2.1 The Lindblom Dispersion Relation}
[cite_start]In Chapter 1, we established the vacuum as a discrete LC lattice[cite: 582, 584]. [cite_start]We now derive the relationship between signal frequency and propagation velocity, identifying the mechanical origin of rest mass as a hardware limitation[cite: 649].

\subsection{2.1.1 Derivation from Discrete Kirchhoff Laws}
[cite_start]Starting from the discrete equations of motion defined by the lattice's fundamental time constant (Eq. 2.1)[cite: 651]:
\begin{equation}
\Lvac \frac{dI_{n}}{dt} = V_{n-1} - V_{n}, \quad \Cvac \frac{dV_{n}}{dt} = I_{n} - I_{n+1}
\end{equation}

[cite_start]Substituting a plane-wave solution $V_{n} = V_{0}e^{i(\omega t - nk\Dx)}$, we obtain the discrete dispersion relation for the vacuum substrate[cite: 652]:
\begin{equation}
\omega(k) = \frac{2}{\sqrt{\Lvac\Cvac}} \sin\left(\frac{k\Dx}{2}\right)
\label{eq:dispersion_relation}
\end{equation}

[cite_start]The Group Velocity ($v_{g}$), representing the speed of energy propagation, is the derivative[cite: 655]:
\begin{equation}
v_{g} = \frac{d\omega}{dk} = \frac{\Dx}{\sqrt{\Lvac\Cvac}} \cos\left(\frac{k\Dx}{2}\right)
\end{equation}

[cite_start]Defining the continuum speed of light as $c = \Dx/\sqrt{\Lvac\Cvac}$ and the cutoff frequency as $\Wcut = 2/\sqrt{\Lvac\Cvac}$, we recover the \textbf{Lindblom Dispersion Relation}[cite: 658]:
\begin{equation}
v_{g}(\omega) = c\sqrt{1 - \left(\frac{\omega}{\Wcut}\right)^{2}}
\label{eq:v_group_final}
\end{equation}



\subsection{2.1.2 Identifying Rest Mass: The Back-EMF Effect}
[cite_start]Equation 2.11 reveals two critical regimes that define the physical nature of energy within the hardware[cite: 662]:
\begin{itemize}
    [cite_start]\item \textbf{Linear Regime ($\omega \ll \Wcut$)}: The lattice appears smooth[cite: 665]; [cite_start]$v_{g} \approx c$[cite: 665]. [cite_start]This is the regime of the photon[cite: 665].
    [cite_start]\item \textbf{Saturation Regime ($\omega \rightarrow \Wcut$)}: As the frequency approaches the Nyquist limit of the LC nodes, $v_{g} \rightarrow 0$[cite: 666]. [cite_start]The energy packet becomes a standing wave[cite: 666].
\end{itemize}

[cite_start]\textbf{Conclusion}: Rest Mass is the state of high-frequency flux trapped by the \textbf{Bandwidth Saturation} of the vacuum lattice[cite: 667]. [cite_start]Inertia is the mechanical \textbf{Back-EMF} generated by the lattice inductors when attempting to shift the phase of this saturated standing wave[cite: 667].

\section{2.2 Gravity as Metric Strain}
[cite_start]General Relativity's "curvature" is recast as the mechanical strain of the hardware components[cite: 669].

\subsection{2.2.1 The LCT Strain Tensor}
[cite_start]A massive object—defined as a localized region of high-frequency saturation—imposes a stress load on the surrounding lattice[cite: 671]. [cite_start]We define the vacuum state using the Strain Tensor $\epsilon_{\mu\nu}$[cite: 672]:
\begin{equation}
\epsilon_{\mu\nu} = \frac{\Delta\Lvac}{\Lvac} \approx \frac{h_{\mu\nu}}{2}
\label{eq:strain_tensor}
\end{equation}

[cite_start]For a static mass $M$, the radial strain $\epsilon_{rr}$ physically stretches the grid nodes ($\Dx$)[cite: 675]:
\begin{equation}
\epsilon_{rr}(r) \approx \frac{2GM}{rc^{2}}
\label{eq:radial_strain}
\end{equation}

[cite_start]This stretch increases the distributed inductance per unit length ($\Lvac' = \Lvac(1+\epsilon)$)[cite: 678]. [cite_start]Because the phase velocity is $v = 1/\sqrt{\Lvac'\Cvac'}$, the speed of light drops near the mass[cite: 678]. [cite_start]Time dilation is therefore the physical lengthening of the signal path through these stretched hardware nodes[cite: 679].

\section{2.3 Reconciling Strain and Sink Flow}
[cite_start]The Schwarzschild metric is recovered by substituting the flow velocity $v_{0}(r) = -\sqrt{2GM/r}$ into the \textbf{Acoustic Metric}[cite: 681, 682]:
\begin{equation}
ds^{2} = -\left(1 - \frac{v_{0}^{2}}{c^{2}}\right)c^{2}dt^{2} + \left(1 - \frac{v_{0}^{2}}{c^{2}}\right)^{-1}dr^{2} + r^{2}d\Omega^{2}
\label{eq:schwarzschild_acoustic}
\end{equation}

\section{2.4 Computational Module: Gravitational Lensing}
[cite_start]By modulating lattice node density according to $\epsilon_{rr}(r)$, the FDTD simulation below demonstrates the wavefront bending toward the mass, matching Einstein's prediction[cite: 685, 686].

\begin{simbox}[Gravitational Lensing]
\begin{lstlisting}[language=Python]
import numpy as np
def simulate_lensing():
    Nx, Ny = 600, 400; Nt = 1200; dt = 0.5
    X, Y = np.meshgrid(np.arange(Nx), np.arange(Ny), indexing='ij')
    # Distance from mass center
    R = np.sqrt((X - Nx//2)**2 + (Y - (Ny//2+50))**2)
    # Metric Strain defines variable speed v = c/n
    n_map = 1.0 + 20.0 / (np.sqrt(R**2 + 10.0))
    v_map = 1.0 / n_map
    u = np.zeros((Nx, Ny)); u_prev = np.zeros((Nx, Ny))
    for t in range(Nt):
        lap = (np.roll(u,1,0) + np.roll(u,-1,0) + np.roll(u,1,1) + np.roll(u,-1,1) - 4*u)
        u_next = 2*u - u_prev + (v_map * dt)**2 * lap
        if t < 100: u_next[5, Ny//2-50] += np.sin(0.6*t)
        u_prev, u = u.copy(), u_next.copy()
    return u
\end{lstlisting}
\end{simbox}

\section{2.5 Exhaustive Problems and Exercises}
\begin{problembox}[Chapter 2 Signal Dynamics]
\begin{enumerate}
    [cite_start]\item \textbf{Group Velocity at Saturation}: Using the Lindblom Dispersion Relation, calculate the group velocity for a signal at $0.99\Wcut$[cite: 704]. [cite_start]If this energy packet represents an electron, what is its Lorentz factor $\gamma$[cite: 704]?
    [cite_start]\item \textbf{Refractive Index of a Black Hole}: Given the Metric Strain $\epsilon_{rr}(r) \approx 2GM/rc^2$, derive the local refractive index $n(r)$ of the vacuum[cite: 705]. [cite_start]Prove that at $r = 2GM/c^2$, the index $n \rightarrow \infty$, causing $v_p \rightarrow 0$ (The Event Horizon)[cite: 706].
    [cite_start]\item \textbf{Energy Packet Momentum}: Using the relation $p = \hbar k$, show that as $\omega \rightarrow \Wcut$, the momentum $p$ becomes singular while $v_g$ vanishes[cite: 707]. [cite_start]Reconcile this with the relativistic definition of momentum[cite: 708].
    [cite_start]\item \textbf{Time Dilation via Signal Path}: Derive the time-dilation factor $\Delta t'$ purely by calculating the delay in signal update rates across a region of metric strain $\epsilon$[cite: 709].
\end{enumerate}
\end{problembox}

\section{2.6 Transition to the Quantum Layer}
[cite_start]Having established how mass and gravity emerge from the hardware constraints of the lattice, we move to the \textbf{Quantum Layer} (Chapter 3) to analyze the hydrodynamic pilot-wave behavior of these high-frequency packets[cite: 711].