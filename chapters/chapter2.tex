\chapter{The Signal Layer: Gravity and Mass}

\section{The Lindblom Dispersion Relation (Mass)}
Standard physics assumes a linear dispersion relation ($E=hf$). LCT applies \textbf{Nyquist Sampling Theory} to the vacuum lattice. As a signal's local excitation rate $\omega$ approaches the resonant frequency of the lattice node ($\omega_{cutoff}$), the Inductive Reactance ($X_L$) becomes non-linear.



\begin{equation}
v_g(\omega) = c \cdot \sqrt{1 - \left(\frac{\omega}{\omega_{cutoff}}\right)^2}
\end{equation}

\begin{itemize}
    \item \textbf{Regime A ($\omega \ll \omega_{cutoff}$):} Linear response. $v_g \approx c$. (Massless Radiation).
    \item \textbf{Regime B ($\omega \to \omega_{cutoff}$):} Saturation. The node's Slew Rate is exceeded. The Group Velocity $v_g \to 0$. The energy packet becomes a localized \textbf{Standing Wave}.
\end{itemize}
\textbf{Conclusion:} Rest Mass is identified as \textbf{High-Frequency Flux trapped by the Bandwidth Limit of the Vacuum.} Inertia is the Back-EMF generated when an external force attempts to change the phase of this standing wave.

\section{Gravity: Adiabatic Impedance Matching}
A massive object loads the surrounding vacuum, creating a smooth gradient of inductance ($\nabla L$).
\begin{equation}
\frac{dZ}{dx} \ll \frac{Z}{\lambda}
\end{equation}
This satisfies the condition for an \textbf{Adiabatic Tapered Transmission Line}. This gradient creates an \textbf{effective refractive index} $n_{eff}(x)$, bending the trajectory to minimize Phase Accumulation (Action).



Because the gradient is adiabatic, \textbf{Reflection is Zero}. Gravity is a lossless refractive process, preserving quantum coherence across cosmic distances.

\section{Deriving the Schwarzschild Metric (Hydrodynamic Limit)}
We model gravity as a radial "sink flow" of the vacuum substrate toward a massive object. The velocity of the vacuum flow $v_0$ is given by:
\begin{equation}
v_0(r) = - \sqrt{\frac{2GM}{r}} \hat{r}
\end{equation}
Substituting this flow field into the acoustic metric line element:
\begin{equation}
ds^2 \approx - \left(1 - \frac{v_0^2}{c^2}\right) c^2 dt^2 + \left(1 - \frac{v_0^2}{c^2}\right)^{-1} dr^2 + r^2 d\Omega^2
\end{equation}
This exactly recovers the **Schwarzschild Metric**, demonstrating that General Relativity is the hydrodynamic limit of a flowing vacuum. The "Event Horizon" corresponds to the radius where the inflow velocity $|v_0|$ exceeds the sound speed $c$ of the lattice.

\section{Computational Module: The Lensing Simulation}
We utilized the Finite-Difference Time-Domain (FDTD) method to simulate a photon pulse passing through a gravitational potential.
\begin{itemize}
    \item **Setup:** A 2D lattice with a variable refractive index $n(r) = 1 + GM/rc^2$.
    \item **Result:** The pulse wavefront bent toward the mass center exactly matching the predicted deflection angle $\alpha = 4GM/rc^2$.
    \item **Observation:** No back-scattering was observed, confirming the adiabatic nature of the impedance gradient.
\end{itemize}
*(See Appendix B.1 for the full Python source code.)*

\section{Bridge the Gap: From Einstein to Acoustics}
To the General Relativist, gravity is curvature $R_{\mu\nu}$. To the Fluid Dynamicist, gravity is a velocity potential $\Phi$.
\begin{itemize}
    \item **The Metric:** $g_{\mu\nu}$ is the acoustic metric of the fluid.
    \item **The Horizon:** A sonic boom where escape velocity equals wave speed.
    \item **Lensing:** Refraction through a density gradient.
\end{itemize}
This mapping allows us to simulate Black Holes using analog water table experiments or superfluid helium.