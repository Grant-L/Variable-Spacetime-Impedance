\chapter{The Signal Layer: Gravity and Mass}

\section{The Lindblom Dispersion Relation}
In Chapter 1, we established the vacuum as a discrete LC lattice. We now derive the relationship between signal frequency and propagation velocity, identifying the mechanical origin of mass.

\subsection{Derivation from Discrete Kirchhoff Laws}
Starting from the discrete equations of motion (Eq. 1.1):
\begin{equation}
L \frac{dI_n}{dt} = V_{n-1} - V_n, \quad C \frac{dV_n}{dt} = I_n - I_{n+1}
\end{equation}
Substituting a plane-wave solution $V_n = V_0 e^{i(\omega t - nk \Delta x)}$, we obtain the discrete dispersion relation for the vacuum substrate:
\begin{equation}
\omega(k) = \frac{2}{\sqrt{L_{vac}C_{vac}}} \sin\left(\frac{k \Delta x}{2}\right)
\end{equation}
The \textbf{Group Velocity} ($v_g$), representing the speed of energy propagation, is the derivative:
\begin{equation}
v_g = \frac{d\omega}{dk} = \frac{\Delta x}{\sqrt{L_{vac}C_{vac}}} \cos\left(\frac{k \Delta x}{2}\right)
\end{equation}
Defining the continuum speed of light as $c = \Delta x / \sqrt{L_{vac}C_{vac}}$ and the cutoff frequency as $\omega_{cutoff} = 2/\sqrt{L_{vac}C_{vac}}$, we recover the \textbf{Lindblom Dispersion Relation}:
\begin{equation}
v_g(\omega) = c \sqrt{1 - \left(\frac{\omega}{\omega_{cutoff}}\right)^2}
\label{eq:dispersion}
\end{equation}

\subsection{Identifying Rest Mass}
Equation \ref{eq:dispersion} reveals two critical regimes:
\begin{itemize}
    \item \textbf{Linear Regime ($\omega \ll \omega_{cutoff}$):} The lattice appears smooth; $v_g \approx c$ (Photon behavior).
    \item \textbf{Saturation Regime ($\omega \rightarrow \omega_{cutoff}$):} As the excitation frequency approaches the Nyquist limit of the nodes, $v_g \rightarrow 0$. The energy packet is unable to propagate and becomes a \textbf{Standing Wave}.
\end{itemize}
\textbf{Conclusion:} Rest Mass is not a separate property; it is the state of high-frequency flux trapped by the \textbf{Bandwidth Saturation} of the vacuum lattice. Inertia is the Back-EMF generated by the lattice inductors when attempting to shift the phase of this standing wave.

\section{Gravity as Metric Strain}
General Relativity's "curvature" is recast as the mechanical strain of the vacuum substrate.

\subsection{The LCT Strain Tensor}
A massive object imposes a stress load on the surrounding lattice. We define the vacuum state using the \textbf{Strain Tensor} $\epsilon_{\mu\nu}$:
\begin{equation}
\epsilon_{\mu\nu} = \frac{\Delta L_{vac}}{L_{vac}} \approx \frac{h_{\mu\nu}}{2}
\end{equation}
For a static mass $M$, the radial strain $\epsilon_{rr}$ physically stretches the grid nodes:
\begin{equation}
\epsilon_{rr}(r) \approx \frac{2GM}{rc^2}
\end{equation}
This stretch increases the distributed inductance per unit length ($L' = L_{vac}(1+\epsilon)$). Because the phase velocity is $v = 1/\sqrt{L'C'}$, the speed of light drops near the mass. 

\begin{tcolorbox}[colback=gray!10!white,colframe=black!75!black,title=\textbf{Engineering Note: The Constant Clock}]
The lattice "Update Rate" ($\omega_{vac}$) is invariant. A signal always takes 1 tick to move 1 node. Gravity stretches the nodes; therefore, the signal covers more "physical" distance but takes more "absolute" time. \textit{Time dilation is the lengthening of the signal path.}
\end{tcolorbox}

\section{Reconciling Strain and Sink Flow}
In Section 2.3, we use the "Sink Flow" model ($v_0$) to recover the Schwarzschild metric. To reconcile this with a solid lattice:
\begin{itemize}
    \item \textbf{Metric Strain} describes the \textit{Static Potential} (the stretched state of the grid).
    \item \textbf{Sink Flow} describes the \textit{Phase Velocity} ($v_0 = \nabla S / m$) required to maintain that strain in equilibrium.
\end{itemize}
By substituting the flow velocity $v_0(r) = -\sqrt{2GM/r}$ into the \textbf{Acoustic Metric}:
\begin{equation}
ds^2 = -(1 - \frac{v_0^2}{c^2})c^2dt^2 + (1 - \frac{v_0^2}{c^2})^{-1}dr^2 + r^2d\Omega^2
\end{equation}
We recover the exact Schwarzschild geometry. Gravity is the hydrodynamic limit of light propagating through a strained, flowing vacuum.

\section{Computational Module: The Lensing Simulation}
We verified this via FDTD simulation. By modulating node density according to $\epsilon_{rr}(r)$, we observed the wavefront bend toward the mass, matching Einstein's prediction of $4GM/rc^2$.
*(See Appendix D.1 for source code.)*