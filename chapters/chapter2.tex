\chapter{The Signal Layer: Variable Impedance and Mass Emergence}

\section{2.1 The Lindblom Dispersion Relation}
In Chapter 1, we established the vacuum as a discrete LC lattice. We now derive the relationship between signal frequency and propagation velocity, identifying the mechanical origin of rest mass as a hardware limitation. [cite: 92-94]

\subsection{2.1.1 Derivation from Discrete Kirchhoff Laws}
Starting from the discrete equations of motion defined by the lattice's fundamental time constant (Eq. 1.1): [cite: 60-61, 95-98]
\begin{equation}
\mathcal{L}\frac{dI_{n}}{dt}=V_{n-1}-V_{n}, \quad \mathcal{C}\frac{dV_{n}}{dt}=I_{n}-I_{n+1}
\end{equation}

Substituting a plane-wave solution $V_{n}=V_{0}e^{i(\omega t-nk\Delta x)}$, we obtain the discrete dispersion relation for the vacuum substrate: [cite: 99-101]
\begin{equation}
\omega(k)=\frac{2}{\sqrt{\mathcal{LC}}}\sin\left(\frac{k\Delta x}{2}\right)
\label{eq:dispersion_relation}
\end{equation}

The Group Velocity ($v_{g}$), representing the speed of energy propagation, is the derivative: [cite: 102-104]
\begin{equation}
v_{g}=\frac{d\omega}{dk}=\frac{\Delta x}{\sqrt{\mathcal{LC}}}\cos\left(\frac{k\Delta x}{2}\right)
\end{equation}

Defining the continuum speed of light as $c = \Delta x/\sqrt{\mathcal{LC}}$ and the cutoff frequency as $\omega_{cutoff} = 2/\sqrt{\mathcal{LC}}$, we recover the \textbf{Lindblom Dispersion Relation}: [cite: 105, 107]
\begin{equation}
v_{g}(\omega)=c\sqrt{1-\left(\frac{\omega}{\omega_{cutoff}}\right)^{2}}
\end{equation}



\subsection{2.1.2 Identifying Rest Mass: The Back-EMF Effect}
Equation 2.4 reveals two critical regimes that define the physical nature of energy within the hardware: [cite: 108, 110-111]
\begin{itemize}
    \item \textbf{Linear Regime ($\omega \ll \omega_{cutoff}$)}: The lattice appears smooth; $v_{g} \approx c$. This is the regime of the photon. [cite: 110]
    \item \textbf{Saturation Regime ($\omega \rightarrow \omega_{cutoff}$)}: As the frequency approaches the Nyquist limit of the LC nodes, $v_{g} \rightarrow 0$. The energy packet becomes a standing wave. [cite: 111]
\end{itemize}

\textbf{Conclusion}: Rest Mass is the state of high-frequency flux trapped by the \textbf{Bandwidth Saturation} of the vacuum lattice. [cite: 112] Inertia is the mechanical \textbf{Back-EMF} generated by the lattice inductors when attempting to shift the phase of this saturated standing wave. [cite: 113]

\section{2.2 Gravity as Metric Strain}
General Relativity's "curvature" is recast as the mechanical strain of the hardware components. [cite: 116, 118]

\subsection{2.2.1 The LCT Strain Tensor}
A massive object imposes a stress load on the surrounding lattice. We define the vacuum state using the Strain Tensor $\epsilon_{\mu\nu}$: [cite: 119-121]
\begin{equation}
\epsilon_{\mu\nu} = \frac{\Delta \mathcal{L}}{\mathcal{L}} \approx \frac{h_{\mu\nu}}{2}
\end{equation}

For a static mass $M$, the radial strain $\epsilon_{rr}$ physically stretches the grid nodes ($\Delta x$): [cite: 122-123]
\begin{equation}
\epsilon_{rr}(r) \approx \frac{2GM}{rc^{2}}
\end{equation}

This stretch increases the distributed inductance per unit length ($L' = \mathcal{L}(1+\epsilon)$). [cite: 126] Because the phase velocity is $v = 1/\sqrt{L'C'}$, the speed of light drops near the mass. [cite: 127] Time dilation is the physical lengthening of the signal path. [cite: 131]

\section{2.3 Reconciling Strain and Sink Flow}
The Schwarzschild metric is recovered by substituting the flow velocity $v_{0}(r) = -\sqrt{2GM/r}$ into the \textbf{Acoustic Metric}: [cite: 133, 137-140]
\begin{equation}
ds^{2}=-\left(1-\frac{v_{0}^{2}}{c^{2}}\right)c^{2}dt^{2}+\left(1-\frac{v_{0}^{2}}{c^{2}}\right)^{-1}dr^{2}+r^{2}d\Omega^{2}
\end{equation}

\section{2.4 Computational Module: Gravitational Lensing}
By modulating lattice node density according to $\epsilon_{rr}(r)$, the FDTD simulation below demonstrates the wavefront bending toward the mass, matching Einstein's prediction. 

\begin{verbatim}
import numpy as np
def simulate_lensing():
    Nx, Ny = 600, 400; Nt = 1200; dt = 0.5
    X, Y = np.meshgrid(np.arange(Nx), np.arange(Ny), indexing='ij')
    # Distance from mass center
    R = np.sqrt((X - Nx//2)**2 + (Y - (Ny//2+50))**2)
    # Metric Strain defines variable speed v = c/n
    n_map = 1.0 + 20.0 / (np.sqrt(R**2 + 10.0))
    v_map = 1.0 / n_map
    u = np.zeros((Nx, Ny)); u_prev = np.zeros((Nx, Ny))
    for t in range(Nt):
        lap = (np.roll(u,1,0) + np.roll(u,-1,0) + np.roll(u,1,1) + np.roll(u,-1,1) - 4*u)
        u_next = 2*u - u_prev + (v_map * dt)**2 * lap
        if t < 100: u_next[5, Ny//2-50] += np.sin(0.6*t)
        u_prev, u = u.copy(), u_next.copy()
    return u
\end{verbatim}