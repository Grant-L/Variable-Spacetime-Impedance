\chapter{The Signal Layer: Gravity and Mass}

\section{The Lindblom Dispersion Relation (Mass)}
Standard physics assumes a linear dispersion relation ($E=hf$). LCT applies \textbf{Nyquist Sampling Theory} to the vacuum lattice. As a signal's local excitation rate $\omega$ approaches the resonant frequency of the lattice node ($\omega_{cutoff}$), the Inductive Reactance ($X_L$) becomes non-linear.
\begin{equation}
v_g(\omega) = c \cdot \sqrt{1 - \left(\frac{\omega}{\omega_{cutoff}}\right)^2}
\end{equation}
As $v_g \to 0$, the energy packet becomes a localized \textbf{Standing Wave} (Rest Mass). Inertia is the Back-EMF generated when an external force attempts to change the phase of this standing wave.

\section{Gravity as Metric Strain}
Standard General Relativity describes gravity as geometric curvature. In the LCT hardware framework, we describe it as **Metric Strain** ($\varepsilon$) of the vacuum lattice.
A massive object imposes a stress load on the surrounding vacuum substrate. Because the lattice behaves as an elastic solid, it responds with a radial strain field:
\begin{equation}
\varepsilon_{rr}(r) = \frac{\Delta L_{vac}}{L_{vac}} \approx \frac{2GM}{rc^2}
\end{equation}
This strain physically stretches the grid spacing. To a photon traveling through this region, the increased inductance per unit length ($L' = L_{vac}(1+\varepsilon)$) manifests as a slower propagation velocity.

\section{Deriving the Schwarzschild Metric (Hydrodynamic Limit)}
We model gravity as a radial "sink flow" of the vacuum substrate toward a massive object. The velocity of the vacuum flow $v_0$ is given by:
\begin{equation}
v_0(r) = - \sqrt{\frac{2GM}{r}} \hat{r}
\end{equation}
Substituting this flow field into the acoustic metric line element:
\begin{equation}
ds^2 \approx - \left(1 - \frac{v_0^2}{c^2}\right) c^2 dt^2 + \left(1 - \frac{v_0^2}{c^2}\right)^{-1} dr^2 + r^2 d\Omega^2
\end{equation}
This exactly recovers the **Schwarzschild Metric**, demonstrating that General Relativity is the hydrodynamic limit of a flowing, strained vacuum.

\section{Problems}
\begin{enumerate}
    \item \textbf{Metric Strain:} A massive object induces a radial strain $\varepsilon_{rr}$ on the lattice. Derive the relationship between this strain and the effective refractive index $n(r)$ assuming an isotropic elastic solid.
    \item \textbf{The Event Horizon:} Using the "Sink Flow" model $v(r) = \sqrt{2GM/r}$, calculate the radius $R_s$ at which the vacuum flow velocity equals the lattice sound speed $c_s$. Compare this to the Schwarzschild radius.
    \item \textbf{Lensing Angle:} A photon passes a mass $M$ with impact parameter $b$. Calculate the deflection angle $\alpha$ using the strain gradient $\nabla \varepsilon$.
\end{enumerate}