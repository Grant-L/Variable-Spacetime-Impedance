\chapter{The Signal Layer: Gravity and Mass}

\section{The Lindblom Dispersion Relation (Mass)}
Standard physics assumes a linear dispersion relation ($E=hf$). LCT applies \textbf{Nyquist Sampling Theory} to the vacuum lattice[cite: 94]. As a signal's local excitation rate $\omega$ approaches the resonant frequency of the lattice node ($\omega_{cutoff}$), the Inductive Reactance ($X_L$) becomes non-linear[cite: 95].



\begin{equation}
v_g(\omega) = c \cdot \sqrt{1 - \left(\frac{\omega}{\omega_{cutoff}}\right)^2}
\end{equation}

\begin{itemize}
    \item \textbf{Regime A ($\omega \ll \omega_{cutoff}$):} Linear response. $v_g \approx c$. (Massless Radiation) [cite: 96].
    \item \textbf{Regime B ($\omega \to \omega_{cutoff}$):} Saturation. The node's Slew Rate is exceeded. The Group Velocity $v_g \to 0$. The energy packet becomes a localized \textbf{Standing Wave} (Rest Mass)[cite: 98].
\end{itemize}
\textbf{Conclusion:} Rest Mass is identified as \textbf{High-Frequency Flux trapped by the Bandwidth Limit of the Vacuum}[cite: 98]. Inertia is the Back-EMF generated when an external force attempts to change the phase of this standing wave[cite: 99].

\section{Gravity as Metric Strain}
Standard General Relativity describes gravity as geometric curvature. In the LCT hardware framework, we describe it as \textbf{Metric Strain} ($\varepsilon$) of the vacuum lattice[cite: 101]. 
A massive object imposes a stress load on the surrounding vacuum substrate[cite: 102]. Because the lattice behaves as an elastic solid, it responds with a radial strain field[cite: 103]:
\begin{equation}
\varepsilon_{rr}(r) = \frac{\Delta L_{vac}}{L_{vac}} \approx \frac{2GM}{rc^2}
\end{equation}
This strain physically stretches the grid spacing[cite: 106]. To a photon traveling through this region, the increased inductance per unit length ($L' = L_{vac}(1+\varepsilon)$) manifests as a slower propagation velocity[cite: 106].

\begin{tcolorbox}[colback=gray!10!white,colframe=black!75!black,title=\textbf{Engineering Note: The Constant Clock}]
    In the Lindblom Coupling Theory, the "Update Rate" of the vacuum lattice ($\omega_{vac} = 1/\sqrt{L_{vac}C_{vac}}$) is an invariant constant. A photon \textit{always} takes 1 "tick" to traverse 1 "node."
    
    \textbf{Why does Time Dilation occur?}
    Gravity strains the lattice, physically increasing the distance and inductance between nodes ($L' > L_{vac}$).
    \begin{itemize}
        \item \textbf{The Observer:} Sees the photon moving slower ($v < c$) because it has to charge a larger inductance per unit length.
        \item \textbf{The Photon:} Experiences no change in local time. It is simply traversing a circuit with a higher Impedance Density.
    \end{itemize}
    \textit{Gravity is not the slowing of time; it is the lengthening of the signal path.}
    \end{tcolorbox}

\section{Deriving the Schwarzschild Metric (Hydrodynamic Limit)}
We model gravity as a radial "sink flow" of the vacuum substrate toward a massive object[cite: 108]. The velocity of the vacuum flow $v_0$ is given by[cite: 109, 110]:
\begin{equation}
v_0(r) = - \sqrt{\frac{2GM}{r}} \hat{r}
\end{equation}
Substituting this flow field into the acoustic metric line element[cite: 110, 112]:
\begin{equation}
ds^2 \approx - \left(1 - \frac{v_0^2}{c^2}\right) c^2 dt^2 + \left(1 - \frac{v_0^2}{c^2}\right)^{-1} dr^2 + r^2 d\Omega^2
\end{equation}
This exactly recovers the \textbf{Schwarzschild Metric}, demonstrating that General Relativity is the hydrodynamic limit of a flowing, strained vacuum[cite: 117].

\section{Computational Module: The Lensing Simulation}
We utilized the Finite-Difference Time-Domain (FDTD) method to simulate a photon pulse passing through a strained lattice[cite: 118, 119].
\begin{itemize}
    \item \textbf{Setup:} A 2D lattice where the node density varies according to the strain field $\varepsilon(r)$[cite: 120].
    \item \textbf{Result:} The pulse wavefront bent toward the mass center exactly matching the predicted deflection angle $\alpha = 4GM/rc^2$[cite: 121].
\end{itemize}
*(See Appendix D.1 for the full Python source code.)* [cite: 122]

\subsection{Strong Lensing and the Photon Sphere}
While weak gravity causes minor deflection, the impedance gradient near a Black Hole is so steep that it can trap light.
\begin{itemize}
    \item \textbf{Refractive Index:} We model the Black Hole not as a hole in space, but as a region of extreme optical density: $n(r) \approx (1 - R_s/r)^{-1}$.
    \item \textbf{The Photon Sphere:} At $r = 1.5 R_s$, the impedance gradient perfectly balances the centrifugal force of the photon, allowing light to orbit the mass.
\end{itemize}

\begin{figure}[H]
    \centering
    \includegraphics[width=1.0\textwidth]{sim_black_hole_geodesics.png}
    \caption{\textbf{Simulation D.9: Black Hole Geodesics.} Light rays (Green) passing a Black Hole (Black Dot). Far rays are weakly lensed, while rays crossing the Photon Sphere (Orange Dashed Line) are captured by the diverging refractive index. This replicates General Relativity's "curved spacetime" using only variable vacuum impedance.}
    \label{fig:black_hole_sim}
\end{figure}

\section{Bridge the Gap: From Geometry to Elasticity}
To the General Relativist, gravity is \textbf{Curvature}. To the Mechanical Engineer, gravity is \textbf{Strain}.
\begin{itemize}
    \item \textbf{The Metric ($g_{\mu\nu}$):} The Strain Tensor of the vacuum solid[cite: 125].
    \item \textbf{Geodesics:} The path of least action through a variable-density medium[cite: 126].
    \item \textbf{Gravitational Waves:} Phonons propagating through the lattice stiffness[cite: 127].
\end{itemize}

\section{Problems}
\begin{enumerate}
    \item \textbf{Metric Strain:} A massive object induces a radial strain $\varepsilon_{rr}$ on the lattice[cite: 130]. Derive the relationship between this strain and the effective refractive index $n(r)$ assuming an isotropic elastic solid[cite: 131].
    \item \textbf{The Event Horizon:} Using the "Sink Flow" model $v(r) = \sqrt{2GM/r}$, calculate the radius $R_s$ at which the vacuum flow velocity equals the lattice sound speed $c_s$[cite: 132]. Compare this to the Schwarzschild radius[cite: 133].
    \item \textbf{Lensing Angle:} A photon passes a mass $M$ with impact parameter $b$[cite: 134]. Calculate the deflection angle $\alpha$ using the strain gradient $\nabla \varepsilon$[cite: 135].
\end{enumerate}