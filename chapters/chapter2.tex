% --- Chapter 2: Vacuum Impedance and Transmission Lines ---

In Chapter 1, we derived the unified equations of motion from a theoretical action principle. In this chapter, we transition to the \textbf{Hardware Layer} of the universe. We model the vacuum not as a geometric void, but as a high-frequency \textbf{Discrete 3D Transmission Line}.

By applying the principles of Radio Frequency (RF) engineering and Elasticity Theory to the lattice, we demonstrate that the "constants" of nature ($\epsilon_0$, $\mu_0$, $c$) are actually the constitutive parameters of a distributed LC network.

\section{The Substrate Topology: The 3D LC Network}
We define the \textbf{Vacuum Substrate} at the microscopic scale as a cubic lattice of resonant LC nodes. Each node acts as a discrete oscillator with a characteristic inductance $L_{vac}$ and capacitance $C_{vac}$.



\section{Constitutive Relations and Vacuum Impedance}
The electromagnetic properties of the substrate are dictated by its constitutive parameters. The \textbf{Characteristic Impedance} of the vacuum ($Z_0$) is derived from the ratio of the lattice's inductive and capacitive reactances:

\begin{equation}
Z_0 = \sqrt{\frac{L_{vac}}{C_{vac}}} = \sqrt{\frac{\mu_0}{\epsilon_0}} \approx 376.73 \, \Omega
\end{equation}

In this framework, the speed of light $c$ is the \textbf{Phase Velocity} of a signal traveling through this distributed network:
\begin{equation}
    c = \frac{1}{\sqrt{L_{vac} C_{vac}}}
\end{equation}

\section{Mass as Bandwidth Saturation}
One of the most profound departures from classical physics in LCT is the definition of mass. We apply \textbf{Nyquist Sampling Theory} to the vacuum substrate.

As the local excitation frequency $\omega$ of a signal approaches the resonant cutoff frequency of the lattice node ($\omega_{sat}$), the inductive reactance becomes non-linear. The Group Velocity ($v_g$) of the signal becomes dispersive:

\begin{equation}
v_g(\omega) = c \cdot \sqrt{1 - \left(\frac{\omega}{\omega_{sat}}\right)^2}
\end{equation}

Thus, \textbf{Inertial Mass} is not an intrinsic property of "matter"; it is the physical manifestation of \textbf{Bandwidth Saturation} in the substrate.

\section{Effective Metric Elasticity (Gravity)}
Standard General Relativity describes gravity as geometric curvature. LCT describes it as **Metric Strain**. A massive object places a stress load on the vacuum lattice. Because the lattice is an elastic solid, this stress creates a strain field:
\begin{equation}
    \epsilon_{ij} = \frac{1}{2} (\partial_i u_j + \partial_j u_i)
\end{equation}
This strain dilates the grid spacing. A photon (or gravitational wave) traveling through this strained region must traverse more lattice nodes to cover the same "distance."

\subsection{Lorentz Invariance and GW170817}
Critically, because the vacuum acts as a **Relativistic Solid**, the shear modulus $\mu$ and bulk modulus $\chi$ are coupled such that the propagation speed of transverse waves (light) and shear waves (gravity) remains identical:
\begin{equation}
    c_g = c_{em} = \sqrt{\frac{\mu}{\rho}}
\end{equation}
This ensures compliance with the **GW170817** observation, where gravitational waves and gamma rays arrived simultaneously.

\section{Computational Module: Simulating Metric Strain}
We simulate the strain field around a massive object. The "Grid Dilation" represents the effective Shapiro delay experienced by signals.

\begin{lstlisting}[language=Python, caption=Simulating Gravitational Strain Field]
import numpy as np
import matplotlib.pyplot as plt

def run_strain_sim():
    # Grid Setup
    x = np.linspace(-10, 10, 100)
    y = np.linspace(-10, 10, 100)
    X, Y = np.meshgrid(x, y)
    
    # Mass at center creates Stress
    R = np.sqrt(X**2 + Y**2)
    # Strain field decays as 1/R (simplified)
    Strain = 1.0 / (R + 1.0)
    
    plt.figure(figsize=(6,4))
    plt.pcolormesh(X, Y, Strain, shading='auto', cmap='viridis')
    plt.colorbar(label='Metric Strain $\epsilon$')
    plt.title("Effective Metric Elasticity (Gravity)")
    plt.xlabel("x")
    plt.ylabel("y")
    plt.savefig('strain_sim.png', dpi=300)

if __name__ == "__main__":
    run_strain_sim()
\end{lstlisting}

\begin{figure}[h]
    \centering
    \includegraphics[width=0.7\textwidth]{sim_a_metric.png}
    \caption{\textbf{Gravity as Strain.} The simulation visualizes the dilation of the lattice around a massive object. Signals passing through the yellow/green regions (high strain) experience an effective delay, reproducing the phenomenology of curved spacetime.}
\end{figure}

\section{Experimental Falsification: The Impedance Sideband Test}
LCT predicts that a rotating mass creates a rotating strain vortex. We propose utilizing a Superconducting Niobium Microwave Cavity ($Q > 10^{11}$) to detect these fluctuations as \textbf{Impedance Sidebands} at $-145$ dBc.

\section*{Bridge the Gap: Multidisciplinary Links}
\begin{itemize}
    \item \textbf{For the Engineer:} Gravity is \textbf{Mechanical Strain} on the PCB. If you bend the board (Space), the trace length increases, and the signal takes longer to arrive.
    \item \textbf{For the Physicist:} This is **Elastic Gravity**. The vacuum has a Shear Modulus. $G_{\mu\nu}$ is the Strain Tensor of the substrate.
\end{itemize}