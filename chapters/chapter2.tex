\section{Effective Metric Elasticity (Gravity)}
In Chapter 1, we derived gravity as an acoustic metric. Here, we refine that definition using \textbf{Lattice Elasticity}.

Standard General Relativity describes gravity as geometric curvature. LCT describes it as **Metric Strain**. A massive object places a stress load on the vacuum lattice. Because the lattice is an elastic solid, this stress creates a strain field:
\begin{equation}
    \epsilon_{ij} = \frac{1}{2} (\partial_i u_j + \partial_j u_i)
\end{equation}
This strain dilates the grid spacing. A photon (or gravitational wave) traveling through this strained region must traverse more lattice nodes to cover the same "distance."

\subsection{Lorentz Invariance and GW170817}
Critically, because the vacuum acts as a **Relativistic Solid**, the shear modulus $\mu$ and bulk modulus $\chi$ are coupled such that the propagation speed of transverse waves (light) and shear waves (gravity) remains identical:
\begin{equation}
    c_g = c_{em} = \sqrt{\frac{\mu}{\rho}}
\end{equation}
This ensures compliance with the **GW170817** observation, where gravitational waves and gamma rays arrived simultaneously. Gravity does not "slow down" light relative to gravity; it dilates the metric for both equally.