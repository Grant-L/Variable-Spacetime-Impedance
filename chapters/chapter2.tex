\chapter{The Signal Layer: Variable Impedance and Mass Emergence}
\label{ch:signal_layer}

\section{2.1 The Lindblom Dispersion Relation}
In Chapter 1, we established the vacuum as a discrete LC lattice[cite: 1056, 1060]. We now derive the relationship between signal frequency and propagation velocity, identifying the mechanical origin of rest mass as a hardware limitation[cite: 1124, 1125].

\subsection{2.1.1 Derivation from Discrete Kirchhoff Laws}
Starting from the discrete equations of motion defined by the lattice's fundamental time constant[cite: 1074, 1127]:
\begin{equation}
\Lvac \frac{dI_{n}}{dt} = V_{n-1} - V_{n}, \quad \Cvac \frac{dV_{n}}{dt} = I_{n} - I_{n+1} \quad (4.1)
\end{equation}

Substituting a plane-wave solution $V_{n} = V_{0}e^{i(\omega t - nk\Dx)}$, we obtain the discrete dispersion relation for the vacuum substrate[cite: 1128, 1129]:
\begin{equation}
\omega(k) = \frac{2}{\sqrt{\Lvac\Cvac}} \sin\left(\frac{k\Dx}{2}\right) \quad (4.2)
\end{equation}

The Group Velocity ($v_{g}$), representing the speed of energy propagation, is the derivative[cite: 1131, 1133]:
\begin{equation}
v_{g} = \frac{d\omega}{dk} = \frac{\Dx}{\sqrt{\Lvac\Cvac}} \cos\left(\frac{k\Dx}{2}\right) \quad (4.3)
\end{equation}

Defining $c = \Dx/\sqrt{\Lvac\Cvac}$ and $\Wcut = 2/\sqrt{\Lvac\Cvac}$, we recover the \textbf{Lindblom Dispersion Relation}[cite: 1135, 1136]:
\begin{equation}
v_{g}(\omega) = c\sqrt{1 - \left(\frac{\omega}{\Wcut}\right)^{2}} \quad (4.4)
\end{equation}

\subsection{2.1.2 Identifying Rest Mass: The Back-EMF Effect}
Equation 4.4 reveals two critical regimes[cite: 1139, 1141]:
\begin{itemize}
    \item \textbf{Linear Regime} ($\omega \ll \Wcut$): The lattice appears smooth; $v_{g} \approx c$. This is the regime of the photon[cite: 1143].
    \item \textbf{Saturation Regime} ($\omega \rightarrow \Wcut$): As the frequency approaches the Nyquist limit, $v_{g} \rightarrow 0$. The energy packet becomes a standing wave[cite: 1144, 1145].
\end{itemize}

\textbf{Conclusion}: Rest Mass is high-frequency flux trapped by \textbf{Bandwidth Saturation}[cite: 1087, 1146]. Inertia is the mechanical \textbf{Back-EMF} generated by the lattice inductors when attempting to shift the phase of this saturated standing wave[cite: 1146].

\section{\texorpdfstring{2.2 Gravity as Metric Strain ($\epsilon$)}{2.2 Gravity as Metric Strain}}
General Relativity's "curvature" is recast as the mechanical strain of the hardware components[cite: 1148].

\subsection{2.2.1 The LCT Strain Tensor}
A massive object imposes a stress load on the surrounding lattice[cite: 1150]. We define the vacuum state using the Strain Tensor $\epsilon_{\mu\nu}$[cite: 1151, 1152]:
\begin{equation}
\epsilon_{\mu\nu} = \frac{\Delta\Lvac}{\Lvac} \approx \frac{h_{\mu\nu}}{2} \quad (4.5)
\end{equation}

For a static mass $M$, the radial strain $\epsilon_{rr}$ physically stretches the grid nodes[cite: 1154, 1155]:
\begin{equation}
\epsilon_{rr}(r) \approx \frac{2GM}{rc^{2}} \quad (4.6)
\end{equation}

\section{2.3 Reconciling Strain and Sink Flow}
The Schwarzschild metric is recovered by substituting the flow velocity $v_{0}(r) = -\sqrt{2GM/r}$ into the \textbf{Acoustic Metric}[cite: 1160, 1161]:
\begin{equation}
ds^{2} = -\left(1 - \frac{v_{0}^{2}}{c^{2}}\right)c^{2}dt^{2} + \left(1 - \frac{v_{0}^{2}}{c^{2}}\right)^{-1}dr^{2} + r^{2}d\Omega^{2} \quad (4.7)
\end{equation}

\section{2.4 Computational Module: Gravitational Lensing}
By modulating lattice node density according to $\epsilon_{rr}(r)$, the simulation demonstrates wavefront bending[cite: 1164].

\begin{simbox}[Gravitational Lensing]
\begin{lstlisting}[language=Python]
import numpy as np
def simulate_lensing():
    Nx, Ny = 600, 400; Nt = 1200; dt = 0.5
    X, Y = np.meshgrid(np.arange(Nx), np.arange(Ny), indexing='ij')
    R = np.sqrt((X - Nx//2)**2 + (Y - (Ny//2+50))**2)
    n_map = 1.0 + 20.0 / (np.sqrt(R**2 + 10.0))
    v_map = 1.0 / n_map
    u = np.zeros((Nx, Ny)); u_prev = np.zeros((Nx, Ny))
    for t in range(Nt):
        lap = (np.roll(u,1,0) + np.roll(u,-1,0) + np.roll(u,1,1) + np.roll(u,-1,1) - 4*u)
        u_next = 2*u - u_prev + (v_map * dt)**2 * lap
        if t < 100: u_next[5, Ny//2-50] += np.sin(0.6*t)
        u_prev, u = u.copy(), u_next.copy()
    return u
\end{lstlisting}
\end{simbox}

\section{2.5 Exhaustive Problems and Exercises}
\begin{problembox}[Chapter 2 Signal Dynamics]
\begin{enumerate}
    \item \textbf{Group Velocity at Saturation}: Calculate $v_g$ for a signal at $0.99\Wcut$. What is its Lorentz factor $\gamma$? [cite: 1184, 1185]
    \item \textbf{Refractive Index of a Black Hole}: Derive $n(r)$ from $\epsilon_{rr}(r)$. Prove that at $r = 2GM/c^2$, $n \rightarrow \infty$[cite: 1186, 1187].
    \item \textbf{Energy Packet Momentum}: Show that as $\omega \rightarrow \Wcut$, momentum $p$ becomes singular while $v_g$ vanishes[cite: 1188].
    \item \textbf{Time Dilation via Signal Path}: Derive $\Delta t'$ by calculating signal update delays across strain $\epsilon$[cite: 1190].
\end{enumerate}
\end{problembox}

\section{2.6 Transition to the Quantum Layer}
Having established how mass and gravity emerge from hardware constraints, we move to the \textbf{Quantum Layer} (Chapter 3)[cite: 1192, 1193].