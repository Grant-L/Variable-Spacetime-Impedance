\chapter*{Glossary of Terms}
\addcontentsline{toc}{chapter}{Glossary of Terms}
\markboth{Glossary of Terms}{}

This text employs a specific lexicon to unify concepts from Electrical Engineering, Fluid Dynamics, and Theoretical Physics. The following table serves as a translation matrix for the multidisciplinary reader.

\section*{The LCT Dictionary}

\begin{center}
\begin{tabular}{@{}p{0.25\textwidth} p{0.1\textwidth} p{0.6\textwidth}@{}}
\toprule
\textbf{Term} & \textbf{Symbol} & \textbf{Definition \& Analog} \\ \midrule

\textbf{Vacuum Order Parameter} & $\Psi$ & The complex scalar field defining the local state of the substrate. Its magnitude $\rho$ represents lattice excitation (amplitude density), and its phase $S$ represents signal flow. \\
\textit{(Analog)} & & \textit{Superfluids: Macroscopic Wavefunction | QM: Probability Amplitude} \\ \addlinespace

\textbf{Discrete Vacuum Substrate} & $\Omega$ & The physical medium of the universe, modeled as a high-frequency, superconducting 3D LC lattice. \\
\textit{(Analog)} & & \textit{Solid State: Crystal Lattice | EE: 3D Transmission Line} \\ \addlinespace

\textbf{Lattice Constitutive Parameter} & $\chi$ & The "Stiffness" or Bulk Modulus of the vacuum. It measures the lattice's resistance to density fluctuations. \\
\textit{(Analog)} & & \textit{Mechanics: Young's Modulus | GR: Inverse Gravitational Constant ($1/G$)} \\ \addlinespace

\textbf{Vacuum Impedance} & $Z_0$ & The ratio of transverse electric to magnetic potential in the lattice. Defined by $\sqrt{L/C}$. \\
\textit{(Analog)} & & \textit{RF Engineering: Characteristic Impedance ($Z_0$) | Optics: Refractive Index} \\ \addlinespace

\textbf{Breakdown Wavelength} & $\lambda_{min}$ & The minimum spatial wavelength the lattice can propagate before dielectric saturation occurs. \\
\textit{(Analog)} & & \textit{Signal Processing: Nyquist Limit | QFT: UV Cutoff (Planck Length)} \\ \addlinespace

\textbf{Topological Nucleation} & — & The mechanical failure of the lattice under extreme phase stress ($2\pi$ twist), fracturing the substrate to create a vortex-antivortex pair. \\
\textit{(Analog)} & & \textit{Material Science: Fracture/Yielding | QFT: Schwinger Pair Production} \\ \addlinespace

\textbf{Phase Bridge} & — & A continuous topological flux tube connecting two entangled defects. It transmits tension (correlation) instantly via topology, but information at speed $c_s$. \\
\textit{(Analog)} & & \textit{Topology: Wormhole (Einstein-Rosen) | Network Theory: Dedicated Bus Line} \\ \addlinespace

\textbf{Cosmological Impedance Evolution} & $\beta$ & The secular drift of lattice parameters ($c_s$, $Z_0$) over cosmic time due to the cooling/hardening of the substrate. \\
\textit{(Analog)} & & \textit{Signal Processing: Clock Drift | Cosmology: Tired Light (Refined)} \\ \addlinespace

\textbf{Vacuum Reynolds Number} & $Re_{vac}$ & A dimensionless ratio determining the stability of the pilot wave. High $Re_{vac}$ leads to turbulence (decoherence). \\
\textit{(Analog)} & & \textit{Fluid Dynamics: Reynolds Number | QM: Decoherence Threshold} \\ \addlinespace

\textbf{Tri-Vortex Molecule} & $p^+$ & The topological structure of the proton, consisting of three bound $n=+1$ vortices. Explains the frequency-dependent radius measurement. \\
\textit{(Analog)} & & \textit{Hydrodynamics: Vortex Knot | Particle Physics: Baryon (Quark Triplet)} \\
\bottomrule
\end{tabular}
\end{center}