\chapter*{Nomenclature and Fundamental Constants}
\addcontentsline{toc}{chapter}{Nomenclature and Fundamental Constants}

\section*{The LCT Hardware Constants}
The following table summarizes the constitutive parameters of the vacuum hardware layer. All emergent units (Volts, Amps, Meters) are derived from the synchronization of these lattice components. [cite: 15, 16]

\begin{table}[h!]
\centering
\begin{tabular}{|c|l|l|l|}
\hline
\textbf{Symbol} & \textbf{Name} & \textbf{Value (LCT)} & \textbf{Physical Equivalent} \\ \hline
$\Lvac$ & Lattice Inductance & $\approx 1.257 \mu\text{H/m}$ & $\mu_0$ (Vacuum Permeability) [cite: 15, 90] \\ \hline
$\Cvac$ & Lattice Capacitance & $\approx 8.854 \text{pF/m}$ & $\epsilon_0$ (Vacuum Permittivity) [cite: 15, 90] \\ \hline
$\Zvac$ & Characteristic Impedance & $\approx 376.73 \Omega$ & $\sqrt{\Lvac/\Cvac}$ [cite: 15, 88] \\ \hline
$\Delta x$ & Lattice Pitch & $\sim 10^{-35} \text{m}$ & The discrete nodal spacing [cite: 15, 94] \\ \hline
$\Wcut$ & Cutoff Frequency & $2/\sqrt{\Lvac\Cvac}$ & The Nyquist limit of the substrate [cite: 15, 122] \\ \hline
$\epsilon_{\mu\nu}$ & Metric Strain Tensor & Dimensionless & Physical nodal displacement [cite: 15, 137] \\ \hline
$Q$ & Quantum Potential & Joules & Internal vacuum pressure [cite: 15, 182, 188] \\ \hline
\end{tabular}
\caption{Foundational variables of the Lindblom Coupling Theory.}
\end{table}