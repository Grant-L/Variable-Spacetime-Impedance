\chapter{Mathematical Proofs and Formalism}
\label{app:proofs}

\section{A.1 The Discrete-to-Continuum Limit (Kirchhoff)}
To bridge the gap between electrical engineering and field theory, we expand the derivation in Section 1.2.2. Consider the 3D discrete lattice where each node $(i, j, k)$ is connected by inductors $\Lvac$ and capacitors $\Cvac$. [cite: 42, 43]

The nodal current balance (Kirchhoff’s Current Law) at node $n$ is: [cite: 44]
\begin{equation}
\Cvac \frac{dV_{n}}{dt} = I_{n} - I_{n+1}
\end{equation}

Differentiating with respect to time and substituting the Inductive Voltage law $\Lvac\frac{dI}{dt} = \Delta V$: [cite: 44, 47]
\begin{equation}
\Lvac\Cvac \frac{d^2 V_{n}}{dt^2} = V_{n-1} - 2V_{n} + V_{n+1}
\end{equation}

In the limit where the lattice spacing $\Delta x \to 0$, we define the spatial second derivative: [cite: 49]
\begin{equation}
\frac{V_{n-1} - 2V_{n} + V_{n+1}}{\Delta x^2} \to \frac{\partial^2 V}{\partial x^2}
\end{equation}

Substituting this into the discrete equation: [cite: 52]
\begin{equation}
\frac{\Lvac\Cvac}{\Delta x^2} \frac{\partial^2 V}{\partial t^2} = \frac{\partial^2 V}{\partial x^2}
\end{equation}

Setting $c^2 = \frac{\Delta x^2}{\Lvac\Cvac}$, we recover the standard Wave Equation. [cite: 52, 53]

\section{A.2 The Madelung Internal Pressure ($Q$)}
In Chapter 3, the Quantum Potential $Q$ was identified as internal vacuum pressure. [cite: 183, 188] We derive this by substituting the polar form of the wavefunction $\psi = \sqrt{\rho}e^{iS/\hbar}$ into the Schrödinger Equation. [cite: 181, 182]

Separating the real part yields the **Quantum Hamilton-Jacobi Equation**: [cite: 182]
\begin{equation}
\frac{\partial S}{\partial t} + \frac{(\nabla S)^2}{2m} + V + Q = 0
\end{equation}

Where $Q$ is defined as: [cite: 182]
\begin{equation}
Q = -\frac{\hbar^2}{2m} \frac{\nabla^2 \sqrt{\rho}}{\sqrt{\rho}}
\end{equation}

In LCT, $Q$ is the **elastic potential energy density** of the lattice nodes being displaced. [cite: 188, 189] This term accounts for the "pressure" that prevents the collapse of the electron into the nucleus, providing a mechanical basis for the Bohr Radius. [cite: 224, 228]

\section{A.3 Impedance Clamping and Parity Violation}
We derive the directional impedance encounters of helical pulses. [cite: 292, 296] The impedance $Z$ is modified by the alignment of the vortex winding $m$ and the momentum vector $k$. [cite: 296, 304]

\begin{equation}
Z_{eff}(\sigma, m, k) = \Zvac \exp \left( \sigma \cdot (m \cdot k) \right)
\end{equation}

As $\sigma \to \Wcut$, the impedance for right-handed configurations ($m \cdot k > 0$) approaches the hardware slew limit. [cite: 308, 310] This reflects the energy back into the substrate, acting as a mechanical filter that creates the observed Parity Violation in weak interactions. [cite: 315, 316]