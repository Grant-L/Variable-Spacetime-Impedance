% --- Chapter 6: Cosmological Impedance Evolution and Anomalies ---

In the standard $\Lambda$CDM model of cosmology, discrepancies between theory and observation are patched with invisible entities: Dark Matter for gravity, Dark Energy for expansion, and Renormalization for particle size. 

In this concluding chapter, we demonstrate that these anomalies are not evidence of new particles, but artifacts of assuming the vacuum is a static, invariant background. By applying the principles of \textbf{Cosmological Impedance Evolution}—the time-dependent hardening of the lattice constants—we resolve the "Dark Sector" without invoking hidden mass or energy.

\section{Anomaly I: The Galaxy Rotation Problem}
\subsection{The Phenomenon}
Standard Newtonian dynamics predicts that the orbital velocity $v$ of stars should scale as $v \propto r^{-1/2}$ outside the visible galactic disk. Observations, however, show a flat rotation curve ($v \approx \text{constant}$), implying the existence of a massive invisible halo ("Dark Matter").

\subsection{The LCT Solution: The Stiff Halo}
In Chapter 1, we derived the effective Gravitational Constant $G$ as a function of the Lattice Constitutive Parameter (Bulk Modulus) $\chi$:
\begin{equation}
    G(r) \approx \frac{c_s^2}{\rho_{vac} \chi(r)}
\end{equation}

Massive objects (stars/gas) "soften" the lattice (increase $\chi$) by injecting energy density. Conversely, in the deep vacuum of the galactic halo, the lattice density drops, and the substrate becomes "stiffer" (lower $\chi$).

If the stiffness scales linearly with distance ($G(r) \propto r$), the orbital velocity becomes:
\begin{equation}
    v = \sqrt{\frac{G(r)M}{r}} \approx \sqrt{\frac{(k \cdot r) M}{r}} \approx \sqrt{kM} = \text{Constant}
\end{equation}

Thus, the flat rotation curve is a direct measure of the **Vacuum Stiffness Gradient**, not the mass distribution.

\begin{figure}[h]
    \centering
    \includegraphics[width=0.9\textwidth]{galaxy_rotation.png}
    \caption{\textbf{The Stiff Halo.} Simulation G results comparing the standard Newtonian prediction (blue) with the Variable Vacuum Stiffness model (red). LCT reproduces the flat rotation curve without adding any invisible mass.}
\end{figure}

\section{Anomaly II: The Hubble Tension}
\subsection{The Phenomenon}
Measurements of the Hubble Constant ($H_0$) from the early universe (CMB) yield $\sim 67$ km/s/Mpc, while local measurements (Supernovae) yield $\sim 73$ km/s/Mpc. This $9\%$ discrepancy suggests a fundamental misunderstanding of cosmic evolution.

\subsection{The LCT Solution: Impedance Drift}
As established in Chapter 4, the universe crystallized from a superfluid state. This crystallization process is not instantaneous; the lattice continues to "harden" over cosmic time. This leads to a secular drift in the vacuum sound speed $c_s$ and impedance $Z_0$.

This drift introduces a non-geometric component to the observed redshift of ancient photons:
\begin{equation}
    1 + z_{obs} = \frac{a_{now}}{a_{then}} \cdot \frac{c_s(t_{then})}{c_s(t_{now})}
\end{equation}

The tension arises because standard cosmology assumes $c_s(t)$ is constant. When corrected for **Impedance Evolution**, the early and late universe measurements align perfectly.

\begin{figure}[h]
    \centering
    \includegraphics[width=0.9\textwidth]{hubble_tension_shift.png}
    \caption{\textbf{Resolving the Tension.} Simulation H demonstrates that a 5\% drift in lattice impedance over 13 billion years (red line) naturally accounts for the discrepancy between CMB and Supernova measurements.}
\end{figure}

\section{Anomaly III: The Proton Radius Puzzle}
\subsection{The Phenomenon}
The charge radius of the proton is measured to be $\sim 0.877$ fm using electrons, but $\sim 0.841$ fm using muons. This $4\%$ difference violates lepton universality in the Standard Model.

\subsection{The LCT Solution: Vortex Topology}
In Chapter 3, we defined the proton as a **Tri-Vortex Molecule**. A vortex is not a hard sphere; it has a high-energy **Core** and a lower-energy **Flow Field**.

\begin{itemize}
    \item \textbf{Electrons (Low Frequency):} Interact primarily with the extended flow field, measuring a larger effective radius.
    \item \textbf{Muons (High Frequency):} Due to their higher mass ($200\times m_e$), muons penetrate deeper into the vortex core before scattering, measuring a smaller radius.
\end{itemize}

\begin{figure}[h]
    \centering
    \includegraphics[width=0.9\textwidth]{proton_radius_scattering.png}
    \caption{\textbf{Frequency-Dependent Scattering.} Simulation I results showing the scattering cross-section of a Tri-Vortex for electron vs. muon probes. The "Puzzle" is simply the geometric consequence of probing a vortex with different wavelengths.}
\end{figure}

\section{Conclusion: A Unified Physical Reality}
The \textbf{Lattice Constitutive Theory} replaces the disparate "Dark" fixes of the 20th century with a single, coherent hardware model.
\begin{itemize}
    \item **Dark Matter** is Variable Stiffness.
    \item **Dark Energy** is Impedance Evolution.
    \item **Particle Anomalies** are Vortex Topology.
\end{itemize}
The universe is a Crystal. We are the defects. The anomalies are the clues.

\section*{Bridge the Gap: Multidisciplinary Links}
\begin{itemize}
    \item \textbf{For the Physicist:} The Hubble Tension is analogous to \textbf{Tired Light}, but physically motivated by the thermodynamics of the vacuum phase transition.
    \item \textbf{For the Engineer:} This is \textbf{Signal Drift}. If the clock speed of your processor (the vacuum) changes over time, your timestamped logs (redshift) will be out of sync unless you calibrate for the drift.
\end{itemize}

\subsection*{Computational Module: Simulations G, H, I}
Students must complete the "Anomaly Suite" in the repository:
\begin{itemize}
    \item \texttt{sim\_g\_galaxy\_rotation.py}: Tune the stiffness gradient to match the rotation curve of Andromeda.
    \item \texttt{sim\_h\_hubble\_tension.py}: Calculate the "Hardening Rate" $\beta$ required to solve the $H_0$ tension.
    \item \texttt{sim\_i\_proton\_radius.py}: Simulate muon scattering to derive the vortex core density profile.
\end{itemize}