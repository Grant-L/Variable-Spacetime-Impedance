\chapter{Observational Signatures: Solving the Dark Sector}

\section{Introduction: Anomalies as Clues}
The Standard Model of Cosmology ($\Lambda$CDM) faces two major crises: the nature of Dark Matter and the Hubble Tension[cite: 275]. LCT proposes that these are not due to invisible particles, but are artifacts of the vacuum's fluid dynamics[cite: 276].

\section{Dark Matter: The Vortex Lattice}
Standard Cold Dark Matter (CDM) postulates a halo of invisible particles[cite: 278]. LCT identifies the "Halo" as a region of \textbf{Quantum Turbulence} in the vacuum substrate[cite: 279].



\begin{itemize}
    \item \textbf{The Mechanism:} The rotating galaxy drags the local vacuum[cite: 280]. However, because the vacuum is a superfluid, it cannot rotate as a rigid body[cite: 281]. Instead, it forms a quantized \textbf{Vortex Lattice} similar to an Abrikosov lattice in a Type-II superconductor[cite: 282].
    \item \textbf{Vortex Density:} The galaxy creates a dense array of microscopic vortices[cite: 283]. The energy density of this lattice acts as effective mass[cite: 284].
\end{itemize}

\section{Explaining Flat Rotation Curves}
A single vortex has a velocity profile $v \propto 1/r$ (Keplerian), which fails to explain galactic rotation[cite: 286]. However, a \textbf{Vortex Lattice} creates a macroscopic "texture" where the vortex area density $n_{v}$ scales with the galactic stress[cite: 287].
\begin{equation}
v_{rot} \approx \frac{\hbar}{m} \sqrt{2\pi n_{v}(r)} [cite: 288]
\end{equation}
If the vacuum responds to shear stress by maintaining a constant vorticity per unit area (Quantum Turbulence equilibrium), the resulting rotation curve is \textbf{flat} ($v \approx const$), exactly matching observations without requiring exotic particles.

\section{The Hubble Tension: A Vacuum Phase Transition}
LCT explains the $H_{0}$ mismatch as a \textbf{Vacuum Phase Transition} (Crystallization) at redshift $z \approx 10$, releasing latent heat (Dark Energy) that boosted late-universe expansion.

\section{Problems}
\begin{enumerate}
    \item \textbf{Vortex Lattice Rotation:} A galactic halo creates a vortex lattice with area density $n_{v}(r) \propto 1/r$. Show that the resulting rotational velocity profile $v_{rot}$ is constant (Flat Rotation Curve).
    \item \textbf{Lensing Asymmetry:} Calculate the time delay difference $\Delta t$ for a photon passing pro-grade vs. retro-grade through a rotating frame-dragging vortex with angular momentum $J$[cite: 299].
    \item \textbf{Hubble Mismatch:} If Early Dark Energy acted only between $z=10$ and $z=8$, how would this shift the inferred value of $H_{0}$ from the CMB peak compared to Supernovae measurements[cite: 300]?
\end{enumerate}