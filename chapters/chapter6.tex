\chapter{Observational Signatures: Solving the Dark Sector}

\section{Introduction: Anomalies as Clues}
The Standard Model of Cosmology ($\Lambda$CDM) faces two major crises: the nature of Dark Matter and the Hubble Tension. LCT proposes that these are not due to invisible particles or new fields, but are artifacts of the vacuum's fluid dynamics.

\section{Dark Matter: The Vortex Halo}
Standard Cold Dark Matter (CDM) models the galactic halo as a spherical cloud of weakly interacting particles (WIMPs). The gravitational field is purely radial.
LCT proposes that the "Halo" is a region of high vacuum vorticity—a **Superfluid Vortex** dragged by the rotating galaxy.
\begin{itemize}
    \item **Mechanism:** The rotating mass of the galaxy drags the vacuum substrate (Frame Dragging), creating a vortex of high impedance.
    \item **Mass Equivalent:** The energy density of this vortex field $E \propto (\nabla \theta)^2$ acts as effective mass, flattening the rotation curves of outer stars.
\end{itemize}

\section{Prediction: Rotational Lensing Asymmetry}
Because the vacuum halo has angular momentum, it induces a strong **Gravitomagnetic** effect. This leads to a distinct falsifiable prediction for strong gravitational lensing:
\begin{enumerate}
    \item **Pro-Grade Lensing:** Light passing on the side of the galaxy spinning *towards* the observer travels "downstream" (lower effective index). It is deflected *less*.
    \item **Retro-Grade Lensing:** Light passing on the side spinning *away* travels "upstream" (higher effective index). It is deflected *more*.
\end{enumerate}
**The Signature:** We predict that Einstein Rings around rotating galaxies will be slightly **elliptical or skewed** along the axis of rotation, a feature not predicted by scalar Dark Matter models.

\section{The Hubble Tension: A Vacuum Phase Transition}
Measurements of the Hubble Constant ($H_0$) from the Early Universe (CMB) and Late Universe (Supernovae) disagree by $\approx 5\sigma$.
LCT explains this as a **Vacuum Phase Transition**.
\begin{itemize}
    \item **Mechanism:** As the universe cooled, the vacuum lattice underwent a freezing event (crystallization) at redshift $z \approx 10$.
    \item **Latent Heat:** This transition released latent energy (Dark Energy) into the lattice, boosting the expansion rate of the late universe.
\end{itemize}
This "Early Dark Energy" solution naturally resolves the tension without requiring fine-tuned scalar fields.

\section{Bridge the Gap: From Astronomy to Hydrodynamics}
To the Astronomer, a Galaxy Halo is a mass distribution. To the Fluid Dynamicist, it is a **Rankine Vortex**.
\begin{itemize}
    \item **Core:** Solid-body rotation (The visible galaxy).
    \item **Halo:** Irrotational flow where velocity drops as $1/r$ (The Dark Matter halo).
\end{itemize}
This mapping suggests that Dark Matter is simply the kinetic energy of the vacuum fluid itself.