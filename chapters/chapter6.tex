\chapter{Observational Signatures: Solving the Dark Sector}

\section{Introduction: Anomalies as Clues}
The Standard Model of Cosmology ($\Lambda$CDM) faces crises in the nature of Dark Matter and the Hubble Tension[cite: 506]. LCT proposes these are not hidden particles, but artifacts of the vacuum's fluid dynamics[cite: 507].

\section{Dark Matter: The Vortex Lattice}
LCT identifies the galactic "Dark Matter Halo" as a region of \textbf{Quantum Turbulence} in the superfluid vacuum substrate[cite: 510]. 
\begin{itemize}
    \item \textbf{The Mechanism:} A rotating galaxy drags the local vacuum[cite: 511]. As a superfluid, the vacuum cannot rotate as a rigid body and instead forms a quantized \textbf{Vortex Lattice} (Abrikosov lattice)[cite: 512, 513].
    \item \textbf{Effective Mass:} The kinetic energy density of this array of microscopic vortices acts as effective mass[cite: 514, 515].
\end{itemize}

\section{Explaining Flat Rotation Curves}
A single vortex velocity profile ($v \propto 1/r$) fails to explain galactic rotation[cite: 517]. However, a macroscopic Vortex Lattice maintains a constant vorticity per unit area[cite: 518, 521]. 
\begin{equation}
v_{rot} \approx \frac{\hbar}{m} \sqrt{2\pi n_v(r)}
\end{equation}
The result is a flat rotation curve ($v \approx const$) that matches observed galactic data without requiring Dark Matter particles[cite: 521, 523].

\begin{figure}[H]
    \centering
    \includegraphics[width=0.9\linewidth]{simulation_dark_matter_rotation.png}
    \caption{Simulation D.3.2: Comparison of Newtonian gravity (Red) vs. LCT Vortex Lattice (Blue) against observed galactic data[cite: 545, 546].}
    \label{fig:rotation}
\end{figure}

\section{The Hubble Tension: A Vacuum Phase Transition}
LCT explains the $H_0$ mismatch as a result of a \textbf{Late-Time Phase Transition}[cite: 551]. At redshift $z \approx 10$, the vacuum underwent a further "crystallization" event, releasing latent heat (Dark Energy) that boosted the late-universe expansion rate[cite: 551].