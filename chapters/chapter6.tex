\chapter{6 The Weak Layer: Chirality as a Filter}

\section{6.1 Introduction: The Vacuum as a Polarized TVS}
Standard particle physics treats chirality as an abstract quantum number and parity violation as an empirical law. LCT proposes that the vacuum acts as a non-linear, directional impedance filter, analogous to a specialized \textbf{Polarized Transient Voltage Suppressor (TVS)}. The "Weak Interaction" is identified as the mechanical response of the hardware lattice to topologically incompatible screw directions.

\section{6.2 Helicity and Mechanical Impedance}
A propagating particle in LCT is a helical vortex pulse. As established in Chapter 2, this propagation induces a backlog of \textbf{Metric Strain}: the compression of nodes ahead and the stretching of nodes behind the wavefront [cite: 118-123, 135].

\subsection{6.2.1 The Impedance Clamping Equation}
We define the \textbf{Coupling Efficiency} ($\eta$) of a propagating helix into the strained hardware lattice. The effective impedance ($Z_{eff}$) encountered by a vortex with winding $m$ and propagation vector $k$ is given by the \textbf{Impedance Clamping Equation}:

\begin{equation}
Z_{eff} = Z_0 \cdot e^{\sigma (m \cdot k)}
\label{eq:clamping}
\end{equation}

Where:
\begin{itemize}
    \item \textbf{$Z_0$}: The baseline characteristic impedance of free space ($\approx 376.73\Omega$)[cite: 40, 69].
    \item \textbf{$\sigma$}: The local \textbf{Metric Strain Constant}.
    \item \textbf{$m \cdot k$}: The alignment of the vortex winding (chirality) with its direction of travel.
\end{itemize}

\section{6.3 The Slew Rate Threshold}
The lattice update rate, defined by the hardware time constant $1/\sqrt{\mathcal{LC}}$ (Eq. 1.1), imposes a maximum rate of change for phase flux [cite: 66, 128-130]. If the "screw pitch" of a vortex exceeds this limit, the node fails to update, presenting an effectively infinite impedance.

\begin{equation}
\left| \frac{d\theta}{dt} \right| > \omega_{cutoff}
\label{eq:slew_limit}
\end{equation}

This \textbf{Slew Rate Limit} "clamps" the signal, forcing incompatible configurations into evanescent, non-propagating modes.

\section{6.4 Chirality as a Lossless Filter}
Unlike standard dissipative engineering components, the vacuum filter is \textbf{Lossless and Elastic}. 
\begin{itemize}
    \item \textbf{Energy Storage}: The energy of a rejected configuration (e.g., a right-handed neutrino) is stored reversibly as elastic metric strain ($\epsilon$) [cite: 121, 358-359].
    \item \textbf{Reflection}: Incompatible configurations are reflected by the impedance barrier rather than absorbed.
    \item \textbf{Parity Violation}: This mechanism explains why only left-handed neutrinos are observed; the vacuum's intrinsic hardware bias acts as a discriminator that reflects all other configurations.
\end{itemize}

\section{6.5 Bridge the Gap: From Weak Force to Surge Protection}
To the Particle Physicist, the Weak Force is mediated by $W$ and $Z$ bosons [cite: 289-290]. To the Engineer, it is the \textbf{Automated Surge Protection} of the vacuum lattice.
\begin{itemize}
    \item \textbf{$W^{\pm}$ Bosons}: Localized lattice "breakdown" events that allow a change in winding number $n$[cite: 86, 275].
    \item \textbf{$Z^0$ Boson}: A common-mode impedance spike that mediates neutral current interactions without altering the topology [cite: 21, 70-73].
    \item \textbf{Chirality}: The "Key-and-Lock" mechanical fit of a vortex screw into the strained vacuum substrate.
\end{itemize}