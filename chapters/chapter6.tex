\chapter{Observational Signatures: Solving the Dark Sector}

\section{Introduction: Anomalies as Clues}
The Standard Model of Cosmology ($\Lambda$CDM) faces two major crises: the nature of Dark Matter and the Hubble Tension. LCT proposes that these are not due to invisible particles, but are artifacts of the vacuum's fluid dynamics.

\section{Dark Matter: The Vortex Lattice}
Standard Cold Dark Matter (CDM) postulates a halo of invisible particles. LCT identifies the "Halo" as a region of **Quantum Turbulence** in the vacuum substrate.
\begin{itemize}
    \item **The Mechanism:** The rotating galaxy drags the local vacuum. However, because the vacuum is a superfluid, it cannot rotate as a rigid body. Instead, it forms a quantized **Vortex Lattice** (similar to an Abrikosov lattice in a Type-II superconductor).
    \item **Vortex Density:** The galaxy creates a dense array of microscopic vortices. The energy density of this lattice acts as effective mass.
\end{itemize}

\section{Explaining Flat Rotation Curves}
A single vortex has a velocity profile $v \propto 1/r$ (Keplerian), which fails to explain galactic rotation.
However, a **Vortex Lattice** creates a macroscopic "texture" where the vortex area density $n_v$ scales with the galactic stress.
\begin{equation}
v_{rot} \approx \frac{\hbar}{m} \sqrt{2\pi n_v(r)}
\end{equation}
If the vacuum responds to shear stress by maintaining a constant vorticity per unit area (Quantum Turbulence equilibrium), the resulting rotation curve is **flat** ($v \approx const$), exactly matching observations without requiring exotic particles.

\section{Prediction: The Lensing Signature}
While the rotation curve mimics CDM, the **Lensing Signature** differs.
\begin{itemize}
    \item **CDM:** Smooth, continuous lensing gradient.
    \item **LCT:** The halo is "granular" at the microscopic scale. High-frequency gravitational waves or gamma rays passing through the halo should experience **Scintillation** (twinkling) due to scattering off the individual vortex cores in the lattice.
\end{itemize}

\section{The Hubble Tension: A Vacuum Phase Transition}
LCT explains the $H_0$ mismatch as a **Vacuum Phase Transition** (Crystallization) at redshift $z \approx 10$, releasing latent heat (Dark Energy) that boosted late-universe expansion.