\chapter*{Glossary and Acronyms}
\addcontentsline{toc}{chapter}{Glossary and Acronyms}

\section*{G.1 Core LCT Acronyms}
\begin{table}[h!]
\centering
\begin{tabular}{|l|l|l|}
\hline
\textbf{Acronym} & \textbf{Full Term} & \textbf{LCT Definition} \\ \hline
[cite_start]\textbf{B-EMF} & Back-Electromotive Force & Mechanical precursor to \textbf{Inertia}; resistance to flux change[cite: 130]. \\ \hline
[cite_start]\textbf{FDTD} & Finite-Difference Time-Domain & Numerical method used to solve discrete vacuum equations[cite: 151]. \\ \hline
[cite_start]\textbf{GL} & Ginzburg-Landau & Relaxation equation for modeling topological assembly[cite: 260]. \\ \hline
[cite_start]\textbf{LCT} & Lindblom Coupling Theory & Hardware-oriented unified field theory modeling the vacuum[cite: 29]. \\ \hline
[cite_start]\textbf{TVS} & Transient Voltage Suppressor & Analogy for the \textbf{Weak Interaction} and its directional clamping[cite: 289]. \\ \hline
[cite_start]\textbf{ZPE} & Zero-Point Energy & Oscillating tension of the vacuum lattice ground state[cite: 459]. \\ \hline
\end{tabular}
\caption{LCT specialized acronyms and engineering definitions.}
\end{table}

\section*{G.2 Exhaustive Glossary of Terms}

\subsection*{A}
\begin{itemize}
    [cite_start]\item \textbf{Abrikosov Lattice}: A quantized vortex lattice formed in the superfluid vacuum; the LCT mechanism for \textbf{Dark Matter}[cite: 395].
    [cite_start]\item \textbf{Acoustic Metric}: A fluid-mechanical model recovering the Schwarzschild metric via a flowing medium[cite: 148].
    [cite_start]\item \textbf{Alcubierre Metric}: A warp-drive solution recast as a localized \textbf{Impedance Bubble}[cite: 438, 440].
\end{itemize}

\subsection*{B}
\begin{itemize}
    [cite_start]\item \textbf{Bandwidth Saturation}: The state where a lattice node reaches its maximum update frequency ($\Wcut$); the origin of \textbf{Rest Mass}[cite: 64, 129].
    [cite_start]\item \textbf{Bohr Radius}: The stable orbital distance where electron wake resonance balances Coulomb attraction[cite: 228].
\end{itemize}

\subsection*{C}
\begin{itemize}
    [cite_start]\item \textbf{Casimir Effect}: Modeled as a \textbf{Band-Stop Filter} within the noisy vacuum substrate[cite: 230].
    [cite_start]\item \textbf{Characteristic Impedance ($\Zvac$)}: The ratio of vacuum voltage to current, $\approx 376.73\Omega$[cite: 15].
    [cite_start]\item \textbf{Chirality Filter}: Directional impedance reflecting right-handed configurations, explaining \textbf{Parity Violation}[cite: 311, 315].
    [cite_start]\item \textbf{Compton Frequency ($\omega_c$)}: The natural oscillation frequency of a particle soliton on the hardware lattice[cite: 191].
\end{itemize}

\subsection*{I}
\begin{itemize}
    [cite_start]\item \textbf{Impedance Clamping}: Non-linear mechanical response where a vortex encounters infinite impedance[cite: 307, 310].
\end{itemize}

\subsection*{K}
\begin{itemize}
    [cite_start]\item \textbf{Kibble-Zurek Mechanism}: The process trapping topological defects (matter) during a vacuum \textbf{Quench}[cite: 351, 355].
\end{itemize}

\subsection*{L}
\begin{itemize}
    [cite_start]\item \textbf{Lattice Inductance ($\Lvac$)}: The inertial component resisting flux changes ($\mu_0$)[cite: 15, 36].
    [cite_start]\item \textbf{Lattice Capacitance ($\Cvac$)}: The elastic component storing potential energy via strain ($\epsilon_0$)[cite: 15, 40].
    [cite_start]\item \textbf{Lindblom Dispersion Relation}: Defines the drop in propagation speed as energy saturates hardware[cite: 103, 123].
\end{itemize}

\subsection*{M}
\begin{itemize}
    [cite_start]\item \textbf{Madelung Transformation}: Mapping revealing the Schrödinger equation as superfluid motion[cite: 181, 189].
    [cite_start]\item \textbf{Metric Strain ($\epsilon$)}: Physical stretching or compression of vacuum nodes creating a refractive index gradient[cite: 132, 139].
\end{itemize}

\subsection*{P}
\begin{itemize}
    [cite_start]\item \textbf{Phase Bridge}: A high-tension flux tube connecting entangled topological defects; the mechanism for \textbf{Non-Locality}[cite: 336, 341].
    [cite_start]\item \textbf{Pilot Wave}: The standing-wave "memory field" guiding a particle soliton through interference[cite: 178, 192].
\end{itemize}

\subsection*{S}
\begin{itemize}
    [cite_start]\item \textbf{Schwinger Limit}: The dielectric breakdown threshold of the vacuum ($\approx 10^{18}$ V/m)[cite: 86].
    [cite_start]\item \textbf{Slew Rate Limit}: The maximum frequency at which a lattice node can update, defining $c$[cite: 64, 310].
\end{itemize}

\subsection*{T}
\begin{itemize}
    [cite_start]\item \textbf{Topological Defect}: A stable vortex or knot in the vacuum phase field identified as a particle[cite: 241, 255].
\end{itemize}