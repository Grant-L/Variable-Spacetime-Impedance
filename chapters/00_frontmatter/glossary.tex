\chapter*{Glossary and Acronyms}
\addcontentsline{toc}{chapter}{Glossary and Acronyms}

\section*{G.1 Core SVF Acronyms}
The following acronyms facilitate the translation of vacuum hardware dynamics into observable physics within the \textbf{Stochastic Vacuum Framework}.

\begin{table}[h!]
\centering
\small
\begin{tabular}{|p{1.5cm}|p{3cm}|p{5.5cm}|}
\hline
\textbf{Acronym} & \textbf{Full Term} & \textbf{SVF Definition} \\ \hline
\textbf{B-EMF} & Back-Electromotive Force & Mechanical precursor to \textbf{Inertia}; resistance to flux change. \\ \hline
\textbf{CBE} & Chiral Bias Equation & The law governing spin-dependent metric impedance. \\ \hline
\textbf{FDTD} & Finite-Difference Time-Domain & Numerical method used to solve discrete vacuum equations. \\ \hline
\textbf{MA} & Discrete Amorphous Manifold & The stochastic hardware substrate of the vacuum. \\ \hline
\textbf{SVF} & Stochastic Vacuum Framework & The hardware-oriented unified field theory modeling the vacuum. \\ \hline
\textbf{TVS} & Transient Voltage Suppressor & Analogy for the \textbf{Weak Interaction} and its directional clamping. \\ \hline
\textbf{VSI} & Variable Spacetime Impedance & The effect of metric strain on vacuum propagation. \\ \hline
\end{tabular}
\caption{SVF specialized acronyms and engineering definitions.}
\end{table}

\section*{G.2 Exhaustive Glossary of Terms}

\subsection*{A}
\begin{itemize}
    \item \textbf{Abrikosov Lattice}: A quantized vortex lattice formed in the superfluid vacuum. In SVF, the elastic stiffness of this lattice is the mechanical origin of the flat rotation curves often attributed to \textbf{Dark Matter}.
    \item \textbf{Acoustic Metric}: A fluid-mechanical model recovering the\\ Schwarzschild metric via a flowing medium.
    \item \textbf{Amorphous Topological Glass}: The hardware state of the vacuum; a disordered Voronoi lattice formed during the \textbf{Global Quench}.
\end{itemize}

\subsection*{B}
\begin{itemize}
    \item \textbf{Bandwidth Saturation}: The state where a lattice node reaches its maximum update frequency ($\omega_{\text{cutoff}}$); the mechanical origin of \textbf{Rest Mass}.
    \item \textbf{Bohr Radius}: The stable orbital distance where electron wake resonance balances Coulomb attraction.
\end{itemize}



\subsection*{C}
\begin{itemize}
    \item \textbf{Casimir Effect}: Modeled as a \textbf{Band-Stop Filter} excluding noise modes from a cavity, creating a pressure deficit.
    \item \textbf{Characteristic Impedance ($Z_0$)}: The ratio of vacuum potential to flux ($\approx 376.73\,\Omega$), derived from nodal inductance and capacitance.
    \item \textbf{Chirality Filter}: The mechanism where the lattice reflects right-handed helical signals, explaining \textbf{Parity Violation}.
\end{itemize}

\subsection*{D}
\begin{itemize}
    \item \textbf{Dark Energy}: The **Latent Heat** released by the vacuum substrate as it relaxes toward a lower-energy ground state.
    \item \textbf{Dark Matter}: The rotational stiffness ($k$) of the vacuum's radial impedance gradient, resisting galactic shear without extra particles.
\end{itemize}

\subsection*{E}
\begin{itemize}
    \item \textbf{Event Horizon}: A boundary of **Total Internal Reflection** where the local refractive index $\chi \to \infty$.
\end{itemize}

\subsection*{G}
\begin{itemize}
    \item \textbf{Global Quench}: The primordial phase transition where the fluid vacuum "froze" into its current amorphous glass state.
\end{itemize}

\subsection*{I}
\begin{itemize}
    \item \textbf{Impedance Clamping}: Non-linear mechanical response where a signal\\ (e.g., a right-handed neutrino) encounters effectively infinite impedance.
\end{itemize}

\subsection*{L}
\begin{itemize}
    \item \textbf{Lattice Inductance ($L$)}: The inertial component of the vacuum resisting flux changes ($\mu_0$).
    \item \textbf{Lattice Capacitance ($C$)}: The elastic component of the vacuum storing potential energy via strain ($\epsilon_0$).
    \item \textbf{Lattice Dispersion Relation}: Defines the drop in propagation speed as signal energy saturates hardware bandwidth ($v_g \to 0$ as $\omega \to \omega_{cut}$).
\end{itemize}



\subsection*{M}
\begin{itemize}
    \item \textbf{Madelung Transformation}: A mapping revealing the Schrödinger equation as the motion of a classical superfluid with internal pressure.
    \item \textbf{Metric Strain ($\epsilon$)}: The physical stretching or compression of vacuum nodes, creating a refractive index gradient (Gravity).
\end{itemize}

\subsection*{N}
\begin{itemize}
    \item \textbf{Nodal Jitter}: The high-frequency background noise of the amorphous lattice, identified as the source of **Heisenberg Uncertainty**.
\end{itemize}

\subsection*{P}
\begin{itemize}
    \item \textbf{Phase Bridge}: A high-tension flux tube connecting entangled topological defects; the mechanism for **Confinement**.
    \item \textbf{Pilot Wave}: The standing-wave "memory field" guiding a particle soliton through interference patterns.
    \item \textbf{Proton}: A stable **Trefoil Knot** in the vacuum phase field, composed of three entangled singularities.
\end{itemize}

\subsection*{S}
\begin{itemize}
    \item \textbf{Slew Rate Limit}: The maximum frequency at which a lattice node can update, defining the global speed limit $c$.
\end{itemize}

\subsection*{T}
\begin{itemize}
    \item \textbf{Topological Helicity}: Replaces "Winding Number." The quantized, self-reinforcing phase twist of a defect, identified as electric charge ($q$).
\end{itemize}



\subsection*{W}
\begin{itemize}
    \item \textbf{Weinberg Angle ($\theta_W$)}: A thermal-mechanical property derived from the lattice's **Directional Bias** and effective energy density.
\end{itemize}