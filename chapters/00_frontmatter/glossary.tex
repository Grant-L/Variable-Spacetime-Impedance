\chapter*{Glossary and Acronyms}
\addcontentsline{toc}{chapter}{Glossary and Acronyms}

\section*{G.1 Core LCT Acronyms}
The following acronyms facilitate the translation of vacuum hardware dynamics into observable physics.

\begin{table}[h!]
\centering
\begin{tabular}{|l|l|l|}
\hline
\textbf{Acronym} & \textbf{Full Term} & \textbf{LCT Definition} \\ \hline
\textbf{B-EMF} & Back-Electromotive Force & Mechanical precursor to \textbf{Inertia}; resistance to flux change. \\ \hline
\textbf{FDTD} & Finite-Difference Time-Domain & Numerical method used to solve discrete vacuum equations. \\ \hline
\textbf{GL} & Ginzburg-Landau & Relaxation equation for modeling topological assembly. \\ \hline
\textbf{LCT} & Lindblom Coupling Theory & Hardware-oriented unified field theory modeling the vacuum. \\ \hline
\textbf{TVS} & Transient Voltage Suppressor & Analogy for the \textbf{Weak Interaction} and its directional clamping. \\ \hline
\textbf{ZPE} & Zero-Point Energy & Oscillating tension of the vacuum lattice ground state. \\ \hline
\end{tabular}
\caption{LCT specialized acronyms and engineering definitions.}
\end{table}

\section*{G.2 Exhaustive Glossary of Terms}

\subsection*{A}
\begin{itemize}
    \item \textbf{Abrikosov Lattice}: A quantized vortex lattice formed in the superfluid vacuum; the LCT mechanism for \textbf{Dark Matter}.
    \item \textbf{Acoustic Metric}: A fluid-mechanical model recovering the Schwarzschild metric via a flowing medium.
    \item \textbf{Alcubierre Metric}: A warp-drive solution recast as a localized \textbf{Impedance Bubble}.
\end{itemize}

\subsection*{B}
\begin{itemize}
    \item \textbf{Bandwidth Saturation}: The state where a lattice node reaches its maximum update frequency ($\omega_{\text{cutoff}}$); the origin of \textbf{Rest Mass}.
    \item \textbf{Bohr Radius}: The stable orbital distance where electron wake resonance balances Coulomb attraction.
\end{itemize}

\subsection*{C}
\begin{itemize}
    \item \textbf{Casimir Effect}: Modeled as a \textbf{Band-Stop Filter} within the noisy vacuum substrate.
    \item \textbf{Characteristic Impedance ($Z_0$)}: The ratio of vacuum voltage to current, $\approx 376.73\,\Omega$.
    \item \textbf{Chirality Filter}: Directional impedance reflecting right-handed configurations, explaining \textbf{Parity Violation}.
    \item \textbf{Compton Frequency ($\omega_c$)}: The natural oscillation frequency of a particle soliton on the hardware lattice.
\end{itemize}

\subsection*{I}
\begin{itemize}
    \item \textbf{Impedance Clamping}: Non-linear mechanical response where a vortex encounters effectively infinite impedance.
\end{itemize}

\subsection*{K}
\begin{itemize}
    \item \textbf{Kibble-Zurek Mechanism}: The process trapping topological defects (matter) during a vacuum \textbf{Quench}.
\end{itemize}

\subsection*{L}
\begin{itemize}
    \item \textbf{Lattice Inductance ($L$)}: The inertial component of the vacuum resisting flux changes ($\mu_0$).
    \item \textbf{Lattice Capacitance ($C$)}: The elastic component of the vacuum storing potential energy via strain ($\epsilon_0$).
    \item \textbf{Lindblom Dispersion Relation}: Defines the drop in propagation speed as signal energy saturates hardware bandwidth.
\end{itemize}

\subsection*{M}
\begin{itemize}
    \item \textbf{Madelung Transformation}: A mapping revealing the Schrödinger equation as the motion of a classical superfluid.
    \item \textbf{Metric Strain ($\epsilon$)}: The physical stretching or compression of vacuum nodes, creating a refractive index gradient.
\end{itemize}

\subsection*{P}
\begin{itemize}
    \item \textbf{Phase Bridge}: A high-tension flux tube connecting entangled topological defects; the mechanism for \textbf{Non-Locality}.
    \item \textbf{Pilot Wave}: The standing-wave "memory field" guiding a particle soliton through interference patterns.
\end{itemize}

\subsection*{S}
\begin{itemize}
    \item \textbf{Schwinger Limit}: The dielectric breakdown threshold of the vacuum substrate ($\approx 10^{18}$ V/m).
    \item \textbf{Slew Rate Limit}: The maximum frequency at which a lattice node can update, defining the global limit $c$.
\end{itemize}

\subsection*{T}
\begin{itemize}
    \item \textbf{Topological Defect}: A stable vortex or knot in the vacuum phase field, identified as a discrete particle.
\end{itemize}