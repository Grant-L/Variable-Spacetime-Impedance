\chapter*{Glossary and Acronyms}
\addcontentsline{toc}{chapter}{Glossary and Acronyms}

\section*{G.1 Core SVF Acronyms}
The following acronyms facilitate the translation of vacuum hardware dynamics into observable physics within the \textbf{Stochastic Vacuum Framework}.

\begin{tabularx}{\textwidth}{|l|l|X|}
\hline
\textbf{Acronym} & \textbf{Full Term} & \textbf{SVF Definition} \\ \hline
\textbf{B-EMF} & Back-Electromotive Force & The mechanical precursor to \textbf{Inertia}; the inductive resistance to flux change. \\ \hline
\textbf{CBE} & Chiral Bias Equation & The law governing spin-dependent metric impedance in the manifold. \\ \hline
\textbf{FDTD} & Finite-Difference Time-Domain & Numerical method used to solve discrete vacuum equations in nodal simulations. \\ \hline
\textbf{MA} & Amorphous Manifold & The discrete, stochastic hardware substrate of the vacuum. \\ \hline
\textbf{SVF} & Stochastic Vacuum Framework & The overarching hardware-first theory modeling the vacuum as a transmission medium. \\ \hline
\textbf{TVS} & Transient Voltage Suppressor & An electrical analogy for the \textbf{Weak Interaction} and its high-frequency damping. \\ \hline
\textbf{VSI} & Variable Spacetime Impedance & The localized shift in vacuum properties caused by metric strain. \\ \hline
\end{tabularx}

\section*{G.2 Exhaustive Glossary of Terms}

\subsection*{A}
\begin{itemize}
    \item \textbf{Abrikosov Lattice}: A quantized vortex lattice. In SVF, its elastic stiffness ($k$) provides the mechanical origin of \textbf{Dark Matter} rotation curves.
    \item \textbf{Acoustic Metric}: A fluid-mechanical model that recovers General Relativistic effects (like the Schwarzschild metric) via a flowing medium.
    \item \textbf{Amorphous Topological Glass}: The hardware state of the vacuum; a disordered Voronoi-like lattice formed during the \textbf{Global Quench}.
\end{itemize}



\subsection*{B}
\begin{itemize}
    \item \textbf{Bandwidth Saturation}: The state where a lattice node reaches its maximum update frequency ($\Wcut$). This process clamps energy into a localized standing wave, creating \textbf{Rest Mass}.
    \item \textbf{Bohr Radius}: The stable distance where the electron’s wake-field resonance balances the substrate’s Coulomb-equivalent potential.
\end{itemize}

\subsection*{C}
\begin{itemize}
    \item \textbf{Casimir Effect}: Modeled as a \textbf{Band-Stop Filter} that excludes stochastic noise modes between plates, resulting in a pressure deficit.
    \item \textbf{Characteristic Impedance ($\Zvac$)}: The baseline ratio of vacuum potential to flux ($\approx 376.73\,\Omega$).
    \item \textbf{Chirality Filter}: The mechanical bias of the $M_A$ lattice that reflects right-handed topological twists, explaining \textbf{Parity Violation}.
\end{itemize}



\subsection*{D--G}
\begin{itemize}
    \item \textbf{Dark Energy}: The **Latent Heat** released by the manifold as it relaxes from a high-saturation state toward a lower-energy equilibrium.
    \item \textbf{Event Horizon}: A boundary of **Total Internal Reflection** where the local refractive index $\chi \to \infty$.
    \item \textbf{Global Quench}: The primordial transition where the vacuum substrate "froze" into its current high-impedance, amorphous state.
\end{itemize}

\subsection*{I--N}
\begin{itemize}
    \item \textbf{Impedance Clamping}: A non-linear response where a signal (e.g., a right-handed neutrino) encounters an impedance spike that prevents propagation.
    \item \textbf{Metric Strain ($\epsilon$)}: The physical displacement of nodes, creating a refractive index gradient perceived as gravity.
    \item \textbf{Nodal Jitter}: The high-frequency stochastic noise of the lattice, identified as the source of **Heisenberg Uncertainty**.
\end{itemize}

\subsection*{P--T}
\begin{itemize}
    \item \textbf{Phase Bridge}: A high-tension flux tube connecting entangled topological defects; the mechanism for **Confinement**.
    \item \textbf{Pilot Wave}: The localized impedance wake generated by a moving soliton, guiding its deterministic path.
    \item \textbf{Topological Helicity ($h$)}: The quantized phase-twist of a defect, identified as \textbf{Electric Charge}.
\end{itemize}



\subsection*{W}
\begin{itemize}
    \item \textbf{Weinberg Angle ($\theta_W$)}: A mechanical property derived from the manifold's chiral bias orientation and its effective energy density.
\end{itemize}