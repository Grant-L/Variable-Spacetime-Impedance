% --- LCT Master Setup (The BIOS) ---

% 1. Core LaTeX Hardware
\usepackage[utf8]{inputenc}
\usepackage[T1]{fontenc}
\usepackage{amsmath, amssymb, amsfonts, amsthm} % Essential Math
\usepackage{bm}                                 % Bold Greek for Tensors
\usepackage{siunitx}                            % For Z_0, L, and C values
\usepackage{graphicx}                           % For simulation PNGs
\usepackage{cleveref}                           % Intelligent cross-referencing

% 2. Styling Containers (The Boxes)
\usepackage{tcolorbox}
\tcbuselibrary{skins, breakable}

% Example Environment: Blue for worked derivations
\newtcolorbox[auto counter, number within=chapter]{examplebox}[1][]{
    colback=blue!5!white, colframe=blue!75!black, fonttitle=\bfseries,
    title=Example \thetcbcounter: #1, enhanced, breakable, attach title to upper, after title={:\enskip}}

% Problem Environment: Red for theory challenges
\newtcolorbox[auto counter, number within=chapter]{problembox}[1][]{
    colback=red!5!white, colframe=red!75!black, fonttitle=\bfseries,
    title=Problem \thetcbcounter: #1, enhanced, breakable}

% Simulation Environment: Green for FDTD modules
\newtcolorbox[auto counter, number within=chapter]{simbox}[1][]{
    colback=green!5!white, colframe=green!50!black, fonttitle=\bfseries,
    title=Computational Module \thetcbcounter: #1, enhanced, breakable}

% 3. LCT Specific Macros (The "Software" Layer)
\newcommand{\Zvac}{Z_0}
\newcommand{\Wcut}{\omega_{\text{cutoff}}}
\newcommand{\epsmn}{\epsilon_{\mu\nu}}
\newcommand{\lplanck}{l_P}
\newcommand{\psiorder}{\psi(\mathbf{r}, t)}