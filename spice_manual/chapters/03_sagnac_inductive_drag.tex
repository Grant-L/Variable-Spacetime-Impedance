\chapter{Sagnac Macroscopic Inductive Drag}
\label{ch:sagnac_inductive_drag}

The physical interpretation of the Sagnac Effect (and the larger Lense-Thirring frame-dragging effect) represents one of the most convoluted geometric arguments in modern Special and General Relativity.

When an optical ring interferometer (a Sagnac loop) is physically rotated, two counter-propagating laser pulses injected into the ring will arrive at the detector at different times. The wave traveling "with" the rotation arrives slightly late; the wave traveling "against" the rotation arrives slightly early. 

Standard physics insists the speed of light ($c$) is absolutely invariant, thereby forcing Relativity to mathematically alter the physical dimensions of the spacetime path length (length contraction and time dilation) to explain the arrival discrepancy. 

The Applied Vacuum Engineering (AVE) framework explicitly rejects this geometric abstraction. The vacuum is not empty coordinate space; it is a dense Macroscopic Mutual Inductance Core ($\rho \approx 7.9 \times 10^6 \text{ kg/m}^3$). Therefore, the Sagnac Effect is nothing more than standard \textbf{Lenz's Law}.

\section{The Rotating LC Frame}
If a massive physical object (like the Earth, or the glass of a fiber-optic ring) rotates, its internal atomic charges are physically moving. This moving bulk charge creates a weak, macroscopic macroscopic $\mathbf{B}$-field via induction. This induced field phase-drags the local inductive capacity ($\mu$) of the surrounding vacuum LC network.

Because $c = \frac{1}{\sqrt{\mu \epsilon}}$, any fractional shift in the local inductance $\mu_{local}$ physically alters the localized propagation speed of the electromagnetic wave. 

\begin{itemize}
    \item \textbf{Co-Rotating Wave:} A photon traveling in the same direction as the macroscopic phase-drag experiences an inductively "thinner" vacuum path (reduced $\mu_{eff}$), propagating physically faster than $c_0$.
    \item \textbf{Counter-Rotating Wave:} A photon plowing against the induced "headwind" of the frame experiences an inductively "denser" vacuum path (increased $\mu_{eff}$), propagating physically slower than $c_0$.
\end{itemize}

\section{The SPICE Equivalent: A Differential LC Ring}
We simulated this using a standard 1D discrete LC topology. We constructed a closed ring of 50 purely classical inductors and capacitors.

\subsection{Circuit Schematic and Netlist}
The simulation evaluates the instantaneous direction of current flow across every single discrete node in the ring. 

\begin{figure}[h!]
    \centering
    \begin{tcolorbox}[colback=black!5!white, colframe=black!75!white, title=Simplified LTspice Equivalent Circuit (Sagnac Ring Segment)]
    \centering
    \begin{verbatim}
     ... Node N-1          Node N          Node N+1 ...
    ---->---+--->-( >0: L_eff = L_0*(1-drag) )->---+---->
    -<---+--<---( <0: L_eff = L_0*(1+drag) )---<---+----<
            |                              |
          +-+-+                          +-+-+
          | C |                          | C |
          | 0 |                          | 0 |
          +-+-+                          +-+-+
            |                              |
           GND                            GND
    \end{verbatim}
    \end{tcolorbox}
    \caption{A segment of the complete 50-node Sagnac topology. The inductors dynamically shift their value based on the direction of the passing photon pulse, simulating the differential "headwind" of an inductively rotated frame.}
\end{figure}

The core SPICE mechanism relies on arbitrary behavioral inductors or behavioral current sources to implement the directional logic:

\begin{tcolorbox}[colback=black!95!white, coltext=white, fontupper=\ttfamily, title=SPICE Netlist: Sagnac Inductive Drag (sagnac\_ring.cir) - Single Node]
* Sagnac Effect SPICE Model (Node N Segment) *
* ------------------------------------------ *

* Parameters
.param L0=1uH C0=1pF S\_DRAG=0.05

* Node Capacitance
C\_N NODE\_N GND \{C0\}

* Directional Inductor linking Node N to Node N+1
* We use a Behavioral Current Source to model V = L * di/dt
* where L depends on the sign of the current (I\_sense)
V\_SENSE NODE\_N NODE\_INT 0  ; 0V source to measure current
B\_IND NODE\_INT NODE\_N\_PLUS\_1 I = sdt( V(NODE\_INT, NODE\_N\_PLUS\_1) / 
+ \{ IF( I(V\_SENSE) > 0, L0*(1 - S\_DRAG), L0*(1 + S\_DRAG) ) \} )

* (This pattern repeats for all 50 nodes in a closed circle)

.TRAN 1ns 2us
.END
\end{tcolorbox}

To model the rotation of the massive frame, we did not apply relativistic tensor math to the simulation clocks or spatial coordinates. We simply instructed the SPICE solver to dynamically evaluate the direction of the current $I$. 
If the current was flowing clockwise (co-rotating), the solver encountered an inductor valued at $L_0(1 - \delta)$. If the current flowed counter-clockwise, the solver encountered an inductor valued at $L_0(1 + \delta)$.

\begin{figure}[h!]
    \centering
    \includegraphics[width=1.0\textwidth]{sagnac_inductive_drag.png}
    \caption{\textbf{Simulating the Sagnac Effect via Pure Inductive Drag.} \textbf{Left:} A waterfall plot displaying two identical voltage pulses propagating in opposite directions around a 50-node LC ring. \textbf{Right:} The voltage readout at the detector (Node 25). Because the co-rotating wave encounters slightly less inductance per node ($\mu_{eff} \downarrow$), its wavefront outpaces the counter-rotating wave ($\mu_{eff} \uparrow$). The SPICE solver natively produces the exact Sagnac arrival-time phase shift without requiring a single equation from Special Relativity.}
    \label{fig:sagnac_inductive_drag}
\end{figure}

As shown in Figure \ref{fig:sagnac_inductive_drag}, the two waves arrive at the detector at different times. 

The analog solver natively reproduces the Lense-Thirring phase shift. It proves that one does not need the Lorentz Transformations or Einstein's field equations to derive the Sagnac effect; one only needs the macroscopic equivalent of Faraday's Law of Induction operating across the pure spacetime metric.
