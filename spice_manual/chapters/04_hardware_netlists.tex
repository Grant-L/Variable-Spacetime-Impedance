\chapter{Hardware Netlists: PONDER-01 and the EE Bench}
\label{ch:hardware_netlists}

While the previous chapters utilized abstract mathematical SPICE topologies to prove theoretical dynamics (decay, autoresonance, Sagnac), this chapter documents the explicit, physical LTspice netlists generated to construct actual tabletop AVE hardware. These are not thought experiments---they are engineering blueprints with component values derived directly from the zero-parameter framework.

\section{The EE Bench: Dielectric Yield Plateau}

The EE Bench (detailed in Book 4) utilizes a $100\ \mu$m sub-millimeter vacuum gap driven to $V_{yield} \approx 43.65\text{ kV}$. The fundamental objective is to measure the asymptotic plateau of the effective capacitance ($C_{eff}$) as the localized metric approaches its absolute structural strain limit.

Standard electromagnetism predicts a perfectly linear capacitance: $C_{meas} = C_0$ at all voltages until catastrophic arc-discharge. The AVE framework predicts a smooth, measurable rolloff governed by Axiom 4:
\begin{equation}
    C_{eff}(V) = C_0 \sqrt{1 - \left(\frac{V}{V_{yield}}\right)^2}
\end{equation}

This non-linear saturation is detectable with a precision LCR meter well before any spark occurs. The ``Anomaly Window'' (approximately $0.85 \times V_{yield}$ to $V_{yield}$) represents the measurable regime where $C_{eff}$ deviates by more than $10\%$ from the linear baseline.

\subsection{EE Bench SPICE Netlist}

The SPICE model evaluates the non-linear capacitance using a behavioral charge equation ($Q = C_{eff} \times V$):

\begin{tcolorbox}[colback=black!95!white, coltext=white, fontupper=\ttfamily, title=SPICE Netlist: EE Bench Yield Plateau (ee\_bench.cir)]
* EE Bench Dielectric Yield Shift SPICE Model *
* -------------------------------------------- *

* Parameters
.param C0=10pF V\_yield=43650

* DC Sweep Source (0 to 45 kV)
V\_SWEEP N\_GAP GND DC 0

* Non-Linear Vacuum Capacitance
* Q = C\_eff * V = C0 * sqrt(1 - (V/V\_yield)\string^2) * V
B1 N\_GAP GND Q = \{C0 *
+ sqrt(1 - min((V(N\_GAP)/V\_yield)**2, 0.999))\}
+ * V(N\_GAP)

* Parasitic series resistance (connector + trace)
R\_PAR N\_GAP GND 1G

.DC V\_SWEEP 0 45000 100
.PROBE I(V\_SWEEP)
.END
\end{tcolorbox}

The DC sweep from $0$ to $45\text{ kV}$ in $100\text{ V}$ steps produces a charge accumulation curve $Q(V)$ whose slope ($dQ/dV = C_{eff}$) smoothly deviates from linear above $\sim 37\text{ kV}$. Plotting $C_{meas}/C_0$ vs.\  $V$ reveals the characteristic AVE saturation plateau.

\section{PONDER-01: Cascaded Transmission-Line Thrust Model}

The PONDER-01 experimental thruster utilizes an asymmetric FR4/Air dielectric stack to intentionally unbalance the vacuum's thermodynamic acoustic modes. By driving the stack with an extreme $100\text{ MHz}$, $30\text{ kV}$ RF sine wave, it actively pumps acoustic ``phonons'' from the background vacuum matrix into the heavier FR4 substrate.

Because the system is geometrically asymmetric, the acoustic energy cannot rebound cleanly; it is trapped by the impedance mismatch at the boundary layer, generating continuous unidirectional Ponderomotive thrust.

\subsection{Impedance Mismatch at Each Boundary}

Each air layer presents an impedance of $Z_0 = \sqrt{\mu_0/\varepsilon_0} \approx 376.7\ \Omega$, while each FR4 layer presents $Z_{FR4} = Z_0/\sqrt{\varepsilon_r} \approx 181.6\ \Omega$ (with $\varepsilon_r = 4.3$). The resulting reflection coefficient at each boundary:
\begin{equation}
    \Gamma = \frac{Z_{FR4} - Z_0}{Z_{FR4} + Z_0} \approx -0.349
\end{equation}

This $34.9\%$ reflection at every air/FR4 interface creates a cascading series of partial reflections that geometrically trap RF energy in the stack---precisely the mechanism that generates the asymmetric $\nabla |E|^2$ gradient responsible for ponderomotive thrust.

\subsection{PONDER-01 SPICE Netlist}

The SPICE topology maps each sub-millimeter physical layer into its equivalent lumped LC element. The asymmetric He-4 emitter tip is modeled using the topological coordinates of the alpha particle nucleus:

\begin{tcolorbox}[colback=black!95!white, coltext=white, fontupper=\ttfamily, title=SPICE Netlist: PONDER-01 Cascaded Stack (ponder\_01\_stack.cir)]
* PONDER-01 Asymmetric Transmission-Line Model *
* --------------------------------------------- *

* Parameters
.param L\_AIR=0.33nH C\_AIR=2.36fF
.param L\_FR4=0.33nH C\_FR4=10.14fF
.param V\_DRIVE=30000 F\_DRIVE=100MEG

* VHF Drive Source (100 MHz, 30 kV)
V1 NODE\_0 GND SINE(0 \{V\_DRIVE\} \{F\_DRIVE\})

* Layer 1: Air (100 um)
L1 NODE\_0 NODE\_1 \{L\_AIR\}
C1 NODE\_1 GND \{C\_AIR\}

* Layer 2: FR4 (100 um)
L2 NODE\_1 NODE\_2 \{L\_FR4\}
C2 NODE\_2 GND \{C\_FR4\}

* Layer 3: Air
L3 NODE\_2 NODE\_3 \{L\_AIR\}
C3 NODE\_3 GND \{C\_AIR\}

* Layer 4: FR4
L4 NODE\_3 NODE\_4 \{L\_FR4\}
C4 NODE\_4 GND \{C\_FR4\}

* ... (Repeat for 20 total layers) ...

* Layer 19: Air
L19 NODE\_18 NODE\_19 \{L\_AIR\}
C19 NODE\_19 GND \{C\_AIR\}

* Layer 20: FR4 (Collector)
L20 NODE\_19 NODE\_20 \{L\_FR4\}
C20 NODE\_20 GND \{C\_FR4\}

* Termination (Collector grounded through load)
R\_LOAD NODE\_20 GND 50

.TRAN 0.1n 100n
.PROBE V(NODE\_0) V(NODE\_10) V(NODE\_20)
.END
\end{tcolorbox}

The transient simulation tracks the voltage waveform at the emitter (Node 0), mid-stack (Node 10), and collector (Node 20). The asymmetric buildup of $|E|^2$ across the stack is directly proportional to the ponderomotive thrust force.

\begin{figure}[h!]
    \centering
    \includegraphics[width=1.0\textwidth]{hardware_netlist_overview.png}
    \caption{\textbf{Hardware Netlist Engineering Overview.} \textbf{Left:} The EE Bench capacitance plateau---standard EM predicts flat $C_{eff}/C_0 = 1$ (red dashed) while AVE predicts smooth saturation rolloff (cyan) approaching $V_{yield} = 43.65\text{ kV}$. The yellow band marks the detectable anomaly window. \textbf{Center:} The PONDER-01 cascaded LC stack impedance, alternating between air ($Z_0 \approx 377\ \Omega$) and FR4 ($Z \approx 182\ \Omega$) layers. \textbf{Right:} The resulting ponderomotive gradient $\nabla |E|^2$ across the stack, showing asymmetric force concentration at each dielectric boundary.}
    \label{fig:hardware_netlist_overview}
\end{figure}

\section{Component Derivation from Zero Parameters}

Every component value in the above netlists is derived from the four AVE axioms with zero free parameters:

\begin{center}
\begin{tabular}{|l|c|l|}
\hline
Component & Value & Derivation \\ \hline
$V_{yield}$ & 43.65 kV & $\sqrt{\alpha} \times m_e c^2 / e$ \\
$C_{AIR}$ & 2.36 fF & $\varepsilon_0 \times A_{layer} / d_{layer}$ \\
$C_{FR4}$ & 10.14 fF & $\varepsilon_r \times \varepsilon_0 \times A_{layer} / d_{layer}$ \\
$L_{AIR}$ & 0.33 nH & $\mu_0 \times d_{layer} / A_{layer}$ \\
$Z_0$ & 376.7 $\Omega$ & $\sqrt{\mu_0 / \varepsilon_0}$ \\
$f_{drive}$ & 100 MHz & VHF resonance of stack \\ \hline
\end{tabular}
\end{center}

No fitting, no tuning, no empirical calibration. The SPICE solver runs against the raw physical constants of the universe.
