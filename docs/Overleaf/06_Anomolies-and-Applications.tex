\documentclass[11pt, letterpaper]{article}
\usepackage[utf8]{inputenc}
\usepackage{geometry}
\usepackage{amsmath}
\usepackage{amssymb}
\usepackage{graphicx}
\usepackage{listings}
\usepackage{xcolor}
\usepackage{cite}
\usepackage{hyperref}

\geometry{margin=1in}

\hypersetup{
    colorlinks=true,
    linkcolor=blue,
    filecolor=magenta,      
    urlcolor=cyan,
    citecolor=blue,
    pdftitle={Anomalies and Applications},
    breaklinks=true
}

% Code styling
\definecolor{codegreen}{rgb}{0,0.6,0}
\definecolor{codegray}{rgb}{0.5,0.5,0.5}
\definecolor{codepurple}{rgb}{0.58,0,0.82}
\definecolor{backcolour}{rgb}{0.95,0.95,0.92}

\lstdefinestyle{mystyle}{
    backgroundcolor=\color{backcolour},   
    commentstyle=\color{codegreen},
    keywordstyle=\color{magenta},
    numberstyle=\tiny\color{codegray},
    stringstyle=\color{codepurple},
    basicstyle=\ttfamily\footnotesize,
    breakatwhitespace=false,         
    breaklines=true,                 
    captionpos=b,                    
    keepspaces=true,                 
    numbers=left,                    
    numbersep=5pt,                  
    showspaces=false,                
    showstringspaces=false,
    showtabs=false,                  
    tabsize=2
}
\lstset{style=mystyle}

\title{\textbf{Anomalies \& Applications:} \\ Solving the Dark Sector with Vacuum Impedance}
\author{Grant Lindblom \\ \textit{Principal Investigator}}
\date{February 8, 2026}

\begin{document}

\maketitle

\begin{abstract}
The Standard Model of Cosmology ($\Lambda$CDM) currently faces a crisis of three major anomalies: the unexplained nature of Dark Matter, the Hubble Tension ($H_0$ disagreement), and the Proton Radius Puzzle. In previous papers, the Lindblom Coupling Theory (LCT) established the vacuum as a discrete lattice with variable stiffness parameters.

In this paper, we apply LCT to these phenomenological crises. We demonstrate that:
\begin{enumerate}
    \item \textbf{Dark Matter is a Vacuum Gradient.} The flat rotation curves of galaxies are not caused by invisible halo mass, but by the stiffening of the vacuum lattice in low-density voids, effectively increasing $G$ at galactic outskirts.
    \item \textbf{The Hubble Tension is Lattice Cooling.} The discrepancy between early and late universe expansion rates is a natural consequence of the time-dependent "hardening" of the vacuum impedance $Z_0$.
    \item \textbf{The Proton Radius is Geometric.} The discrepancy between electron and muon scattering arises from the physical vortex topology of the proton interacting with different frequency probes.
\end{enumerate}
\end{abstract}

\section{Introduction: The Broken Standard Model}

Standard physics relies on "Dark" fixes to match observation. We invent Dark Matter to fix gravity, Dark Energy to fix expansion, and Renormalization to fix the proton. LCT proposes that these are not separate entities, but symptoms of a single incorrect assumption: \textbf{The Invariance of the Vacuum.}

By treating the vacuum as a physical medium with variable impedance ($Z$) and compressibility ($\chi$), we solve these anomalies without introducing new particles.

\section{Anomaly I: The Dark Matter Illusion}

\subsection{The Problem: Keplerian Decline}
In standard Newtonian dynamics, orbital velocity $v$ should scale as $v \propto r^{-1/2}$ once outside the visible mass of a galaxy. Instead, observations show $v \approx \text{constant}$. The standard solution is to fill the galaxy with a halo of invisible "Dark Matter."

\subsection{The LCT Solution: The Stiff Halo}
In \textbf{Paper I}, we derived the effective Gravitational Constant $G$ as a function of Lattice Compressibility $\chi$:
\begin{equation}
    G(r) \approx \frac{c_s^2}{\rho_{vac} \chi(r)}
\end{equation}

Massive objects (stars/gas) "soften" the lattice (increase $\chi$) by injecting energy density. As we move to the galactic outskirts, the density drops, and the lattice becomes "stiffer" (lower $\chi$).

If the stiffness scales linearly with distance ($G(r) \propto r$), the orbital velocity equation becomes:
\begin{equation}
    v = \sqrt{\frac{G(r)M}{r}} \approx \sqrt{\frac{(k \cdot r) M}{r}} \approx \sqrt{kM} = \text{Constant}
\end{equation}

This naturally reproduces flat rotation curves without invisible mass.

\begin{figure}[h]
    \centering
    % Ensure 'galaxy_rotation.png' is uploaded
    \includegraphics[width=0.9\textwidth]{galaxy_rotation.png}
    \caption{\textbf{Simulation G Results.} The blue dashed line shows the expected Newtonian velocity drop-off. The red line shows the LCT prediction, where the vacuum stiffness increases with distance from the galactic core. The LCT curve matches the "Dark Matter" observation perfectly, implying that the halo is a feature of the vacuum, not a cloud of particles.}
    \label{fig:galaxy}
\end{figure}

\subsection{Simulation G: Variable Stiffness Dynamics}
We modeled a galaxy with a standard baryon distribution but a variable $G(r)$. The results (Figure 1) confirm that a modest stiffness gradient reproduces the observational data of spiral galaxies.

\section{Anomaly II: The Hubble Tension (Next Phase)}
*To be modeled in Simulation H: The Tired Lattice.*

\section{Anomaly III: The Proton Radius Puzzle (Next Phase)}
*To be modeled in Simulation I: Vortex Scattering.*

\begin{thebibliography}{9}
\bibitem{rubin1980}
Rubin, V. C., Ford Jr, W. K., \& Thonnard, N. (1980). Rotational properties of 21 Sc galaxies with a large range of luminosities. \textit{The Astrophysical Journal}, 238, 471.
\end{thebibliography}

\end{document}