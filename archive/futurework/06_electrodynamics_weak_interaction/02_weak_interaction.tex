\section{The Weak Interaction: Micropolar Cutoff Dynamics}

The Weak Force is profoundly unique in the Standard Model because it is extraordinarily short-ranged ($\approx 10^{-18}$ m) and is mediated by massively heavy gauge bosons ($W \approx 80.4$ GeV, $Z \approx 91.2$ GeV). The Standard Model heuristically explains this via spontaneous symmetry breaking and the mathematically abstract Higgs Mechanism. The AVE framework derives this natively and mechanically from the \textbf{Characteristic Cutoff Scale} of a trace-reversed Cosserat continuum.

\subsection{Rigorous Derivation: The Cosserat Cutoff Length}
In Chapter 1, we mathematically established that to prevent catastrophic causality violations (superluminal longitudinal P-waves), the vacuum substrate must act structurally as a \textbf{Trace-Reversed Cosserat Solid}. A Cosserat solid natively possesses an independent microrotational couple-stress stiffness ($\gamma_c$) alongside its standard macroscopic shear modulus ($G_{vac}$).

In classical solid mechanics, the ratio of the microrotational bending stiffness to the macroscopic shear modulus rigidly defines a fundamental \textbf{Characteristic Length Scale} ($l_c$). This length scale dictates the maximum spatial extent to which localized couple-stresses (isolated twists) can propagate before the intrinsic ambient stiffness of the solid completely damps them out:
\begin{equation}
    l_c = \sqrt{\frac{\gamma_c}{G_{vac}}}
\end{equation}
We formally identify this exact mechanical decay length ($l_c$) as the physical origin of the Weak Force range ($r_W \approx 10^{-18}$ m). 

\subsection{Mechanical Origin of the Yukawa Potential}
Why does the Weak Force die off so rapidly while Electromagnetism possesses infinite range? 

Electromagnetism operates \textit{above} the vacuum's acoustic mass gap (it is massless), allowing the signal to propagate freely as a standard inverse-square Laplace field. However, static Weak interactions lack the immense kinetic energy required to overcome the ambient Cosserat rotational stiffness. 

In wave mechanics, any physical excitation operating \textit{below} a medium's natural cutoff frequency cannot physically propagate; it is mathematically forced to become an \textbf{Evanescent Wave} that decays exponentially. Because the Weak Force operates below the Cosserat cutoff frequency, its static field equation mathematically transforms from the standard Laplace equation ($\nabla^2 \theta = 0$) to the massive Helmholtz equation:
\begin{equation}
    \nabla^2 \theta - \frac{1}{l_c^2}\theta = 0
\end{equation}
The unique spherically symmetric solution to this damped equation natively yields the exact \textbf{Yukawa Potential}:
\begin{equation}
    V_{weak}(r) \propto \frac{e^{-r/l_c}}{r}
\end{equation}
The Weak Force is short-range exclusively because it is mathematically and physically evanescent (see Figure \ref{fig:weak_yukawa}).

\begin{figure}[htbp]
    \centering
    \includegraphics[width=0.9\textwidth]{chapters/06_electrodynamics_weak_interaction/simulations/outputs/weak_yukawa_cutoff.png}
    \caption{\textbf{Mechanical Origin of the Weak Force Cutoff.} The $\mathcal{M}_A$ Cosserat vacuum acts as a strict high-pass mechanical filter. Massless electromagnetism operates above the gap, propagating infinitely. The Weak interaction lacks the energy to overcome the Cosserat rotational mass gap ($\gamma_c$). Operating below the spatial cutoff, it propagates purely as a mechanical Evanescent Wave, perfectly reproducing the exponential decay of the Yukawa Potential without Higgs fields.}
    \label{fig:weak_yukawa}
\end{figure}