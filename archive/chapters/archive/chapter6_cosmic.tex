\chapter[Cosmic Evolution]{Cosmic Evolution: The Cosmic Quench and Metric Aging}
\label{ch:cosmic_evolution}

\section{The Quench Hypothesis}
The \textbf{Stochastic Vacuum Framework (SVF)} rejects the assumption that the fundamental constants of nature ($\mu_0, \epsilon_0, c$) are static throughout the history of the universe. Instead, we propose the \textbf{Cosmic Quench}: a thermodynamic and mechanical relaxation of the $M_A$ substrate from its primordial high-saturation state. 

In the early universe ($z \gg 10$), the lattice nodes were in a state of near-total saturation due to high flux density. This resulted in low \textbf{Metric Impedance} and significantly higher propagation speeds. As the manifold expanded, the flux density diluted, allowing the nodes to transition into their modern, high-impedance "locked" ground state.

\section{The Impedance Evolution Equation}
The background \textbf{Characteristic Impedance} ($\Zvac$) of the vacuum is a function of the cosmic scale factor $a(t)$. We model this evolution as a relaxation curve:

\begin{equation}
    \Zvac(t) = \Zvac^{(modern)} \left( 1 - e^{-\gamma / a(t)} \right)
\end{equation}

Where:
\begin{itemize}
    \item $\Zvac^{(modern)} \approx 376.73 \, \Omega$ is the currently measured vacuum impedance.
    \item $a(t)$ is the expansion scale factor.
    \item $\gamma$ is the \textbf{Quench Constant}, representing the lattice relaxation rate.
\end{itemize}



\section{Variable Speed of Light and the Horizon Problem}
Because $c = 1/\sqrt{\Lvac \Cvac}$, the SVF framework naturally resolves the \textbf{Horizon Problem} without requiring the ad-hoc addition of an "Inflation" field. In the high-saturation early epoch, the slew rate of the nodes was orders of magnitude higher than the modern value. This allowed for thermal equilibrium to be established across the entire manifold before the quench "throttled" the global propagation speed to its current value.

\section{Metric Aging and Radioactive Decay}
VSI posits that the rate of radioactive decay is not an immutable constant, but a frequency-dependent lattice response. The decay constant $\lambda$ is inversely proportional to the background metric impedance:

\begin{equation}
    \lambda(t) \propto \frac{1}{\Zvac(t)}
\end{equation}



This implies that radioactive clocks (e.g., Carbon-14, Uranium-Lead) ran faster in the low-impedance past. Recalibrating these chronometers against the \textbf{Impedance Evolution Curve} is a primary requirement for means-testing the historical accuracy of the SVF framework.

\section{The Stability of the Fine Structure Constant ($\alpha$)}
To pass the "Spectroscopic Audit," SVF requires that the Fine Structure Constant $\alpha = \frac{e^2}{2 \epsilon_0 h c}$ remain relatively stable over cosmic time. In this framework, $\epsilon_0$ and $c$ shift in a coupled ratio dictated by the node geometry. As $\Cvac$ ($\epsilon_0$) increases during the quench, the global slew rate ($c$) decreases proportionally. This ensuring that while the "hardware speed" changes, the ratio defining atomic transition energies remains consistent with observations of distant quasars.

\section{Exercises}
\begin{problembox}[Cosmic Evolution Challenges]
\begin{enumerate}
    \item \textbf{The Redshift Correction}: Derive the relationship between cosmological redshift $z$ and the shifting impedance $\Zvac(t)$. 
    \item \textbf{High-Flux Biology}: Calculate the required $\Zvac$ value in a low-impedance epoch that would allow biological structures to maintain double their modern skeletal stress limit.
    \item \textbf{Quench Rate}: Given the measured stability of $c$ over the last 100 years, calculate the upper bound for the modern Quench Constant $\gamma$.
\end{enumerate}
\end{problembox}