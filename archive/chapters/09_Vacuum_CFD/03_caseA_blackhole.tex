\section{Case Study A: The Black Hole as a Trans-Sonic Sink}
\label{sec:black_hole_cfd}

General Relativity describes a Black Hole as a geometric singularity. VCFD describes it as a **Trans-Sonic Fluid Sink**.

\subsection{The River Model}
We adopt the Gullstrand-Painlevé coordinate system, often called the "River Model" of gravity. Space flows into the black hole like a river falling into a waterfall.
\begin{equation}
    v_{flow}(r) = -\sqrt{\frac{2GM}{r}}
\end{equation}
The speed of light ($c$) is the speed of sound ($c_s$) in this river.

\subsection{The Sonic Horizon}
The Event Horizon is physically identified as the **Sonic Point** (Mach 1).
\begin{itemize}
    \item \textbf{Outside ($r > R_s$):} The river moves slower than sound ($v_{flow} < c$). Light can swim upstream and escape.
    \item \textbf{Horizon ($r = R_s$):} The river moves at the speed of sound ($v_{flow} = c$). Light trying to escape is frozen in place (Standing Wave).
    \item \textbf{Inside ($r < R_s$):} The river is supersonic ($v_{flow} > c$). All signals are swept inward to the singularity.
\end{itemize}

\begin{figure}[ht]
    \centering
    \includegraphics[width=0.8\textwidth]{assets/sim_outputs/black_hole_cfd.png}
    \caption{\textbf{CFD Simulation of an Event Horizon.} The streamlines show the vacuum fluid flowing into the sink. The blue dashed line marks the Sonic Horizon (Mach 1), where the inflow velocity equals the wave propagation speed $c$. Inside this boundary, the flow is supersonic, and no signal can propagate outward.}
    \label{fig:black_hole_cfd}
\end{figure}