\section{Macroscopic Topology: Ball Lightning and Plasmoids}
\label{sec:ball_lightning}

Ball Lightning (BL) remains one of the few macroscopic phenomena that defies standard explanation. It appears as a glowing, floating orb ($10-100$ cm) that persists for seconds or minutes before decaying or exploding.

Standard plasma physics struggles to explain BL stability: the virial theorem predicts that a ball of plasma should expand and dissipate in milliseconds due to thermal pressure. VSI resolves this by identifying BL as a **Macro-Scale Topological Soliton**.

\subsection{The Inductive Containment Mechanism}
In VSI, a plasmoid is a self-organized torus of magnetic flux. Stability is not maintained by external pressure, but by **Lattice Saturation**.
\begin{itemize}
    \item \textbf{High Winding Number ($N \gg 1$):} The core of the plasmoid possesses immense helicity (Twist).
    \item \textbf{Inductive Crowding:} As derived in Section 4.3.3 ($m \propto N^9$), the high winding density saturates the local vacuum dielectric ($U \to U_{sat}$).
    \item \textbf{Self-Confinement:} The lattice stiffness ($Z$) spikes at the boundary of the soliton. The plasmoid effectively "digs a hole" in the vacuum metric, creating a potential well that traps the plasma against thermal expansion.
\end{itemize}

\subsection{Decay Modes}
The VSI model predicts two distinct decay modes, matching observations:
\begin{enumerate}
    \item \textbf{Silent Decay:} The winding number $N$ unwinds gradually via vacuum friction ($\alpha$), releasing heat slowly.
    \item \textbf{Explosive Unraveling:} If the topological integrity is breached (e.g., passing through a conductor), the knot snaps. The stored inductive energy ($E = \frac{1}{2}LI^2$) is released instantly as a shockwave, similar to a macroscopic "particle annihilation."
\end{enumerate}

\begin{figure}[ht]
    \centering
    \includegraphics[width=0.8\textwidth]{assets/sim_outputs/ball_lightning_sim.png}
    \caption{\textbf{Ball Lightning Simulation.} The plot shows a cross-section of a stable VSI Plasmoid. The \textbf{Red Core} represents the high-helicity plasma trapped in a self-generated potential well. The \textbf{Blue Boundary} shows the Inductive Saturation wall that prevents thermal dissipation, allowing the structure to persist for seconds.}
    \label{fig:ball_lightning}
\end{figure}