\chapter*{Introduction}
\addcontentsline{toc}{chapter}{Introduction}

This document presents experimental protocols for testing Variable Spacetime Impedance Theory (VSIT). The theory makes specific, quantitative predictions about the behavior of the vacuum as a physical medium, and these experiments are designed to test those predictions through direct measurement and observation.

\section*{Experimental Philosophy}

The experiments described in this document follow a systematic approach:

\begin{itemize}
    \item \textbf{Reproducibility:} All protocols include sufficient detail for independent replication.
    \item \textbf{Quantitative Predictions:} Each experiment tests specific, measurable quantities predicted by VSIT.
    \item \textbf{Control Experiments:} Standard model predictions are explicitly compared with VSIT predictions.
    \item \textbf{Systematic Uncertainties:} Measurement techniques and error analysis are documented for each protocol.
\end{itemize}

\section*{Organization}

This document is organized by experimental category:
\begin{itemize}
    \item \textbf{Tabletop Experiments:} Laboratory-scale tests that can be performed with standard equipment.
    \item \textbf{Observational Tests:} Tests using astronomical and cosmological observations.
    \item \textbf{Computational Simulations:} Numerical tests of theoretical predictions.
\end{itemize}

Each experimental protocol includes:
\begin{enumerate}
    \item Theoretical background and VSIT predictions
    \item Experimental setup and procedures
    \item Measurement techniques and data analysis
    \item Expected results and comparison with standard model
    \item Discussion of systematic uncertainties and limitations
\end{enumerate}
