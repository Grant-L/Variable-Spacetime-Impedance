\title{\textbf{Experimental Protocols} \\ \large \textit{Testing Variable Spacetime Impedance Theory}}
\author{Grant Lindblom}
\date{2026}

\maketitle

\vfill
\noindent\textbf{Experimental Protocols: Testing Variable Spacetime Impedance Theory} \\
This document presents experimental designs and protocols for testing predictions of Variable Spacetime Impedance Theory through tabletop experiments, observational tests, and computational simulations.

\begin{abstract}
    Variable Spacetime Impedance Theory (VSIT) makes specific, falsifiable predictions about the behavior of the vacuum as a physical medium. This document outlines experimental protocols designed to test these predictions through direct measurement and observation.
    
    The experiments described herein are designed to probe:
    \begin{itemize}
        \item \textbf{Vacuum Impedance Variations:} Direct measurement of spatial and temporal variations in the characteristic impedance of free space.
        \item \textbf{Gravitational Coupling:} Tests of the relationship between impedance gradients and gravitational effects.
        \item \textbf{Topological Defects:} Detection and characterization of vacuum lattice defects and their interactions.
        \item \textbf{Non-Linear Dielectric Response:} Measurement of vacuum saturation effects at high field strengths.
    \end{itemize}
    
    Each experimental protocol includes detailed setup procedures, measurement techniques, expected results under VSIT, and comparison with standard model predictions. These experiments are designed to be reproducible and provide clear, quantitative tests of the theory's predictions.
\end{abstract}
