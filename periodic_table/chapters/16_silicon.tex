\chapter{Silicon-28 ($^{28}_{14}\text{Si}$): The Seven-Alpha Bipyramid}
\label{ch:silicon}

With 14 protons and 14 neutrons bounding into absolute symmetry, Silicon-28 ($Z=14, A=28$) fully closes the next perfect scalar topological shell. Like Oxygen ($4\alpha$), Neon ($5\alpha$), and Magnesium ($6\alpha$), Silicon constructs flawlessly from exactly 7 structurally isolated Alpha geometries ($7\alpha$).

The most thermodynamically stable spatial arrangement for 7 mutually repulsive, identical geometric nodes bound across a contiguous primary sphere is a \textbf{Pentagonal Bipyramid}. This architecture locks 5 Alpha clusters equidistantly across an equatorial $Z=0$ ring, bound axially by 2 pole Alpha clusters holding strictly at $Z = \pm R$.

\section{Symmetric Core Collapse}
By running the $M_{ij}$ array to target the empirical CODATA Nuclear Mass of Silicon-28 ($26053.188$ MeV), the $7\alpha$ solver generates a mathematically pure, highly symmetric envelope. 

In asymmetric nucleonic shells, the partial valences exhibit massive reactive levers (like the $351d$ Fluorine whip or even the moderate $53.1d$ Aluminum halo). When the shell completes perfectly as with the $7\alpha$ Pentagonal Bipyramid, the geometric envelope collapses beautifully: exactly **$R_{bipyramid} = 80.174d$**.

To hit the exact $26053.188$ MeV parameter limit binding the 28 nucleons, the solver coordinates precisely 378 dynamic interconnected $LC$ network elements. Across every individual iteration, the fundamental topological rule of Variable-Spacetime Impedance holds perfectly true: closed integer sub-cluster sets produce symmetric, highly-stable geometries. Incomplete sequences generate large-scale asymmetric moments responsible for reactive electronegativity.

\begin{figure}[htbp]
    \centering
    \includegraphics[width=0.8\textwidth]{figures/silicon_28_dynamic_flux.png}
    \caption{The $Z=0$ Equatorial cross-layer for Silicon-28. The 5 primary Alpha macro-clusters position exactly 72 degrees apart. Only the nodes in the pure equator are held solidly; the remaining 2 Alpha poles exist above and below the viewing plane in deep vacuum shadow.}
    \label{fig:silicon_28_density}
\end{figure}

\begin{figure}[htbp]
    \centering
    \includegraphics[width=0.9\textwidth]{figures/circuit_si28.pdf}
    \caption{The Pentagonal Bipyramid core network schematic for Silicon-28. This 378-element inductive matrix connects the 20-nucleon equatorial band directly into the 8 nucleons bounded at the symmetric poles.}
    \label{fig:circuit_si28}
\end{figure}
