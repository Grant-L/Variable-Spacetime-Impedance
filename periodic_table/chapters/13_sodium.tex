\chapter{Sodium-23 ($^{23}_{11}\text{Na}$): The Alkali Halogen Paradox}
\label{ch:sodium}

Sodium-23 ($Z=11, A=23$) kicks off the Alkali Metal series following the perfectly balanced Neon-20 ($5\alpha$) noble gas. The structural geometry of Sodium provides a spectacular proof of the AVE framework's mechanical rigidity, because Sodium-23 is geometrically almost identical to Fluorine-19, yet exists chemically on the exact opposite side of the periodic table. Let us examine the topological paradox.

We established in Chapter \ref{ch:fluorine} that Fluorine-19 consists of a stable $4\alpha$ Oxygen core acting as an impassable void, forcing a 3-nucleon Tritium ($^3\text{H}$) halo to bind externally at an extreme distance of $351.019d$.

When we move past Neon-20 (a fully symmetrical $5\alpha$ Triangular Bipyramid core), the next stable isotope is Sodium-23. The configuration? A stable Neon-20 core acting as an impassable void, forcing a 3-nucleon Tritium ($^3\text{H}$) halo to bind externally.

\section{The Core Proximity Effect ($351d$ vs $50d$)}
Despite sharing the exact same $^3\text{H}$ structural origin, Fluorine and Sodium are chemical opposites (an extreme Halogen vs an extreme Alkali metal). If electronegativity is just a function of possessing a Tritium bound halo, how can this be true?

The variable $R_{halo}$ holds the absolute mechanical answer.

By mapping the empirical CODATA nuclear mass of Sodium-23 ($21409.214$ MeV) across the $23$-node $M_{ij}$ array, the optimization engine perfectly snaps the Tritium triangle at exactly $R_{halo} = 50.733d$ directly above the Neon-20 Bipyramid's North Pole.

\begin{itemize}
    \item \textbf{Fluorine-19 ($R_{halo} = 351d$):} The $4\alpha$ core is relatively small and weakly inductive. The Tritium halo is violently repelled far out into space, creating a massive, unstable 170fm reactive whip. Thus, Fluorine acts as a profound receiver.
    \item \textbf{Sodium-23 ($R_{halo} = 50d$):} The $5\alpha$ core is massively dense and highly inductive. The extreme mutual attraction rips the Tritium halo deep down towards the core pole. At $50d$, the Tritium triangle is strapped tightly against the core array. Because the halo is bound so rigidly, it acts not as a distant reactive whip, but as a hard asymmetric localized bulge. This short moment arm defines Alkali metal stripping dynamics.
\end{itemize}

\begin{figure}[htbp]
    \centering
    \includegraphics[width=0.8\textwidth]{figures/sodium_23_dynamic_flux.png}
    \caption{The topological vacuum flux slice for Sodium ($^{23}_{11}\text{Na}$) highlighting the polar geometric offset. The highly inductive Neon-20 ($5\alpha$) Bipyramid core pulls the Tritium array in to an incredibly tight $50.733d$ radius, differentiating its chemical reactivity from the Fluorine $351d$ geometry.}
    \label{fig:sodium_23_density}
\end{figure}

\begin{figure}[htbp]
    \centering
    \includegraphics[width=0.9\textwidth]{figures/circuit_na23.pdf}
    \caption{The Equivalent AC Network representation for Sodium-23. The $253$-point matrix creates distinct bandpass filters for the 20-node core and the highly-coupled 3-node polar halo array.}
    \label{fig:circuit_na23}
\end{figure}
