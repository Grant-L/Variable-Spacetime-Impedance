\chapter{Z=2: Helium}
\label{ch:helium}

\section{The Alpha Particle ($^4He$)}
The Helium-4 nucleus (the Alpha Particle) forms the first perfectly symmetrical closed topological knot shell in the AVE framework.

By structurally interlocking two $6^3_2$ protons and two corresponding neutrons, the resulting macro-knot minimizes external geometric strain. It forms an exceptionally tight, quasi-spherical localized "hardness" zone within the vacuum lattice. This geometry natively explains the immense binding energy per nucleon observed in Alpha particles and their tendency to be spontaneously ejected as unified blocks during heavy-element decay.

\section{Continuous Vacuum Strain (Topological Mass)}
While the core of the nucleon is a discrete topological knot, its geometric presence induces a continuous refractive strain upon the surrounding vacuum metric (the origin of gravitation). By treating the $6^3_2$ knot centers as Faddeev-Skyrme defect cores, we can calculate the 2D spatial gradient of this strain.

\begin{figure}[htbp]
    \centering
    \begin{minipage}{0.48\textwidth}
        \centering
        \includegraphics[width=\textwidth]{figures/helium_4_density_z_pos.png}
        \caption{Vacuum strain density slice at $Z=0.85$, intersecting the two upper proton knot centers.}
        \label{fig:he_density_pos}
    \end{minipage}\hfill
    \begin{minipage}{0.48\textwidth}
        \centering
        \includegraphics[width=\textwidth]{figures/helium_4_density_equator.png}
        \caption{Equatorial vacuum strain density ($Z=0.0$). The discrete knots visually blend into a unified macroscopic gravitational well.}
        \label{fig:he_density_equator}
    \end{minipage}
\end{figure}

The vector flux arrows in Figures \ref{fig:he_density_pos} and \ref{fig:he_density_equator} explicitly trace the spatial gradient of the packing fraction $p_c$ towards the knot centroids, visualizing the macroscopic topological ``gravity'' emerging from discrete chiral geometry.

\section{Electrical Engineering Equivalent: Polyphase Resonant Transformer}
Because the four discrete $6^3_2$ topological defects lock into a perfectly symmetrical tetrahedron, Helium-4 acts conceptually identically to a \textbf{Polyphase Resonant Transformer} in classic Electrical Engineering.

Every primary inductive load (nucleon) is equally coupled to every other load in the core via mutual spatial inductance ($M \propto 1/d_{core}$). No new symbols or mathematics are required to map this behavior; standard dashed mutual coupling arrows perfectly describe the gravitational/strong force flux interlocking the geometry. Because the circuit is symmetrically balanced, the total stored reactive energy is vastly minimized, producing the immense \"Binding Energy\" (Mass Defect) observed empirically.

\begin{figure}[htbp]
    \centering
    \includegraphics[width=0.8\textwidth]{figures/circuit_he4.png}
    \caption{\textbf{Equivalent EE Circuit for Helium-4.} A symmetrically balanced, 4-node fully-coupled polyphase inductive network. The identical mutual coupling $M$ minimizes the total network impedance, resulting in extreme stability.}
    \label{fig:he4_circuit}
\end{figure}
