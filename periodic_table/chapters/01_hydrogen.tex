\chapter{Z=1: Hydrogen}
\label{ch:hydrogen}

\section{Protium ($^1H$)}
The simplest possible atomic state consists of a singular $6^3_2$ Borromean proton defect anchored by the $3_1$ trefoil electron defect orbiting its refractive gravity well. 

\begin{figure}[htbp]
    \centering
    \includegraphics[width=0.8\textwidth]{figures/hydrogen_1_density.png}
    \caption{\textbf{Protium Vacuum Flux.} The continuous, symmetric $1/r$ vacuum strain and flux streamplot generated by a single $6^3_2$ localized topological defect. This isotropic gradient constitutes the classical electrical and gravitational fields.}
    \label{fig:h1_density}
\end{figure}

\section{Electrical Engineering Equivalent: The Coupled Tank}
In terms of classical Electrical Engineering, the Protium topology acts natively as a loosely-coupled dual-tank Resonant LC Circuit. 

The massive $6^3_2$ Borromean core forms the primary localized Reactive Inductive load. The orbiting $3_1$ trefoil electron forms the secondary phase tank. The surrounding spatial gravitational field structurally provides the geometric mutual inductance ($M_{orbit} \propto 1/r_{Bohr}$) connecting the two.

\begin{figure}[htbp]
    \centering
    \includegraphics[width=0.7\textwidth]{figures/circuit_h1.png}
    \caption{\textbf{Equivalent EE Circuit for Hydrogen-1.} A primary inductive core coupled via mutual geometric inductance to a secondary orbital tank.}
    \label{fig:h1_circuit}
\end{figure}


\section{Deuterium ($^2H$)}
The addition of a neutron ($6^3_2 + \text{axial twist}$) geometrically links with the proton, forming a heavily anisotropic "dumbbell" defect. This drastically alters the local spatial drag and acoustic cross-section.

\section{Tritium ($^3H$)}
The topological strain of interlocking three $6^3_2$ defects forces the overall knot into a state of severe internal mechanical tension, spontaneously unraveling (beta decaying) to stabilize the local topology.

\section{Topological Area of Interest: Stellar Compression \& S-Parameters}
When analyzing the Protium $^1H$ topology purely as an EE Resonant Tank, its exceptionally small geometrical footprint translates directly into an exceedingly small $S_{11}$ scattering cross-section ($2.27 d^2$). 

In practical applied physics, this explains why initiating Hydrogen fusion requires such staggering pressure and temperature (e.g., the core of a star, or a Tokamak reactor). Because the $S_{11}$ cross-section is so compact, the probability of two autonomous Protium tanks successfully colliding their continuous metric fields to induce an inductive topological merge (fusion) is statistically poor without massive kinetic confinement forcing their boundaries to overlap.
