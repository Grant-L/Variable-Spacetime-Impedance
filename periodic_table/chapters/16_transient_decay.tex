\chapter{Transient Resonance Decay}
\label{ch:transient_decay}

\section{Overview: Topological Fission as Impedance Mismatch}
In earlier chapters, we rigorously established that stable nucleonic structures (from $Z=1$ to $Z=14$) maintain physical integrity via synchronized Phase-Locked Mutually Inductive networks ($M_{ij}$). A ``stable isotope'' is simply a network geometry whose aggregate spatial impedance successfully isolates the standing wave topologies from the surrounding vacuum dielectric. 

Radioactive decay, therefore, is not a probabilistic die roll, but a strictly deterministic failure mode in macroscopic Electromagnetic Circuit Theory. When the capacitive tension across a specific topological bridge (like the $\alpha-\alpha$ linkage in Beryllium-8) exceeds the Dielectric Yield Limit ($V_{yield}$), the $M_{ij}$ coupling suffers a fast-transient rupture. This effectively "snaps" the inductive bond, allowing the discrete, semi-stable topologies to violently repel and scatter as lower-energy phase-locked clusters (e.g., $^8\text{Be} \rightarrow 2\alpha$).

\section{Beryllium-8: Alpha-Decay Rupture Limit}
Beryllium-8 ($Z=4, A=8$) is the simplest example of topological $\alpha$-fission. It structurally forms as two distinct tetrahedral $\alpha$-matrices connected via an equatorial bridge. Because both $\alpha$ matrices are inherently stable (each representing a closed $Q \rightarrow \infty$ LC tank), the inter-matrix $M_{bridge}$ coupling carries massive phase tension.

If we pulse the topology with a transient voltage stepping above the local structural limit:
\begin{equation}
    V_{\text{transient}} > V_{\text{critical}} \implies \text{Dielectric Breakdown}
\end{equation}
The cross-coupling parameter $K_{bridge} \rightarrow 0$. Without the binding mutual inductance, the bare capacitive repulsion between the two highly charged $4\alpha$ macro-blocks dominates, ejecting the two Alpha topologies in opposite Euclidean vectors.

\section{Tritium: Beta-Minus Resonant Overload}
Tritium ($^3\text{H}$) represents a distinct failure loop known conventionally as $\beta^-$-decay. Structurally, it forms a 3-nucleon topological loop consisting of 1 Proton node and 2 Neutron nodes. 
As previously derived, a Neutron is merely a composite topology containing an internalized secondary $3_1$ electron knot. The $A=3$ tri-node geometry forces asymmetric inductance coupling across the loop ($M_{pn} \neq M_{nn}$). 

This asymmetry acts as a parasitic resonant drain. As the internal AC flux circulates through the Tritium mesh, energy progressively builds up disproportionately in one of the Neutron nodes until it surpasses the topological boundary limit confining the internalized $3_1$ electron knot. 

Upon reaching $V_{yield}$, the topology violently ejects the electron (along with a reactive back-EMF wave interpreted as an antineutrino, $\bar{\nu}_e$) to stabilize the geometric loop into a symmetric, Phase-Locked $^3\text{He}$ architecture.

\section{FDTD SPICE Simulation of the Decay Transient}
To physically prove this topological mechanism, our engine leverages standard `.TRAN` SPICE step-voltage directives. By plotting the voltage flux across an unstable simulated Beryllium-8 LC network over $0.1\text{ns}$ to $200\text{ns}$, the classical `.cir` netlist demonstrates recursive energy accumulation at the bridging nodes right up until complete unrecoverable state divergence (Decay).
