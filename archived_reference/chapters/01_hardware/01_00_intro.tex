\section{The Shift from Geometry to Hardware}

Theoretical physics has reached a juncture where the mathematical complexity of our models has outpaced our mechanical understanding of the phenomena they describe. For a century, we have accepted geometric abstractions (curved spacetime) and probabilistic outcomes (wavefunctions) as fundamental truths, rather than as sophisticated approximations of an underlying physical reality.

\textbf{Variable Spacetime Impedance (VSI)} is a departure from this trend. It is a framework for the next era of physics—one where the cosmos is understood not as a mathematical ghost, but as a physical, constitutive hardware substrate.

\subsection{The Discrete Amorphous Manifold ($M_A$)}
The central thesis of this work is that the vacuum is a \textbf{Discrete Amorphous Manifold} ($M_{A}$) governed by finite inductive and capacitive limits. By redefining the fundamental constants of nature ($c, G, \hbar, \epsilon_0$) as the bulk engineering properties of this substrate, we move from a \textit{descriptive} physics to an \textit{operational} one.

In this framework, the "laws of physics" are simply the constitutive relations of the hardware:
\begin{itemize}
    \item \textbf{Inertia} is the Back-EMF of the lattice inductance.
    \item \textbf{Gravity} is the impedance-matched refraction of flux.
    \item \textbf{Mass} is a topological defect (knot) stored in the lattice memory.
\end{itemize}