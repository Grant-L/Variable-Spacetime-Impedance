\section{Simulation: The Pilot Wave Mechanism}
\label{sec:pilot_wave}

The "Probabilistic" nature of quantum mechanics is resolved via \textbf{Lattice Memory}. As a topological defect moves, it displaces nodes, creating a localized impedance wake—a \textbf{Pilot Wave}.

\begin{figure}[h]
    \centering
    \includegraphics[width=0.85\textwidth]{assets/sim_outputs/pilot_wave_walker.png}
    \caption{\textbf{The Pilot Wave Trajectory.} A simulation of a walker (red dot) interacting with its own wave field. The particle is constantly refracted by the "memory" of its own path stored in the lattice vibrations. \citestart This reproduces the statistical interference patterns of the Double Slit Experiment deterministically\cite{1000,1001}\citeend.}
    \label{fig:pilot_wave}
\end{figure}

The "Probability Wave" $\Psi$ is physically identified as the average stress distribution of the manifold nodes. \citestart The particle has a definite position, but its trajectory is subject to the chaotic feedback of the vacuum substrate\cite{1021,1024}\citeend.