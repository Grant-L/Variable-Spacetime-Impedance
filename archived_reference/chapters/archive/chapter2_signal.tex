\chapter[The Signal Layer]{The Signal Layer: Variable Impedance and Mass Emergence}
\label{ch:signal_layer}

\section{The Vacuum Dispersion Relation}
In Chapter 1, we established the vacuum as a physical transmission medium composed of discrete LC nodes. We now derive the relationship between signal frequency and propagation velocity, identifying the mechanical origin of rest mass and relativistic scaling as a direct result of hardware bandwidth limitations.

\subsection{Derivation from Discrete Kirchhoff Laws}
Starting from the discrete equations of motion for the \textbf{Discrete Amorphous Manifold ($M_A$)}, we treat the vacuum as a transmission line where each node possesses an intrinsic inductance ($\Lvac$) and capacitance ($\Cvac$):
\begin{align}
    \Lvac \frac{dI_n}{dt} &= V_{n-1} - V_n \\
    \Cvac \frac{dV_n}{dt} &= I_n - I_{n+1}
\end{align}

By substituting a plane-wave solution $V_n = V_0 e^{i(\omega t - nk\lp)}$, we obtain the discrete dispersion relation for the vacuum substrate:
\begin{equation}
    \omega(k) = \frac{2}{\sqrt{\Lvac \Cvac}} \sin\left(\frac{k \lp}{2}\right)
\end{equation}

The \textbf{Group Velocity} ($v_g$), representing the speed of energy propagation through the hardware nodes, is derived as:
\begin{equation}
    v_g = \frac{d\omega}{dk} = \frac{\lp}{\sqrt{\Lvac \Cvac}} \cos\left(\frac{k \lp}{2}\right)
\end{equation}



Defining the global speed limit $c = \lp/\sqrt{\Lvac \Cvac}$, we observe that as the signal frequency $\omega$ approaches the hardware \textbf{Saturation Frequency} ($\Wcut$), the propagation speed $v_g$ drops toward zero.

\subsection{Relativistic Scaling as Bandwidth Limiting}
The familiar Lorentz factor $\gamma = 1/\sqrt{1 - v^2/c^2}$ emerges not from a geometric property of "spacetime," but from the hardware's approach to its Nyquist limit. We rewrite the velocity relation in terms of frequency:
\begin{equation}
    v_g = c \sqrt{1 - \left(\frac{\omega}{\Wcut}\right)^2}
\end{equation}
When a topological defect is accelerated, its internal oscillation frequency $\omega$ increases. As $\omega \to \Wcut$, the hardware becomes increasingly "loaded," requiring more update cycles to process the twist, which macroscopically manifests as a decrease in velocity and an increase in effective mass.

\section{The Origin of Inertia as Back-EMF}
In classical mechanics, inertia is an axiom ($F=ma$). In the SVF framework, inertia is an emergent \textbf{Back-Electromotive Force (B-EMF)}. Because the manifold is inductive ($\Lvac = \mu_0$), any attempt to change the flux state of a node (acceleration) is met with an opposing force generated by the lattice.

\begin{axiombox}[The Inertial B-EMF]
    Inertia is the manifold's inductive resistance to the change in flux density associated with an accelerating topological defect. The "Force" required to move a mass is simply the work required to overcome the lattice B-EMF:
    $$\mathcal{E}_{back} = -\Lvac \frac{d\Phi}{dt}$$
\end{axiombox}

\section{Gravity as Metric Refraction}
In the SVF framework, gravity is not the curvature of an empty void, but a localized gradient in the \textbf{Variable Spacetime Impedance}. Massive bodies "load" the surrounding nodes of $M_A$, increasing the local \textbf{Metric Strain} ($\epsilon$).



This strain alters the local refractive index $\chi$ of the vacuum:
\begin{equation}
    \chi = 1 + \epsilon \approx 1 + \frac{2GM}{rc^2}
\end{equation}
Light passing near a massive body slows down because the nodes in that region are saturated and require more update cycles to process the same flux. This explains the \textbf{Shapiro Delay} and gravitational lensing as simple refraction through a variable-impedance medium ($v = c/\chi$).

\section{Time Dilation as Lattice Latency}
Time is the rate of nodal updates. In a high-impedance zone (high gravity or high velocity), nodes must dedicate a higher percentage of their "hardware cycles" to maintaining the saturation state of the mass. Consequently, fewer cycles are available for external signal propagation. 

An observer in a high-strain zone perceives time moving slower because the hardware is running at a higher \textbf{Lattice Latency}. The "flow" of time is the global clock-rate of the manifold minus the local processing load.

\section{Exercises}
\begin{problembox}[Chapter 2 Signal Dynamics]
\begin{enumerate}
    \item \textbf{The Black Hole Limit}: Prove that at an "Event Horizon," the metric strain $\epsilon$ is sufficient to force the group velocity $v_g \to 0$. 
    \item \textbf{Redshift Derivation}: Show that a signal entering a region of high lattice impedance must undergo a frequency shift to maintain phase continuity across node boundaries.
    \item \textbf{Latency Calculation}: Calculate the additional processing latency (in seconds) incurred by a node at the surface of the Earth compared to a node in deep space.
\end{enumerate}
\end{problembox}