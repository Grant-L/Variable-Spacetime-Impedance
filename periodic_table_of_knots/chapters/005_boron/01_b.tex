\section{Boron ($Z=5$): Spatial Crowding and Trigonal Resonance}
With the addition of the fifth electron in Boron ($Z=5$), the $n=2$ harmonic track is forced to accommodate three separate Trefoil solitons. In the standard orbital model, this marks the abrupt introduction of the $p$-orbital subshell. 

In the AVE Topological hierarchy, $p$-orbitals are mathematically identical to $s$-orbitals; the distinction is merely a geometric consequence of spatial crowding. The three outer Trefoils repel one another's continuous metric strain fields, sliding along the $n=2$ boundary until they hit the lowest energy equilibrium: a strictly $120^\circ$ trigonal planar resonance. The physical topology of the elements natively adapts its internal phase-locking to minimize global elastodynamic tension.

\begin{figure}[h]
    \centering
    \includegraphics[width=0.85\textwidth]{chapters/005_boron/simulations/outputs/boron_strain.png}
    \caption{Boron's topological hierarchy. Three outer solitons achieve minimal impedance by spacing themselves at exact $120^\circ$ intervals along the $n=2$ harmonic.}
\end{figure}