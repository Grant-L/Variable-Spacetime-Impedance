\section{Helium ($^4$He) and Phase-Locked Spin Pairing}

With the foundational ground state of Protium established as a continuous LC standing wave, the framework seamlessly scales to multi-electron atomic structures. The Helium-4 nucleus is an Alpha particle, structurally formed by two protons and two neutrons interlocking into a highly symmetric, deeply bound crystalline tensor core. 

Possessing a nuclear charge of $Z=2$, the induced refractive gradient of the spatial metric is significantly steeper than in Protium. This macroscopic elastodynamic tension dynamically pulls the geometric standing wave boundary inward. Shielded marginally by their mutual topological wake ($Z_{eff} \approx 1.70$), the geometric Bohr radius is squeezed to $r_{He} \approx a_0 / 1.70$.

To satisfy macroscopic electrical neutrality, two $3_1$ Trefoil knots (electrons) must surf this inner track. In standard quantum models, these electrons are permitted to share the $1s$ orbital only by possessing anti-aligned ``spin.'' In the AVE topological hierarchy, spin is physically identified as the topological helicity (chirality) of the knot. 

By possessing opposite topological chiralities and maintaining a strict $180^\circ$ phase-locked antipodal separation along the continuous orbital track, the two Trefoil solitons minimize their mutual spatial strain. Their collective LC wake forms a perfectly balanced continuous standing wave.

\begin{figure}[h]
    \centering
    \includegraphics[width=0.85\textwidth]{chapters/002_helium/simulations/outputs/helium_topological_strain.png}
    \caption{The metric strain field of Helium ($^4$He). Two phase-locked Trefoil knots maintain a $180^\circ$ antipodal orbit, collectively saturating the local $\mathcal{M}_A$ metric and structurally defining the $1s^2$ closed shell without invoking quantum probability amplitudes.}
    \label{fig:helium_strain}
\end{figure}

Crucially, because both solitons are highly localized sources of metric strain ($\propto 1/\sqrt{1-V^2}$), their superimposed spatial tensor footprint pushes the localized $\mathcal{M}_A$ metric along the $1s$ track to the absolute threshold of dielectric saturation ($V_{tot} \to 1.0$). The spatial capacitance diverges, and the local RF impedance drops toward zero. The $1s$ orbital is now physically, structurally, and topologically ``full.''