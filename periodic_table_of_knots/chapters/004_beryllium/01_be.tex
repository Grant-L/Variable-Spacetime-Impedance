\section{Beryllium ($Z=4$): Perpendicular Harmonic Phase-Locking}
In Lithium, the third electron was expelled to the $n=2$ harmonic boundary to prevent dielectric rupture of the $\mathcal{M}_A$ vacuum. In Beryllium ($Z=4$), the increased nuclear gradient pulls this $n=2$ boundary slightly inward. When the fourth macroscopic electron is introduced, it must occupy the $n=2$ track alongside the third electron. 

To prevent their localized spatial wakes from inducing an Axiom 4 impedance mismatch, the two outer Trefoil knots naturally assume an antipodal ($180^\circ$) separation. Crucially, to avoid passing through the dense metric wake generated by the highly saturated $1s^2$ inner pair, the $2s^2$ electrons phase-lock perpendicularly ($90^\circ$ offset) to the inner shell's axis of resonance. This classical spatial self-organization computationally guarantees structural stability without invoking statistical exchange-correlation limits.

\begin{figure}[h]
    \centering
    \includegraphics[width=0.85\textwidth]{chapters/004_beryllium/simulations/outputs/beryllium_strain.png}
    \caption{The continuous metric strain of Beryllium. The outer $n=2$ pair maintains an antipodal phase-lock, physically offset from the inner $1s$ pair's topological wake.}
\end{figure}